%% Generated by Sphinx.
\def\sphinxdocclass{report}
\documentclass[letterpaper,10pt,spanish]{sphinxmanual}
\ifdefined\pdfpxdimen
   \let\sphinxpxdimen\pdfpxdimen\else\newdimen\sphinxpxdimen
\fi \sphinxpxdimen=.75bp\relax

\usepackage[utf8]{inputenc}
\ifdefined\DeclareUnicodeCharacter
 \ifdefined\DeclareUnicodeCharacterAsOptional
  \DeclareUnicodeCharacter{"00A0}{\nobreakspace}
  \DeclareUnicodeCharacter{"2500}{\sphinxunichar{2500}}
  \DeclareUnicodeCharacter{"2502}{\sphinxunichar{2502}}
  \DeclareUnicodeCharacter{"2514}{\sphinxunichar{2514}}
  \DeclareUnicodeCharacter{"251C}{\sphinxunichar{251C}}
  \DeclareUnicodeCharacter{"2572}{\textbackslash}
 \else
  \DeclareUnicodeCharacter{00A0}{\nobreakspace}
  \DeclareUnicodeCharacter{2500}{\sphinxunichar{2500}}
  \DeclareUnicodeCharacter{2502}{\sphinxunichar{2502}}
  \DeclareUnicodeCharacter{2514}{\sphinxunichar{2514}}
  \DeclareUnicodeCharacter{251C}{\sphinxunichar{251C}}
  \DeclareUnicodeCharacter{2572}{\textbackslash}
 \fi
\fi
\usepackage{cmap}
\usepackage[T1]{fontenc}
\usepackage{amsmath,amssymb,amstext}
\usepackage{babel}
\usepackage{times}
\usepackage[Sonny]{fncychap}
\usepackage[dontkeepoldnames]{sphinx}

\usepackage{geometry}

% Include hyperref last.
\usepackage{hyperref}
% Fix anchor placement for figures with captions.
\usepackage{hypcap}% it must be loaded after hyperref.
% Set up styles of URL: it should be placed after hyperref.
\urlstyle{same}

\addto\captionsspanish{\renewcommand{\figurename}{Figura}}
\addto\captionsspanish{\renewcommand{\tablename}{Tabla}}
\addto\captionsspanish{\renewcommand{\literalblockname}{Lista}}

\addto\captionsspanish{\renewcommand{\literalblockcontinuedname}{continued from previous page}}
\addto\captionsspanish{\renewcommand{\literalblockcontinuesname}{continues on next page}}

\addto\extrasspanish{\def\pageautorefname{página}}

\setcounter{tocdepth}{1}



\title{Creemos en la Web Documentation}
\date{24 de agosto de 2018}
\release{1.0.0}
\author{Lara Garbero Tais, Mariano Guerra}
\newcommand{\sphinxlogo}{\vbox{}}
\renewcommand{\releasename}{Versión}
\makeindex

\begin{document}
\ifnum\catcode`\"=\active\shorthandoff{"}\fi
\maketitle
\sphinxtableofcontents
\phantomsection\label{\detokenize{index::doc}}



\chapter{Introducción}
\label{\detokenize{introduccion:introduccion}}\label{\detokenize{introduccion::doc}}\label{\detokenize{introduccion:creemos-en-la-web}}
Este es el primero de lo que espero sera una serie de artículos sobre como aprender a hacer paginas web para personas sin ningún conocimiento previo de tecnología o programación.

Si sabes de tecnología recomendaselo a la mayor cantidad de personas posible, no sabemos cuantos grandes diseñadores y programadores web se encuentran escondidos por ahí.

Si te recomendaron esto y pensás que no es para vos, la cuestión es que si es para vos, y si algo es confuso no es tu culpa, es miá, así que contactame y decime que parte no esta clara así mejoramos esta guiá para todos.

Este articulo es una introducción a la herramienta que vamos a usar para los siguientes ejemplos, un proyecto llamado \sphinxhref{https://thimble.mozilla.org/es/}{Mozilla Thimble}, el cual facilita el proceso de crear, compartir y remixar paginas de otros.

Para los que prefieren ver videos, acá hay uno con el mismo contenido que este articulo:



Lo primero que vamos a necesitar hacer es crear una cuenta en Mozilla Thimble, visitando \sphinxurl{https://thimble.mozilla.org/es/}, vamos a ver algo similar a la siguiente imagen:

\begin{figure}[htbp]
\centering

\noindent\sphinxincludegraphics{{01-landing}.png}
\end{figure}


\section{Creando una cuenta}
\label{\detokenize{introduccion:creando-una-cuenta}}
\begin{figure}[htbp]
\centering
\capstart

\noindent\sphinxincludegraphics{{02-sign-up}.png}
\caption{Click en \sphinxstylestrong{Crea una cuenta} en la parte superior derecha}\label{\detokenize{introduccion:id1}}\end{figure}

Llena el formulario para crear una nueva cuenta:
\begin{itemize}
\item {} 
Nombre de usuario

\item {} 
Tu dirección de correo electrónico

\item {} 
Contraseña (al menos una mayúscula, una minúscula y un numero)

\end{itemize}

\begin{figure}[htbp]
\centering

\noindent\sphinxincludegraphics{{03-sign-up-form}.png}
\end{figure}

Luego de crear la cuenta debería ir directamente a la pagina principal de tu
cuenta.


\section{Iniciando sesión}
\label{\detokenize{introduccion:iniciando-sesion}}
Si en otro momento querés acceder de nuevo desde la pagina principal de Mozilla
Thimble:

\begin{figure}[htbp]
\centering
\capstart

\noindent\sphinxincludegraphics{{04-sign-in}.png}
\caption{Click en \sphinxstylestrong{Inicia sesión}}\label{\detokenize{introduccion:id2}}\end{figure}

\begin{figure}[htbp]
\centering
\capstart

\noindent\sphinxincludegraphics{{05-sign-in-form}.png}
\caption{Llena el formulario con tu usuario y contraseña}\label{\detokenize{introduccion:id3}}\end{figure}


\section{Espacio de trabajo principal}
\label{\detokenize{introduccion:espacio-de-trabajo-principal}}
Espacio de trabajo vacío una vez que ingresamos a nuestra cuenta, para trabajar
en un nuevo proyecto hay que hacer click en el botón verde \sphinxstylestrong{Crear un nuevo proyecto}

\begin{figure}[htbp]
\centering

\noindent\sphinxincludegraphics{{06-workspace}.png}
\end{figure}

El espacio de trabajo de un proyecto tiene 3 paneles, de izquierda a derecha:
\begin{itemize}
\item {} 
Explorador de archivos de proyecto

\item {} 
Editor de código

\item {} 
Vista previa del proyecto

\end{itemize}

\begin{figure}[htbp]
\centering

\noindent\sphinxincludegraphics{{07-new-project}.png}
\end{figure}

Para hacer visible el proyecto en la web y compartirlo con un enlace, hacemos
click en el botón blanco \sphinxstylestrong{Publicar} arriba a la derecha en el entorno de
trabajo de nuestro proyecto.

\begin{figure}[htbp]
\centering

\noindent\sphinxincludegraphics{{08-publish-project}.png}
\end{figure}

Ingresamos la descripción de nuestro proyecto y hacemos click en el botón verde
\sphinxstylestrong{Publicar}

\begin{figure}[htbp]
\centering

\noindent\sphinxincludegraphics{{09-publish-project}.png}
\end{figure}

Al finalizar el proceso podemos hacer click o copiar el enlace a nuestro
proyecto publico en la web.

Si hacemos click en el botón rojo \sphinxstylestrong{Eliminar la versión publicada*} nuestro
proyecto ya no sera accesible en la web.

\begin{figure}[htbp]
\centering

\noindent\sphinxincludegraphics{{10-publish-project}.png}
\end{figure}

\begin{figure}[htbp]
\centering
\capstart

\noindent\sphinxincludegraphics{{11-public-project}.png}
\caption{Nuestro proyecto de prueba en la web.}\label{\detokenize{introduccion:id4}}\end{figure}

Si hacemos click en el botón verde \sphinxstylestrong{Remix} de cualquier proyecto publicado
con Thimble, vamos a poder acceder a el desde nuestro entorno de trabajo, hacer
nuestros cambios y publicar nuestros cambios.

\begin{figure}[htbp]
\centering

\noindent\sphinxincludegraphics{{12-remix-project}.png}
\end{figure}

Podemos buscar proyectos para modificar en la página principal de Thimble

\begin{figure}[htbp]
\centering

\noindent\sphinxincludegraphics{{13-remix-landing}.png}
\end{figure}

Si vamos a la sección \sphinxtitleref{Mezcla un proyecto para comenzar...} podemos buscar por tema o filtrar por etiqueta, si seleccionamos la etiqueta \sphinxstylestrong{html} podemos ver proyectos
que usen principalmente HTML, buscamos el proyecto \sphinxtitleref{Keep Calm and Carry On} y clickeamos en el botón verde \sphinxstylestrong{Mezclar}.

\begin{figure}[htbp]
\centering

\noindent\sphinxincludegraphics{{14-remix-landing}.png}
\end{figure}

Esto hará una copia del proyecto en nuestro usuario y abrirá el editor.

\begin{figure}[htbp]
\centering

\noindent\sphinxincludegraphics{{15-remix-change}.png}
\end{figure}

Podemos cambiar el texto del poster a

\fvset{hllines={, ,}}%
\begin{sphinxVerbatim}[commandchars=\\\{\}]
Calma\PYG{p}{\PYGZlt{}}\PYG{n+nt}{br}\PYG{p}{\PYGZgt{}}
\PYG{p}{\PYGZlt{}}\PYG{n+nt}{span}\PYG{p}{\PYGZgt{}}y\PYG{p}{\PYGZlt{}}\PYG{p}{/}\PYG{n+nt}{span}\PYG{p}{\PYGZgt{}}
crea\PYG{p}{\PYGZlt{}}\PYG{n+nt}{br}\PYG{p}{\PYGZgt{}} en la web
\end{sphinxVerbatim}

\begin{figure}[htbp]
\centering

\noindent\sphinxincludegraphics{{16-remix-change}.png}
\end{figure}

Y luego republicarlo con nuestros cambios


\chapter{HTML}
\label{\detokenize{html::doc}}\label{\detokenize{html:html}}
Crear paginas web involucra normalmente 3 "lenguajes" (formas que tenemos los humanos de decirle a una computadora que queremos que haga).

El único necesario es el que vamos a cubrir en esta sección: HTML

HTML permite describir el contenido de una pagina web que un programa especial
llamado normalmente navegador web interpreta y muestra en la pantalla.

El contenido de un archivo HTML es texto con un formato especial, pero que podemos
inspeccionar y editar con cualquier editor de texto.

Normalmente cuando la gente quiere crear un documento de texto que tenga cierto
formato usa un programa como Microsoft Word, Google Docs o Libre Office Writer,
en estos programas le indicamos con acciones al editor que partes del texto tienen
que formato, algo como lo siguiente:

\begin{figure}[htbp]
\centering

\noindent\sphinxincludegraphics{{01-editor-visual}.gif}
\end{figure}

Para indicarle al editor que una linea es un titulo, la seleccionamos, demarcando
los limites y luego le indicamos que queremos que se muestre como un titulo.

Los párrafos simplemente los separamos con saltos de linea, si queremos texto
en negrita o itálica, al igual que con el titulo, seleccionamos e indicamos que
formato queremos para la selección.

HTML es un lenguaje que inicialmente fue pensado para escribir en editores de
texto y luego evoluciono para ser generado por programas, que toman datos de
una base de datos y generan como salida texto en formato HTML.

Si bien existen algunos editores visuales para HTML, normalmente este se edita
a mano con editores de texto, y eso es lo que vamos a hacer.


\section{La pagina mas simple del mundo}
\label{\detokenize{html:la-pagina-mas-simple-del-mundo}}
Escribí lo siguiente en la barra de dirección de tu navegador:

\fvset{hllines={, ,}}%
\begin{sphinxVerbatim}[commandchars=\\\{\}]
data:text/html, hola mundo
\end{sphinxVerbatim}

Felicitaciones! acabas de crear tu primera pagina web!

Normalmente lo que escribimos en la barra de direcciones del navegador es la
ubicación de la pagina web, la primera parte (http: o https:) le indica al
navegador que lo que sigue es la ubicación de la pagina que queremos ver y que
la puede solicitar usando el "protocolo" %
\begin{footnote}[1]\sphinxAtStartFootnote
Un protocolo es un acuerdo entre dos o mas partes que establece la forma en la que se van a comunicar, en este caso establece como un navegador solicita un documento HTML y como el otro se lo envía.
%
\end{footnote} HTTP (Protocolo de Transferencia
de Hiper Texto), lo que sigue es la dirección de la pagina, similar a la
ubicación de un archivo en tu computadora, pero empezando con la pagina web que
contiene la pagina.

En este caso le decimos que le vamos a indicar la pagina directamente, y que
esta en formato HTML, luego escribimos el contenido de la misma.

Si bien no es una forma ideal de crear paginas web, a veces es útil para tareas
especificas, por ejemplo:

Selector de colores:

\fvset{hllines={, ,}}%
\begin{sphinxVerbatim}[commandchars=\\\{\}]
data:text/html, \PYG{p}{\PYGZlt{}}\PYG{n+nt}{input} \PYG{n+na}{type}\PYG{o}{=}\PYG{l+s}{\PYGZdq{}color\PYGZdq{}}\PYG{p}{\PYGZgt{}}
\end{sphinxVerbatim}

Calendario:

\fvset{hllines={, ,}}%
\begin{sphinxVerbatim}[commandchars=\\\{\}]
data:text/html, \PYG{p}{\PYGZlt{}}\PYG{n+nt}{input} \PYG{n+na}{type}\PYG{o}{=}\PYG{l+s}{\PYGZdq{}date\PYGZdq{}}\PYG{p}{\PYGZgt{}}
\end{sphinxVerbatim}

Bloc de notas:

\fvset{hllines={, ,}}%
\begin{sphinxVerbatim}[commandchars=\\\{\}]
data:text/html, \PYG{p}{\PYGZlt{}}\PYG{n+nt}{body} \PYG{n+na}{contenteditable} \PYG{n+na}{style}\PYG{o}{=}\PYG{l+s}{\PYGZdq{}max\PYGZhy{}width:60rem;margin:0 auto;padding:4rem;\PYGZdq{}}\PYG{p}{\PYGZgt{}}bloc de notas
\end{sphinxVerbatim}

No te preocupes por la parte \sphinxtitleref{style="max-width:60rem;margin:0 auto;padding:4rem;"} eso es el segundo lenguaje que vamos a ver en la próxima sección.


\section{Nuestra primera pagina web}
\label{\detokenize{html:nuestra-primera-pagina-web}}
Nuestra primera pagina web va a intentar replicar el ejemplo de Google Docs que
vimos mas arriba, con esta pagina vamos a cubrir los principales elementos de
HTML.

Empezamos creando un nuevo proyecto en Thimble, si tenes dudas de como hacerlo
revisa la sección anterior que contiene una introducción a Thimble.

El nuevo proyecto comienza con un contenido por defecto:

\fvset{hllines={, ,}}%
\begin{sphinxVerbatim}[commandchars=\\\{\}]
\PYG{c+cp}{\PYGZlt{}!DOCTYPE html\PYGZgt{}}
\PYG{p}{\PYGZlt{}}\PYG{n+nt}{html}\PYG{p}{\PYGZgt{}}
  \PYG{p}{\PYGZlt{}}\PYG{n+nt}{head}\PYG{p}{\PYGZgt{}}
        \PYG{p}{\PYGZlt{}}\PYG{n+nt}{meta} \PYG{n+na}{charset}\PYG{o}{=}\PYG{l+s}{\PYGZdq{}utf\PYGZhy{}8\PYGZdq{}}\PYG{p}{\PYGZgt{}}
        \PYG{p}{\PYGZlt{}}\PYG{n+nt}{meta} \PYG{n+na}{name}\PYG{o}{=}\PYG{l+s}{\PYGZdq{}viewport\PYGZdq{}} \PYG{n+na}{content}\PYG{o}{=}\PYG{l+s}{\PYGZdq{}width=device\PYGZhy{}width, initial\PYGZhy{}scale=1\PYGZdq{}}\PYG{p}{\PYGZgt{}}
        \PYG{p}{\PYGZlt{}}\PYG{n+nt}{title}\PYG{p}{\PYGZgt{}}Made with Thimble\PYG{p}{\PYGZlt{}}\PYG{p}{/}\PYG{n+nt}{title}\PYG{p}{\PYGZgt{}}
        \PYG{p}{\PYGZlt{}}\PYG{n+nt}{link} \PYG{n+na}{rel}\PYG{o}{=}\PYG{l+s}{\PYGZdq{}stylesheet\PYGZdq{}} \PYG{n+na}{href}\PYG{o}{=}\PYG{l+s}{\PYGZdq{}style.css\PYGZdq{}}\PYG{p}{\PYGZgt{}}
  \PYG{p}{\PYGZlt{}}\PYG{p}{/}\PYG{n+nt}{head}\PYG{p}{\PYGZgt{}}
  \PYG{p}{\PYGZlt{}}\PYG{n+nt}{body}\PYG{p}{\PYGZgt{}}
        \PYG{p}{\PYGZlt{}}\PYG{n+nt}{h1}\PYG{p}{\PYGZgt{}}Welcome to Thimble\PYG{p}{\PYGZlt{}}\PYG{p}{/}\PYG{n+nt}{h1}\PYG{p}{\PYGZgt{}}
        \PYG{p}{\PYGZlt{}}\PYG{n+nt}{p}\PYG{p}{\PYGZgt{}}
          Make something \PYG{p}{\PYGZlt{}}\PYG{n+nt}{strong}\PYG{p}{\PYGZgt{}}amazing\PYG{p}{\PYGZlt{}}\PYG{p}{/}\PYG{n+nt}{strong}\PYG{p}{\PYGZgt{}} with the web!
        \PYG{p}{\PYGZlt{}}\PYG{p}{/}\PYG{n+nt}{p}\PYG{p}{\PYGZgt{}}
  \PYG{p}{\PYGZlt{}}\PYG{p}{/}\PYG{n+nt}{body}\PYG{p}{\PYGZgt{}}
\PYG{p}{\PYGZlt{}}\PYG{p}{/}\PYG{n+nt}{html}\PYG{p}{\PYGZgt{}}
\end{sphinxVerbatim}

Lo vamos a borrar y escribir el siguiente texto:

\fvset{hllines={, ,}}%
\begin{sphinxVerbatim}[commandchars=\\\{\}]
Esto es un título

Esto es un párrafo, la siguiente palabra es en negrita, la siguiente en itálica

Esto es otro párrafo

Una lista no ordenada:

Manzana
Durazno
Banana

Una lista ordenada:

Uno
Dos
Tres
\end{sphinxVerbatim}

El resultado debería ser similar al siguiente:

\begin{figure}[htbp]
\centering

\noindent\sphinxincludegraphics{{02-plain-text}.png}
\end{figure}

Si la vista previa no se actualiza automáticamente podes hacer click en el
botón de refrescar vista previa.

Como podemos ver la vista previa muestra todo el texto junto y sin formato, algo así:

\fvset{hllines={, ,}}%
\begin{sphinxVerbatim}[commandchars=\\\{\}]
\PYG{n}{Esto} \PYG{n}{es} \PYG{n}{un} \PYG{n}{título} \PYG{n}{Esto} \PYG{n}{es} \PYG{n}{un} \PYG{n}{párrafo}\PYG{p}{,} \PYG{n}{la} \PYG{n}{siguiente} \PYG{n}{palabra} \PYG{n}{es} \PYG{n}{en} \PYG{n}{negrita}\PYG{p}{,} \PYG{n}{la} \PYG{n}{siguiente} \PYG{n}{en} \PYG{n}{itálica} \PYG{n}{Esto} \PYG{n}{es} \PYG{n}{otro} \PYG{n}{párrafo} \PYG{n}{Una} \PYG{n}{lista} \PYG{n}{no} \PYG{n}{ordenada}\PYG{p}{:} \PYG{n}{Manzana} \PYG{n}{Durazno} \PYG{n}{Banana} \PYG{n}{Una} \PYG{n}{lista} \PYG{n}{ordenada}\PYG{p}{:} \PYG{n}{Uno} \PYG{n}{Dos} \PYG{n}{Tres}
\end{sphinxVerbatim}

Sin formato es esperable ya que no le indicamos ninguno, pero porque todo junto?

Porque HTML "junta" todos los espacios y saltos de lineas a un solo espacio, si queremos especificar algo distinto lo tenemos que hacer explícitamente.

Esto nos va a permitir estructurar el documento con claridad y estructura sabiendo
que el navegador no va a reflejar nuestros espacios y saltos de linea en el documento final.

Ya tenemos el contenido de nuestra primera pagina web, es un avance! pero no
es muy diferente a un documento de texto, como agregamos el formato?

En el ejemplo de Google Docs mas arriba para indicar el formato de las
distintas partes lo que hacíamos era indicar el principio y el fin de la
sección a la que le queriamos aplicar formato con el mouse y luego
seleccionamos una operación en el menú para indicarle que tipo de formato
queremos.

En HTML es casi igual, salvo que no tenemos mouse ni menú :)

Como HTML es un formato de texto, tenemos que trabajar con lo que tenemos,
pero la forma es muy parecida, primero indicamos el principio y final de una
sección a la que le queremos aplicar una operación y luego le indicamos cual es
esa operación.

En HTML el indicador para el inicio de una sección es \textless{}\textgreater{} y el indicador de fin es \textless{}/\textgreater{}

Pero eso no funciona porque todavía tenemos que indicar que operación aplicar al texto entre \textless{}\textgreater{} y \textless{}/\textgreater{}, para eso escribimos la operación entre el \textless{} y el \textgreater{}.

Empecemos con el titulo, si notas en la animación de Google Docs, la operación se llama \sphinxstylestrong{Heading 1}, es medio largo para escribir todo eso cada vez que queremos un titulo de nivel 1, así que lo acortamos a \sphinxtitleref{h1}.

Cambiemos

\fvset{hllines={, ,}}%
\begin{sphinxVerbatim}[commandchars=\\\{\}]
Esto es un título
\end{sphinxVerbatim}

Por

\fvset{hllines={, ,}}%
\begin{sphinxVerbatim}[commandchars=\\\{\}]
\PYG{p}{\PYGZlt{}}\PYG{n+nt}{h1}\PYG{p}{\PYGZgt{}}Esto es un título\PYG{p}{\PYGZlt{}}\PYG{p}{/}\PYG{n+nt}{h1}\PYG{p}{\PYGZgt{}}
\end{sphinxVerbatim}

El resultado debería verse algo así:

\begin{figure}[htbp]
\centering

\noindent\sphinxincludegraphics{{03-h1}.png}
\end{figure}

Tenemos el titulo!

Y ese es el primer \sphinxstylestrong{tag} (etiqueta en inglés) que aprendimos:
\begin{description}
\item[{h1}] \leavevmode
Formatea el texto delimitado como un titulo de nivel 1

\end{description}

Si digo titulo de nivel 1 podemos imaginarnos que hay mas niveles, es como
el indice de un libro, las secciones tienen sub secciones y cada sección tiene
un titulo de un nivel mas alto.

En HTML tenemos 6 niveles: h1, h2, h3, h4, h5, h6.

Continuemos.

La siguiente linea dice:

\fvset{hllines={, ,}}%
\begin{sphinxVerbatim}[commandchars=\\\{\}]
Esto es un párrafo, la siguiente palabra es en negrita, la siguiente en itálica
\end{sphinxVerbatim}

Como le indicamos que es un párrafo? como con el titulo, lo rodeamos de una etiqueta de apertura y una de cierre y le indicamos que es un párrafo, pero de nuevo, escribir párrafo en español o ingles por cada párrafo es bastante largo así que lo vamos a abreviar a \sphinxstylestrong{p}

\fvset{hllines={, ,}}%
\begin{sphinxVerbatim}[commandchars=\\\{\}]
\PYG{p}{\PYGZlt{}}\PYG{n+nt}{p}\PYG{p}{\PYGZgt{}}Esto es un párrafo, la siguiente palabra es en negrita, la siguiente en itálica\PYG{p}{\PYGZlt{}}\PYG{p}{/}\PYG{n+nt}{p}\PYG{p}{\PYGZgt{}}
\end{sphinxVerbatim}

El resultado debería verse similar al siguiente:

\begin{figure}[htbp]
\centering

\noindent\sphinxincludegraphics{{04-p}.png}
\end{figure}

Notaras que ahora el primer párrafo tiene espacio con respecto al titulo y al
resto del texto, ahora hacemos lo mismo con los siguientes párrafos:

\fvset{hllines={, ,}}%
\begin{sphinxVerbatim}[commandchars=\\\{\}]
\PYG{p}{\PYGZlt{}}\PYG{n+nt}{h1}\PYG{p}{\PYGZgt{}}Esto es un título\PYG{p}{\PYGZlt{}}\PYG{p}{/}\PYG{n+nt}{h1}\PYG{p}{\PYGZgt{}}

\PYG{p}{\PYGZlt{}}\PYG{n+nt}{p}\PYG{p}{\PYGZgt{}}Esto es un párrafo, la siguiente palabra es en negrita, la siguiente en itálica\PYG{p}{\PYGZlt{}}\PYG{p}{/}\PYG{n+nt}{p}\PYG{p}{\PYGZgt{}}

\PYG{p}{\PYGZlt{}}\PYG{n+nt}{p}\PYG{p}{\PYGZgt{}}Esto es otro párrafo\PYG{p}{\PYGZlt{}}\PYG{p}{/}\PYG{n+nt}{p}\PYG{p}{\PYGZgt{}}

\PYG{p}{\PYGZlt{}}\PYG{n+nt}{p}\PYG{p}{\PYGZgt{}}
  Una lista no ordenada:

Manzana
Durazno
Banana
\PYG{p}{\PYGZlt{}}\PYG{p}{/}\PYG{n+nt}{p}\PYG{p}{\PYGZgt{}}

\PYG{p}{\PYGZlt{}}\PYG{n+nt}{p}\PYG{p}{\PYGZgt{}}
Una lista ordenada:

Uno
Dos
Tres
\PYG{p}{\PYGZlt{}}\PYG{p}{/}\PYG{n+nt}{p}\PYG{p}{\PYGZgt{}}
\end{sphinxVerbatim}

Como veras los saltos de linea y los espacios no afectan el formato final.

\begin{figure}[htbp]
\centering

\noindent\sphinxincludegraphics{{05-p}.png}
\end{figure}

Tenemos el titulo y los párrafos, ya casi que podemos escribir un cuento en HTML :)

Pero un poco mas de formato no vendría mal, sigamos con negrita e itálica.

No hay nada de magia, es igual a las anteriores, rodeamos la sección que queremos
formatear y le indicamos que formato queremos aplicarle.

A ver si te podes imaginar que identificador lleva negrita (\sphinxstylestrong{b} old en ingles)
e itálica (\sphinxstylestrong{i} talic en ingles)?

\fvset{hllines={, ,}}%
\begin{sphinxVerbatim}[commandchars=\\\{\}]
\PYG{p}{\PYGZlt{}}\PYG{n+nt}{p}\PYG{p}{\PYGZgt{}}Esto es un párrafo, la siguiente palabra es en \PYG{p}{\PYGZlt{}}\PYG{n+nt}{b}\PYG{p}{\PYGZgt{}}negrita\PYG{p}{\PYGZlt{}}\PYG{p}{/}\PYG{n+nt}{b}\PYG{p}{\PYGZgt{}}, la siguiente en \PYG{p}{\PYGZlt{}}\PYG{n+nt}{i}\PYG{p}{\PYGZgt{}}itálica\PYG{p}{\PYGZlt{}}\PYG{p}{/}\PYG{n+nt}{i}\PYG{p}{\PYGZgt{}}\PYG{p}{\PYGZlt{}}\PYG{p}{/}\PYG{n+nt}{p}\PYG{p}{\PYGZgt{}}
\end{sphinxVerbatim}

El resultado debería verse algo así:

\begin{figure}[htbp]
\centering

\noindent\sphinxincludegraphics{{06-bi}.png}
\end{figure}

Como podes ver en el párrafo, podemos tener tags/etiquetas dentro de otros tags/etiquetas, en este caso tenemos tags negrita e itálica dentro del tag de párrafo.

Ya casi estamos!

Solo faltan las listas, en este caso tenemos que indicar dos cosas distintas con tags:
\begin{enumerate}
\item {} 
Que queremos una lista
\begin{itemize}
\item {} 
Ordenada: Numerada

\item {} 
No Ordenada: Sin Numeración

\end{itemize}

\item {} 
Cuales son los elementos de la lista

\end{enumerate}

Empecemos con la lista no ordenada, en ingles \sphinxstylestrong{u} nordered \sphinxstylestrong{l} ist, ya te podes imaginar como se identifica el tag:

\fvset{hllines={, ,}}%
\begin{sphinxVerbatim}[commandchars=\\\{\}]
\PYG{p}{\PYGZlt{}}\PYG{n+nt}{ul}\PYG{p}{\PYGZgt{}}
Manzana
Durazno
Banana
\PYG{p}{\PYGZlt{}}\PYG{p}{/}\PYG{n+nt}{ul}\PYG{p}{\PYGZgt{}}
\end{sphinxVerbatim}

Pero esto no es suficiente, todavía le tenemos que decir cuales son los elementos de la lista (\sphinxstylestrong{l} ist \sphinxstylestrong{i} tem en ingles):

\fvset{hllines={, ,}}%
\begin{sphinxVerbatim}[commandchars=\\\{\}]
\PYG{p}{\PYGZlt{}}\PYG{n+nt}{ul}\PYG{p}{\PYGZgt{}}
        \PYG{p}{\PYGZlt{}}\PYG{n+nt}{li}\PYG{p}{\PYGZgt{}}Manzana\PYG{p}{\PYGZlt{}}\PYG{p}{/}\PYG{n+nt}{li}\PYG{p}{\PYGZgt{}}
        \PYG{p}{\PYGZlt{}}\PYG{n+nt}{li}\PYG{p}{\PYGZgt{}}Durazno\PYG{p}{\PYGZlt{}}\PYG{p}{/}\PYG{n+nt}{li}\PYG{p}{\PYGZgt{}}
        \PYG{p}{\PYGZlt{}}\PYG{n+nt}{li}\PYG{p}{\PYGZgt{}}Banana\PYG{p}{\PYGZlt{}}\PYG{p}{/}\PYG{n+nt}{li}\PYG{p}{\PYGZgt{}}
\PYG{p}{\PYGZlt{}}\PYG{p}{/}\PYG{n+nt}{ul}\PYG{p}{\PYGZgt{}}
\end{sphinxVerbatim}

De nuevo tenemos tags dentro de otro tag, en este caso el tag \sphinxstylestrong{li} (list
item) dentro del tag \sphinxstylestrong{ul} (unordered list)

Para la lista ordenada es igual, pero en lugar de indicar que queremos una lista
no ordenada, le indicamos que queremos una ordenada (\sphinxstylestrong{o} rdered \sphinxstylestrong{l} list en ingles)

\fvset{hllines={, ,}}%
\begin{sphinxVerbatim}[commandchars=\\\{\}]
\PYG{p}{\PYGZlt{}}\PYG{n+nt}{ol}\PYG{p}{\PYGZgt{}}
        \PYG{p}{\PYGZlt{}}\PYG{n+nt}{li}\PYG{p}{\PYGZgt{}}Uno\PYG{p}{\PYGZlt{}}\PYG{p}{/}\PYG{n+nt}{li}\PYG{p}{\PYGZgt{}}
        \PYG{p}{\PYGZlt{}}\PYG{n+nt}{li}\PYG{p}{\PYGZgt{}}Dos\PYG{p}{\PYGZlt{}}\PYG{p}{/}\PYG{n+nt}{li}\PYG{p}{\PYGZgt{}}
        \PYG{p}{\PYGZlt{}}\PYG{n+nt}{li}\PYG{p}{\PYGZgt{}}Tres\PYG{p}{\PYGZlt{}}\PYG{p}{/}\PYG{n+nt}{li}\PYG{p}{\PYGZgt{}}
\PYG{p}{\PYGZlt{}}\PYG{p}{/}\PYG{n+nt}{ol}\PYG{p}{\PYGZgt{}}
\end{sphinxVerbatim}

El código completo:

\fvset{hllines={, ,}}%
\begin{sphinxVerbatim}[commandchars=\\\{\}]
\PYG{p}{\PYGZlt{}}\PYG{n+nt}{h1}\PYG{p}{\PYGZgt{}}Esto es un título\PYG{p}{\PYGZlt{}}\PYG{p}{/}\PYG{n+nt}{h1}\PYG{p}{\PYGZgt{}}

\PYG{p}{\PYGZlt{}}\PYG{n+nt}{p}\PYG{p}{\PYGZgt{}}Esto es un párrafo, la siguiente palabra es en \PYG{p}{\PYGZlt{}}\PYG{n+nt}{b}\PYG{p}{\PYGZgt{}}negrita\PYG{p}{\PYGZlt{}}\PYG{p}{/}\PYG{n+nt}{b}\PYG{p}{\PYGZgt{}}, la siguiente en \PYG{p}{\PYGZlt{}}\PYG{n+nt}{i}\PYG{p}{\PYGZgt{}}itálica\PYG{p}{\PYGZlt{}}\PYG{p}{/}\PYG{n+nt}{i}\PYG{p}{\PYGZgt{}}\PYG{p}{\PYGZlt{}}\PYG{p}{/}\PYG{n+nt}{p}\PYG{p}{\PYGZgt{}}

\PYG{p}{\PYGZlt{}}\PYG{n+nt}{p}\PYG{p}{\PYGZgt{}}Esto es otro párrafo\PYG{p}{\PYGZlt{}}\PYG{p}{/}\PYG{n+nt}{p}\PYG{p}{\PYGZgt{}}

\PYG{p}{\PYGZlt{}}\PYG{n+nt}{p}\PYG{p}{\PYGZgt{}}
  Una lista no ordenada:
\PYG{p}{\PYGZlt{}}\PYG{p}{/}\PYG{n+nt}{p}\PYG{p}{\PYGZgt{}}

\PYG{p}{\PYGZlt{}}\PYG{n+nt}{ul}\PYG{p}{\PYGZgt{}}
  \PYG{p}{\PYGZlt{}}\PYG{n+nt}{li}\PYG{p}{\PYGZgt{}}Manzana\PYG{p}{\PYGZlt{}}\PYG{p}{/}\PYG{n+nt}{li}\PYG{p}{\PYGZgt{}}
  \PYG{p}{\PYGZlt{}}\PYG{n+nt}{li}\PYG{p}{\PYGZgt{}}Durazno\PYG{p}{\PYGZlt{}}\PYG{p}{/}\PYG{n+nt}{li}\PYG{p}{\PYGZgt{}}
  \PYG{p}{\PYGZlt{}}\PYG{n+nt}{li}\PYG{p}{\PYGZgt{}}Banana\PYG{p}{\PYGZlt{}}\PYG{p}{/}\PYG{n+nt}{li}\PYG{p}{\PYGZgt{}}
\PYG{p}{\PYGZlt{}}\PYG{p}{/}\PYG{n+nt}{ul}\PYG{p}{\PYGZgt{}}


\PYG{p}{\PYGZlt{}}\PYG{n+nt}{p}\PYG{p}{\PYGZgt{}}Una lista ordenada:\PYG{p}{\PYGZlt{}}\PYG{p}{/}\PYG{n+nt}{p}\PYG{p}{\PYGZgt{}}

\PYG{p}{\PYGZlt{}}\PYG{n+nt}{ol}\PYG{p}{\PYGZgt{}}
  \PYG{p}{\PYGZlt{}}\PYG{n+nt}{li}\PYG{p}{\PYGZgt{}}Uno\PYG{p}{\PYGZlt{}}\PYG{p}{/}\PYG{n+nt}{li}\PYG{p}{\PYGZgt{}}
  \PYG{p}{\PYGZlt{}}\PYG{n+nt}{li}\PYG{p}{\PYGZgt{}}Dos\PYG{p}{\PYGZlt{}}\PYG{p}{/}\PYG{n+nt}{li}\PYG{p}{\PYGZgt{}}
  \PYG{p}{\PYGZlt{}}\PYG{n+nt}{li}\PYG{p}{\PYGZgt{}}Tres\PYG{p}{\PYGZlt{}}\PYG{p}{/}\PYG{n+nt}{li}\PYG{p}{\PYGZgt{}}
\PYG{p}{\PYGZlt{}}\PYG{p}{/}\PYG{n+nt}{ol}\PYG{p}{\PYGZgt{}}
\end{sphinxVerbatim}

Que debería verse similar a esto:

\begin{figure}[htbp]
\centering

\noindent\sphinxincludegraphics{{07-lists}.png}
\end{figure}

Y con esto replicamos el documento de Google Docs y sabemos un poco mas cual es
el contenido de los archivos .doc, .odt y similares, solo que esos están en
formato binario (unos y ceros) ya que las computadoras los prefieren en lugar
del formato texto, que preferimos los humanos :)


\chapter{HTML atributos y meta datos}
\label{\detokenize{html-atributos-y-metadatos::doc}}\label{\detokenize{html-atributos-y-metadatos:html-atributos-y-meta-datos}}
En la sección anterior vimos como crear una pagina web que replicaba un
documento de Google Docs con formato básico, en esta sección vamos a aprender
sobre dos conceptos con nombres tenebrosos pero que como todo lo tenebroso
cuando se los entiende resulta ser bastante inofensivo.

Necesitamos saber sobre atributos y meta datos para poder resolver dos problemas
en nuestra pagina web:
\begin{enumerate}
\item {} 
Como le digo al navegador cual es el titulo de mi pagina web sin que lo muestre en el contenido del documento?

\item {} 
Como le indico información extra sobre un tag sin que se vea en el documento? por ejemplo, como separo el texto de un enlace y la dirección a la que apunta?

\end{enumerate}

Para el primero vamos a usar meta datos, que son datos sobre el documento pero
que no son parte del contenido del mismo.

Para lo segundo vamos a usar atributos, que son datos sobre los tags pero no
son parte del contenido.


\section{Meta datos}
\label{\detokenize{html-atributos-y-metadatos:meta-datos}}
Para separar los meta datos del documento del contenido hacemos lo usual,
rodeamos los metadatos en un tag y el contenido en otro.

En la terminología HTML un documento esta separado en dos partes, la \sphinxstyleemphasis{cabeza} o
\sphinxstyleemphasis{cabecera} (\sphinxstylestrong{head} en ingles), la cual contiene los datos sobre el documento
y el cuerpo (\sphinxstylestrong{body}), el cual tiene el contenido del mismo.

Como estos dos son parte de un documento HTML, ambos están contenidos en el tag
raíz de todo documento HTML, el tag html.

Veamos el documento correctamente estructurado mas simple que podamos tener:

\fvset{hllines={, ,}}%
\begin{sphinxVerbatim}[commandchars=\\\{\}]
\PYG{p}{\PYGZlt{}}\PYG{n+nt}{html}\PYG{p}{\PYGZgt{}}
    \PYG{p}{\PYGZlt{}}\PYG{n+nt}{head}\PYG{p}{\PYGZgt{}}\PYG{p}{\PYGZlt{}}\PYG{p}{/}\PYG{n+nt}{head}\PYG{p}{\PYGZgt{}}
    \PYG{p}{\PYGZlt{}}\PYG{n+nt}{body}\PYG{p}{\PYGZgt{}}\PYG{p}{\PYGZlt{}}\PYG{p}{/}\PYG{n+nt}{body}\PYG{p}{\PYGZgt{}}
\PYG{p}{\PYGZlt{}}\PYG{p}{/}\PYG{n+nt}{html}\PYG{p}{\PYGZgt{}}
\end{sphinxVerbatim}

Si queremos definir el titulo del documento, el cual se va a mostrar en la
barra de títulos del navegador cuando la pagina tiene el foco y también en la
pestaña de la pagina en el navegador, usamos el tag \sphinxstylestrong{title}.

\fvset{hllines={, ,}}%
\begin{sphinxVerbatim}[commandchars=\\\{\}]
\PYG{p}{\PYGZlt{}}\PYG{n+nt}{html}\PYG{p}{\PYGZgt{}}
    \PYG{p}{\PYGZlt{}}\PYG{n+nt}{head}\PYG{p}{\PYGZgt{}}
        \PYG{p}{\PYGZlt{}}\PYG{n+nt}{title}\PYG{p}{\PYGZgt{}}Mi Pagina\PYG{p}{\PYGZlt{}}\PYG{p}{/}\PYG{n+nt}{title}\PYG{p}{\PYGZgt{}}
    \PYG{p}{\PYGZlt{}}\PYG{p}{/}\PYG{n+nt}{head}\PYG{p}{\PYGZgt{}}

    \PYG{p}{\PYGZlt{}}\PYG{n+nt}{body}\PYG{p}{\PYGZgt{}}
    \PYG{p}{\PYGZlt{}}\PYG{p}{/}\PYG{n+nt}{body}\PYG{p}{\PYGZgt{}}
\PYG{p}{\PYGZlt{}}\PYG{p}{/}\PYG{n+nt}{html}\PYG{p}{\PYGZgt{}}
\end{sphinxVerbatim}

Otro tag que es recomendable poner en el documento si no queremos tener
problemas con como el navegador muestra las tildes y la ñ es un tag que le
indica como interpretar tildes y otros caracteres especiales, utf-8 es un
estándar que define como interpretar caracteres especiales y es el mas usado
actualmente, simplemente copialo en todos tus documentos para evitar dolores de
cabeza.

\fvset{hllines={, ,}}%
\begin{sphinxVerbatim}[commandchars=\\\{\}]
\PYG{p}{\PYGZlt{}}\PYG{n+nt}{html}\PYG{p}{\PYGZgt{}}
    \PYG{p}{\PYGZlt{}}\PYG{n+nt}{head}\PYG{p}{\PYGZgt{}}
        \PYG{p}{\PYGZlt{}}\PYG{n+nt}{meta} \PYG{n+na}{charset}\PYG{o}{=}\PYG{l+s}{\PYGZdq{}utf\PYGZhy{}8\PYGZdq{}}\PYG{p}{\PYGZgt{}}
        \PYG{p}{\PYGZlt{}}\PYG{n+nt}{title}\PYG{p}{\PYGZgt{}}Mi Pagina\PYG{p}{\PYGZlt{}}\PYG{p}{/}\PYG{n+nt}{title}\PYG{p}{\PYGZgt{}}
    \PYG{p}{\PYGZlt{}}\PYG{p}{/}\PYG{n+nt}{head}\PYG{p}{\PYGZgt{}}

    \PYG{p}{\PYGZlt{}}\PYG{n+nt}{body}\PYG{p}{\PYGZgt{}}
    \PYG{p}{\PYGZlt{}}\PYG{p}{/}\PYG{n+nt}{body}\PYG{p}{\PYGZgt{}}
\PYG{p}{\PYGZlt{}}\PYG{p}{/}\PYG{n+nt}{html}\PYG{p}{\PYGZgt{}}
\end{sphinxVerbatim}

Para terminar, dado que HTML es un estándar que ha evolucionado en el tiempo,
el navegador soporta múltiples versiones y si no se le indica la versión intenta
adivinar.

Normalmente adivina bien, pero para facilitarle el trabajo y evitar confusiones
podemos ser buenas personas e indicarselo explícitamente.

\fvset{hllines={, ,}}%
\begin{sphinxVerbatim}[commandchars=\\\{\}]
\PYG{c+cp}{\PYGZlt{}!doctype html\PYGZgt{}}
\PYG{p}{\PYGZlt{}}\PYG{n+nt}{html}\PYG{p}{\PYGZgt{}}
    \PYG{p}{\PYGZlt{}}\PYG{n+nt}{head}\PYG{p}{\PYGZgt{}}
        \PYG{p}{\PYGZlt{}}\PYG{n+nt}{meta} \PYG{n+na}{charset}\PYG{o}{=}\PYG{l+s}{\PYGZdq{}utf\PYGZhy{}8\PYGZdq{}}\PYG{p}{\PYGZgt{}}
        \PYG{p}{\PYGZlt{}}\PYG{n+nt}{title}\PYG{p}{\PYGZgt{}}Mi Pagina\PYG{p}{\PYGZlt{}}\PYG{p}{/}\PYG{n+nt}{title}\PYG{p}{\PYGZgt{}}
    \PYG{p}{\PYGZlt{}}\PYG{p}{/}\PYG{n+nt}{head}\PYG{p}{\PYGZgt{}}

    \PYG{p}{\PYGZlt{}}\PYG{n+nt}{body}\PYG{p}{\PYGZgt{}}
    \PYG{p}{\PYGZlt{}}\PYG{p}{/}\PYG{n+nt}{body}\PYG{p}{\PYGZgt{}}
\PYG{p}{\PYGZlt{}}\PYG{p}{/}\PYG{n+nt}{html}\PYG{p}{\PYGZgt{}}
\end{sphinxVerbatim}

La primera linea le indica que el documento es de tipo HTML 5, la ultima versión
del estándar.

No te preocupes en memorizar estos tags, yo simplemente copio de algún documento
anterior las partes comunes, nunca empiezo de cero :)

La evolución de HTML y su énfasis en mantener compatibilidad hace que todavía
puedas visitar la primera pagina web publicada en 1991: \sphinxurl{http://info.cern.ch/hypertext/WWW/TheProject.html}


\section{Atributos}
\label{\detokenize{html-atributos-y-metadatos:atributos}}
Ahora al segundo problema, como indicamos información sobre un tag que no es
el contenido principal, por ejemplo, si queremos crear un enlace a \sphinxurl{https://google.com} pero queremos que el texto del enlace diga "Google", como hacemos esto?

Quizás lo notaste en algunos de los ejemplos hasta ahora.

Para hacer eso usamos lo que se llaman atributos, que son información extra que
agregamos a un tag, la mayoría son opcionales, de manera que los vamos agregando
y aprendiendo a medida que los vamos necesitando.

Veamos como resolver el problema del enlace.

\fvset{hllines={, ,}}%
\begin{sphinxVerbatim}[commandchars=\\\{\}]
\PYG{p}{\PYGZlt{}}\PYG{n+nt}{a} \PYG{n+na}{href}\PYG{o}{=}\PYG{l+s}{\PYGZdq{}https://google.com\PYGZdq{}}\PYG{p}{\PYGZgt{}}Google\PYG{p}{\PYGZlt{}}\PYG{p}{/}\PYG{n+nt}{a}\PYG{p}{\PYGZgt{}}
\end{sphinxVerbatim}

Que se ve así:



Los atributos van luego del identificador del tag de apertura, separados por espacios, primero va el nombre del atributo, luego un \sphinxtitleref{=} y luego el valor, normalmente entre comillas.

Por si no te diste cuenta, acabamos de aprender un nuevo tag, el tag \sphinxstylestrong{a} (de
\sphinxstylestrong{a} nchor que significa ancla en ingles) con su atributo \sphinxstylestrong{href} (de \sphinxstylestrong{h}
ypertext \sphinxstylestrong{ref} erence en ingles).

Ya que aprendimos un tag nuevo, aprendamos otro similar y muy útil, el tag para mostrar imágenes %
\begin{footnote}[1]\sphinxAtStartFootnote
“DOF Example” by Owen Byrne is licensed under CC BY 2.0
%
\end{footnote}:

\fvset{hllines={, ,}}%
\begin{sphinxVerbatim}[commandchars=\\\{\}]
\PYG{p}{\PYGZlt{}}\PYG{n+nt}{img} \PYG{n+na}{src}\PYG{o}{=}\PYG{l+s}{\PYGZdq{}http://marianoguerra.org/galleries/cew/3/cube.jpg\PYGZdq{}}\PYG{p}{\PYGZgt{}}
\end{sphinxVerbatim}

Así se ve:



Agreguemos mas atributos, uno para el tooltip (title), y dos para definir el alto (height) y el ancho (width).

\fvset{hllines={, ,}}%
\begin{sphinxVerbatim}[commandchars=\\\{\}]
\PYG{p}{\PYGZlt{}}\PYG{n+nt}{img} \PYG{n+na}{title}\PYG{o}{=}\PYG{l+s}{\PYGZdq{}un cubo\PYGZdq{}} \PYG{n+na}{width}\PYG{o}{=}\PYG{l+s}{\PYGZdq{}200\PYGZdq{}} \PYG{n+na}{height}\PYG{o}{=}\PYG{l+s}{\PYGZdq{}200\PYGZdq{}} \PYG{n+na}{src}\PYG{o}{=}\PYG{l+s}{\PYGZdq{}http://marianoguerra.org/galleries/cew/3/cube.jpg\PYGZdq{}}\PYG{p}{\PYGZgt{}}
\end{sphinxVerbatim}

El resultado:



Ahora pongamos todo junto en una pagina:

\fvset{hllines={, ,}}%
\begin{sphinxVerbatim}[commandchars=\\\{\}]
\PYG{c+cp}{\PYGZlt{}!doctype html\PYGZgt{}}
\PYG{p}{\PYGZlt{}}\PYG{n+nt}{html}\PYG{p}{\PYGZgt{}}
    \PYG{p}{\PYGZlt{}}\PYG{n+nt}{head}\PYG{p}{\PYGZgt{}}
        \PYG{p}{\PYGZlt{}}\PYG{n+nt}{meta} \PYG{n+na}{charset}\PYG{o}{=}\PYG{l+s}{\PYGZdq{}utf\PYGZhy{}8\PYGZdq{}}\PYG{p}{\PYGZgt{}}
        \PYG{p}{\PYGZlt{}}\PYG{n+nt}{title}\PYG{p}{\PYGZgt{}}Mi Pagina\PYG{p}{\PYGZlt{}}\PYG{p}{/}\PYG{n+nt}{title}\PYG{p}{\PYGZgt{}}
    \PYG{p}{\PYGZlt{}}\PYG{p}{/}\PYG{n+nt}{head}\PYG{p}{\PYGZgt{}}

    \PYG{p}{\PYGZlt{}}\PYG{n+nt}{body}\PYG{p}{\PYGZgt{}}
        \PYG{p}{\PYGZlt{}}\PYG{n+nt}{p}\PYG{p}{\PYGZgt{}}Un link a \PYG{p}{\PYGZlt{}}\PYG{n+nt}{a} \PYG{n+na}{href}\PYG{o}{=}\PYG{l+s}{\PYGZdq{}https://google.com\PYGZdq{}}\PYG{p}{\PYGZgt{}}Google\PYG{p}{\PYGZlt{}}\PYG{p}{/}\PYG{n+nt}{a}\PYG{p}{\PYGZgt{}}\PYG{p}{\PYGZlt{}}\PYG{p}{/}\PYG{n+nt}{p}\PYG{p}{\PYGZgt{}}

        \PYG{p}{\PYGZlt{}}\PYG{n+nt}{p}\PYG{p}{\PYGZgt{}}Una imagen:\PYG{p}{\PYGZlt{}}\PYG{p}{/}\PYG{n+nt}{p}\PYG{p}{\PYGZgt{}}

        \PYG{p}{\PYGZlt{}}\PYG{n+nt}{img} \PYG{n+na}{title}\PYG{o}{=}\PYG{l+s}{\PYGZdq{}un cubo\PYGZdq{}} \PYG{n+na}{width}\PYG{o}{=}\PYG{l+s}{\PYGZdq{}200\PYGZdq{}} \PYG{n+na}{height}\PYG{o}{=}\PYG{l+s}{\PYGZdq{}200\PYGZdq{}} \PYG{n+na}{src}\PYG{o}{=}\PYG{l+s}{\PYGZdq{}http://marianoguerra.org/galleries/cew/3/cube.jpg\PYGZdq{}}\PYG{p}{\PYGZgt{}}
    \PYG{p}{\PYGZlt{}}\PYG{p}{/}\PYG{n+nt}{body}\PYG{p}{\PYGZgt{}}
\PYG{p}{\PYGZlt{}}\PYG{p}{/}\PYG{n+nt}{html}\PYG{p}{\PYGZgt{}}
\end{sphinxVerbatim}

Que se ve algo así:

\begin{figure}[htbp]
\centering

\noindent\sphinxincludegraphics{{01-page}.png}
\end{figure}

Notar que subí la imagen al proyecto, haciendo click en el icono de
archivo y seleccionando \sphinxtitleref{Subir un archivo...}, la dirección de la imagen es
simplemente el nombre del archivo ya que esta en el mismo lugar que la pagina
que la muestra.

\begin{figure}[htbp]
\centering

\noindent\sphinxincludegraphics{{02-upload}.png}
\end{figure}


\chapter{CSS y cosas por el estilo}
\label{\detokenize{css-y-cosas-por-el-estilo::doc}}\label{\detokenize{css-y-cosas-por-el-estilo:css-y-cosas-por-el-estilo}}
En la sección previa vimos como usar HTML para definir la estructura y el
contenido de nuestras paginas web, pero seamos sinceros, el aspecto deja
bastante que desear.

Para poder hacer nuestras paginas mas agradables vamos a aprender el segundo
lenguaje: CSS, el cual sirve para describir el aspecto y disposición del
contenido que definamos con HTML.


\section{Apagando las luces}
\label{\detokenize{css-y-cosas-por-el-estilo:apagando-las-luces}}
Vamos a empezar cambiando un poco de color, que pasa si queremos hacer una
pagina "invertida"? donde el fondo es oscuro y el texto claro.

Creemos un proyecto nuevo con el siguiente contenido:

\fvset{hllines={, ,}}%
\begin{sphinxVerbatim}[commandchars=\\\{\}]
\PYG{c+cp}{\PYGZlt{}!doctype html\PYGZgt{}}
\PYG{p}{\PYGZlt{}}\PYG{n+nt}{html}\PYG{p}{\PYGZgt{}}
  \PYG{p}{\PYGZlt{}}\PYG{n+nt}{head}\PYG{p}{\PYGZgt{}}
        \PYG{p}{\PYGZlt{}}\PYG{n+nt}{meta} \PYG{n+na}{charset}\PYG{o}{=}\PYG{l+s}{\PYGZdq{}utf\PYGZhy{}8\PYGZdq{}}\PYG{p}{\PYGZgt{}}
        \PYG{p}{\PYGZlt{}}\PYG{n+nt}{title}\PYG{p}{\PYGZgt{}}Mi Pagina\PYG{p}{\PYGZlt{}}\PYG{p}{/}\PYG{n+nt}{title}\PYG{p}{\PYGZgt{}}
  \PYG{p}{\PYGZlt{}}\PYG{p}{/}\PYG{n+nt}{head}\PYG{p}{\PYGZgt{}}

  \PYG{p}{\PYGZlt{}}\PYG{n+nt}{body}\PYG{p}{\PYGZgt{}}
        \PYG{p}{\PYGZlt{}}\PYG{n+nt}{h1}\PYG{p}{\PYGZgt{}}Esto es un título\PYG{p}{\PYGZlt{}}\PYG{p}{/}\PYG{n+nt}{h1}\PYG{p}{\PYGZgt{}}

        \PYG{p}{\PYGZlt{}}\PYG{n+nt}{p}\PYG{p}{\PYGZgt{}}Esto es un párrafo, la siguiente palabra es en \PYG{p}{\PYGZlt{}}\PYG{n+nt}{b}\PYG{p}{\PYGZgt{}}negrita\PYG{p}{\PYGZlt{}}\PYG{p}{/}\PYG{n+nt}{b}\PYG{p}{\PYGZgt{}}, la siguiente en \PYG{p}{\PYGZlt{}}\PYG{n+nt}{i}\PYG{p}{\PYGZgt{}}itálica\PYG{p}{\PYGZlt{}}\PYG{p}{/}\PYG{n+nt}{i}\PYG{p}{\PYGZgt{}}\PYG{p}{\PYGZlt{}}\PYG{p}{/}\PYG{n+nt}{p}\PYG{p}{\PYGZgt{}}

        \PYG{p}{\PYGZlt{}}\PYG{n+nt}{p}\PYG{p}{\PYGZgt{}}Esto es otro párrafo\PYG{p}{\PYGZlt{}}\PYG{p}{/}\PYG{n+nt}{p}\PYG{p}{\PYGZgt{}}

        \PYG{p}{\PYGZlt{}}\PYG{n+nt}{p}\PYG{p}{\PYGZgt{}}
          Una lista no ordenada:
        \PYG{p}{\PYGZlt{}}\PYG{p}{/}\PYG{n+nt}{p}\PYG{p}{\PYGZgt{}}

        \PYG{p}{\PYGZlt{}}\PYG{n+nt}{ul}\PYG{p}{\PYGZgt{}}
          \PYG{p}{\PYGZlt{}}\PYG{n+nt}{li}\PYG{p}{\PYGZgt{}}Manzana\PYG{p}{\PYGZlt{}}\PYG{p}{/}\PYG{n+nt}{li}\PYG{p}{\PYGZgt{}}
          \PYG{p}{\PYGZlt{}}\PYG{n+nt}{li}\PYG{p}{\PYGZgt{}}Durazno\PYG{p}{\PYGZlt{}}\PYG{p}{/}\PYG{n+nt}{li}\PYG{p}{\PYGZgt{}}
          \PYG{p}{\PYGZlt{}}\PYG{n+nt}{li}\PYG{p}{\PYGZgt{}}Banana\PYG{p}{\PYGZlt{}}\PYG{p}{/}\PYG{n+nt}{li}\PYG{p}{\PYGZgt{}}
        \PYG{p}{\PYGZlt{}}\PYG{p}{/}\PYG{n+nt}{ul}\PYG{p}{\PYGZgt{}}


        \PYG{p}{\PYGZlt{}}\PYG{n+nt}{p}\PYG{p}{\PYGZgt{}}Una lista ordenada:\PYG{p}{\PYGZlt{}}\PYG{p}{/}\PYG{n+nt}{p}\PYG{p}{\PYGZgt{}}

        \PYG{p}{\PYGZlt{}}\PYG{n+nt}{ol}\PYG{p}{\PYGZgt{}}
          \PYG{p}{\PYGZlt{}}\PYG{n+nt}{li}\PYG{p}{\PYGZgt{}}Uno\PYG{p}{\PYGZlt{}}\PYG{p}{/}\PYG{n+nt}{li}\PYG{p}{\PYGZgt{}}
          \PYG{p}{\PYGZlt{}}\PYG{n+nt}{li}\PYG{p}{\PYGZgt{}}Dos\PYG{p}{\PYGZlt{}}\PYG{p}{/}\PYG{n+nt}{li}\PYG{p}{\PYGZgt{}}
          \PYG{p}{\PYGZlt{}}\PYG{n+nt}{li}\PYG{p}{\PYGZgt{}}Tres\PYG{p}{\PYGZlt{}}\PYG{p}{/}\PYG{n+nt}{li}\PYG{p}{\PYGZgt{}}
        \PYG{p}{\PYGZlt{}}\PYG{p}{/}\PYG{n+nt}{ol}\PYG{p}{\PYGZgt{}}

        \PYG{p}{\PYGZlt{}}\PYG{n+nt}{p}\PYG{p}{\PYGZgt{}}Un link a \PYG{p}{\PYGZlt{}}\PYG{n+nt}{a} \PYG{n+na}{href}\PYG{o}{=}\PYG{l+s}{\PYGZdq{}https://google.com\PYGZdq{}}\PYG{p}{\PYGZgt{}}Google\PYG{p}{\PYGZlt{}}\PYG{p}{/}\PYG{n+nt}{a}\PYG{p}{\PYGZgt{}}\PYG{p}{\PYGZlt{}}\PYG{p}{/}\PYG{n+nt}{p}\PYG{p}{\PYGZgt{}}

        \PYG{p}{\PYGZlt{}}\PYG{n+nt}{p}\PYG{p}{\PYGZgt{}}Una imagen:\PYG{p}{\PYGZlt{}}\PYG{p}{/}\PYG{n+nt}{p}\PYG{p}{\PYGZgt{}}

        \PYG{p}{\PYGZlt{}}\PYG{n+nt}{img} \PYG{n+na}{title}\PYG{o}{=}\PYG{l+s}{\PYGZdq{}un cubo\PYGZdq{}} \PYG{n+na}{width}\PYG{o}{=}\PYG{l+s}{\PYGZdq{}200\PYGZdq{}} \PYG{n+na}{height}\PYG{o}{=}\PYG{l+s}{\PYGZdq{}200\PYGZdq{}}  \PYG{n+na}{src}\PYG{o}{=}\PYG{l+s}{\PYGZdq{}cube.jpg\PYGZdq{}}\PYG{p}{\PYGZgt{}}
  \PYG{p}{\PYGZlt{}}\PYG{p}{/}\PYG{n+nt}{body}\PYG{p}{\PYGZgt{}}
\PYG{p}{\PYGZlt{}}\PYG{p}{/}\PYG{n+nt}{html}\PYG{p}{\PYGZgt{}}
\end{sphinxVerbatim}

Para empezar tenemos que hacer que el fondo sea oscuro, es decir el cuerpo
(body) de la pagina debe tener el color de fondo negro.

Como le decimos a un tag cosas que no son su contenido?

Con atributos, en este caso el atributo es bastante poderoso, y su valor es un lenguaje en si mismo!

Cambiamos el tag de apertura del cuerpo de

\fvset{hllines={, ,}}%
\begin{sphinxVerbatim}[commandchars=\\\{\}]
\PYG{p}{\PYGZlt{}}\PYG{n+nt}{body}\PYG{p}{\PYGZgt{}}
\end{sphinxVerbatim}

A:

\fvset{hllines={, ,}}%
\begin{sphinxVerbatim}[commandchars=\\\{\}]
\PYG{p}{\PYGZlt{}}\PYG{n+nt}{body} \PYG{n+na}{style}\PYG{o}{=}\PYG{l+s}{\PYGZdq{}background\PYGZhy{}color: black;\PYGZdq{}}\PYG{p}{\PYGZgt{}}
\end{sphinxVerbatim}

El contenido del atributo style es uno o mas pares de valores, cada par separado
por \sphinxstylestrong{;}, cada par a su lado izquierdo tiene la "llave" y del lado derecho el "valor".

Este concepto de pares llave valor se va a repetir mucho en el mundo informático
así que si prestas atención lo vas a ver en muchos lados, para empezar, ya los
conocés de los atributos de los tags :)

Los pares llave/valor en CSS sirven para especificar distintas propiedades del
tag en el que están definidos.

En este caso le estamos diciendo que el color de fondo \sphinxhref{https://developer.mozilla.org/es/docs/Web/CSS/background-color}{background-color} (background significa
fondo en ingles) tiene el valor negro (black).

El resultado, si bien es lo que queremos, nos agrega un nuevo desafío:

\begin{figure}[htbp]
\centering
\capstart

\noindent\sphinxincludegraphics{{01-bg-color}.png}
\caption{El texto no se lee!}\label{\detokenize{css-y-cosas-por-el-estilo:id1}}\end{figure}

Ahora tenemos que hacer que el texto sea un color claro, empezamos con el titulo:

\fvset{hllines={, ,}}%
\begin{sphinxVerbatim}[commandchars=\\\{\}]
\PYG{p}{\PYGZlt{}}\PYG{n+nt}{h1} \PYG{n+na}{style}\PYG{o}{=}\PYG{l+s}{\PYGZdq{}color: white;\PYGZdq{}}\PYG{p}{\PYGZgt{}}Esto es un título\PYG{p}{\PYGZlt{}}\PYG{p}{/}\PYG{n+nt}{h1}\PYG{p}{\PYGZgt{}}
\end{sphinxVerbatim}

Para definir el color del contenido de un tag (no el fondo), usamos la llave
\sphinxhref{https://developer.mozilla.org/es/docs/Web/CSS/color}{color} (que por suerte no tenemos que traducir :)

Continuamos con el primer párrafo:

\fvset{hllines={, ,}}%
\begin{sphinxVerbatim}[commandchars=\\\{\}]
\PYG{p}{\PYGZlt{}}\PYG{n+nt}{p} \PYG{n+na}{style}\PYG{o}{=}\PYG{l+s}{\PYGZdq{}color: white;\PYGZdq{}}\PYG{p}{\PYGZgt{}}Esto es un párrafo, la siguiente palabra es en \PYG{p}{\PYGZlt{}}\PYG{n+nt}{b}\PYG{p}{\PYGZgt{}}negrita\PYG{p}{\PYGZlt{}}\PYG{p}{/}\PYG{n+nt}{b}\PYG{p}{\PYGZgt{}}, la siguiente en \PYG{p}{\PYGZlt{}}\PYG{n+nt}{i}\PYG{p}{\PYGZgt{}}itálica\PYG{p}{\PYGZlt{}}\PYG{p}{/}\PYG{n+nt}{i}\PYG{p}{\PYGZgt{}}\PYG{p}{\PYGZlt{}}\PYG{p}{/}\PYG{n+nt}{p}\PYG{p}{\PYGZgt{}}
\end{sphinxVerbatim}

Y el resultado se ve algo así:

\begin{figure}[htbp]
\centering

\noindent\sphinxincludegraphics{{02-color}.png}
\end{figure}

Si sos como yo, ya estarás pensando: "Esto va a llevar un buen tiempo y mucha repetición!".

Si, eso pensé yo.

Pero quizás notaste que la palabra \sphinxstylestrong{negrita} y la palabra \sphinxstyleemphasis{itálica} ahora
también tienen color blanco.

Esto no es un accidente, cuando un valor se define en CSS para un tag, los tags
descendientes "heredan" ese valor si tiene sentido, el color de fondo y el
color del texto por suerte son unos de ellos.

Ahora bien, donde podríamos poner el color de texto para hacer el menor esfuerzo posible?

En el mismo lugar que definimos el color de fondo.

\fvset{hllines={, ,}}%
\begin{sphinxVerbatim}[commandchars=\\\{\}]
\PYG{p}{\PYGZlt{}}\PYG{n+nt}{body} \PYG{n+na}{style}\PYG{o}{=}\PYG{l+s}{\PYGZdq{}background\PYGZhy{}color: black; color: white;\PYGZdq{}}\PYG{p}{\PYGZgt{}}
\end{sphinxVerbatim}

El resultado es lo que esperábamos:

\begin{figure}[htbp]
\centering

\noindent\sphinxincludegraphics{{03-color}.png}
\end{figure}


\section{Gustos específicos}
\label{\detokenize{css-y-cosas-por-el-estilo:gustos-especificos}}
Ahora digamos que se nos ocurre que queremos que la palabra \sphinxstylestrong{negrita},
\sphinxstyleemphasis{itálica} y los elementos impares de las listas tienen que tener fondo blanco y
texto rojo.

Intentemos lo:

\fvset{hllines={, ,}}%
\begin{sphinxVerbatim}[commandchars=\\\{\}]
\PYG{c+cp}{\PYGZlt{}!doctype html\PYGZgt{}}
\PYG{p}{\PYGZlt{}}\PYG{n+nt}{html}\PYG{p}{\PYGZgt{}}
  \PYG{p}{\PYGZlt{}}\PYG{n+nt}{head}\PYG{p}{\PYGZgt{}}
        \PYG{p}{\PYGZlt{}}\PYG{n+nt}{meta} \PYG{n+na}{charset}\PYG{o}{=}\PYG{l+s}{\PYGZdq{}utf\PYGZhy{}8\PYGZdq{}}\PYG{p}{\PYGZgt{}}
        \PYG{p}{\PYGZlt{}}\PYG{n+nt}{title}\PYG{p}{\PYGZgt{}}Mi Pagina\PYG{p}{\PYGZlt{}}\PYG{p}{/}\PYG{n+nt}{title}\PYG{p}{\PYGZgt{}}
  \PYG{p}{\PYGZlt{}}\PYG{p}{/}\PYG{n+nt}{head}\PYG{p}{\PYGZgt{}}

  \PYG{p}{\PYGZlt{}}\PYG{n+nt}{body} \PYG{n+na}{style}\PYG{o}{=}\PYG{l+s}{\PYGZdq{}background\PYGZhy{}color: black; color: white;\PYGZdq{}}\PYG{p}{\PYGZgt{}}
        \PYG{p}{\PYGZlt{}}\PYG{n+nt}{h1}\PYG{p}{\PYGZgt{}}Esto es un título\PYG{p}{\PYGZlt{}}\PYG{p}{/}\PYG{n+nt}{h1}\PYG{p}{\PYGZgt{}}

        \PYG{p}{\PYGZlt{}}\PYG{n+nt}{p}\PYG{p}{\PYGZgt{}}Esto es un párrafo, la siguiente palabra es en \PYG{p}{\PYGZlt{}}\PYG{n+nt}{b} \PYG{n+na}{style}\PYG{o}{=}\PYG{l+s}{\PYGZdq{}background\PYGZhy{}color: white; color: red;\PYGZdq{}}\PYG{p}{\PYGZgt{}}negrita\PYG{p}{\PYGZlt{}}\PYG{p}{/}\PYG{n+nt}{b}\PYG{p}{\PYGZgt{}}, la siguiente en \PYG{p}{\PYGZlt{}}\PYG{n+nt}{i} \PYG{n+na}{style}\PYG{o}{=}\PYG{l+s}{\PYGZdq{}background\PYGZhy{}color: white; color: red;\PYGZdq{}}\PYG{p}{\PYGZgt{}}itálica\PYG{p}{\PYGZlt{}}\PYG{p}{/}\PYG{n+nt}{i}\PYG{p}{\PYGZgt{}}\PYG{p}{\PYGZlt{}}\PYG{p}{/}\PYG{n+nt}{p}\PYG{p}{\PYGZgt{}}

        \PYG{p}{\PYGZlt{}}\PYG{n+nt}{p}\PYG{p}{\PYGZgt{}}Esto es otro párrafo\PYG{p}{\PYGZlt{}}\PYG{p}{/}\PYG{n+nt}{p}\PYG{p}{\PYGZgt{}}

        \PYG{p}{\PYGZlt{}}\PYG{n+nt}{p}\PYG{p}{\PYGZgt{}}
          Una lista no ordenada:
        \PYG{p}{\PYGZlt{}}\PYG{p}{/}\PYG{n+nt}{p}\PYG{p}{\PYGZgt{}}

        \PYG{p}{\PYGZlt{}}\PYG{n+nt}{ul}\PYG{p}{\PYGZgt{}}
          \PYG{p}{\PYGZlt{}}\PYG{n+nt}{li} \PYG{n+na}{style}\PYG{o}{=}\PYG{l+s}{\PYGZdq{}background\PYGZhy{}color: white; color: red;\PYGZdq{}}\PYG{p}{\PYGZgt{}}Manzana\PYG{p}{\PYGZlt{}}\PYG{p}{/}\PYG{n+nt}{li}\PYG{p}{\PYGZgt{}}
          \PYG{p}{\PYGZlt{}}\PYG{n+nt}{li}\PYG{p}{\PYGZgt{}}Durazno\PYG{p}{\PYGZlt{}}\PYG{p}{/}\PYG{n+nt}{li}\PYG{p}{\PYGZgt{}}
          \PYG{p}{\PYGZlt{}}\PYG{n+nt}{li} \PYG{n+na}{style}\PYG{o}{=}\PYG{l+s}{\PYGZdq{}background\PYGZhy{}color: white; color: red;\PYGZdq{}}\PYG{p}{\PYGZgt{}}Banana\PYG{p}{\PYGZlt{}}\PYG{p}{/}\PYG{n+nt}{li}\PYG{p}{\PYGZgt{}}
        \PYG{p}{\PYGZlt{}}\PYG{p}{/}\PYG{n+nt}{ul}\PYG{p}{\PYGZgt{}}


        \PYG{p}{\PYGZlt{}}\PYG{n+nt}{p}\PYG{p}{\PYGZgt{}}Una lista ordenada:\PYG{p}{\PYGZlt{}}\PYG{p}{/}\PYG{n+nt}{p}\PYG{p}{\PYGZgt{}}

        \PYG{p}{\PYGZlt{}}\PYG{n+nt}{ol}\PYG{p}{\PYGZgt{}}
          \PYG{p}{\PYGZlt{}}\PYG{n+nt}{li} \PYG{n+na}{style}\PYG{o}{=}\PYG{l+s}{\PYGZdq{}background\PYGZhy{}color: white; color: red;\PYGZdq{}}\PYG{p}{\PYGZgt{}}Uno\PYG{p}{\PYGZlt{}}\PYG{p}{/}\PYG{n+nt}{li}\PYG{p}{\PYGZgt{}}
          \PYG{p}{\PYGZlt{}}\PYG{n+nt}{li}\PYG{p}{\PYGZgt{}}Dos\PYG{p}{\PYGZlt{}}\PYG{p}{/}\PYG{n+nt}{li}\PYG{p}{\PYGZgt{}}
          \PYG{p}{\PYGZlt{}}\PYG{n+nt}{li} \PYG{n+na}{style}\PYG{o}{=}\PYG{l+s}{\PYGZdq{}background\PYGZhy{}color: white; color: red;\PYGZdq{}}\PYG{p}{\PYGZgt{}}Tres\PYG{p}{\PYGZlt{}}\PYG{p}{/}\PYG{n+nt}{li}\PYG{p}{\PYGZgt{}}
        \PYG{p}{\PYGZlt{}}\PYG{p}{/}\PYG{n+nt}{ol}\PYG{p}{\PYGZgt{}}

        \PYG{p}{\PYGZlt{}}\PYG{n+nt}{p}\PYG{p}{\PYGZgt{}}Un link a \PYG{p}{\PYGZlt{}}\PYG{n+nt}{a} \PYG{n+na}{href}\PYG{o}{=}\PYG{l+s}{\PYGZdq{}https://google.com\PYGZdq{}}\PYG{p}{\PYGZgt{}}Google\PYG{p}{\PYGZlt{}}\PYG{p}{/}\PYG{n+nt}{a}\PYG{p}{\PYGZgt{}}\PYG{p}{\PYGZlt{}}\PYG{p}{/}\PYG{n+nt}{p}\PYG{p}{\PYGZgt{}}

        \PYG{p}{\PYGZlt{}}\PYG{n+nt}{p}\PYG{p}{\PYGZgt{}}Una imagen:\PYG{p}{\PYGZlt{}}\PYG{p}{/}\PYG{n+nt}{p}\PYG{p}{\PYGZgt{}}

        \PYG{p}{\PYGZlt{}}\PYG{n+nt}{img} \PYG{n+na}{title}\PYG{o}{=}\PYG{l+s}{\PYGZdq{}un cubo\PYGZdq{}} \PYG{n+na}{width}\PYG{o}{=}\PYG{l+s}{\PYGZdq{}200\PYGZdq{}} \PYG{n+na}{height}\PYG{o}{=}\PYG{l+s}{\PYGZdq{}200\PYGZdq{}}  \PYG{n+na}{src}\PYG{o}{=}\PYG{l+s}{\PYGZdq{}cube.jpg\PYGZdq{}}\PYG{p}{\PYGZgt{}}
  \PYG{p}{\PYGZlt{}}\PYG{p}{/}\PYG{n+nt}{body}\PYG{p}{\PYGZgt{}}
\PYG{p}{\PYGZlt{}}\PYG{p}{/}\PYG{n+nt}{html}\PYG{p}{\PYGZgt{}}
\end{sphinxVerbatim}

Eso fue bastante repetitivo...


\section{Gustos específicos, cambiantes}
\label{\detokenize{css-y-cosas-por-el-estilo:gustos-especificos-cambiantes}}
En este momento se nos ocurre que quizás seria mejor si el texto fuera azul en
lugar de rojo.

La idea de tener que cambiar el color en cada lugar nos hace pensar que quizás
el rojo esta bien después de todo...

Pero como siempre en el mundo de la web, si algo es repetitivo y tedioso,
seguro hay una forma de automatizar lo repetitivo.

En este caso lo que nos serviría es indicar todos los tags que comparten un
conjunto de características y especificar en un mismo lugar las características
comunes.

Es como si los tags pertenecieran a una misma clase.

Y resulta que todos los tags pueden tener un atributo para eso, el atributo
\sphinxhref{https://developer.mozilla.org/es/docs/Web/HTML/Atributos\_Globales/class}{class} nos permite definir una lista de palabras separadas por espacios que
describen a que clases pertenece ese tag.

Llamemos a nuestra clase de tags con fondo claro y texto colorido \sphinxstylestrong{llamativo}.

Edita el ejemplo, todos los elementos con \sphinxtitleref{style="background-color: white; color: red;"} ahora tienen que contener el atributo \sphinxstylestrong{class} con el valor \sphinxstylestrong{llamativo}, ejemplo del primero:

\fvset{hllines={, ,}}%
\begin{sphinxVerbatim}[commandchars=\\\{\}]
\PYG{p}{\PYGZlt{}}\PYG{n+nt}{b} \PYG{n+na}{class}\PYG{o}{=}\PYG{l+s}{\PYGZdq{}llamativo\PYGZdq{}}\PYG{p}{\PYGZgt{}}negrita\PYG{p}{\PYGZlt{}}\PYG{p}{/}\PYG{n+nt}{b}\PYG{p}{\PYGZgt{}}
\end{sphinxVerbatim}

Luego de hacer todos los cambios podemos observar que ... no paso nada.

Porque las clases son cosas que usamos nosotros para agrupar tags, ahora
tenemos que de alguna forma decirle al navegador que queremos que todos los
tags con clase \sphinxstylestrong{llamativo} tengan fondo blanco y texto azul.

Para eso vamos a aprender un tag nuevo, el tag \sphinxhref{https://developer.mozilla.org/es/docs/Web/HTML/Elemento/style}{style}, este tag normalmente va en
la cabecera (porque no define contenido del documento) y nos permite
centralizar en un lugar las definiciones de estilo.

Este va a ser el principio del documento:

\fvset{hllines={, ,}}%
\begin{sphinxVerbatim}[commandchars=\\\{\}]
\PYG{c+cp}{\PYGZlt{}!doctype html\PYGZgt{}}
\PYG{p}{\PYGZlt{}}\PYG{n+nt}{html}\PYG{p}{\PYGZgt{}}
  \PYG{p}{\PYGZlt{}}\PYG{n+nt}{head}\PYG{p}{\PYGZgt{}}
        \PYG{p}{\PYGZlt{}}\PYG{n+nt}{meta} \PYG{n+na}{charset}\PYG{o}{=}\PYG{l+s}{\PYGZdq{}utf\PYGZhy{}8\PYGZdq{}}\PYG{p}{\PYGZgt{}}
        \PYG{p}{\PYGZlt{}}\PYG{n+nt}{title}\PYG{p}{\PYGZgt{}}Mi Pagina\PYG{p}{\PYGZlt{}}\PYG{p}{/}\PYG{n+nt}{title}\PYG{p}{\PYGZgt{}}
        \PYG{p}{\PYGZlt{}}\PYG{n+nt}{style}\PYG{p}{\PYGZgt{}}
        \PYG{p}{.}\PYG{n+nc}{llamativo}\PYG{p}{\PYGZob{}}
                \PYG{k}{background\PYGZhy{}color}\PYG{p}{:} \PYG{k+kc}{white}\PYG{p}{;}
                \PYG{k}{color}\PYG{p}{:} \PYG{k+kc}{blue}\PYG{p}{;}
        \PYG{p}{\PYGZcb{}}
        \PYG{p}{\PYGZlt{}}\PYG{p}{/}\PYG{n+nt}{style}\PYG{p}{\PYGZgt{}}
  \PYG{p}{\PYGZlt{}}\PYG{p}{/}\PYG{n+nt}{head}\PYG{p}{\PYGZgt{}}
\end{sphinxVerbatim}

La parte que nos interesa y es nueva es el contenido del tag

\fvset{hllines={, ,}}%
\begin{sphinxVerbatim}[commandchars=\\\{\}]
\PYG{p}{.}\PYG{n+nc}{llamativo}\PYG{p}{\PYGZob{}}
        \PYG{k}{background\PYGZhy{}color}\PYG{p}{:} \PYG{k+kc}{white}\PYG{p}{;}
        \PYG{k}{color}\PYG{p}{:} \PYG{k+kc}{blue}\PYG{p}{;}
\PYG{p}{\PYGZcb{}}
\end{sphinxVerbatim}

El nombre de nuestra clase esta ahí, pero empieza con un punto?

Si, para decirle al navegador que \sphinxstylestrong{llamativo} es una clase de tags en nuestro documento.

Luego de decir para que cosa queremos definir el estilo, llamado \sphinxstyleemphasis{selector} en
la jerga \sphinxstylestrong{CSS} (ya que selecciona el conjunto de tags a los cuales el estilo
aplica) le decimos que estilo aplicar, en nuestro caso y de la misma forma que
en el atributo style, pares de llave/valor separados por \sphinxstylestrong{;}. Por suerte acá
podemos separarlos con saltos de linea y espacios para hacerlo mas legible.

Que pasa si no ponemos el punto? el navegador piensa que nos referimos al
nombre de un tag, veamos un ejemplo.

\fvset{hllines={, ,}}%
\begin{sphinxVerbatim}[commandchars=\\\{\}]
\PYG{p}{.}\PYG{n+nc}{llamativo}\PYG{p}{\PYGZob{}}
        \PYG{k}{background\PYGZhy{}color}\PYG{p}{:} \PYG{k+kc}{white}\PYG{p}{;}
        \PYG{k}{color}\PYG{p}{:} \PYG{k+kc}{blue}\PYG{p}{;}
\PYG{p}{\PYGZcb{}}

\PYG{n+nt}{body}\PYG{p}{\PYGZob{}}
        \PYG{k}{background\PYGZhy{}color}\PYG{p}{:} \PYG{k+kc}{black}\PYG{p}{;}
        \PYG{k}{color}\PYG{p}{:} \PYG{k+kc}{white}\PYG{p}{;}
\PYG{p}{\PYGZcb{}}
\end{sphinxVerbatim}

Y así podemos centralizar todo el estilo de la pagina en la cabecera y separar
claramente el contenido de su presentación, algo que es una buena costumbre en
el desarrollo web.

\begin{figure}[htbp]
\centering

\noindent\sphinxincludegraphics{{04-style}.png}
\end{figure}


\section{Gustos cambiantes, en muchos lugares}
\label{\detokenize{css-y-cosas-por-el-estilo:gustos-cambiantes-en-muchos-lugares}}
Con lo que aprendimos hasta ahora ya podrías tener tu pagina personal, tu blog
o una pagina con cuentos o historias.

Imaginemos que con el tiempo tu pagina web crece y tiene 10 documentos distintos,
todos con el mismo estilo en la cabecera.

Y un día decidís cambiar el estilo de tu pagina, querés algo mas claro.

Ahí es cuando dándote cuenta que vas a tener que hacer cambios en 10 documentos, el estilo oscuro actual no es tan mala idea después de todo...

A menos que haya otra forma de evitar la repetición.

Por suerte la hay, y quizás ya la notaste al ver en tus proyectos de Thimble un
archivo con un nombre familiar que todavía no mencionamos.

El misterioso \sphinxstylestrong{style.css}.

Si lo abrís vas a ver un contenido familiar con algunas cosas nuevas, el mio tiene esto:

\fvset{hllines={, ,}}%
\begin{sphinxVerbatim}[commandchars=\\\{\}]
\PYG{c}{/* Fonts from Google Fonts \PYGZhy{} more at https://fonts.google.com */}
\PYG{p}{@}\PYG{k}{import} \PYG{n+nt}{url}\PYG{o}{(}\PYG{l+s+s1}{\PYGZsq{}https://fonts.googleapis.com/css?family=Open+Sans:400,700\PYGZsq{}}\PYG{o}{)}\PYG{p}{;}
\PYG{p}{@}\PYG{k}{import} \PYG{n+nt}{url}\PYG{o}{(}\PYG{l+s+s1}{\PYGZsq{}https://fonts.googleapis.com/css?family=Merriweather:400,700\PYGZsq{}}\PYG{o}{)}\PYG{p}{;}

\PYG{n+nt}{body} \PYG{p}{\PYGZob{}}
  \PYG{k}{background\PYGZhy{}color}\PYG{p}{:} \PYG{k+kc}{white}\PYG{p}{;}
  \PYG{k}{font\PYGZhy{}family}\PYG{p}{:} \PYG{l+s+s2}{\PYGZdq{}Open Sans\PYGZdq{}}\PYG{p}{,} \PYG{k+kc}{sans\PYGZhy{}serif}\PYG{p}{;}
  \PYG{k}{padding}\PYG{p}{:} \PYG{l+m+mi}{5}\PYG{k+kt}{px} \PYG{l+m+mi}{25}\PYG{k+kt}{px}\PYG{p}{;}
  \PYG{k}{font\PYGZhy{}size}\PYG{p}{:} \PYG{l+m+mi}{18}\PYG{k+kt}{px}\PYG{p}{;}
  \PYG{k}{margin}\PYG{p}{:} \PYG{l+m+mi}{0}\PYG{p}{;}
  \PYG{k}{color}\PYG{p}{:} \PYG{l+m+mh}{\PYGZsh{}444}\PYG{p}{;}
\PYG{p}{\PYGZcb{}}

\PYG{n+nt}{h1} \PYG{p}{\PYGZob{}}
  \PYG{k}{font\PYGZhy{}family}\PYG{p}{:} \PYG{l+s+s2}{\PYGZdq{}Merriweather\PYGZdq{}}\PYG{p}{,} \PYG{k+kc}{serif}\PYG{p}{;}
  \PYG{k}{font\PYGZhy{}size}\PYG{p}{:} \PYG{l+m+mi}{32}\PYG{k+kt}{px}\PYG{p}{;}
\PYG{p}{\PYGZcb{}}
\end{sphinxVerbatim}

Con lo que aprendimos e ignorando las primeras dos lineas podemos ver que es un
archivo que contiene CSS y que define el estilo para el tag body y para los
títulos.

Pero ese estilo no se esta aplicando, porque no lo incluimos en nuestro
documento.

Para incluirlo vamos a aprender un tag nuevo, que hace muchas cosas distintas
pero su tarea habitual es incluir archivos de estilo en documentos HTML.

Si agregamos el siguiente tag:

\fvset{hllines={, ,}}%
\begin{sphinxVerbatim}[commandchars=\\\{\}]
\PYG{p}{\PYGZlt{}}\PYG{n+nt}{link} \PYG{n+na}{href}\PYG{o}{=}\PYG{l+s}{\PYGZdq{}style.css\PYGZdq{}} \PYG{n+na}{rel}\PYG{o}{=}\PYG{l+s}{\PYGZdq{}stylesheet\PYGZdq{}}\PYG{p}{\PYGZgt{}}
\end{sphinxVerbatim}

Después del tag \sphinxstyleemphasis{style} en la cabecera podemos ver como de pronto el estilo
contenido en ese archivo se aplica al documento!

Antes de ver que sucedió y donde esta nuestro fondo negro veamos los dos
atributos del tag \sphinxhref{https://developer.mozilla.org/es/docs/Web/HTML/Elemento/link}{link}:
\begin{description}
\item[{href}] \leavevmode
Atributo que indica la ubicación del archivo de estilo a cargar, ya lo conocíamos del tag \sphinxstyleemphasis{a}

\item[{rel}] \leavevmode
Como vimos mas arriba, \sphinxstyleemphasis{link} es un tag polifacético, y para saber cual es la \sphinxstylestrong{rel} acion del archivo referenciado con el actual, se lo tenemos que indicar.
En este caso le decimos que la relación es de una hoja de estilo (\sphinxstylestrong{S} tyle \sphinxstylestrong{s} heet), de ahí las dos \sphinxstyleemphasis{s} en CSS

\end{description}

Ahora tendrás una de dos preguntas, o las dos:

Y nuestro fondo oscuro?

Y la C en CSS que significa?

Resulta que las dos preguntas tienen mas o menos la misma respuesta, la C en
CSS es de Cascada, osea que CSS en español significa hojas de estilo en
cascada.

Y donde esta la cascada? en la forma en la que el navegador interpreta los
estilos que definimos para nuestro documento.

En nuestro documento primero le decimos que el fondo del tag body es negro y
después cargamos un archivo CSS que le dice que el fondo es claro.

El navegador interpreta los estilos dándole la razón al ultimo que lo declaro y
al mas especifico.

En este caso, el ultimo en declarar el color de fondo del documento es el archivo (esta mas abajo en el documento HTML).

Y lo de mas especifico? bueno, el color del texto esta definido en varios lugares,
en el tag style para el tag body, y en el archivo style.css para el tag body. en ese caso sabemos que el ultimo gana.

Pero sin embargo los tags con clase \sphinxstylestrong{llamativo} son azules, como decide el
navegador que el azul le gana al negro? Porque el atributo class es mas
especifico que el tag body.

De esta manera podemos hacer definiciones generales "a grandes rasgos" al
principio de nuestras hojas de estilo e irlas refinando mas abajo,
redefiniendolas para casos mas particulares e incluso en otras hojas de estilo
especificas para ciertos documentos.

Esto es bastante información y con el tiempo lo vamos a ir aprendiendo a medida
que lo usamos.

Pero antes de terminar, movamos nuestro estilo al archivo style.css y dejemos
el documento HTML libre de CSS mas que la referencia a style.css, el cual queda
así:

\fvset{hllines={, ,}}%
\begin{sphinxVerbatim}[commandchars=\\\{\}]
\PYG{n+nt}{body} \PYG{p}{\PYGZob{}}
  \PYG{k}{background\PYGZhy{}color}\PYG{p}{:} \PYG{k+kc}{white}\PYG{p}{;}
  \PYG{k}{color}\PYG{p}{:} \PYG{l+m+mh}{\PYGZsh{}444}\PYG{p}{;}

  \PYG{k}{font\PYGZhy{}family}\PYG{p}{:} \PYG{n}{helvetica}\PYG{p}{;}
  \PYG{k}{font\PYGZhy{}size}\PYG{p}{:} \PYG{l+m+mi}{18}\PYG{k+kt}{px}\PYG{p}{;}

  \PYG{k}{padding}\PYG{p}{:} \PYG{l+m+mi}{5}\PYG{k+kt}{px} \PYG{l+m+mi}{25}\PYG{k+kt}{px}\PYG{p}{;}
  \PYG{k}{margin}\PYG{p}{:} \PYG{l+m+mi}{0}\PYG{p}{;}
\PYG{p}{\PYGZcb{}}

\PYG{n+nt}{h1} \PYG{p}{\PYGZob{}}
  \PYG{k}{font\PYGZhy{}size}\PYG{p}{:} \PYG{l+m+mi}{32}\PYG{k+kt}{px}\PYG{p}{;}
\PYG{p}{\PYGZcb{}}

\PYG{p}{.}\PYG{n+nc}{llamativo}\PYG{p}{\PYGZob{}}
  \PYG{k}{background\PYGZhy{}color}\PYG{p}{:} \PYG{k+kc}{white}\PYG{p}{;}
  \PYG{k}{color}\PYG{p}{:} \PYG{k+kc}{blue}\PYG{p}{;}
\PYG{p}{\PYGZcb{}}
\end{sphinxVerbatim}

En el CSS hay algunas llaves nuevas, font-family define la fuente del texto,
font-size su tamaño, las otras dos (padding y margin) las vamos a ver en
próximas secciones.

El principio de nuestro documento queda así:

\fvset{hllines={, ,}}%
\begin{sphinxVerbatim}[commandchars=\\\{\}]
\PYG{c+cp}{\PYGZlt{}!doctype html\PYGZgt{}}
\PYG{p}{\PYGZlt{}}\PYG{n+nt}{html}\PYG{p}{\PYGZgt{}}
  \PYG{p}{\PYGZlt{}}\PYG{n+nt}{head}\PYG{p}{\PYGZgt{}}
        \PYG{p}{\PYGZlt{}}\PYG{n+nt}{meta} \PYG{n+na}{charset}\PYG{o}{=}\PYG{l+s}{\PYGZdq{}utf\PYGZhy{}8\PYGZdq{}}\PYG{p}{\PYGZgt{}}
        \PYG{p}{\PYGZlt{}}\PYG{n+nt}{title}\PYG{p}{\PYGZgt{}}Mi Pagina\PYG{p}{\PYGZlt{}}\PYG{p}{/}\PYG{n+nt}{title}\PYG{p}{\PYGZgt{}}
        \PYG{p}{\PYGZlt{}}\PYG{n+nt}{link} \PYG{n+na}{href}\PYG{o}{=}\PYG{l+s}{\PYGZdq{}style.css\PYGZdq{}} \PYG{n+na}{rel}\PYG{o}{=}\PYG{l+s}{\PYGZdq{}stylesheet\PYGZdq{}}\PYG{p}{\PYGZgt{}}
  \PYG{p}{\PYGZlt{}}\PYG{p}{/}\PYG{n+nt}{head}\PYG{p}{\PYGZgt{}}
\end{sphinxVerbatim}

Ya observaras uno de los beneficios de separar contenido de presentación:
cambiamos completamente el aspecto de la pagina sin tocar su contenido.

El proyecto quedo así:

\begin{figure}[htbp]
\centering

\noindent\sphinxincludegraphics{{05-link}.png}
\end{figure}


\chapter{Lo espacial es invisible a los ojos}
\label{\detokenize{lo-espacial-es-invisible-a-los-ojos::doc}}\label{\detokenize{lo-espacial-es-invisible-a-los-ojos:lo-espacial-es-invisible-a-los-ojos}}
En la sección previa terminamos con una pagina que se veía así:

\begin{figure}[htbp]
\centering

\noindent\sphinxincludegraphics{{05-link}.png}
\end{figure}

Que tendríamos que hacer si quisiéramos que la palabra lista tuviera la clase \sphinxtitleref{llamativo}?

Hasta ahora para aplicar una clase a una palabra o conjunto de palabras aprovechábamos un tag que ya estaba ahí (\sphinxstylestrong{b}, \sphinxstylestrong{i} o \sphinxstylestrong{li}), pero dos de ellos aplican
un formato que no queremos, y el otro solo funciona con listas.

Como podemos hacer para definir una selección para aplicarle atributos pero que
en si no signifique nada?

Para eso hay un tag llamado \sphinxhref{https://developer.mozilla.org/es/docs/Web/HTML/Elemento/span}{span} que en ingles significa algo así como
tramo, palmo, lapso o mas detallado: la extensión completa de algo de punta a
punta; la cantidad de espacio que cubre algo.

Osea, demarca el comienzo y fin de algo, su extensión y nada mas. Este tag
no significa nada en si mismo, solo lo usamos para demarcar una región a la
cual queremos referirnos para algo.

Si, muy difuso, vamos a algo mas tangible, en nuestro caso queremos usar span
para demarcar la palabra \sphinxtitleref{lista} y aplicarle la clase \sphinxtitleref{llamativo}:

\fvset{hllines={, ,}}%
\begin{sphinxVerbatim}[commandchars=\\\{\}]
\PYG{p}{\PYGZlt{}}\PYG{n+nt}{p}\PYG{p}{\PYGZgt{}}
    Una \PYG{p}{\PYGZlt{}}\PYG{n+nt}{span} \PYG{n+na}{class}\PYG{o}{=}\PYG{l+s}{\PYGZdq{}llamativo\PYGZdq{}}\PYG{p}{\PYGZgt{}}lista\PYG{p}{\PYGZlt{}}\PYG{p}{/}\PYG{n+nt}{span}\PYG{p}{\PYGZgt{}} no ordenada:
\PYG{p}{\PYGZlt{}}\PYG{p}{/}\PYG{n+nt}{p}\PYG{p}{\PYGZgt{}}
\end{sphinxVerbatim}

Y

\fvset{hllines={, ,}}%
\begin{sphinxVerbatim}[commandchars=\\\{\}]
\PYG{p}{\PYGZlt{}}\PYG{n+nt}{p}\PYG{p}{\PYGZgt{}}Una \PYG{p}{\PYGZlt{}}\PYG{n+nt}{span} \PYG{n+na}{class}\PYG{o}{=}\PYG{l+s}{\PYGZdq{}llamativo\PYGZdq{}}\PYG{p}{\PYGZgt{}}lista\PYG{p}{\PYGZlt{}}\PYG{p}{/}\PYG{n+nt}{span}\PYG{p}{\PYGZgt{}} ordenada:\PYG{p}{\PYGZlt{}}\PYG{p}{/}\PYG{n+nt}{p}\PYG{p}{\PYGZgt{}}
\end{sphinxVerbatim}

Logrado nuestro objetivo pasamos al siguiente desafió, queremos resaltar el segundo párrafo poniéndole un borde, para eso vamos a usar un conjunto de atributos CSS nuevos:

\fvset{hllines={, ,}}%
\begin{sphinxVerbatim}[commandchars=\\\{\}]
\PYG{p}{\PYGZlt{}}\PYG{n+nt}{p} \PYG{n+na}{style}\PYG{o}{=}\PYG{l+s}{\PYGZdq{}border\PYGZhy{}width: 1px; border\PYGZhy{}style: solid; border\PYGZhy{}color: red;\PYGZdq{}}\PYG{p}{\PYGZgt{}}Esto es otro párrafo\PYG{p}{\PYGZlt{}}\PYG{p}{/}\PYG{n+nt}{p}\PYG{p}{\PYGZgt{}}
\end{sphinxVerbatim}

Es un poco largo para solo decir "quiero un border rojo, de 1 pixel de ancho y solido", por suerte hay una versión abreviada:

\fvset{hllines={, ,}}%
\begin{sphinxVerbatim}[commandchars=\\\{\}]
\PYG{p}{\PYGZlt{}}\PYG{n+nt}{p} \PYG{n+na}{style}\PYG{o}{=}\PYG{l+s}{\PYGZdq{}border: 1px solid red;\PYGZdq{}}\PYG{p}{\PYGZgt{}}Esto es otro párrafo\PYG{p}{\PYGZlt{}}\PYG{p}{/}\PYG{n+nt}{p}\PYG{p}{\PYGZgt{}}
\end{sphinxVerbatim}

No se vos, pero a mi me parece que el texto esta muy pegado al borde rojo,
preferiría que tuviera un poco mas de espacio.

Para eso usamos una de dos formas de darle mas "espacio" a un tag, una es el
espacio "interno" (entre el borde y el contenido) y otro es el espacio
"externo" (entre el borde y sus vecinos).

El espacio interno en ingles se llama \sphinxhref{https://developer.mozilla.org/es/docs/Web/CSS/padding}{padding} (se traduce "relleno"), el
espacio externo en ingles se llama \sphinxhref{https://developer.mozilla.org/es/docs/Web/CSS/margin}{margin} (se traduce "margen").

Estos dos atributos los vas a usar mucho en el día a día y como se usan mucho
hay formas abreviadas y mas especificas de usarlos, empecemos con la abreviada
que es la que nos sirve a nosotros.

\fvset{hllines={, ,}}%
\begin{sphinxVerbatim}[commandchars=\\\{\}]
\PYG{p}{\PYGZlt{}}\PYG{n+nt}{p} \PYG{n+na}{style}\PYG{o}{=}\PYG{l+s}{\PYGZdq{}padding: 8px; border: 1px solid red;\PYGZdq{}}\PYG{p}{\PYGZgt{}}Esto es otro párrafo\PYG{p}{\PYGZlt{}}\PYG{p}{/}\PYG{n+nt}{p}\PYG{p}{\PYGZgt{}}
\end{sphinxVerbatim}

Eso esta un poco mejor!

Ya que estamos haciéndonos espacio, demosle un poco mas de espacio exterior:

\fvset{hllines={, ,}}%
\begin{sphinxVerbatim}[commandchars=\\\{\}]
\PYG{p}{\PYGZlt{}}\PYG{n+nt}{p} \PYG{n+na}{style}\PYG{o}{=}\PYG{l+s}{\PYGZdq{}margin: 8px; padding: 8px; border: 1px solid red;\PYGZdq{}}\PYG{p}{\PYGZgt{}}Esto es otro párrafo\PYG{p}{\PYGZlt{}}\PYG{p}{/}\PYG{n+nt}{p}\PYG{p}{\PYGZgt{}}
\end{sphinxVerbatim}

Un poco mejor, pero si bien le da espacio con respecto al párrafo anterior y el
siguiente, también tiene un margen izquierdo que lo "desalinea" con respecto
al resto de los tags.

Para poder especificar margenes y rellenos con mas nivel de detalle vamos a
necesitar usar las versiones menos abreviadas, ambas aplican tanto a \sphinxtitleref{margin}
como a \sphinxtitleref{padding} así que las vemos juntas:

Primero la que ya vimos:

\fvset{hllines={, ,}}%
\begin{sphinxVerbatim}[commandchars=\\\{\}]
\PYG{n+nt}{margin}\PYG{o}{:} \PYG{n+nt}{8px}\PYG{o}{;}
\PYG{n+nt}{padding}\PYG{o}{:} \PYG{n+nt}{8px}\PYG{o}{;}
\end{sphinxVerbatim}

Esto aplica a los 4 lados del tag, el siguiente nos permite especificar dos
valores:

\fvset{hllines={, ,}}%
\begin{sphinxVerbatim}[commandchars=\\\{\}]
\PYG{n+nt}{margin}\PYG{o}{:} \PYG{n+nt}{8px} \PYG{n+nt}{0px}\PYG{o}{;}
\PYG{n+nt}{padding}\PYG{o}{:} \PYG{n+nt}{8px} \PYG{n+nt}{0px}\PYG{o}{;}
\end{sphinxVerbatim}

El primer valor es para arriba y abajo, el segundo valor para la izquierda y la
derecha.

El no abreviado nos permite especificar los cuatro valores:

\fvset{hllines={, ,}}%
\begin{sphinxVerbatim}[commandchars=\\\{\}]
\PYG{n+nt}{margin}\PYG{o}{:} \PYG{n+nt}{8px} \PYG{n+nt}{0px} \PYG{n+nt}{0px} \PYG{n+nt}{0px}\PYG{o}{;}
\PYG{n+nt}{padding}\PYG{o}{:} \PYG{n+nt}{8px} \PYG{n+nt}{0px} \PYG{n+nt}{0px} \PYG{n+nt}{0px}\PYG{o}{;}
\end{sphinxVerbatim}

Y el orden es como las agujas del reloj:

\fvset{hllines={, ,}}%
\begin{sphinxVerbatim}[commandchars=\\\{\}]
\PYG{n+nt}{margin}\PYG{o}{:} \PYG{o}{\PYGZlt{}}\PYG{n+nt}{arriba}\PYG{o}{\PYGZgt{}} \PYG{o}{\PYGZlt{}}\PYG{n+nt}{derecha}\PYG{o}{\PYGZgt{}} \PYG{o}{\PYGZlt{}}\PYG{n+nt}{abajo}\PYG{o}{\PYGZgt{}} \PYG{o}{\PYGZlt{}}\PYG{n+nt}{izquierda}\PYG{o}{\PYGZgt{}}\PYG{o}{;}
\PYG{n+nt}{padding}\PYG{o}{:} \PYG{o}{\PYGZlt{}}\PYG{n+nt}{arriba}\PYG{o}{\PYGZgt{}} \PYG{o}{\PYGZlt{}}\PYG{n+nt}{derecha}\PYG{o}{\PYGZgt{}} \PYG{o}{\PYGZlt{}}\PYG{n+nt}{abajo}\PYG{o}{\PYGZgt{}} \PYG{o}{\PYGZlt{}}\PYG{n+nt}{izquierda}\PYG{o}{\PYGZgt{}}\PYG{o}{;}
\end{sphinxVerbatim}

Pero que pasa si solo queremos especificar uno de ellos? tenemos que poner en 0
a todos los otros siempre? Por suerte no, hay otra forma de ser aun mas
especifico, diciendole cual margen o relleno queremos definir:

\fvset{hllines={, ,}}%
\begin{sphinxVerbatim}[commandchars=\\\{\}]
\PYG{n+nt}{margin\PYGZhy{}top}\PYG{o}{:} \PYG{n+nt}{8px}\PYG{o}{;}    \PYG{c}{/* top: arriba */}
\PYG{n+nt}{margin\PYGZhy{}right}\PYG{o}{:} \PYG{n+nt}{0px}\PYG{o}{;}  \PYG{c}{/* right: derecha */}
\PYG{n+nt}{margin\PYGZhy{}bottom}\PYG{o}{:} \PYG{n+nt}{0px}\PYG{o}{;} \PYG{c}{/* bottom: abajo */}
\PYG{n+nt}{margin\PYGZhy{}left}\PYG{o}{:} \PYG{n+nt}{0px}\PYG{o}{;}   \PYG{c}{/* left: izquierda */}
\end{sphinxVerbatim}

Como veras a la derecha de las definiciones puse las traducciones entre \sphinxtitleref{/*} y
\sphinxtitleref{*/}, si escribís eso en tu CSS va a funcionar. Esto es porque son comentarios,
una forma de agregar notas en el código CSS que el navegador ignora ya que es
para los humanos.

Si querés escribir comentarios en HTML es un poco distinto:

\fvset{hllines={, ,}}%
\begin{sphinxVerbatim}[commandchars=\\\{\}]
\PYG{c}{\PYGZlt{}!\PYGZhy{}\PYGZhy{}}\PYG{c}{ Esto es un comentario, puede ir en cualquier lado y el navegador}
\PYG{c}{     lo va a ignorar, puede extenderse mas de una linea.}

\PYG{c}{     suele ser útil cuando queremos esconder un tag pero no borrarlo.}
\PYG{c}{\PYGZhy{}\PYGZhy{}\PYGZgt{}}
\end{sphinxVerbatim}

Volviendo al tag \sphinxtitleref{span}, este tag sirve solo para rodear segmentos de texto,
que pasa si queremos agregar un borde con relleno a los primeros 3 párrafos?

Agregar borde a cada párrafo no sirve, porque lo que queremos es "rodear" los
3 párrafos con un borde y el tag \sphinxtitleref{span} solo sirve para rodear texto.

Obviamente este problema se resuelve con un nuevo tag, se llama \sphinxhref{https://developer.mozilla.org/es/docs/Web/HTML/Elemento/div}{div} y cumple
la misma función que \sphinxtitleref{span} pero se diferencia en que dentro de el puede haber
otros tags.

Probemoslo rodeando los 3 primeros párrafos con un borde verde y relleno de 8
pixeles:

\fvset{hllines={, ,}}%
\begin{sphinxVerbatim}[commandchars=\\\{\}]
\PYG{p}{\PYGZlt{}}\PYG{n+nt}{div} \PYG{n+na}{style}\PYG{o}{=}\PYG{l+s}{\PYGZdq{}border: 1px solid green; padding: 8px\PYGZdq{}}\PYG{p}{\PYGZgt{}}
  \PYG{p}{\PYGZlt{}}\PYG{n+nt}{p}\PYG{p}{\PYGZgt{}}Esto es un párrafo, la siguiente palabra es en
     \PYG{p}{\PYGZlt{}}\PYG{n+nt}{b} \PYG{n+na}{class}\PYG{o}{=}\PYG{l+s}{\PYGZdq{}llamativo\PYGZdq{}}\PYG{p}{\PYGZgt{}}negrita\PYG{p}{\PYGZlt{}}\PYG{p}{/}\PYG{n+nt}{b}\PYG{p}{\PYGZgt{}}, la siguiente en
     \PYG{p}{\PYGZlt{}}\PYG{n+nt}{i} \PYG{n+na}{class}\PYG{o}{=}\PYG{l+s}{\PYGZdq{}llamativo\PYGZdq{}}\PYG{p}{\PYGZgt{}}itálica\PYG{p}{\PYGZlt{}}\PYG{p}{/}\PYG{n+nt}{i}\PYG{p}{\PYGZgt{}}
  \PYG{p}{\PYGZlt{}}\PYG{p}{/}\PYG{n+nt}{p}\PYG{p}{\PYGZgt{}}

  \PYG{p}{\PYGZlt{}}\PYG{n+nt}{p} \PYG{n+na}{style}\PYG{o}{=}\PYG{l+s}{\PYGZdq{}margin: 8px; padding: 8px; border: 1px solid red;\PYGZdq{}}\PYG{p}{\PYGZgt{}}
    Esto es otro párrafo
  \PYG{p}{\PYGZlt{}}\PYG{p}{/}\PYG{n+nt}{p}\PYG{p}{\PYGZgt{}}

  \PYG{p}{\PYGZlt{}}\PYG{n+nt}{p}\PYG{p}{\PYGZgt{}}
    Una \PYG{p}{\PYGZlt{}}\PYG{n+nt}{span} \PYG{n+na}{class}\PYG{o}{=}\PYG{l+s}{\PYGZdq{}llamativo\PYGZdq{}}\PYG{p}{\PYGZgt{}}lista\PYG{p}{\PYGZlt{}}\PYG{p}{/}\PYG{n+nt}{span}\PYG{p}{\PYGZgt{}} no ordenada:
  \PYG{p}{\PYGZlt{}}\PYG{p}{/}\PYG{n+nt}{p}\PYG{p}{\PYGZgt{}}
\PYG{p}{\PYGZlt{}}\PYG{p}{/}\PYG{n+nt}{div}\PYG{p}{\PYGZgt{}}
\end{sphinxVerbatim}

El resultado queda algo así (tené en cuenta que esta sección ya tiene CSS así
que en Thimble se va a ver un poco distinto):



Con lo que aprendimos en esta sección podemos arreglar una cosa que no quedaba
del todo bien.

Quizás notaste que al aplicar la clase \sphinxtitleref{llamativo} a los ítems de la lista, el
marcador de la izquierda también se volvió \sphinxtitleref{llamativo}, si queremos que solo el
contenido del ítem sea \sphinxtitleref{llamativo} y no el ítem completo, podemos hacer uso de
nuestro nuevo amigo el tag \sphinxtitleref{span} para aplicar la clase solo al contenido.

Pasamos de:

\fvset{hllines={, ,}}%
\begin{sphinxVerbatim}[commandchars=\\\{\}]
\PYG{p}{\PYGZlt{}}\PYG{n+nt}{ul}\PYG{p}{\PYGZgt{}}
  \PYG{p}{\PYGZlt{}}\PYG{n+nt}{li} \PYG{n+na}{class}\PYG{o}{=}\PYG{l+s}{\PYGZdq{}llamativo\PYGZdq{}}\PYG{p}{\PYGZgt{}}Manzana\PYG{p}{\PYGZlt{}}\PYG{p}{/}\PYG{n+nt}{li}\PYG{p}{\PYGZgt{}}
  \PYG{p}{\PYGZlt{}}\PYG{n+nt}{li}\PYG{p}{\PYGZgt{}}Durazno\PYG{p}{\PYGZlt{}}\PYG{p}{/}\PYG{n+nt}{li}\PYG{p}{\PYGZgt{}}
  \PYG{p}{\PYGZlt{}}\PYG{n+nt}{li} \PYG{n+na}{class}\PYG{o}{=}\PYG{l+s}{\PYGZdq{}llamativo\PYGZdq{}}\PYG{p}{\PYGZgt{}}Banana\PYG{p}{\PYGZlt{}}\PYG{p}{/}\PYG{n+nt}{li}\PYG{p}{\PYGZgt{}}
\PYG{p}{\PYGZlt{}}\PYG{p}{/}\PYG{n+nt}{ul}\PYG{p}{\PYGZgt{}}
\end{sphinxVerbatim}



A:

\fvset{hllines={, ,}}%
\begin{sphinxVerbatim}[commandchars=\\\{\}]
\PYG{p}{\PYGZlt{}}\PYG{n+nt}{ul}\PYG{p}{\PYGZgt{}}
  \PYG{p}{\PYGZlt{}}\PYG{n+nt}{li}\PYG{p}{\PYGZgt{}}\PYG{p}{\PYGZlt{}}\PYG{n+nt}{span} \PYG{n+na}{class}\PYG{o}{=}\PYG{l+s}{\PYGZdq{}llamativo\PYGZdq{}}\PYG{p}{\PYGZgt{}}Manzana\PYG{p}{\PYGZlt{}}\PYG{n+nt}{span}\PYG{p}{\PYGZgt{}}\PYG{p}{\PYGZlt{}}\PYG{p}{/}\PYG{n+nt}{li}\PYG{p}{\PYGZgt{}}
  \PYG{p}{\PYGZlt{}}\PYG{n+nt}{li}\PYG{p}{\PYGZgt{}}Durazno\PYG{p}{\PYGZlt{}}\PYG{p}{/}\PYG{n+nt}{li}\PYG{p}{\PYGZgt{}}
  \PYG{p}{\PYGZlt{}}\PYG{n+nt}{li}\PYG{p}{\PYGZgt{}}\PYG{p}{\PYGZlt{}}\PYG{n+nt}{span} \PYG{n+na}{class}\PYG{o}{=}\PYG{l+s}{\PYGZdq{}llamativo\PYGZdq{}}\PYG{p}{\PYGZgt{}}Banana\PYG{p}{\PYGZlt{}}\PYG{n+nt}{span}\PYG{p}{\PYGZgt{}}\PYG{p}{\PYGZlt{}}\PYG{p}{/}\PYG{n+nt}{li}\PYG{p}{\PYGZgt{}}
\PYG{p}{\PYGZlt{}}\PYG{p}{/}\PYG{n+nt}{ul}\PYG{p}{\PYGZgt{}}
\end{sphinxVerbatim}



Y Problema resuelto.


\chapter{Reusando estilo (de otros)}
\label{\detokenize{reusando-estilo-de-otros:reusando-estilo-de-otros}}\label{\detokenize{reusando-estilo-de-otros::doc}}
En las secciones anteriores aprendimos como reusar el estilo guardándolo en un archivo separado y cargándolo en múltiples paginas.

A medida que vamos creando mas y mas paginas empiezan a surgir cosas comunes que se usan en casi cualquier proyecto.

Si miras con atención las paginas que visitas vas a ver que hay ciertos componentes que se repiten.

De esto te podrás imaginar que con tanto reusar estilos y llevarlos de un proyecto a otro a alguien se le habrá ocurrido la idea de hacer un archivo de estilos lo suficientemente genérico para que pueda ser el estilo inicial de mucha gente que quiere crear una pagina web nueva pero no quiere estar definiendo de cero todo cada vez que empieza.

Imaginas bien!

Hay varios, pero el mas conocido se llama bootstrap, y nos permite empezar nuestras paginas sin tener que preocuparnos por detalles que al fin del día no son tan importantes.

Mas interesante, es que como este proyecto estandariza los nombres de clases para cada estilo, hay gente que crea nuevos estilos que se cargan "sobre" bootstrap y que le cambian el aspecto sin que nosotros tengamos que hacer nada mas que agregar una linea a nuestro HTML!

Podes ver algunos ejemplos en \sphinxurl{https://bootswatch.com/}


\section{Incluyendo Bootstrap}
\label{\detokenize{reusando-estilo-de-otros:incluyendo-bootstrap}}
Vamos a ver como usar esto, empecemos con un proyecto nuevo en thimble con una
pagina básica:

\fvset{hllines={, ,}}%
\begin{sphinxVerbatim}[commandchars=\\\{\}]
\PYG{c+cp}{\PYGZlt{}!doctype html\PYGZgt{}}
\PYG{p}{\PYGZlt{}}\PYG{n+nt}{html}\PYG{p}{\PYGZgt{}}
  \PYG{p}{\PYGZlt{}}\PYG{n+nt}{head}\PYG{p}{\PYGZgt{}}
        \PYG{p}{\PYGZlt{}}\PYG{n+nt}{meta} \PYG{n+na}{charset}\PYG{o}{=}\PYG{l+s}{\PYGZdq{}utf\PYGZhy{}8\PYGZdq{}}\PYG{p}{\PYGZgt{}}
        \PYG{p}{\PYGZlt{}}\PYG{n+nt}{title}\PYG{p}{\PYGZgt{}}Mi Pagina\PYG{p}{\PYGZlt{}}\PYG{p}{/}\PYG{n+nt}{title}\PYG{p}{\PYGZgt{}}
  \PYG{p}{\PYGZlt{}}\PYG{p}{/}\PYG{n+nt}{head}\PYG{p}{\PYGZgt{}}

  \PYG{p}{\PYGZlt{}}\PYG{n+nt}{body}\PYG{p}{\PYGZgt{}}
        \PYG{p}{\PYGZlt{}}\PYG{n+nt}{h1}\PYG{p}{\PYGZgt{}}Esto es un título\PYG{p}{\PYGZlt{}}\PYG{p}{/}\PYG{n+nt}{h1}\PYG{p}{\PYGZgt{}}

        \PYG{p}{\PYGZlt{}}\PYG{n+nt}{p}\PYG{p}{\PYGZgt{}}Esto es un párrafo, la siguiente palabra es en \PYG{p}{\PYGZlt{}}\PYG{n+nt}{b}\PYG{p}{\PYGZgt{}}negrita\PYG{p}{\PYGZlt{}}\PYG{p}{/}\PYG{n+nt}{b}\PYG{p}{\PYGZgt{}}, la siguiente en \PYG{p}{\PYGZlt{}}\PYG{n+nt}{i}\PYG{p}{\PYGZgt{}}itálica\PYG{p}{\PYGZlt{}}\PYG{p}{/}\PYG{n+nt}{i}\PYG{p}{\PYGZgt{}}\PYG{p}{\PYGZlt{}}\PYG{p}{/}\PYG{n+nt}{p}\PYG{p}{\PYGZgt{}}

        \PYG{p}{\PYGZlt{}}\PYG{n+nt}{p}\PYG{p}{\PYGZgt{}}Esto es otro párrafo\PYG{p}{\PYGZlt{}}\PYG{p}{/}\PYG{n+nt}{p}\PYG{p}{\PYGZgt{}}

        \PYG{p}{\PYGZlt{}}\PYG{n+nt}{p}\PYG{p}{\PYGZgt{}}
          Una lista no ordenada:
        \PYG{p}{\PYGZlt{}}\PYG{p}{/}\PYG{n+nt}{p}\PYG{p}{\PYGZgt{}}

        \PYG{p}{\PYGZlt{}}\PYG{n+nt}{ul}\PYG{p}{\PYGZgt{}}
          \PYG{p}{\PYGZlt{}}\PYG{n+nt}{li}\PYG{p}{\PYGZgt{}}Manzana\PYG{p}{\PYGZlt{}}\PYG{p}{/}\PYG{n+nt}{li}\PYG{p}{\PYGZgt{}}
          \PYG{p}{\PYGZlt{}}\PYG{n+nt}{li}\PYG{p}{\PYGZgt{}}Durazno\PYG{p}{\PYGZlt{}}\PYG{p}{/}\PYG{n+nt}{li}\PYG{p}{\PYGZgt{}}
          \PYG{p}{\PYGZlt{}}\PYG{n+nt}{li}\PYG{p}{\PYGZgt{}}Banana\PYG{p}{\PYGZlt{}}\PYG{p}{/}\PYG{n+nt}{li}\PYG{p}{\PYGZgt{}}
        \PYG{p}{\PYGZlt{}}\PYG{p}{/}\PYG{n+nt}{ul}\PYG{p}{\PYGZgt{}}

        \PYG{p}{\PYGZlt{}}\PYG{n+nt}{p}\PYG{p}{\PYGZgt{}}Una lista ordenada:\PYG{p}{\PYGZlt{}}\PYG{p}{/}\PYG{n+nt}{p}\PYG{p}{\PYGZgt{}}

        \PYG{p}{\PYGZlt{}}\PYG{n+nt}{ol}\PYG{p}{\PYGZgt{}}
          \PYG{p}{\PYGZlt{}}\PYG{n+nt}{li}\PYG{p}{\PYGZgt{}}Uno\PYG{p}{\PYGZlt{}}\PYG{p}{/}\PYG{n+nt}{li}\PYG{p}{\PYGZgt{}}
          \PYG{p}{\PYGZlt{}}\PYG{n+nt}{li}\PYG{p}{\PYGZgt{}}Dos\PYG{p}{\PYGZlt{}}\PYG{p}{/}\PYG{n+nt}{li}\PYG{p}{\PYGZgt{}}
          \PYG{p}{\PYGZlt{}}\PYG{n+nt}{li}\PYG{p}{\PYGZgt{}}Tres\PYG{p}{\PYGZlt{}}\PYG{p}{/}\PYG{n+nt}{li}\PYG{p}{\PYGZgt{}}
        \PYG{p}{\PYGZlt{}}\PYG{p}{/}\PYG{n+nt}{ol}\PYG{p}{\PYGZgt{}}

        \PYG{p}{\PYGZlt{}}\PYG{n+nt}{p}\PYG{p}{\PYGZgt{}}Un link a \PYG{p}{\PYGZlt{}}\PYG{n+nt}{a} \PYG{n+na}{href}\PYG{o}{=}\PYG{l+s}{\PYGZdq{}https://google.com\PYGZdq{}}\PYG{p}{\PYGZgt{}}Google\PYG{p}{\PYGZlt{}}\PYG{p}{/}\PYG{n+nt}{a}\PYG{p}{\PYGZgt{}}\PYG{p}{\PYGZlt{}}\PYG{p}{/}\PYG{n+nt}{p}\PYG{p}{\PYGZgt{}}

  \PYG{p}{\PYGZlt{}}\PYG{p}{/}\PYG{n+nt}{body}\PYG{p}{\PYGZgt{}}
\PYG{p}{\PYGZlt{}}\PYG{p}{/}\PYG{n+nt}{html}\PYG{p}{\PYGZgt{}}
\end{sphinxVerbatim}

Nuestra pagina de siempre, no muy linda de ver:

\begin{figure}[htbp]
\centering

\noindent\sphinxincludegraphics{{01-base}.png}
\end{figure}

Ahora agregamos la hoja de estilo de bootstrap en el \textless{}head\textgreater{} de la pagina,
debajo del tag \textless{}title\textgreater{}:

\fvset{hllines={, ,}}%
\begin{sphinxVerbatim}[commandchars=\\\{\}]
\PYG{p}{\PYGZlt{}}\PYG{n+nt}{link} \PYG{n+na}{rel}\PYG{o}{=}\PYG{l+s}{\PYGZdq{}stylesheet\PYGZdq{}} \PYG{n+na}{href}\PYG{o}{=}\PYG{l+s}{\PYGZdq{}https://stackpath.bootstrapcdn.com/bootstrap/4.1.0/css/bootstrap.min.css\PYGZdq{}}\PYG{p}{\PYGZgt{}}
\end{sphinxVerbatim}

Y podemos ver que ya cambio un poco:

\begin{figure}[htbp]
\centering

\noindent\sphinxincludegraphics{{02-base-bootstrap}.png}
\end{figure}


\section{Un tour por algunos componentes de bootstrap}
\label{\detokenize{reusando-estilo-de-otros:un-tour-por-algunos-componentes-de-bootstrap}}
Pero si fuera solo por eso entonces no seria tan útil, esto es solo el
contenido base, bootstrap nos provee con muchos componentes estandard para
usar, empecemos por las alertas:


\subsection{Alertas}
\label{\detokenize{reusando-estilo-de-otros:alertas}}
Agreguemos el siguiente HTML al \textless{}body\textgreater{} de nuestro proyecto:

\fvset{hllines={, ,}}%
\begin{sphinxVerbatim}[commandchars=\\\{\}]
\PYG{p}{\PYGZlt{}}\PYG{n+nt}{h2}\PYG{p}{\PYGZgt{}}Alertas\PYG{p}{\PYGZlt{}}\PYG{p}{/}\PYG{n+nt}{h2}\PYG{p}{\PYGZgt{}}

\PYG{p}{\PYGZlt{}}\PYG{n+nt}{div} \PYG{n+na}{class}\PYG{o}{=}\PYG{l+s}{\PYGZdq{}alert alert\PYGZhy{}primary\PYGZdq{}} \PYG{n+na}{role}\PYG{o}{=}\PYG{l+s}{\PYGZdq{}alert\PYGZdq{}}\PYG{p}{\PYGZgt{}}
  Alerta principal (primary)
\PYG{p}{\PYGZlt{}}\PYG{p}{/}\PYG{n+nt}{div}\PYG{p}{\PYGZgt{}}
\PYG{p}{\PYGZlt{}}\PYG{n+nt}{div} \PYG{n+na}{class}\PYG{o}{=}\PYG{l+s}{\PYGZdq{}alert alert\PYGZhy{}secondary\PYGZdq{}} \PYG{n+na}{role}\PYG{o}{=}\PYG{l+s}{\PYGZdq{}alert\PYGZdq{}}\PYG{p}{\PYGZgt{}}
  Alerta secundario (secondary)
\PYG{p}{\PYGZlt{}}\PYG{p}{/}\PYG{n+nt}{div}\PYG{p}{\PYGZgt{}}
\PYG{p}{\PYGZlt{}}\PYG{n+nt}{div} \PYG{n+na}{class}\PYG{o}{=}\PYG{l+s}{\PYGZdq{}alert alert\PYGZhy{}success\PYGZdq{}} \PYG{n+na}{role}\PYG{o}{=}\PYG{l+s}{\PYGZdq{}alert\PYGZdq{}}\PYG{p}{\PYGZgt{}}
  Alerta exito (success)
\PYG{p}{\PYGZlt{}}\PYG{p}{/}\PYG{n+nt}{div}\PYG{p}{\PYGZgt{}}
\PYG{p}{\PYGZlt{}}\PYG{n+nt}{div} \PYG{n+na}{class}\PYG{o}{=}\PYG{l+s}{\PYGZdq{}alert alert\PYGZhy{}danger\PYGZdq{}} \PYG{n+na}{role}\PYG{o}{=}\PYG{l+s}{\PYGZdq{}alert\PYGZdq{}}\PYG{p}{\PYGZgt{}}
  Alerta peligro (danger)
\PYG{p}{\PYGZlt{}}\PYG{p}{/}\PYG{n+nt}{div}\PYG{p}{\PYGZgt{}}
\PYG{p}{\PYGZlt{}}\PYG{n+nt}{div} \PYG{n+na}{class}\PYG{o}{=}\PYG{l+s}{\PYGZdq{}alert alert\PYGZhy{}warning\PYGZdq{}} \PYG{n+na}{role}\PYG{o}{=}\PYG{l+s}{\PYGZdq{}alert\PYGZdq{}}\PYG{p}{\PYGZgt{}}
  Alerta advertencia (warning)
\PYG{p}{\PYGZlt{}}\PYG{p}{/}\PYG{n+nt}{div}\PYG{p}{\PYGZgt{}}
\PYG{p}{\PYGZlt{}}\PYG{n+nt}{div} \PYG{n+na}{class}\PYG{o}{=}\PYG{l+s}{\PYGZdq{}alert alert\PYGZhy{}info\PYGZdq{}} \PYG{n+na}{role}\PYG{o}{=}\PYG{l+s}{\PYGZdq{}alert\PYGZdq{}}\PYG{p}{\PYGZgt{}}
  Alerta informacion (info)
\PYG{p}{\PYGZlt{}}\PYG{p}{/}\PYG{n+nt}{div}\PYG{p}{\PYGZgt{}}
\PYG{p}{\PYGZlt{}}\PYG{n+nt}{div} \PYG{n+na}{class}\PYG{o}{=}\PYG{l+s}{\PYGZdq{}alert alert\PYGZhy{}light\PYGZdq{}} \PYG{n+na}{role}\PYG{o}{=}\PYG{l+s}{\PYGZdq{}alert\PYGZdq{}}\PYG{p}{\PYGZgt{}}
  Alerta claro (light)
\PYG{p}{\PYGZlt{}}\PYG{p}{/}\PYG{n+nt}{div}\PYG{p}{\PYGZgt{}}
\PYG{p}{\PYGZlt{}}\PYG{n+nt}{div} \PYG{n+na}{class}\PYG{o}{=}\PYG{l+s}{\PYGZdq{}alert alert\PYGZhy{}dark\PYGZdq{}} \PYG{n+na}{role}\PYG{o}{=}\PYG{l+s}{\PYGZdq{}alert\PYGZdq{}}\PYG{p}{\PYGZgt{}}
  Alerta oscuro (dark)
\PYG{p}{\PYGZlt{}}\PYG{p}{/}\PYG{n+nt}{div}\PYG{p}{\PYGZgt{}}

\PYG{p}{\PYGZlt{}}\PYG{n+nt}{h2}\PYG{p}{\PYGZgt{}}Alertas con mas contenido\PYG{p}{\PYGZlt{}}\PYG{p}{/}\PYG{n+nt}{h2}\PYG{p}{\PYGZgt{}}

\PYG{p}{\PYGZlt{}}\PYG{n+nt}{div} \PYG{n+na}{class}\PYG{o}{=}\PYG{l+s}{\PYGZdq{}alert alert\PYGZhy{}info\PYGZdq{}} \PYG{n+na}{role}\PYG{o}{=}\PYG{l+s}{\PYGZdq{}alert\PYGZdq{}}\PYG{p}{\PYGZgt{}}
  \PYG{p}{\PYGZlt{}}\PYG{n+nt}{h4} \PYG{n+na}{class}\PYG{o}{=}\PYG{l+s}{\PYGZdq{}alert\PYGZhy{}heading\PYGZdq{}}\PYG{p}{\PYGZgt{}}Titulo\PYG{p}{\PYGZlt{}}\PYG{p}{/}\PYG{n+nt}{h4}\PYG{p}{\PYGZgt{}}
  \PYG{p}{\PYGZlt{}}\PYG{n+nt}{p}\PYG{p}{\PYGZgt{}}Contenido principal.\PYG{p}{\PYGZlt{}}\PYG{p}{/}\PYG{n+nt}{p}\PYG{p}{\PYGZgt{}}
  \PYG{p}{\PYGZlt{}}\PYG{n+nt}{hr}\PYG{p}{\PYGZgt{}}
  \PYG{p}{\PYGZlt{}}\PYG{n+nt}{p} \PYG{n+na}{class}\PYG{o}{=}\PYG{l+s}{\PYGZdq{}mb\PYGZhy{}0\PYGZdq{}}\PYG{p}{\PYGZgt{}}Contenido despues del separador.\PYG{p}{\PYGZlt{}}\PYG{p}{/}\PYG{n+nt}{p}\PYG{p}{\PYGZgt{}}
\PYG{p}{\PYGZlt{}}\PYG{p}{/}\PYG{n+nt}{div}\PYG{p}{\PYGZgt{}}
\end{sphinxVerbatim}

Debería verse algo así:

\begin{figure}[htbp]
\centering

\noindent\sphinxincludegraphics{{03-bs-alerts}.png}
\end{figure}


\section{Badges}
\label{\detokenize{reusando-estilo-de-otros:badges}}
Badges se traduce a medalla o distintivo, es mas fácil entender que son
viéndolos que por la palabra, así que agreguemos el siguiente HTML a nuestra
pagina:

\fvset{hllines={, ,}}%
\begin{sphinxVerbatim}[commandchars=\\\{\}]
\PYG{p}{\PYGZlt{}}\PYG{n+nt}{h2}\PYG{p}{\PYGZgt{}}Badges\PYG{p}{\PYGZlt{}}\PYG{p}{/}\PYG{n+nt}{h2}\PYG{p}{\PYGZgt{}}

\PYG{p}{\PYGZlt{}}\PYG{n+nt}{h3}\PYG{p}{\PYGZgt{}}Estandar\PYG{p}{\PYGZlt{}}\PYG{p}{/}\PYG{n+nt}{h3}\PYG{p}{\PYGZgt{}}

\PYG{p}{\PYGZlt{}}\PYG{n+nt}{div} \PYG{n+na}{class}\PYG{o}{=}\PYG{l+s}{\PYGZdq{}m\PYGZhy{}3\PYGZdq{}}\PYG{p}{\PYGZgt{}}
  \PYG{p}{\PYGZlt{}}\PYG{n+nt}{span} \PYG{n+na}{class}\PYG{o}{=}\PYG{l+s}{\PYGZdq{}badge badge\PYGZhy{}primary\PYGZdq{}}\PYG{p}{\PYGZgt{}}Principal\PYG{p}{\PYGZlt{}}\PYG{p}{/}\PYG{n+nt}{span}\PYG{p}{\PYGZgt{}}
  \PYG{p}{\PYGZlt{}}\PYG{n+nt}{span} \PYG{n+na}{class}\PYG{o}{=}\PYG{l+s}{\PYGZdq{}badge badge\PYGZhy{}secondary\PYGZdq{}}\PYG{p}{\PYGZgt{}}Secundario\PYG{p}{\PYGZlt{}}\PYG{p}{/}\PYG{n+nt}{span}\PYG{p}{\PYGZgt{}}
  \PYG{p}{\PYGZlt{}}\PYG{n+nt}{span} \PYG{n+na}{class}\PYG{o}{=}\PYG{l+s}{\PYGZdq{}badge badge\PYGZhy{}success\PYGZdq{}}\PYG{p}{\PYGZgt{}}Exito\PYG{p}{\PYGZlt{}}\PYG{p}{/}\PYG{n+nt}{span}\PYG{p}{\PYGZgt{}}
  \PYG{p}{\PYGZlt{}}\PYG{n+nt}{span} \PYG{n+na}{class}\PYG{o}{=}\PYG{l+s}{\PYGZdq{}badge badge\PYGZhy{}danger\PYGZdq{}}\PYG{p}{\PYGZgt{}}Peligro\PYG{p}{\PYGZlt{}}\PYG{p}{/}\PYG{n+nt}{span}\PYG{p}{\PYGZgt{}}
  \PYG{p}{\PYGZlt{}}\PYG{n+nt}{span} \PYG{n+na}{class}\PYG{o}{=}\PYG{l+s}{\PYGZdq{}badge badge\PYGZhy{}warning\PYGZdq{}}\PYG{p}{\PYGZgt{}}Advertencia\PYG{p}{\PYGZlt{}}\PYG{p}{/}\PYG{n+nt}{span}\PYG{p}{\PYGZgt{}}
  \PYG{p}{\PYGZlt{}}\PYG{n+nt}{span} \PYG{n+na}{class}\PYG{o}{=}\PYG{l+s}{\PYGZdq{}badge badge\PYGZhy{}info\PYGZdq{}}\PYG{p}{\PYGZgt{}}Informacion\PYG{p}{\PYGZlt{}}\PYG{p}{/}\PYG{n+nt}{span}\PYG{p}{\PYGZgt{}}
  \PYG{p}{\PYGZlt{}}\PYG{n+nt}{span} \PYG{n+na}{class}\PYG{o}{=}\PYG{l+s}{\PYGZdq{}badge badge\PYGZhy{}light\PYGZdq{}}\PYG{p}{\PYGZgt{}}Claro\PYG{p}{\PYGZlt{}}\PYG{p}{/}\PYG{n+nt}{span}\PYG{p}{\PYGZgt{}}
  \PYG{p}{\PYGZlt{}}\PYG{n+nt}{span} \PYG{n+na}{class}\PYG{o}{=}\PYG{l+s}{\PYGZdq{}badge badge\PYGZhy{}dark\PYGZdq{}}\PYG{p}{\PYGZgt{}}Oscuro\PYG{p}{\PYGZlt{}}\PYG{p}{/}\PYG{n+nt}{span}\PYG{p}{\PYGZgt{}}
\PYG{p}{\PYGZlt{}}\PYG{p}{/}\PYG{n+nt}{div}\PYG{p}{\PYGZgt{}}

\PYG{p}{\PYGZlt{}}\PYG{n+nt}{h3}\PYG{p}{\PYGZgt{}}Pill Badges (Pastillas)\PYG{p}{\PYGZlt{}}\PYG{p}{/}\PYG{n+nt}{h3}\PYG{p}{\PYGZgt{}}

\PYG{p}{\PYGZlt{}}\PYG{n+nt}{div} \PYG{n+na}{class}\PYG{o}{=}\PYG{l+s}{\PYGZdq{}m\PYGZhy{}3\PYGZdq{}}\PYG{p}{\PYGZgt{}}
  \PYG{p}{\PYGZlt{}}\PYG{n+nt}{span} \PYG{n+na}{class}\PYG{o}{=}\PYG{l+s}{\PYGZdq{}badge badge\PYGZhy{}pill badge\PYGZhy{}primary\PYGZdq{}}\PYG{p}{\PYGZgt{}}Principal\PYG{p}{\PYGZlt{}}\PYG{p}{/}\PYG{n+nt}{span}\PYG{p}{\PYGZgt{}}
  \PYG{p}{\PYGZlt{}}\PYG{n+nt}{span} \PYG{n+na}{class}\PYG{o}{=}\PYG{l+s}{\PYGZdq{}badge badge\PYGZhy{}pill badge\PYGZhy{}secondary\PYGZdq{}}\PYG{p}{\PYGZgt{}}Secundario\PYG{p}{\PYGZlt{}}\PYG{p}{/}\PYG{n+nt}{span}\PYG{p}{\PYGZgt{}}
  \PYG{p}{\PYGZlt{}}\PYG{n+nt}{span} \PYG{n+na}{class}\PYG{o}{=}\PYG{l+s}{\PYGZdq{}badge badge\PYGZhy{}pill badge\PYGZhy{}success\PYGZdq{}}\PYG{p}{\PYGZgt{}}Exito\PYG{p}{\PYGZlt{}}\PYG{p}{/}\PYG{n+nt}{span}\PYG{p}{\PYGZgt{}}
  \PYG{p}{\PYGZlt{}}\PYG{n+nt}{span} \PYG{n+na}{class}\PYG{o}{=}\PYG{l+s}{\PYGZdq{}badge badge\PYGZhy{}pill badge\PYGZhy{}danger\PYGZdq{}}\PYG{p}{\PYGZgt{}}Peligro\PYG{p}{\PYGZlt{}}\PYG{p}{/}\PYG{n+nt}{span}\PYG{p}{\PYGZgt{}}
  \PYG{p}{\PYGZlt{}}\PYG{n+nt}{span} \PYG{n+na}{class}\PYG{o}{=}\PYG{l+s}{\PYGZdq{}badge badge\PYGZhy{}pill badge\PYGZhy{}warning\PYGZdq{}}\PYG{p}{\PYGZgt{}}Advertencia\PYG{p}{\PYGZlt{}}\PYG{p}{/}\PYG{n+nt}{span}\PYG{p}{\PYGZgt{}}
  \PYG{p}{\PYGZlt{}}\PYG{n+nt}{span} \PYG{n+na}{class}\PYG{o}{=}\PYG{l+s}{\PYGZdq{}badge badge\PYGZhy{}pill badge\PYGZhy{}info\PYGZdq{}}\PYG{p}{\PYGZgt{}}Informacion\PYG{p}{\PYGZlt{}}\PYG{p}{/}\PYG{n+nt}{span}\PYG{p}{\PYGZgt{}}
  \PYG{p}{\PYGZlt{}}\PYG{n+nt}{span} \PYG{n+na}{class}\PYG{o}{=}\PYG{l+s}{\PYGZdq{}badge badge\PYGZhy{}pill badge\PYGZhy{}light\PYGZdq{}}\PYG{p}{\PYGZgt{}}Claro\PYG{p}{\PYGZlt{}}\PYG{p}{/}\PYG{n+nt}{span}\PYG{p}{\PYGZgt{}}
  \PYG{p}{\PYGZlt{}}\PYG{n+nt}{span} \PYG{n+na}{class}\PYG{o}{=}\PYG{l+s}{\PYGZdq{}badge badge\PYGZhy{}pill badge\PYGZhy{}dark\PYGZdq{}}\PYG{p}{\PYGZgt{}}Oscuro\PYG{p}{\PYGZlt{}}\PYG{p}{/}\PYG{n+nt}{span}\PYG{p}{\PYGZgt{}}
\PYG{p}{\PYGZlt{}}\PYG{p}{/}\PYG{n+nt}{div}\PYG{p}{\PYGZgt{}}

\PYG{p}{\PYGZlt{}}\PYG{n+nt}{h3}\PYG{p}{\PYGZgt{}}Links\PYG{p}{\PYGZlt{}}\PYG{p}{/}\PYG{n+nt}{h3}\PYG{p}{\PYGZgt{}}

\PYG{p}{\PYGZlt{}}\PYG{n+nt}{div} \PYG{n+na}{class}\PYG{o}{=}\PYG{l+s}{\PYGZdq{}m\PYGZhy{}3\PYGZdq{}}\PYG{p}{\PYGZgt{}}
  \PYG{p}{\PYGZlt{}}\PYG{n+nt}{a} \PYG{n+na}{href}\PYG{o}{=}\PYG{l+s}{\PYGZdq{}\PYGZsh{}\PYGZdq{}} \PYG{n+na}{class}\PYG{o}{=}\PYG{l+s}{\PYGZdq{}badge badge\PYGZhy{}primary\PYGZdq{}}\PYG{p}{\PYGZgt{}}Principal\PYG{p}{\PYGZlt{}}\PYG{p}{/}\PYG{n+nt}{a}\PYG{p}{\PYGZgt{}}
  \PYG{p}{\PYGZlt{}}\PYG{n+nt}{a} \PYG{n+na}{href}\PYG{o}{=}\PYG{l+s}{\PYGZdq{}\PYGZsh{}\PYGZdq{}} \PYG{n+na}{class}\PYG{o}{=}\PYG{l+s}{\PYGZdq{}badge badge\PYGZhy{}secondary\PYGZdq{}}\PYG{p}{\PYGZgt{}}Secundario\PYG{p}{\PYGZlt{}}\PYG{p}{/}\PYG{n+nt}{a}\PYG{p}{\PYGZgt{}}
  \PYG{p}{\PYGZlt{}}\PYG{n+nt}{a} \PYG{n+na}{href}\PYG{o}{=}\PYG{l+s}{\PYGZdq{}\PYGZsh{}\PYGZdq{}} \PYG{n+na}{class}\PYG{o}{=}\PYG{l+s}{\PYGZdq{}badge badge\PYGZhy{}success\PYGZdq{}}\PYG{p}{\PYGZgt{}}Exito\PYG{p}{\PYGZlt{}}\PYG{p}{/}\PYG{n+nt}{a}\PYG{p}{\PYGZgt{}}
  \PYG{p}{\PYGZlt{}}\PYG{n+nt}{a} \PYG{n+na}{href}\PYG{o}{=}\PYG{l+s}{\PYGZdq{}\PYGZsh{}\PYGZdq{}} \PYG{n+na}{class}\PYG{o}{=}\PYG{l+s}{\PYGZdq{}badge badge\PYGZhy{}danger\PYGZdq{}}\PYG{p}{\PYGZgt{}}Peligro\PYG{p}{\PYGZlt{}}\PYG{p}{/}\PYG{n+nt}{a}\PYG{p}{\PYGZgt{}}
  \PYG{p}{\PYGZlt{}}\PYG{n+nt}{a} \PYG{n+na}{href}\PYG{o}{=}\PYG{l+s}{\PYGZdq{}\PYGZsh{}\PYGZdq{}} \PYG{n+na}{class}\PYG{o}{=}\PYG{l+s}{\PYGZdq{}badge badge\PYGZhy{}warning\PYGZdq{}}\PYG{p}{\PYGZgt{}}Advertencia\PYG{p}{\PYGZlt{}}\PYG{p}{/}\PYG{n+nt}{a}\PYG{p}{\PYGZgt{}}
  \PYG{p}{\PYGZlt{}}\PYG{n+nt}{a} \PYG{n+na}{href}\PYG{o}{=}\PYG{l+s}{\PYGZdq{}\PYGZsh{}\PYGZdq{}} \PYG{n+na}{class}\PYG{o}{=}\PYG{l+s}{\PYGZdq{}badge badge\PYGZhy{}info\PYGZdq{}}\PYG{p}{\PYGZgt{}}Informacion\PYG{p}{\PYGZlt{}}\PYG{p}{/}\PYG{n+nt}{a}\PYG{p}{\PYGZgt{}}
  \PYG{p}{\PYGZlt{}}\PYG{n+nt}{a} \PYG{n+na}{href}\PYG{o}{=}\PYG{l+s}{\PYGZdq{}\PYGZsh{}\PYGZdq{}} \PYG{n+na}{class}\PYG{o}{=}\PYG{l+s}{\PYGZdq{}badge badge\PYGZhy{}light\PYGZdq{}}\PYG{p}{\PYGZgt{}}Claro\PYG{p}{\PYGZlt{}}\PYG{p}{/}\PYG{n+nt}{a}\PYG{p}{\PYGZgt{}}
  \PYG{p}{\PYGZlt{}}\PYG{n+nt}{a} \PYG{n+na}{href}\PYG{o}{=}\PYG{l+s}{\PYGZdq{}\PYGZsh{}\PYGZdq{}} \PYG{n+na}{class}\PYG{o}{=}\PYG{l+s}{\PYGZdq{}badge badge\PYGZhy{}dark\PYGZdq{}}\PYG{p}{\PYGZgt{}}Oscuro\PYG{p}{\PYGZlt{}}\PYG{p}{/}\PYG{n+nt}{a}\PYG{p}{\PYGZgt{}}
\PYG{p}{\PYGZlt{}}\PYG{p}{/}\PYG{n+nt}{div}\PYG{p}{\PYGZgt{}}
\end{sphinxVerbatim}

Como veras rodeo los ejemplos con un div para darle mas margen, pero no uso
style="margin: ..." como hasta acá, sino que uso una clase que bootstrap provee
que estandariza los margenes en 6 niveles (m-0, m-1, ..., m-5). De esta manera
si usamos estas clases en nuestras paginas los margenes serán consistentes y
luego podremos ajustarlos en un solo lugar (la definición de .m-0, ..., .m-5 en
nuestra hoja de estilos)

El resultado es algo así:

\begin{figure}[htbp]
\centering

\noindent\sphinxincludegraphics{{04-badges}.png}
\end{figure}


\subsection{Barra de Navegacion}
\label{\detokenize{reusando-estilo-de-otros:barra-de-navegacion}}
Otro componente muy útil y versátil son las barras de navegación, suelen usarse
en la parte superior de la pagina y en cualquier sección que tiene mas de un
elemento para mostrar, como los tabs del navegador web.

\fvset{hllines={, ,}}%
\begin{sphinxVerbatim}[commandchars=\\\{\}]
\PYG{p}{\PYGZlt{}}\PYG{n+nt}{h2}\PYG{p}{\PYGZgt{}}Barra de Navegacion\PYG{p}{\PYGZlt{}}\PYG{p}{/}\PYG{n+nt}{h2}\PYG{p}{\PYGZgt{}}

\PYG{p}{\PYGZlt{}}\PYG{n+nt}{h3}\PYG{p}{\PYGZgt{}}Usando tags de lista\PYG{p}{\PYGZlt{}}\PYG{p}{/}\PYG{n+nt}{h3}\PYG{p}{\PYGZgt{}}

\PYG{p}{\PYGZlt{}}\PYG{n+nt}{ul} \PYG{n+na}{class}\PYG{o}{=}\PYG{l+s}{\PYGZdq{}nav\PYGZdq{}}\PYG{p}{\PYGZgt{}}
  \PYG{p}{\PYGZlt{}}\PYG{n+nt}{li} \PYG{n+na}{class}\PYG{o}{=}\PYG{l+s}{\PYGZdq{}nav\PYGZhy{}item\PYGZdq{}}\PYG{p}{\PYGZgt{}}
    \PYG{p}{\PYGZlt{}}\PYG{n+nt}{a} \PYG{n+na}{class}\PYG{o}{=}\PYG{l+s}{\PYGZdq{}nav\PYGZhy{}link active\PYGZdq{}} \PYG{n+na}{href}\PYG{o}{=}\PYG{l+s}{\PYGZdq{}\PYGZsh{}\PYGZdq{}}\PYG{p}{\PYGZgt{}}Activo\PYG{p}{\PYGZlt{}}\PYG{p}{/}\PYG{n+nt}{a}\PYG{p}{\PYGZgt{}}
  \PYG{p}{\PYGZlt{}}\PYG{p}{/}\PYG{n+nt}{li}\PYG{p}{\PYGZgt{}}
  \PYG{p}{\PYGZlt{}}\PYG{n+nt}{li} \PYG{n+na}{class}\PYG{o}{=}\PYG{l+s}{\PYGZdq{}nav\PYGZhy{}item\PYGZdq{}}\PYG{p}{\PYGZgt{}}
    \PYG{p}{\PYGZlt{}}\PYG{n+nt}{a} \PYG{n+na}{class}\PYG{o}{=}\PYG{l+s}{\PYGZdq{}nav\PYGZhy{}link\PYGZdq{}} \PYG{n+na}{href}\PYG{o}{=}\PYG{l+s}{\PYGZdq{}\PYGZsh{}\PYGZdq{}}\PYG{p}{\PYGZgt{}}Link\PYG{p}{\PYGZlt{}}\PYG{p}{/}\PYG{n+nt}{a}\PYG{p}{\PYGZgt{}}
  \PYG{p}{\PYGZlt{}}\PYG{p}{/}\PYG{n+nt}{li}\PYG{p}{\PYGZgt{}}
  \PYG{p}{\PYGZlt{}}\PYG{n+nt}{li} \PYG{n+na}{class}\PYG{o}{=}\PYG{l+s}{\PYGZdq{}nav\PYGZhy{}item\PYGZdq{}}\PYG{p}{\PYGZgt{}}
    \PYG{p}{\PYGZlt{}}\PYG{n+nt}{a} \PYG{n+na}{class}\PYG{o}{=}\PYG{l+s}{\PYGZdq{}nav\PYGZhy{}link\PYGZdq{}} \PYG{n+na}{href}\PYG{o}{=}\PYG{l+s}{\PYGZdq{}\PYGZsh{}\PYGZdq{}}\PYG{p}{\PYGZgt{}}Link\PYG{p}{\PYGZlt{}}\PYG{p}{/}\PYG{n+nt}{a}\PYG{p}{\PYGZgt{}}
  \PYG{p}{\PYGZlt{}}\PYG{p}{/}\PYG{n+nt}{li}\PYG{p}{\PYGZgt{}}
  \PYG{p}{\PYGZlt{}}\PYG{n+nt}{li} \PYG{n+na}{class}\PYG{o}{=}\PYG{l+s}{\PYGZdq{}nav\PYGZhy{}item\PYGZdq{}}\PYG{p}{\PYGZgt{}}
    \PYG{p}{\PYGZlt{}}\PYG{n+nt}{a} \PYG{n+na}{class}\PYG{o}{=}\PYG{l+s}{\PYGZdq{}nav\PYGZhy{}link disabled\PYGZdq{}} \PYG{n+na}{href}\PYG{o}{=}\PYG{l+s}{\PYGZdq{}\PYGZsh{}\PYGZdq{}}\PYG{p}{\PYGZgt{}}Inactivo\PYG{p}{\PYGZlt{}}\PYG{p}{/}\PYG{n+nt}{a}\PYG{p}{\PYGZgt{}}
  \PYG{p}{\PYGZlt{}}\PYG{p}{/}\PYG{n+nt}{li}\PYG{p}{\PYGZgt{}}
\PYG{p}{\PYGZlt{}}\PYG{p}{/}\PYG{n+nt}{ul}\PYG{p}{\PYGZgt{}}

\PYG{p}{\PYGZlt{}}\PYG{n+nt}{h3}\PYG{p}{\PYGZgt{}}Usando el tag nav\PYG{p}{\PYGZlt{}}\PYG{p}{/}\PYG{n+nt}{h3}\PYG{p}{\PYGZgt{}}

\PYG{p}{\PYGZlt{}}\PYG{n+nt}{nav} \PYG{n+na}{class}\PYG{o}{=}\PYG{l+s}{\PYGZdq{}nav\PYGZdq{}}\PYG{p}{\PYGZgt{}}
  \PYG{p}{\PYGZlt{}}\PYG{n+nt}{a} \PYG{n+na}{class}\PYG{o}{=}\PYG{l+s}{\PYGZdq{}nav\PYGZhy{}link active\PYGZdq{}} \PYG{n+na}{href}\PYG{o}{=}\PYG{l+s}{\PYGZdq{}\PYGZsh{}\PYGZdq{}}\PYG{p}{\PYGZgt{}}Activo\PYG{p}{\PYGZlt{}}\PYG{p}{/}\PYG{n+nt}{a}\PYG{p}{\PYGZgt{}}
  \PYG{p}{\PYGZlt{}}\PYG{n+nt}{a} \PYG{n+na}{class}\PYG{o}{=}\PYG{l+s}{\PYGZdq{}nav\PYGZhy{}link\PYGZdq{}} \PYG{n+na}{href}\PYG{o}{=}\PYG{l+s}{\PYGZdq{}\PYGZsh{}\PYGZdq{}}\PYG{p}{\PYGZgt{}}Link\PYG{p}{\PYGZlt{}}\PYG{p}{/}\PYG{n+nt}{a}\PYG{p}{\PYGZgt{}}
  \PYG{p}{\PYGZlt{}}\PYG{n+nt}{a} \PYG{n+na}{class}\PYG{o}{=}\PYG{l+s}{\PYGZdq{}nav\PYGZhy{}link\PYGZdq{}} \PYG{n+na}{href}\PYG{o}{=}\PYG{l+s}{\PYGZdq{}\PYGZsh{}\PYGZdq{}}\PYG{p}{\PYGZgt{}}Link\PYG{p}{\PYGZlt{}}\PYG{p}{/}\PYG{n+nt}{a}\PYG{p}{\PYGZgt{}}
  \PYG{p}{\PYGZlt{}}\PYG{n+nt}{a} \PYG{n+na}{class}\PYG{o}{=}\PYG{l+s}{\PYGZdq{}nav\PYGZhy{}link disabled\PYGZdq{}} \PYG{n+na}{href}\PYG{o}{=}\PYG{l+s}{\PYGZdq{}\PYGZsh{}\PYGZdq{}}\PYG{p}{\PYGZgt{}}Inactivo\PYG{p}{\PYGZlt{}}\PYG{p}{/}\PYG{n+nt}{a}\PYG{p}{\PYGZgt{}}
\PYG{p}{\PYGZlt{}}\PYG{p}{/}\PYG{n+nt}{nav}\PYG{p}{\PYGZgt{}}

\PYG{p}{\PYGZlt{}}\PYG{n+nt}{h3}\PYG{p}{\PYGZgt{}}Justificado al centro\PYG{p}{\PYGZlt{}}\PYG{p}{/}\PYG{n+nt}{h3}\PYG{p}{\PYGZgt{}}

\PYG{p}{\PYGZlt{}}\PYG{n+nt}{nav} \PYG{n+na}{class}\PYG{o}{=}\PYG{l+s}{\PYGZdq{}nav justify\PYGZhy{}content\PYGZhy{}center\PYGZdq{}}\PYG{p}{\PYGZgt{}}
  \PYG{p}{\PYGZlt{}}\PYG{n+nt}{a} \PYG{n+na}{class}\PYG{o}{=}\PYG{l+s}{\PYGZdq{}nav\PYGZhy{}link active\PYGZdq{}} \PYG{n+na}{href}\PYG{o}{=}\PYG{l+s}{\PYGZdq{}\PYGZsh{}\PYGZdq{}}\PYG{p}{\PYGZgt{}}Activo\PYG{p}{\PYGZlt{}}\PYG{p}{/}\PYG{n+nt}{a}\PYG{p}{\PYGZgt{}}
  \PYG{p}{\PYGZlt{}}\PYG{n+nt}{a} \PYG{n+na}{class}\PYG{o}{=}\PYG{l+s}{\PYGZdq{}nav\PYGZhy{}link\PYGZdq{}} \PYG{n+na}{href}\PYG{o}{=}\PYG{l+s}{\PYGZdq{}\PYGZsh{}\PYGZdq{}}\PYG{p}{\PYGZgt{}}Link\PYG{p}{\PYGZlt{}}\PYG{p}{/}\PYG{n+nt}{a}\PYG{p}{\PYGZgt{}}
  \PYG{p}{\PYGZlt{}}\PYG{n+nt}{a} \PYG{n+na}{class}\PYG{o}{=}\PYG{l+s}{\PYGZdq{}nav\PYGZhy{}link\PYGZdq{}} \PYG{n+na}{href}\PYG{o}{=}\PYG{l+s}{\PYGZdq{}\PYGZsh{}\PYGZdq{}}\PYG{p}{\PYGZgt{}}Link\PYG{p}{\PYGZlt{}}\PYG{p}{/}\PYG{n+nt}{a}\PYG{p}{\PYGZgt{}}
  \PYG{p}{\PYGZlt{}}\PYG{n+nt}{a} \PYG{n+na}{class}\PYG{o}{=}\PYG{l+s}{\PYGZdq{}nav\PYGZhy{}link disabled\PYGZdq{}} \PYG{n+na}{href}\PYG{o}{=}\PYG{l+s}{\PYGZdq{}\PYGZsh{}\PYGZdq{}}\PYG{p}{\PYGZgt{}}Inactivo\PYG{p}{\PYGZlt{}}\PYG{p}{/}\PYG{n+nt}{a}\PYG{p}{\PYGZgt{}}
\PYG{p}{\PYGZlt{}}\PYG{p}{/}\PYG{n+nt}{nav}\PYG{p}{\PYGZgt{}}

\PYG{p}{\PYGZlt{}}\PYG{n+nt}{h3}\PYG{p}{\PYGZgt{}}Justificado a la derecha\PYG{p}{\PYGZlt{}}\PYG{p}{/}\PYG{n+nt}{h3}\PYG{p}{\PYGZgt{}}

\PYG{p}{\PYGZlt{}}\PYG{n+nt}{nav} \PYG{n+na}{class}\PYG{o}{=}\PYG{l+s}{\PYGZdq{}nav justify\PYGZhy{}content\PYGZhy{}end\PYGZdq{}}\PYG{p}{\PYGZgt{}}
  \PYG{p}{\PYGZlt{}}\PYG{n+nt}{a} \PYG{n+na}{class}\PYG{o}{=}\PYG{l+s}{\PYGZdq{}nav\PYGZhy{}link active\PYGZdq{}} \PYG{n+na}{href}\PYG{o}{=}\PYG{l+s}{\PYGZdq{}\PYGZsh{}\PYGZdq{}}\PYG{p}{\PYGZgt{}}Activo\PYG{p}{\PYGZlt{}}\PYG{p}{/}\PYG{n+nt}{a}\PYG{p}{\PYGZgt{}}
  \PYG{p}{\PYGZlt{}}\PYG{n+nt}{a} \PYG{n+na}{class}\PYG{o}{=}\PYG{l+s}{\PYGZdq{}nav\PYGZhy{}link\PYGZdq{}} \PYG{n+na}{href}\PYG{o}{=}\PYG{l+s}{\PYGZdq{}\PYGZsh{}\PYGZdq{}}\PYG{p}{\PYGZgt{}}Link\PYG{p}{\PYGZlt{}}\PYG{p}{/}\PYG{n+nt}{a}\PYG{p}{\PYGZgt{}}
  \PYG{p}{\PYGZlt{}}\PYG{n+nt}{a} \PYG{n+na}{class}\PYG{o}{=}\PYG{l+s}{\PYGZdq{}nav\PYGZhy{}link\PYGZdq{}} \PYG{n+na}{href}\PYG{o}{=}\PYG{l+s}{\PYGZdq{}\PYGZsh{}\PYGZdq{}}\PYG{p}{\PYGZgt{}}Link\PYG{p}{\PYGZlt{}}\PYG{p}{/}\PYG{n+nt}{a}\PYG{p}{\PYGZgt{}}
  \PYG{p}{\PYGZlt{}}\PYG{n+nt}{a} \PYG{n+na}{class}\PYG{o}{=}\PYG{l+s}{\PYGZdq{}nav\PYGZhy{}link disabled\PYGZdq{}} \PYG{n+na}{href}\PYG{o}{=}\PYG{l+s}{\PYGZdq{}\PYGZsh{}\PYGZdq{}}\PYG{p}{\PYGZgt{}}Inactivo\PYG{p}{\PYGZlt{}}\PYG{p}{/}\PYG{n+nt}{a}\PYG{p}{\PYGZgt{}}
\PYG{p}{\PYGZlt{}}\PYG{p}{/}\PYG{n+nt}{nav}\PYG{p}{\PYGZgt{}}

\PYG{p}{\PYGZlt{}}\PYG{n+nt}{h3}\PYG{p}{\PYGZgt{}}Tabs\PYG{p}{\PYGZlt{}}\PYG{p}{/}\PYG{n+nt}{h3}\PYG{p}{\PYGZgt{}}

\PYG{p}{\PYGZlt{}}\PYG{n+nt}{nav} \PYG{n+na}{class}\PYG{o}{=}\PYG{l+s}{\PYGZdq{}nav nav\PYGZhy{}tabs\PYGZdq{}}\PYG{p}{\PYGZgt{}}
  \PYG{p}{\PYGZlt{}}\PYG{n+nt}{a} \PYG{n+na}{class}\PYG{o}{=}\PYG{l+s}{\PYGZdq{}nav\PYGZhy{}link active\PYGZdq{}} \PYG{n+na}{href}\PYG{o}{=}\PYG{l+s}{\PYGZdq{}\PYGZsh{}\PYGZdq{}}\PYG{p}{\PYGZgt{}}Activo\PYG{p}{\PYGZlt{}}\PYG{p}{/}\PYG{n+nt}{a}\PYG{p}{\PYGZgt{}}
  \PYG{p}{\PYGZlt{}}\PYG{n+nt}{a} \PYG{n+na}{class}\PYG{o}{=}\PYG{l+s}{\PYGZdq{}nav\PYGZhy{}link\PYGZdq{}} \PYG{n+na}{href}\PYG{o}{=}\PYG{l+s}{\PYGZdq{}\PYGZsh{}\PYGZdq{}}\PYG{p}{\PYGZgt{}}Link\PYG{p}{\PYGZlt{}}\PYG{p}{/}\PYG{n+nt}{a}\PYG{p}{\PYGZgt{}}
  \PYG{p}{\PYGZlt{}}\PYG{n+nt}{a} \PYG{n+na}{class}\PYG{o}{=}\PYG{l+s}{\PYGZdq{}nav\PYGZhy{}link\PYGZdq{}} \PYG{n+na}{href}\PYG{o}{=}\PYG{l+s}{\PYGZdq{}\PYGZsh{}\PYGZdq{}}\PYG{p}{\PYGZgt{}}Link\PYG{p}{\PYGZlt{}}\PYG{p}{/}\PYG{n+nt}{a}\PYG{p}{\PYGZgt{}}
  \PYG{p}{\PYGZlt{}}\PYG{n+nt}{a} \PYG{n+na}{class}\PYG{o}{=}\PYG{l+s}{\PYGZdq{}nav\PYGZhy{}link disabled\PYGZdq{}} \PYG{n+na}{href}\PYG{o}{=}\PYG{l+s}{\PYGZdq{}\PYGZsh{}\PYGZdq{}}\PYG{p}{\PYGZgt{}}Inactivo\PYG{p}{\PYGZlt{}}\PYG{p}{/}\PYG{n+nt}{a}\PYG{p}{\PYGZgt{}}
\PYG{p}{\PYGZlt{}}\PYG{p}{/}\PYG{n+nt}{nav}\PYG{p}{\PYGZgt{}}

\PYG{p}{\PYGZlt{}}\PYG{n+nt}{h3}\PYG{p}{\PYGZgt{}}Pills\PYG{p}{\PYGZlt{}}\PYG{p}{/}\PYG{n+nt}{h3}\PYG{p}{\PYGZgt{}}

\PYG{p}{\PYGZlt{}}\PYG{n+nt}{nav} \PYG{n+na}{class}\PYG{o}{=}\PYG{l+s}{\PYGZdq{}nav nav\PYGZhy{}pills\PYGZdq{}}\PYG{p}{\PYGZgt{}}
  \PYG{p}{\PYGZlt{}}\PYG{n+nt}{a} \PYG{n+na}{class}\PYG{o}{=}\PYG{l+s}{\PYGZdq{}nav\PYGZhy{}link active\PYGZdq{}} \PYG{n+na}{href}\PYG{o}{=}\PYG{l+s}{\PYGZdq{}\PYGZsh{}\PYGZdq{}}\PYG{p}{\PYGZgt{}}Activo\PYG{p}{\PYGZlt{}}\PYG{p}{/}\PYG{n+nt}{a}\PYG{p}{\PYGZgt{}}
  \PYG{p}{\PYGZlt{}}\PYG{n+nt}{a} \PYG{n+na}{class}\PYG{o}{=}\PYG{l+s}{\PYGZdq{}nav\PYGZhy{}link\PYGZdq{}} \PYG{n+na}{href}\PYG{o}{=}\PYG{l+s}{\PYGZdq{}\PYGZsh{}\PYGZdq{}}\PYG{p}{\PYGZgt{}}Link\PYG{p}{\PYGZlt{}}\PYG{p}{/}\PYG{n+nt}{a}\PYG{p}{\PYGZgt{}}
  \PYG{p}{\PYGZlt{}}\PYG{n+nt}{a} \PYG{n+na}{class}\PYG{o}{=}\PYG{l+s}{\PYGZdq{}nav\PYGZhy{}link\PYGZdq{}} \PYG{n+na}{href}\PYG{o}{=}\PYG{l+s}{\PYGZdq{}\PYGZsh{}\PYGZdq{}}\PYG{p}{\PYGZgt{}}Link\PYG{p}{\PYGZlt{}}\PYG{p}{/}\PYG{n+nt}{a}\PYG{p}{\PYGZgt{}}
  \PYG{p}{\PYGZlt{}}\PYG{n+nt}{a} \PYG{n+na}{class}\PYG{o}{=}\PYG{l+s}{\PYGZdq{}nav\PYGZhy{}link disabled\PYGZdq{}} \PYG{n+na}{href}\PYG{o}{=}\PYG{l+s}{\PYGZdq{}\PYGZsh{}\PYGZdq{}}\PYG{p}{\PYGZgt{}}Inactivo\PYG{p}{\PYGZlt{}}\PYG{p}{/}\PYG{n+nt}{a}\PYG{p}{\PYGZgt{}}
\PYG{p}{\PYGZlt{}}\PYG{p}{/}\PYG{n+nt}{nav}\PYG{p}{\PYGZgt{}}

\PYG{p}{\PYGZlt{}}\PYG{n+nt}{h3}\PYG{p}{\PYGZgt{}}Tabs Expandidas (Lista)\PYG{p}{\PYGZlt{}}\PYG{p}{/}\PYG{n+nt}{h3}\PYG{p}{\PYGZgt{}}

\PYG{p}{\PYGZlt{}}\PYG{n+nt}{ul} \PYG{n+na}{class}\PYG{o}{=}\PYG{l+s}{\PYGZdq{}nav nav\PYGZhy{}tabs nav\PYGZhy{}fill\PYGZdq{}}\PYG{p}{\PYGZgt{}}
  \PYG{p}{\PYGZlt{}}\PYG{n+nt}{li} \PYG{n+na}{class}\PYG{o}{=}\PYG{l+s}{\PYGZdq{}nav\PYGZhy{}item\PYGZdq{}}\PYG{p}{\PYGZgt{}}
    \PYG{p}{\PYGZlt{}}\PYG{n+nt}{a} \PYG{n+na}{class}\PYG{o}{=}\PYG{l+s}{\PYGZdq{}nav\PYGZhy{}link active\PYGZdq{}} \PYG{n+na}{href}\PYG{o}{=}\PYG{l+s}{\PYGZdq{}\PYGZsh{}\PYGZdq{}}\PYG{p}{\PYGZgt{}}Activo\PYG{p}{\PYGZlt{}}\PYG{p}{/}\PYG{n+nt}{a}\PYG{p}{\PYGZgt{}}
  \PYG{p}{\PYGZlt{}}\PYG{p}{/}\PYG{n+nt}{li}\PYG{p}{\PYGZgt{}}
  \PYG{p}{\PYGZlt{}}\PYG{n+nt}{li} \PYG{n+na}{class}\PYG{o}{=}\PYG{l+s}{\PYGZdq{}nav\PYGZhy{}item\PYGZdq{}}\PYG{p}{\PYGZgt{}}
    \PYG{p}{\PYGZlt{}}\PYG{n+nt}{a} \PYG{n+na}{class}\PYG{o}{=}\PYG{l+s}{\PYGZdq{}nav\PYGZhy{}link\PYGZdq{}} \PYG{n+na}{href}\PYG{o}{=}\PYG{l+s}{\PYGZdq{}\PYGZsh{}\PYGZdq{}}\PYG{p}{\PYGZgt{}}Link\PYG{p}{\PYGZlt{}}\PYG{p}{/}\PYG{n+nt}{a}\PYG{p}{\PYGZgt{}}
  \PYG{p}{\PYGZlt{}}\PYG{p}{/}\PYG{n+nt}{li}\PYG{p}{\PYGZgt{}}
  \PYG{p}{\PYGZlt{}}\PYG{n+nt}{li} \PYG{n+na}{class}\PYG{o}{=}\PYG{l+s}{\PYGZdq{}nav\PYGZhy{}item\PYGZdq{}}\PYG{p}{\PYGZgt{}}
    \PYG{p}{\PYGZlt{}}\PYG{n+nt}{a} \PYG{n+na}{class}\PYG{o}{=}\PYG{l+s}{\PYGZdq{}nav\PYGZhy{}link\PYGZdq{}} \PYG{n+na}{href}\PYG{o}{=}\PYG{l+s}{\PYGZdq{}\PYGZsh{}\PYGZdq{}}\PYG{p}{\PYGZgt{}}Link\PYG{p}{\PYGZlt{}}\PYG{p}{/}\PYG{n+nt}{a}\PYG{p}{\PYGZgt{}}
  \PYG{p}{\PYGZlt{}}\PYG{p}{/}\PYG{n+nt}{li}\PYG{p}{\PYGZgt{}}
  \PYG{p}{\PYGZlt{}}\PYG{n+nt}{li} \PYG{n+na}{class}\PYG{o}{=}\PYG{l+s}{\PYGZdq{}nav\PYGZhy{}item\PYGZdq{}}\PYG{p}{\PYGZgt{}}
    \PYG{p}{\PYGZlt{}}\PYG{n+nt}{a} \PYG{n+na}{class}\PYG{o}{=}\PYG{l+s}{\PYGZdq{}nav\PYGZhy{}link disabled\PYGZdq{}} \PYG{n+na}{href}\PYG{o}{=}\PYG{l+s}{\PYGZdq{}\PYGZsh{}\PYGZdq{}}\PYG{p}{\PYGZgt{}}Inactivo\PYG{p}{\PYGZlt{}}\PYG{p}{/}\PYG{n+nt}{a}\PYG{p}{\PYGZgt{}}
  \PYG{p}{\PYGZlt{}}\PYG{p}{/}\PYG{n+nt}{li}\PYG{p}{\PYGZgt{}}
\PYG{p}{\PYGZlt{}}\PYG{p}{/}\PYG{n+nt}{ul}\PYG{p}{\PYGZgt{}}

\PYG{p}{\PYGZlt{}}\PYG{n+nt}{h3}\PYG{p}{\PYGZgt{}}Pills Expandidas (Nav)\PYG{p}{\PYGZlt{}}\PYG{p}{/}\PYG{n+nt}{h3}\PYG{p}{\PYGZgt{}}

\PYG{p}{\PYGZlt{}}\PYG{n+nt}{nav} \PYG{n+na}{class}\PYG{o}{=}\PYG{l+s}{\PYGZdq{}nav nav\PYGZhy{}pills nav\PYGZhy{}fill\PYGZdq{}}\PYG{p}{\PYGZgt{}}
  \PYG{p}{\PYGZlt{}}\PYG{n+nt}{a} \PYG{n+na}{class}\PYG{o}{=}\PYG{l+s}{\PYGZdq{}nav\PYGZhy{}item nav\PYGZhy{}link active\PYGZdq{}} \PYG{n+na}{href}\PYG{o}{=}\PYG{l+s}{\PYGZdq{}\PYGZsh{}\PYGZdq{}}\PYG{p}{\PYGZgt{}}Activo\PYG{p}{\PYGZlt{}}\PYG{p}{/}\PYG{n+nt}{a}\PYG{p}{\PYGZgt{}}
  \PYG{p}{\PYGZlt{}}\PYG{n+nt}{a} \PYG{n+na}{class}\PYG{o}{=}\PYG{l+s}{\PYGZdq{}nav\PYGZhy{}item nav\PYGZhy{}link\PYGZdq{}} \PYG{n+na}{href}\PYG{o}{=}\PYG{l+s}{\PYGZdq{}\PYGZsh{}\PYGZdq{}}\PYG{p}{\PYGZgt{}}Link\PYG{p}{\PYGZlt{}}\PYG{p}{/}\PYG{n+nt}{a}\PYG{p}{\PYGZgt{}}
  \PYG{p}{\PYGZlt{}}\PYG{n+nt}{a} \PYG{n+na}{class}\PYG{o}{=}\PYG{l+s}{\PYGZdq{}nav\PYGZhy{}item nav\PYGZhy{}link\PYGZdq{}} \PYG{n+na}{href}\PYG{o}{=}\PYG{l+s}{\PYGZdq{}\PYGZsh{}\PYGZdq{}}\PYG{p}{\PYGZgt{}}Link\PYG{p}{\PYGZlt{}}\PYG{p}{/}\PYG{n+nt}{a}\PYG{p}{\PYGZgt{}}
  \PYG{p}{\PYGZlt{}}\PYG{n+nt}{a} \PYG{n+na}{class}\PYG{o}{=}\PYG{l+s}{\PYGZdq{}nav\PYGZhy{}item nav\PYGZhy{}link disabled\PYGZdq{}} \PYG{n+na}{href}\PYG{o}{=}\PYG{l+s}{\PYGZdq{}\PYGZsh{}\PYGZdq{}}\PYG{p}{\PYGZgt{}}Inactivo\PYG{p}{\PYGZlt{}}\PYG{p}{/}\PYG{n+nt}{a}\PYG{p}{\PYGZgt{}}
\PYG{p}{\PYGZlt{}}\PYG{p}{/}\PYG{n+nt}{nav}\PYG{p}{\PYGZgt{}}
\end{sphinxVerbatim}

El resultado es algo así:

\begin{figure}[htbp]
\centering

\noindent\sphinxincludegraphics{{05-nav}.png}
\end{figure}

Luego de explorar los componentes copiando y pegando los ejemplos intenta
modificarlos, agregar mas items, eliminar algunos, reordenarlos etc.

Si te sentís aventurero, intenta mirar un ejemplo y luego tiperarlo por
completo solo mirando el ejemplo cuando te olvides de algo o algo no funcione.

Este ejercicio es una buena forma de memorizar los conceptos básicos y de ver
cuales partes pensabas que entendías pero todavía algún detalle se escapa.


\chapter{Reusando HTML (de otros)}
\label{\detokenize{reusando-html-de-otros::doc}}\label{\detokenize{reusando-html-de-otros:reusando-html-de-otros}}
En la sección anterior vimos como reusar estilo creado por otras personas.

Como vimos, muchas paginas comparten distintos componentes y yendo un nivel
mas arriba, muchas paginas tienen incluso una estructura similar, por ejemplo
blogs, paginas principales de un diario, de un producto, o de un sitio de ventas
online.

Si bien no hay una forma simple de reusar HTML sin acudir a funcionalidades mas
avanzadas que requieran programar, existe lo que usualmente se llaman
templates (plantillas), que son paginas completas o secciones grandes de paginas para copiar
y modificar el contenido de manera de no empezar de cero.

En esta sección vamos a explorar algunos de esos templates, en este caso, los
disponibles en la pagina del proyecto bootstrap: \sphinxurl{https://getbootstrap.com/docs/4.1/examples/}


\section{Cover}
\label{\detokenize{reusando-html-de-otros:cover}}
Probemos uno de los ejemplos llamado \sphinxstyleemphasis{Cover}, copia y pega el siguiente HTML en
un proyecto thimble nuevo:

\fvset{hllines={, ,}}%
\begin{sphinxVerbatim}[commandchars=\\\{\}]
\PYG{c+cp}{\PYGZlt{}!doctype html\PYGZgt{}}
\PYG{p}{\PYGZlt{}}\PYG{n+nt}{html} \PYG{n+na}{lang}\PYG{o}{=}\PYG{l+s}{\PYGZdq{}en\PYGZdq{}}\PYG{p}{\PYGZgt{}}
  \PYG{p}{\PYGZlt{}}\PYG{n+nt}{head}\PYG{p}{\PYGZgt{}}
        \PYG{p}{\PYGZlt{}}\PYG{n+nt}{meta} \PYG{n+na}{charset}\PYG{o}{=}\PYG{l+s}{\PYGZdq{}utf\PYGZhy{}8\PYGZdq{}}\PYG{p}{\PYGZgt{}}
        \PYG{p}{\PYGZlt{}}\PYG{n+nt}{meta} \PYG{n+na}{name}\PYG{o}{=}\PYG{l+s}{\PYGZdq{}viewport\PYGZdq{}} \PYG{n+na}{content}\PYG{o}{=}\PYG{l+s}{\PYGZdq{}width=device\PYGZhy{}width, initial\PYGZhy{}scale=1, shrink\PYGZhy{}to\PYGZhy{}fit=no\PYGZdq{}}\PYG{p}{\PYGZgt{}}

        \PYG{p}{\PYGZlt{}}\PYG{n+nt}{title}\PYG{p}{\PYGZgt{}}Titulo de Pagina\PYG{p}{\PYGZlt{}}\PYG{p}{/}\PYG{n+nt}{title}\PYG{p}{\PYGZgt{}}

        \PYG{p}{\PYGZlt{}}\PYG{n+nt}{link} \PYG{n+na}{rel}\PYG{o}{=}\PYG{l+s}{\PYGZdq{}stylesheet\PYGZdq{}} \PYG{n+na}{href}\PYG{o}{=}\PYG{l+s}{\PYGZdq{}https://stackpath.bootstrapcdn.com/bootstrap/4.1.0/css/bootstrap.min.css\PYGZdq{}}\PYG{p}{\PYGZgt{}}

        \PYG{p}{\PYGZlt{}}\PYG{n+nt}{link} \PYG{n+na}{href}\PYG{o}{=}\PYG{l+s}{\PYGZdq{}style.css\PYGZdq{}} \PYG{n+na}{rel}\PYG{o}{=}\PYG{l+s}{\PYGZdq{}stylesheet\PYGZdq{}}\PYG{p}{\PYGZgt{}}
  \PYG{p}{\PYGZlt{}}\PYG{p}{/}\PYG{n+nt}{head}\PYG{p}{\PYGZgt{}}

  \PYG{p}{\PYGZlt{}}\PYG{n+nt}{body} \PYG{n+na}{class}\PYG{o}{=}\PYG{l+s}{\PYGZdq{}text\PYGZhy{}center\PYGZdq{}}\PYG{p}{\PYGZgt{}}

        \PYG{p}{\PYGZlt{}}\PYG{n+nt}{div} \PYG{n+na}{class}\PYG{o}{=}\PYG{l+s}{\PYGZdq{}cover\PYGZhy{}container d\PYGZhy{}flex w\PYGZhy{}100 h\PYGZhy{}100 p\PYGZhy{}3 mx\PYGZhy{}auto flex\PYGZhy{}column\PYGZdq{}}\PYG{p}{\PYGZgt{}}
          \PYG{p}{\PYGZlt{}}\PYG{n+nt}{header} \PYG{n+na}{class}\PYG{o}{=}\PYG{l+s}{\PYGZdq{}masthead mb\PYGZhy{}auto\PYGZdq{}}\PYG{p}{\PYGZgt{}}
                \PYG{p}{\PYGZlt{}}\PYG{n+nt}{div} \PYG{n+na}{class}\PYG{o}{=}\PYG{l+s}{\PYGZdq{}inner\PYGZdq{}}\PYG{p}{\PYGZgt{}}
                  \PYG{p}{\PYGZlt{}}\PYG{n+nt}{h3} \PYG{n+na}{class}\PYG{o}{=}\PYG{l+s}{\PYGZdq{}masthead\PYGZhy{}brand\PYGZdq{}}\PYG{p}{\PYGZgt{}}Nombre\PYG{p}{\PYGZlt{}}\PYG{p}{/}\PYG{n+nt}{h3}\PYG{p}{\PYGZgt{}}
                  \PYG{p}{\PYGZlt{}}\PYG{n+nt}{nav} \PYG{n+na}{class}\PYG{o}{=}\PYG{l+s}{\PYGZdq{}nav nav\PYGZhy{}masthead justify\PYGZhy{}content\PYGZhy{}center\PYGZdq{}}\PYG{p}{\PYGZgt{}}
                        \PYG{p}{\PYGZlt{}}\PYG{n+nt}{a} \PYG{n+na}{class}\PYG{o}{=}\PYG{l+s}{\PYGZdq{}nav\PYGZhy{}link active\PYGZdq{}} \PYG{n+na}{href}\PYG{o}{=}\PYG{l+s}{\PYGZdq{}\PYGZsh{}\PYGZdq{}}\PYG{p}{\PYGZgt{}}Principal\PYG{p}{\PYGZlt{}}\PYG{p}{/}\PYG{n+nt}{a}\PYG{p}{\PYGZgt{}}
                        \PYG{p}{\PYGZlt{}}\PYG{n+nt}{a} \PYG{n+na}{class}\PYG{o}{=}\PYG{l+s}{\PYGZdq{}nav\PYGZhy{}link\PYGZdq{}} \PYG{n+na}{href}\PYG{o}{=}\PYG{l+s}{\PYGZdq{}\PYGZsh{}\PYGZdq{}}\PYG{p}{\PYGZgt{}}Link 1\PYG{p}{\PYGZlt{}}\PYG{p}{/}\PYG{n+nt}{a}\PYG{p}{\PYGZgt{}}
                        \PYG{p}{\PYGZlt{}}\PYG{n+nt}{a} \PYG{n+na}{class}\PYG{o}{=}\PYG{l+s}{\PYGZdq{}nav\PYGZhy{}link\PYGZdq{}} \PYG{n+na}{href}\PYG{o}{=}\PYG{l+s}{\PYGZdq{}\PYGZsh{}\PYGZdq{}}\PYG{p}{\PYGZgt{}}Link 2\PYG{p}{\PYGZlt{}}\PYG{p}{/}\PYG{n+nt}{a}\PYG{p}{\PYGZgt{}}
                  \PYG{p}{\PYGZlt{}}\PYG{p}{/}\PYG{n+nt}{nav}\PYG{p}{\PYGZgt{}}
                \PYG{p}{\PYGZlt{}}\PYG{p}{/}\PYG{n+nt}{div}\PYG{p}{\PYGZgt{}}
          \PYG{p}{\PYGZlt{}}\PYG{p}{/}\PYG{n+nt}{header}\PYG{p}{\PYGZgt{}}

          \PYG{p}{\PYGZlt{}}\PYG{n+nt}{main} \PYG{n+na}{role}\PYG{o}{=}\PYG{l+s}{\PYGZdq{}main\PYGZdq{}} \PYG{n+na}{class}\PYG{o}{=}\PYG{l+s}{\PYGZdq{}inner cover\PYGZdq{}}\PYG{p}{\PYGZgt{}}
                \PYG{p}{\PYGZlt{}}\PYG{n+nt}{h1} \PYG{n+na}{class}\PYG{o}{=}\PYG{l+s}{\PYGZdq{}cover\PYGZhy{}heading\PYGZdq{}}\PYG{p}{\PYGZgt{}}Título\PYG{p}{\PYGZlt{}}\PYG{p}{/}\PYG{n+nt}{h1}\PYG{p}{\PYGZgt{}}
                \PYG{p}{\PYGZlt{}}\PYG{n+nt}{p} \PYG{n+na}{class}\PYG{o}{=}\PYG{l+s}{\PYGZdq{}lead\PYGZdq{}}\PYG{p}{\PYGZgt{}}Descripción.\PYG{p}{\PYGZlt{}}\PYG{p}{/}\PYG{n+nt}{p}\PYG{p}{\PYGZgt{}}
                \PYG{p}{\PYGZlt{}}\PYG{n+nt}{p} \PYG{n+na}{class}\PYG{o}{=}\PYG{l+s}{\PYGZdq{}lead\PYGZdq{}}\PYG{p}{\PYGZgt{}}
                  \PYG{p}{\PYGZlt{}}\PYG{n+nt}{a} \PYG{n+na}{href}\PYG{o}{=}\PYG{l+s}{\PYGZdq{}\PYGZsh{}\PYGZdq{}} \PYG{n+na}{class}\PYG{o}{=}\PYG{l+s}{\PYGZdq{}btn btn\PYGZhy{}lg btn\PYGZhy{}secondary\PYGZdq{}}\PYG{p}{\PYGZgt{}}Acción Principal\PYG{p}{\PYGZlt{}}\PYG{p}{/}\PYG{n+nt}{a}\PYG{p}{\PYGZgt{}}
                \PYG{p}{\PYGZlt{}}\PYG{p}{/}\PYG{n+nt}{p}\PYG{p}{\PYGZgt{}}
          \PYG{p}{\PYGZlt{}}\PYG{p}{/}\PYG{n+nt}{main}\PYG{p}{\PYGZgt{}}

          \PYG{p}{\PYGZlt{}}\PYG{n+nt}{footer} \PYG{n+na}{class}\PYG{o}{=}\PYG{l+s}{\PYGZdq{}mastfoot mt\PYGZhy{}auto\PYGZdq{}}\PYG{p}{\PYGZgt{}}
                \PYG{p}{\PYGZlt{}}\PYG{n+nt}{div} \PYG{n+na}{class}\PYG{o}{=}\PYG{l+s}{\PYGZdq{}inner\PYGZdq{}}\PYG{p}{\PYGZgt{}}
                  \PYG{p}{\PYGZlt{}}\PYG{n+nt}{p}\PYG{p}{\PYGZgt{}}Template \PYGZdq{}Cover\PYGZdq{} para \PYG{p}{\PYGZlt{}}\PYG{n+nt}{a} \PYG{n+na}{href}\PYG{o}{=}\PYG{l+s}{\PYGZdq{}https://getbootstrap.com/\PYGZdq{}}\PYG{p}{\PYGZgt{}}Bootstrap\PYG{p}{\PYGZlt{}}\PYG{p}{/}\PYG{n+nt}{a}\PYG{p}{\PYGZgt{}}, por \PYG{p}{\PYGZlt{}}\PYG{n+nt}{a} \PYG{n+na}{href}\PYG{o}{=}\PYG{l+s}{\PYGZdq{}https://twitter.com/mdo\PYGZdq{}}\PYG{p}{\PYGZgt{}}@mdo\PYG{p}{\PYGZlt{}}\PYG{p}{/}\PYG{n+nt}{a}\PYG{p}{\PYGZgt{}}.\PYG{p}{\PYGZlt{}}\PYG{p}{/}\PYG{n+nt}{p}\PYG{p}{\PYGZgt{}}
                \PYG{p}{\PYGZlt{}}\PYG{p}{/}\PYG{n+nt}{div}\PYG{p}{\PYGZgt{}}
          \PYG{p}{\PYGZlt{}}\PYG{p}{/}\PYG{n+nt}{footer}\PYG{p}{\PYGZgt{}}
        \PYG{p}{\PYGZlt{}}\PYG{p}{/}\PYG{n+nt}{div}\PYG{p}{\PYGZgt{}}

  \PYG{p}{\PYGZlt{}}\PYG{p}{/}\PYG{n+nt}{body}\PYG{p}{\PYGZgt{}}
\PYG{p}{\PYGZlt{}}\PYG{p}{/}\PYG{n+nt}{html}\PYG{p}{\PYGZgt{}}
\end{sphinxVerbatim}

Y el siguiente CSS en el archivo style.css del proyecto:

\fvset{hllines={, ,}}%
\begin{sphinxVerbatim}[commandchars=\\\{\}]
\PYG{c}{/*}
\PYG{c}{* Globals}
\PYG{c}{*/}

\PYG{c}{/* Links */}
\PYG{n+nt}{a}\PYG{o}{,}
\PYG{n+nt}{a}\PYG{p}{:}\PYG{n+nd}{focus}\PYG{o}{,}
\PYG{n+nt}{a}\PYG{p}{:}\PYG{n+nd}{hover} \PYG{p}{\PYGZob{}}
  \PYG{k}{color}\PYG{p}{:} \PYG{l+m+mh}{\PYGZsh{}fff}\PYG{p}{;}
\PYG{p}{\PYGZcb{}}

\PYG{c}{/* Custom default button */}
\PYG{p}{.}\PYG{n+nc}{btn\PYGZhy{}secondary}\PYG{o}{,}
\PYG{p}{.}\PYG{n+nc}{btn\PYGZhy{}secondary}\PYG{p}{:}\PYG{n+nd}{hover}\PYG{o}{,}
\PYG{p}{.}\PYG{n+nc}{btn\PYGZhy{}secondary}\PYG{p}{:}\PYG{n+nd}{focus} \PYG{p}{\PYGZob{}}
  \PYG{k}{color}\PYG{p}{:} \PYG{l+m+mh}{\PYGZsh{}333}\PYG{p}{;}
  \PYG{k}{text\PYGZhy{}shadow}\PYG{p}{:} \PYG{k+kc}{none}\PYG{p}{;} \PYG{c}{/* Prevent inheritance from {}`body{}` */}
  \PYG{k}{background\PYGZhy{}color}\PYG{p}{:} \PYG{l+m+mh}{\PYGZsh{}fff}\PYG{p}{;}
  \PYG{k}{border}\PYG{p}{:} \PYG{l+m+mf}{.05}\PYG{k+kt}{rem} \PYG{k+kc}{solid} \PYG{l+m+mh}{\PYGZsh{}fff}\PYG{p}{;}
\PYG{p}{\PYGZcb{}}


\PYG{c}{/*}
\PYG{c}{* Base structure}
\PYG{c}{*/}

\PYG{n+nt}{html}\PYG{o}{,}
\PYG{n+nt}{body} \PYG{p}{\PYGZob{}}
  \PYG{k}{height}\PYG{p}{:} \PYG{l+m+mi}{100}\PYG{k+kt}{\PYGZpc{}}\PYG{p}{;}
  \PYG{k}{background\PYGZhy{}color}\PYG{p}{:} \PYG{l+m+mh}{\PYGZsh{}333}\PYG{p}{;}
\PYG{p}{\PYGZcb{}}

\PYG{n+nt}{body} \PYG{p}{\PYGZob{}}
  \PYG{k}{display}\PYG{p}{:} \PYG{n+nb+bp}{\PYGZhy{}ms\PYGZhy{}}\PYG{n}{flexbox}\PYG{p}{;}
  \PYG{k}{display}\PYG{p}{:} \PYG{k+kc}{flex}\PYG{p}{;}
  \PYG{k}{color}\PYG{p}{:} \PYG{l+m+mh}{\PYGZsh{}fff}\PYG{p}{;}
  \PYG{k}{text\PYGZhy{}shadow}\PYG{p}{:} \PYG{l+m+mi}{0} \PYG{l+m+mf}{.05}\PYG{k+kt}{rem} \PYG{l+m+mf}{.1}\PYG{k+kt}{rem} \PYG{n+nb}{rgba}\PYG{p}{(}\PYG{l+m+mi}{0}\PYG{p}{,} \PYG{l+m+mi}{0}\PYG{p}{,} \PYG{l+m+mi}{0}\PYG{p}{,} \PYG{l+m+mf}{.5}\PYG{p}{)}\PYG{p}{;}
  \PYG{k}{box\PYGZhy{}shadow}\PYG{p}{:} \PYG{k+kc}{inset} \PYG{l+m+mi}{0} \PYG{l+m+mi}{0} \PYG{l+m+mi}{5}\PYG{k+kt}{rem} \PYG{n+nb}{rgba}\PYG{p}{(}\PYG{l+m+mi}{0}\PYG{p}{,} \PYG{l+m+mi}{0}\PYG{p}{,} \PYG{l+m+mi}{0}\PYG{p}{,} \PYG{l+m+mf}{.5}\PYG{p}{)}\PYG{p}{;}
\PYG{p}{\PYGZcb{}}

\PYG{p}{.}\PYG{n+nc}{cover\PYGZhy{}container} \PYG{p}{\PYGZob{}}
  \PYG{k}{max\PYGZhy{}width}\PYG{p}{:} \PYG{l+m+mi}{42}\PYG{k+kt}{em}\PYG{p}{;}
\PYG{p}{\PYGZcb{}}


\PYG{c}{/*}
\PYG{c}{* Header}
\PYG{c}{*/}
\PYG{p}{.}\PYG{n+nc}{masthead} \PYG{p}{\PYGZob{}}
  \PYG{k}{margin\PYGZhy{}bottom}\PYG{p}{:} \PYG{l+m+mi}{2}\PYG{k+kt}{rem}\PYG{p}{;}
\PYG{p}{\PYGZcb{}}

\PYG{p}{.}\PYG{n+nc}{masthead\PYGZhy{}brand} \PYG{p}{\PYGZob{}}
  \PYG{k}{margin\PYGZhy{}bottom}\PYG{p}{:} \PYG{l+m+mi}{0}\PYG{p}{;}
\PYG{p}{\PYGZcb{}}

\PYG{p}{.}\PYG{n+nc}{nav\PYGZhy{}masthead} \PYG{p}{.}\PYG{n+nc}{nav\PYGZhy{}link} \PYG{p}{\PYGZob{}}
  \PYG{k}{padding}\PYG{p}{:} \PYG{l+m+mf}{.25}\PYG{k+kt}{rem} \PYG{l+m+mi}{0}\PYG{p}{;}
  \PYG{k}{font\PYGZhy{}weight}\PYG{p}{:} \PYG{l+m+mi}{700}\PYG{p}{;}
  \PYG{k}{color}\PYG{p}{:} \PYG{n+nb}{rgba}\PYG{p}{(}\PYG{l+m+mi}{255}\PYG{p}{,} \PYG{l+m+mi}{255}\PYG{p}{,} \PYG{l+m+mi}{255}\PYG{p}{,} \PYG{l+m+mf}{.5}\PYG{p}{)}\PYG{p}{;}
  \PYG{k}{background\PYGZhy{}color}\PYG{p}{:} \PYG{k+kc}{transparent}\PYG{p}{;}
  \PYG{k}{border\PYGZhy{}bottom}\PYG{p}{:} \PYG{l+m+mf}{.25}\PYG{k+kt}{rem} \PYG{k+kc}{solid} \PYG{k+kc}{transparent}\PYG{p}{;}
\PYG{p}{\PYGZcb{}}

\PYG{p}{.}\PYG{n+nc}{nav\PYGZhy{}masthead} \PYG{p}{.}\PYG{n+nc}{nav\PYGZhy{}link}\PYG{p}{:}\PYG{n+nd}{hover}\PYG{o}{,}
\PYG{p}{.}\PYG{n+nc}{nav\PYGZhy{}masthead} \PYG{p}{.}\PYG{n+nc}{nav\PYGZhy{}link}\PYG{p}{:}\PYG{n+nd}{focus} \PYG{p}{\PYGZob{}}
  \PYG{k}{border\PYGZhy{}bottom\PYGZhy{}color}\PYG{p}{:} \PYG{n+nb}{rgba}\PYG{p}{(}\PYG{l+m+mi}{255}\PYG{p}{,} \PYG{l+m+mi}{255}\PYG{p}{,} \PYG{l+m+mi}{255}\PYG{p}{,} \PYG{l+m+mf}{.25}\PYG{p}{)}\PYG{p}{;}
\PYG{p}{\PYGZcb{}}

\PYG{p}{.}\PYG{n+nc}{nav\PYGZhy{}masthead} \PYG{p}{.}\PYG{n+nc}{nav\PYGZhy{}link} \PYG{o}{+} \PYG{p}{.}\PYG{n+nc}{nav\PYGZhy{}link} \PYG{p}{\PYGZob{}}
  \PYG{k}{margin\PYGZhy{}left}\PYG{p}{:} \PYG{l+m+mi}{1}\PYG{k+kt}{rem}\PYG{p}{;}
\PYG{p}{\PYGZcb{}}

\PYG{p}{.}\PYG{n+nc}{nav\PYGZhy{}masthead} \PYG{p}{.}\PYG{n+nc}{active} \PYG{p}{\PYGZob{}}
  \PYG{k}{color}\PYG{p}{:} \PYG{l+m+mh}{\PYGZsh{}fff}\PYG{p}{;}
  \PYG{k}{border\PYGZhy{}bottom\PYGZhy{}color}\PYG{p}{:} \PYG{l+m+mh}{\PYGZsh{}fff}\PYG{p}{;}
\PYG{p}{\PYGZcb{}}

\PYG{p}{@}\PYG{k}{media} \PYG{o}{(}\PYG{n+nt}{min\PYGZhy{}width}\PYG{o}{:} \PYG{n+nt}{48em}\PYG{o}{)} \PYG{p}{\PYGZob{}}
  \PYG{p}{.}\PYG{n+nc}{masthead\PYGZhy{}brand} \PYG{p}{\PYGZob{}}
        \PYG{k}{float}\PYG{p}{:} \PYG{k+kc}{left}\PYG{p}{;}
  \PYG{p}{\PYGZcb{}}
  \PYG{p}{.}\PYG{n+nc}{nav\PYGZhy{}masthead} \PYG{p}{\PYGZob{}}
        \PYG{k}{float}\PYG{p}{:} \PYG{k+kc}{right}\PYG{p}{;}
  \PYG{p}{\PYGZcb{}}
\PYG{p}{\PYGZcb{}}


\PYG{c}{/*}
\PYG{c}{* Cover}
\PYG{c}{*/}
\PYG{p}{.}\PYG{n+nc}{cover} \PYG{p}{\PYGZob{}}
  \PYG{k}{padding}\PYG{p}{:} \PYG{l+m+mi}{0} \PYG{l+m+mf}{1.5}\PYG{k+kt}{rem}\PYG{p}{;}
\PYG{p}{\PYGZcb{}}
\PYG{p}{.}\PYG{n+nc}{cover} \PYG{p}{.}\PYG{n+nc}{btn\PYGZhy{}lg} \PYG{p}{\PYGZob{}}
  \PYG{k}{padding}\PYG{p}{:} \PYG{l+m+mf}{.75}\PYG{k+kt}{rem} \PYG{l+m+mf}{1.25}\PYG{k+kt}{rem}\PYG{p}{;}
  \PYG{k}{font\PYGZhy{}weight}\PYG{p}{:} \PYG{l+m+mi}{700}\PYG{p}{;}
\PYG{p}{\PYGZcb{}}


\PYG{c}{/*}
\PYG{c}{* Footer}
\PYG{c}{*/}
\PYG{p}{.}\PYG{n+nc}{mastfoot} \PYG{p}{\PYGZob{}}
  \PYG{k}{color}\PYG{p}{:} \PYG{n+nb}{rgba}\PYG{p}{(}\PYG{l+m+mi}{255}\PYG{p}{,} \PYG{l+m+mi}{255}\PYG{p}{,} \PYG{l+m+mi}{255}\PYG{p}{,} \PYG{l+m+mf}{.5}\PYG{p}{)}\PYG{p}{;}
\PYG{p}{\PYGZcb{}}
\end{sphinxVerbatim}

Debería verse algo así:

\begin{figure}[htbp]
\centering

\noindent\sphinxincludegraphics{{01-cover}.png}
\end{figure}

Una vez publicado:

\begin{figure}[htbp]
\centering

\noindent\sphinxincludegraphics{{02-cover-preview}.png}
\end{figure}

Actividades propuestas:
\begin{enumerate}
\item {} 
Intenta cambiar el contenido para presentar un proyecto o producto que te interese.

\item {} 
Intenta cambiarlo para que sea fondo claro con contenido oscuro.
\begin{itemize}
\item {} 
Agregando estilo al final de style.css

\item {} 
Cambiando el estilo existente en style.css

\end{itemize}

\end{enumerate}


\section{Header y Footer}
\label{\detokenize{reusando-html-de-otros:header-y-footer}}
Otro ejemplo muestra como tener una barra de navegación en la parte superior y
un pie de pagina en la parte inferior con el contenido en el centro, copia y
pega el siguiente HTML en un proyecto thimble nuevo:

\fvset{hllines={, ,}}%
\begin{sphinxVerbatim}[commandchars=\\\{\}]
\PYG{c+cp}{\PYGZlt{}!doctype html\PYGZgt{}}
\PYG{p}{\PYGZlt{}}\PYG{n+nt}{html} \PYG{n+na}{lang}\PYG{o}{=}\PYG{l+s}{\PYGZdq{}en\PYGZdq{}}\PYG{p}{\PYGZgt{}}
  \PYG{p}{\PYGZlt{}}\PYG{n+nt}{head}\PYG{p}{\PYGZgt{}}
        \PYG{p}{\PYGZlt{}}\PYG{n+nt}{meta} \PYG{n+na}{charset}\PYG{o}{=}\PYG{l+s}{\PYGZdq{}utf\PYGZhy{}8\PYGZdq{}}\PYG{p}{\PYGZgt{}}
        \PYG{p}{\PYGZlt{}}\PYG{n+nt}{meta} \PYG{n+na}{name}\PYG{o}{=}\PYG{l+s}{\PYGZdq{}viewport\PYGZdq{}} \PYG{n+na}{content}\PYG{o}{=}\PYG{l+s}{\PYGZdq{}width=device\PYGZhy{}width, initial\PYGZhy{}scale=1, shrink\PYGZhy{}to\PYGZhy{}fit=no\PYGZdq{}}\PYG{p}{\PYGZgt{}}

        \PYG{p}{\PYGZlt{}}\PYG{n+nt}{title}\PYG{p}{\PYGZgt{}}Título de Pagina\PYG{p}{\PYGZlt{}}\PYG{p}{/}\PYG{n+nt}{title}\PYG{p}{\PYGZgt{}}

        \PYG{p}{\PYGZlt{}}\PYG{n+nt}{link} \PYG{n+na}{rel}\PYG{o}{=}\PYG{l+s}{\PYGZdq{}stylesheet\PYGZdq{}} \PYG{n+na}{href}\PYG{o}{=}\PYG{l+s}{\PYGZdq{}https://stackpath.bootstrapcdn.com/bootstrap/4.1.0/css/bootstrap.min.css\PYGZdq{}}\PYG{p}{\PYGZgt{}}

        \PYG{p}{\PYGZlt{}}\PYG{n+nt}{link} \PYG{n+na}{href}\PYG{o}{=}\PYG{l+s}{\PYGZdq{}style.css\PYGZdq{}} \PYG{n+na}{rel}\PYG{o}{=}\PYG{l+s}{\PYGZdq{}stylesheet\PYGZdq{}}\PYG{p}{\PYGZgt{}}
  \PYG{p}{\PYGZlt{}}\PYG{p}{/}\PYG{n+nt}{head}\PYG{p}{\PYGZgt{}}

  \PYG{p}{\PYGZlt{}}\PYG{n+nt}{body}\PYG{p}{\PYGZgt{}}

        \PYG{p}{\PYGZlt{}}\PYG{n+nt}{header}\PYG{p}{\PYGZgt{}}
          \PYG{p}{\PYGZlt{}}\PYG{n+nt}{nav} \PYG{n+na}{class}\PYG{o}{=}\PYG{l+s}{\PYGZdq{}navbar navbar\PYGZhy{}expand\PYGZhy{}md navbar\PYGZhy{}dark fixed\PYGZhy{}top bg\PYGZhy{}dark\PYGZdq{}}\PYG{p}{\PYGZgt{}}
                \PYG{p}{\PYGZlt{}}\PYG{n+nt}{a} \PYG{n+na}{class}\PYG{o}{=}\PYG{l+s}{\PYGZdq{}navbar\PYGZhy{}brand\PYGZdq{}} \PYG{n+na}{href}\PYG{o}{=}\PYG{l+s}{\PYGZdq{}\PYGZsh{}\PYGZdq{}}\PYG{p}{\PYGZgt{}}Nombre\PYG{p}{\PYGZlt{}}\PYG{p}{/}\PYG{n+nt}{a}\PYG{p}{\PYGZgt{}}
          \PYG{p}{\PYGZlt{}}\PYG{p}{/}\PYG{n+nt}{nav}\PYG{p}{\PYGZgt{}}
        \PYG{p}{\PYGZlt{}}\PYG{p}{/}\PYG{n+nt}{header}\PYG{p}{\PYGZgt{}}

        \PYG{p}{\PYGZlt{}}\PYG{n+nt}{main} \PYG{n+na}{role}\PYG{o}{=}\PYG{l+s}{\PYGZdq{}main\PYGZdq{}} \PYG{n+na}{class}\PYG{o}{=}\PYG{l+s}{\PYGZdq{}container\PYGZdq{}}\PYG{p}{\PYGZgt{}}
          \PYG{p}{\PYGZlt{}}\PYG{n+nt}{h1} \PYG{n+na}{class}\PYG{o}{=}\PYG{l+s}{\PYGZdq{}mt\PYGZhy{}5\PYGZdq{}}\PYG{p}{\PYGZgt{}}Título\PYG{p}{\PYGZlt{}}\PYG{p}{/}\PYG{n+nt}{h1}\PYG{p}{\PYGZgt{}}
          \PYG{p}{\PYGZlt{}}\PYG{n+nt}{p} \PYG{n+na}{class}\PYG{o}{=}\PYG{l+s}{\PYGZdq{}lead\PYGZdq{}}\PYG{p}{\PYGZgt{}}Descripción.\PYG{p}{\PYGZlt{}}\PYG{p}{/}\PYG{n+nt}{p}\PYG{p}{\PYGZgt{}}
        \PYG{p}{\PYGZlt{}}\PYG{p}{/}\PYG{n+nt}{main}\PYG{p}{\PYGZgt{}}

        \PYG{p}{\PYGZlt{}}\PYG{n+nt}{footer} \PYG{n+na}{class}\PYG{o}{=}\PYG{l+s}{\PYGZdq{}footer\PYGZdq{}}\PYG{p}{\PYGZgt{}}
          \PYG{p}{\PYGZlt{}}\PYG{n+nt}{div} \PYG{n+na}{class}\PYG{o}{=}\PYG{l+s}{\PYGZdq{}container\PYGZdq{}}\PYG{p}{\PYGZgt{}}
                \PYG{p}{\PYGZlt{}}\PYG{n+nt}{span} \PYG{n+na}{class}\PYG{o}{=}\PYG{l+s}{\PYGZdq{}text\PYGZhy{}muted\PYGZdq{}}\PYG{p}{\PYGZgt{}}Esta sección se suele llamar \PYGZdq{}footer\PYGZdq{} (pie de pagina).\PYG{p}{\PYGZlt{}}\PYG{p}{/}\PYG{n+nt}{span}\PYG{p}{\PYGZgt{}}
          \PYG{p}{\PYGZlt{}}\PYG{p}{/}\PYG{n+nt}{div}\PYG{p}{\PYGZgt{}}
        \PYG{p}{\PYGZlt{}}\PYG{p}{/}\PYG{n+nt}{footer}\PYG{p}{\PYGZgt{}}

  \PYG{p}{\PYGZlt{}}\PYG{p}{/}\PYG{n+nt}{body}\PYG{p}{\PYGZgt{}}
\PYG{p}{\PYGZlt{}}\PYG{p}{/}\PYG{n+nt}{html}\PYG{p}{\PYGZgt{}}
\end{sphinxVerbatim}

El contenido del archivo style.css del proyecto:

\fvset{hllines={, ,}}%
\begin{sphinxVerbatim}[commandchars=\\\{\}]
\PYG{c}{/* Sticky footer styles}
\PYG{c}{\PYGZhy{}\PYGZhy{}\PYGZhy{}\PYGZhy{}\PYGZhy{}\PYGZhy{}\PYGZhy{}\PYGZhy{}\PYGZhy{}\PYGZhy{}\PYGZhy{}\PYGZhy{}\PYGZhy{}\PYGZhy{}\PYGZhy{}\PYGZhy{}\PYGZhy{}\PYGZhy{}\PYGZhy{}\PYGZhy{}\PYGZhy{}\PYGZhy{}\PYGZhy{}\PYGZhy{}\PYGZhy{}\PYGZhy{}\PYGZhy{}\PYGZhy{}\PYGZhy{}\PYGZhy{}\PYGZhy{}\PYGZhy{}\PYGZhy{}\PYGZhy{}\PYGZhy{}\PYGZhy{}\PYGZhy{}\PYGZhy{}\PYGZhy{}\PYGZhy{}\PYGZhy{}\PYGZhy{}\PYGZhy{}\PYGZhy{}\PYGZhy{}\PYGZhy{}\PYGZhy{}\PYGZhy{}\PYGZhy{}\PYGZhy{} */}
\PYG{n+nt}{html} \PYG{p}{\PYGZob{}}
  \PYG{k}{position}\PYG{p}{:} \PYG{k+kc}{relative}\PYG{p}{;}
  \PYG{k}{min\PYGZhy{}height}\PYG{p}{:} \PYG{l+m+mi}{100}\PYG{k+kt}{\PYGZpc{}}\PYG{p}{;}
\PYG{p}{\PYGZcb{}}
\PYG{n+nt}{body} \PYG{p}{\PYGZob{}}
  \PYG{c}{/* Margin bottom by footer height */}
  \PYG{k}{margin\PYGZhy{}bottom}\PYG{p}{:} \PYG{l+m+mi}{60}\PYG{k+kt}{px}\PYG{p}{;}
\PYG{p}{\PYGZcb{}}
\PYG{p}{.}\PYG{n+nc}{footer} \PYG{p}{\PYGZob{}}
  \PYG{k}{position}\PYG{p}{:} \PYG{k+kc}{absolute}\PYG{p}{;}
  \PYG{k}{bottom}\PYG{p}{:} \PYG{l+m+mi}{0}\PYG{p}{;}
  \PYG{k}{width}\PYG{p}{:} \PYG{l+m+mi}{100}\PYG{k+kt}{\PYGZpc{}}\PYG{p}{;}
  \PYG{c}{/* Set the fixed height of the footer here */}
  \PYG{k}{height}\PYG{p}{:} \PYG{l+m+mi}{60}\PYG{k+kt}{px}\PYG{p}{;}
  \PYG{k}{line\PYGZhy{}height}\PYG{p}{:} \PYG{l+m+mi}{60}\PYG{k+kt}{px}\PYG{p}{;} \PYG{c}{/* Vertically center the text there */}
  \PYG{k}{background\PYGZhy{}color}\PYG{p}{:} \PYG{l+m+mh}{\PYGZsh{}f5f5f5}\PYG{p}{;}
\PYG{p}{\PYGZcb{}}


\PYG{c}{/* Custom page CSS}
\PYG{c}{\PYGZhy{}\PYGZhy{}\PYGZhy{}\PYGZhy{}\PYGZhy{}\PYGZhy{}\PYGZhy{}\PYGZhy{}\PYGZhy{}\PYGZhy{}\PYGZhy{}\PYGZhy{}\PYGZhy{}\PYGZhy{}\PYGZhy{}\PYGZhy{}\PYGZhy{}\PYGZhy{}\PYGZhy{}\PYGZhy{}\PYGZhy{}\PYGZhy{}\PYGZhy{}\PYGZhy{}\PYGZhy{}\PYGZhy{}\PYGZhy{}\PYGZhy{}\PYGZhy{}\PYGZhy{}\PYGZhy{}\PYGZhy{}\PYGZhy{}\PYGZhy{}\PYGZhy{}\PYGZhy{}\PYGZhy{}\PYGZhy{}\PYGZhy{}\PYGZhy{}\PYGZhy{}\PYGZhy{}\PYGZhy{}\PYGZhy{}\PYGZhy{}\PYGZhy{}\PYGZhy{}\PYGZhy{}\PYGZhy{}\PYGZhy{} */}
\PYG{c}{/* Not required for template or sticky footer method. */}

\PYG{n+nt}{body} \PYG{o}{\PYGZgt{}} \PYG{p}{.}\PYG{n+nc}{container} \PYG{p}{\PYGZob{}}
  \PYG{k}{padding}\PYG{p}{:} \PYG{l+m+mi}{60}\PYG{k+kt}{px} \PYG{l+m+mi}{15}\PYG{k+kt}{px} \PYG{l+m+mi}{0}\PYG{p}{;}
\PYG{p}{\PYGZcb{}}

\PYG{p}{.}\PYG{n+nc}{footer} \PYG{o}{\PYGZgt{}} \PYG{p}{.}\PYG{n+nc}{container} \PYG{p}{\PYGZob{}}
  \PYG{k}{padding\PYGZhy{}right}\PYG{p}{:} \PYG{l+m+mi}{15}\PYG{k+kt}{px}\PYG{p}{;}
  \PYG{k}{padding\PYGZhy{}left}\PYG{p}{:} \PYG{l+m+mi}{15}\PYG{k+kt}{px}\PYG{p}{;}
\PYG{p}{\PYGZcb{}}
\end{sphinxVerbatim}

Debería verse algo así:

\begin{figure}[htbp]
\centering

\noindent\sphinxincludegraphics{{03-nav-footer}.png}
\end{figure}

Una vez publicado:

\begin{figure}[htbp]
\centering

\noindent\sphinxincludegraphics{{04-nav-footer-preview}.png}
\end{figure}

Actividades propuestas:
\begin{enumerate}
\item {} 
Intenta cambiar el contenido para presentar un proyecto o producto que te interese.

\item {} 
Intenta cambiarlo para que sea fondo claro con contenido oscuro.
\begin{itemize}
\item {} 
Agregando estilo al final de style.css

\item {} 
Cambiando el estilo existente en style.css

\end{itemize}

\end{enumerate}


\section{Pagina Principal}
\label{\detokenize{reusando-html-de-otros:pagina-principal}}
Esta pagina es bastante mas larga así que vamos a probar una forma nueva.

Podes ver el resultado visitando \sphinxurl{https://marianoguerra.github.io/creemos-en-la-web/paginas/landing/}

En la pagina principal hace click derecho en cada imagen y selecciona la opción
"Descargar imagen como..." o similar.

Descarga todas las imágenes.

Hace click en la pagina y presiona las teclas Ctrl y "u" a la vez, esto debería
abrirte una ventana nueva con el HTML la pagina. Otra forma de hacerlo en algunos
navegadores es haciendo click derecho con el mouse sobre la pagina y seleccionando
la opción "Ver Código" o similar.

Copia el HTML en un proyecto nuevo de thimble.

Visita la dirección: \sphinxurl{https://marianoguerra.github.io/creemos-en-la-web/paginas/landing/style.css}

Copia el CSS en el archivo style.css del proyecto.

Agrega las imágenes que descargaste a un nuevo directorio llamado \sphinxtitleref{img},
un video que muestra como:



Actividades propuestas:
\begin{enumerate}
\item {} 
Intenta cambiar el contenido para presentar un proyecto o producto que te interese.

\item {} 
Intenta cambiarlo las imágenes
\begin{itemize}
\item {} 
Manteniendo los nombres de las imágenes existentes (subiendo nuevas imágenes con nombres existentes)

\item {} 
Cambiando el nombre de las imágenes en el HTML (subiendo nuevas imágenes con nombres nuevos)

\end{itemize}

\end{enumerate}


\chapter{Colores}
\label{\detokenize{colores::doc}}\label{\detokenize{colores:colores}}
Estamos creando una pagina y vemos o nos imaginamos un color que queremos usar,
como se lo comunicamos a la computadora?

Las computadoras son buenas manipulando números, los colores, al menos como
nosotros los usamos al comunicarnos, no tienen mucho de números.

En la búsqueda por encontrar un compromiso que funcionara tanto para humanos
como para computadoras surgieron diferentes "representaciones", formas de
describir unívocamente a que color nos referimos.

En esta sección vamos a explorar estas distintas representaciones.

Una herramienta online para elegir colores es \sphinxhref{https://mdn.mozillademos.org/en-US/docs/Web/CSS/CSS\_Colors/Color\_picker\_tool\$samples/ColorPicker\_Tool?revision=1310905}{Mozilla Color Picker} podes usar esta o buscar alguna con la que ya hayas trabajado, intenta que permita ver las representaciones que vamos a explorar en esta sección.


\section{Usando colores}
\label{\detokenize{colores:usando-colores}}
Los siguientes son algunos de los atributos CSS que requieren colores, todos
soportan todas las representaciones que vamos a explorar a continuación.
\begin{itemize}
\item {} 
\sphinxhref{https://developer.mozilla.org/es/docs/Web/CSS/color}{color}

\item {} 
\sphinxhref{https://developer.mozilla.org/es/docs/Web/CSS/background-color}{background-color}

\item {} 
\sphinxhref{https://developer.mozilla.org/es/docs/Web/CSS/border-color}{border-color}

\item {} 
\sphinxhref{https://developer.mozilla.org/es/docs/Web/CSS/outline-color}{outline-color}

\item {} 
\sphinxhref{https://developer.mozilla.org/es/docs/Web/CSS/text-decoration-color}{text-decoration-color}

\item {} 
\sphinxhref{https://developer.mozilla.org/es/docs/Web/CSS/text-emphasis-color}{text-emphasis-color}

\item {} 
\sphinxhref{https://developer.mozilla.org/es/docs/Web/CSS/text-shadow}{text-shadow}

\end{itemize}

Un ejemplo que algunos de los atributos listados:

\fvset{hllines={, ,}}%
\begin{sphinxVerbatim}[commandchars=\\\{\}]
\PYG{p}{\PYGZlt{}}\PYG{n+nt}{span} \PYG{n+na}{style}\PYG{o}{=}\PYG{l+s}{\PYGZdq{}border: 1px solid; text\PYGZhy{}decoration: underline; color: red; background\PYGZhy{}color: lightgrey; border\PYGZhy{}color: blue; text\PYGZhy{}decoration\PYGZhy{}color: green; text\PYGZhy{}shadow: yellow 0.6em 0.6em; padding: 1em;\PYGZdq{}}\PYG{p}{\PYGZgt{}}Hello Color\PYG{p}{\PYGZlt{}}\PYG{p}{/}\PYG{n+nt}{span}\PYG{p}{\PYGZgt{}}
\end{sphinxVerbatim}




\section{Colores con nombres}
\label{\detokenize{colores:colores-con-nombres}}
La mas fácil y la que hemos usado hasta ahora es simplemente dar el nombre del
color en ingles cuando necesitamos referirnos a el.

Esta representación es simple para los humanos pero tiene un par de limitaciones:
\begin{itemize}
\item {} 
Hay que mantener una lista de nombres a colores

\item {} 
No todos los colores tienen nombre

\item {} 
Como me acuerdo de tantos nombres?

\item {} 
Tampoco queremos una lista de colores eterna

\item {} 
Cuando busco un color, como lo busco rápido en la tabla?

\item {} 
Cuando digo rojo, que rojo es?

\end{itemize}

A continuación la lista de colores




\section{RGB: Combinando Rojo, Verde y Azul}
\label{\detokenize{colores:rgb-combinando-rojo-verde-y-azul}}
\begin{figure}[htbp]
\centering
\capstart

\noindent\sphinxincludegraphics{{RGB_illumination}.jpg}
\caption{Fuente: \sphinxurl{https://en.wikipedia.org/wiki/File:RGB\_illumination.jpg}}\label{\detokenize{colores:id1}}\end{figure}

Otra forma de especificar colores es describir una mezcla de colores "básicos",
en este caso rojo (Red), verde (Green) y azul (Blue).


\subsection{Como lo describimos?}
\label{\detokenize{colores:como-lo-describimos}}
Necesitamos indicar la cantidad de cada color en la combinación, lo podemos
hacer de dos formas:
\begin{itemize}
\item {} 
Con números
\begin{itemize}
\item {} 
0: nada de color

\item {} 
255: máximo de color

\end{itemize}

\item {} 
Como porcentaje
\begin{itemize}
\item {} 
0\%: nada de color

\item {} 
100\%: máximo de color

\end{itemize}

\end{itemize}


\subsection{Como lo escribimos?}
\label{\detokenize{colores:como-lo-escribimos}}
La forma mas fácil de escribirlo si sabemos los valores individuales y dado
que hay múltiples formas de indicarlo es:
\begin{itemize}
\item {} 
Indicar de que forma vamos a describir el color, en nuestro caso rgb

\item {} 
Indicar los 3 valores

\end{itemize}

Veamos algunos ejemplos




\subsection{Alternativa: Hexadecimal}
\label{\detokenize{colores:alternativa-hexadecimal}}
Este formato suele estar disponible en herramientas que usan o manipulan
colores, es compacta pero difícil de interpretar a menos que entiendas la
numeración hexadecimal fluidamente.

La mencionamos porque te la vas a encontrar en muchos lugares y hay que saber
que es un formato de color valido, que lo podes usar en lugar de cualquiera de
los otros y que hay muchas herramientas que la usan.

Nota muy resumida para los curiosos:

En hexadecimal en lugar de contar del 0 al 9 y al quedarnos sin dígitos ponemos
un 1 adelante y empezamos de nuevo como lo hacemos en decimal, se cuenta del 0
a la F, porque tiene como base 16 y no 10 como el decimal, contando en
hexadecimal seria algo como: 0, 1, 2, 3, 4, 5, 6, 7, 8, 9, A, B, C, D, E, F,
10, 11 ... 1F, 20 ... 2F, 30 ... F0 ... FF, 100 etc.

Lo que cambia es que el 255 de antes se convierte en FF y ponemos los 3 valores
todos juntos luego del símbolo \#, que indica que estamos usando este formato,
el formato debe tener o 3 o 6 caracteres, de ser necesario llenamos los valores
restantes con 0.

Algunos ejemplos de la sección anterior convertidos a formato hexadecimal:




\section{HSL: Tono, Saturación y "Liviandad/Brillo"}
\label{\detokenize{colores:hsl-tono-saturacion-y-liviandad-brillo}}
En esta representación tenemos 3 valores, llamados H (Hue), S (Saturation), L (Lightness).
\begin{description}
\item[{H}] \leavevmode
Numero de 0 a 360 (angulo en un circulo de colores), indica el tono base que queremos

\item[{S}] \leavevmode
Porcentaje de saturación del color elegido, 0\% va a ser un gris, 100\% va a ser el color puro.

\item[{L}] \leavevmode
Porcentaje de liviandad/brillo del color elegido, 0\% va a ser negro, 100\% va a
ser blanco, 50\% (y 100\% de saturación) va a ser el color puro.

\end{description}

Algunos colores "puros":



Cambiando la saturación:



Cambiando el "brillo":




\section{Alpha: Transparencia/Opacidad}
\label{\detokenize{colores:alpha-transparencia-opacidad}}
Un ultimo componente del color común a todos los que vimos es la
transparencia/opacidad, es decir, cuanto del contenido que se encuentra
"detrás" del color que estamos mostrando es visible.

La transparencia/opacidad va de 0\% (ausencia absoluta de color, como un vidrio) a 100\% (color solido, nada se "transluce")

Cuando indicamos la transparencia en rgb y hsl puede ser:
\begin{itemize}
\item {} 
Numero entre 0 y 1

\item {} 
Porcentaje de 0\% a 100\%

\end{itemize}

Ejemplos:

\fvset{hllines={, ,}}%
\begin{sphinxVerbatim}[commandchars=\\\{\}]
\PYG{c}{/* transparente */}
\PYG{n+nt}{rgba}\PYG{o}{(}\PYG{n+nt}{255}\PYG{o}{,} \PYG{n+nt}{0}\PYG{o}{,} \PYG{n+nt}{153}\PYG{o}{,} \PYG{n+nt}{0}\PYG{o}{)}
\PYG{n+nt}{rgba}\PYG{o}{(}\PYG{n+nt}{255}\PYG{o}{,} \PYG{n+nt}{0}\PYG{o}{,} \PYG{n+nt}{153}\PYG{o}{,} \PYG{n+nt}{0}\PYG{o}{\PYGZpc{}}\PYG{o}{)}

\PYG{c}{/* semi translucido */}
\PYG{n+nt}{rgba}\PYG{o}{(}\PYG{n+nt}{255}\PYG{o}{,} \PYG{n+nt}{0}\PYG{o}{,} \PYG{n+nt}{153}\PYG{o}{,} \PYG{n+nt}{0}\PYG{p}{.}\PYG{n+nc}{5}\PYG{o}{)}
\PYG{n+nt}{rgba}\PYG{o}{(}\PYG{n+nt}{255}\PYG{o}{,} \PYG{n+nt}{0}\PYG{o}{,} \PYG{n+nt}{153}\PYG{o}{,} \PYG{n+nt}{50}\PYG{o}{\PYGZpc{}}\PYG{o}{)}

\PYG{c}{/* opaco */}
\PYG{n+nt}{rgba}\PYG{o}{(}\PYG{n+nt}{255}\PYG{o}{,} \PYG{n+nt}{0}\PYG{o}{,} \PYG{n+nt}{153}\PYG{o}{,} \PYG{n+nt}{1}\PYG{o}{)}
\PYG{n+nt}{rgba}\PYG{o}{(}\PYG{n+nt}{255}\PYG{o}{,} \PYG{n+nt}{0}\PYG{o}{,} \PYG{n+nt}{153}\PYG{o}{,} \PYG{n+nt}{100}\PYG{o}{\PYGZpc{}}\PYG{o}{)}
\end{sphinxVerbatim}

\fvset{hllines={, ,}}%
\begin{sphinxVerbatim}[commandchars=\\\{\}]
\PYG{c}{/* transparente */}
\PYG{n+nt}{hsla}\PYG{o}{(}\PYG{n+nt}{240}\PYG{o}{,} \PYG{n+nt}{100}\PYG{o}{\PYGZpc{}}\PYG{o}{,} \PYG{n+nt}{50}\PYG{o}{\PYGZpc{}}\PYG{o}{,} \PYG{n+nt}{0}\PYG{o}{)}
\PYG{n+nt}{hsla}\PYG{o}{(}\PYG{n+nt}{240}\PYG{o}{,} \PYG{n+nt}{100}\PYG{o}{\PYGZpc{}}\PYG{o}{,} \PYG{n+nt}{50}\PYG{o}{\PYGZpc{}}\PYG{o}{,} \PYG{n+nt}{0}\PYG{o}{\PYGZpc{}}\PYG{o}{)}

\PYG{c}{/* semi translucido */}
\PYG{n+nt}{hsla}\PYG{o}{(}\PYG{n+nt}{240}\PYG{o}{,} \PYG{n+nt}{100}\PYG{o}{\PYGZpc{}}\PYG{o}{,} \PYG{n+nt}{50}\PYG{o}{\PYGZpc{}}\PYG{o}{,} \PYG{n+nt}{0}\PYG{p}{.}\PYG{n+nc}{5}\PYG{o}{)}
\PYG{n+nt}{hsla}\PYG{o}{(}\PYG{n+nt}{240}\PYG{o}{,} \PYG{n+nt}{100}\PYG{o}{\PYGZpc{}}\PYG{o}{,} \PYG{n+nt}{50}\PYG{o}{\PYGZpc{}}\PYG{o}{,} \PYG{n+nt}{50}\PYG{o}{\PYGZpc{}}\PYG{o}{)}

\PYG{c}{/* opaco */}
\PYG{n+nt}{hsla}\PYG{o}{(}\PYG{n+nt}{240}\PYG{o}{,} \PYG{n+nt}{100}\PYG{o}{\PYGZpc{}}\PYG{o}{,} \PYG{n+nt}{50}\PYG{o}{\PYGZpc{}}\PYG{o}{,} \PYG{n+nt}{1}\PYG{o}{)}
\PYG{n+nt}{hsla}\PYG{o}{(}\PYG{n+nt}{240}\PYG{o}{,} \PYG{n+nt}{100}\PYG{o}{\PYGZpc{}}\PYG{o}{,} \PYG{n+nt}{50}\PYG{o}{\PYGZpc{}}\PYG{o}{,} \PYG{n+nt}{100}\PYG{o}{\PYGZpc{}}\PYG{o}{)}
\end{sphinxVerbatim}

En el formato hexadecimal simplemente agregamos dos dígitos mas al final entre
00 (0\%) y FF (100\%) indicando el nivel de transparencia

\fvset{hllines={, ,}}%
\begin{sphinxVerbatim}[commandchars=\\\{\}]
\PYG{c}{/* transparente */}
\PYG{p}{\PYGZsh{}}\PYG{n+nn}{FFFFFF00}

\PYG{c}{/* semi translucido */}
\PYG{p}{\PYGZsh{}}\PYG{n+nn}{FFFFFF80}

\PYG{c}{/* opaco */}
\PYG{p}{\PYGZsh{}}\PYG{n+nn}{FFFFFFFF}
\end{sphinxVerbatim}

Y así llegamos al final, no hace falta que intentes aprender, entender o
memorizar esto, solo saber que hay distintas formas de especificar colores, mas
o menos cuales son, para el resto esta tu motor de búsqueda amigo.


\chapter{Haciendo Lugar}
\label{\detokenize{haciendo-lugar::doc}}\label{\detokenize{haciendo-lugar:haciendo-lugar}}
En la sección anterior vimos como indicar que queremos mas espacio entre el
borde y otros elementos (margin/margen), en el borde (border) o entre el borde
y el contenido interno (padding/relleno).

Cuando yo era joven y el pasto era mas verde, las paginas web se hacían
asumiendo una o dos resoluciones de pantalla, ya que los navegadores solo
funcionaban en PCs y todas las PCs tenían una resolución casi estándar (800
pixeles de ancho por 600 de alto).

En esas épocas mas simples la unidad que se usaba para indicar
espacio/distancia eran los pixeles, una unidad absoluta que se refiere a cada
puntito de la pantalla que puede mostrar un color.

Como todas las pantallas tenían cantidades similares de pixeles horizontales y
verticales, la cosa funcionaba bastante bien para todos.

No hace falta que les cuente que hoy la web se accede de una cantidad
impresionante de dispositivos y resoluciones.

Para poder escribir estilo que se adapte a la resolución de cada dispositivo
usamos otras unidades, llamadas relativas, ya que son relativas a algo presente
en la pagina, las mas usadas son relativas al ancho del elemento que contiene
nuestro tag o al ancho de una letra en el texto.


\section{Pixeles}
\label{\detokenize{haciendo-lugar:pixeles}}
Esta es la unidad mas común, un pixel es un punto en la pantalla, el cual puede
mostrar un color, la cantidad horizontal y vertical de pixeles en una pantalla
se llama resolución, por ejemplo, una pantalla de una laptop puede tener 1280
pixeles de ancho por 800 de alto, lo que se escribe abreviado 1280x800.

Cuando especificamos el estilo de un tag y le decimos que su borde, margen o
relleno es de 15px (15 pixeles), le estamos diciendo que queremos 15 puntos de
espacio, pero no sabemos cuantos pixeles tiene la pantalla en total, asi que
eso puede ser bastante para la pantalla de un celular básico que puede tener
una resolución de 320x240 o muy poco para la pantalla de una computadora de
escritorio avanzada que pueden tener 3840x2160 (mas de 10 veces mas resolución!).

Idealmente vamos a usar esta unidad muy poco, yo lo uso solo para especificar
el ancho de los bordes y muchas veces no debería :)


\section{em}
\label{\detokenize{haciendo-lugar:em}}
Esta es la unidad mas usada y la mas recomendada, su nombre según tengo
entendido viene de lejos, 1em es el alto en pixeles de la letra M mayúscula, si
nombramos la letra M "eme" en ingles es "em".

el alto de que M mayúscula? de la M si estuviera en el tag en el que estamos
actualmente, osea que esta unidad es relativa al tamaño de texto del tag en el
que nos encontramos, el cual lo puede haber establecido cualquiera de los tags
padres, o ninguno, siendo asi el estándar de 16px.


\section{rem}
\label{\detokenize{haciendo-lugar:rem}}
Pero que pasa si no sabemos que estilos se aplicaron a tags padres y queremos
estar mas seguros del tamaño que vamos a obtener? para eso existe la unidad
rem, que es el alto de la letra M según el tamaño de texto definido en la base
del documento (el tag \textless{}body\textgreater{}), de no estar definido va a ser también 16px.

Su nombre viene de "root em" (em de la raíz).

Esto nos permite saber que no importa que tamaños de fuente se hayan redefinido
hasta nuestro tag, su tamaño va a ser siempre fijo relativo al tamaño base del
texto del documento.


\section{Porcentajes}
\label{\detokenize{haciendo-lugar:porcentajes}}
Otra unidad útil, pero usada normalmente cuando estamos definiendo tamaños de
la estructura de nuestro documento como ancho de columnas es el porcentaje \%.

El porcentaje se refiere al ancho del tag que contiene al tag actual, si
decimos que el ancho de nuestro tag es 50\%, este ocupara la mitad del tag
padre, si decimos que el margen horizontal es de 5\%, el padding horizontal es
del 2.5\%, entonces nos queda 75\% para el contenido:

5\% margen izquierdo + 2.5\% padding izquierdo + 2.5\% padding derecho + 5\% margen derecho = 15\%


\section{Ejemplos que no se entiende nada!}
\label{\detokenize{haciendo-lugar:ejemplos-que-no-se-entiende-nada}}
Las unidades de espacio se entienden mas usandolas y a puro prueba y error, la
recomendación es usar em y rem siempre que se pueda, porcentajes cuando estamos
definiendo posicionamiento de cosas en la pagina y pixeles si tenes una buena
razón.

El div azul que contiene a todos los otros divs establece el tamaño de la
fuente a 16 pixeles:

\fvset{hllines={, ,}}%
\begin{sphinxVerbatim}[commandchars=\\\{\}]
\PYG{p}{\PYGZlt{}}\PYG{n+nt}{div} \PYG{n+na}{style}\PYG{o}{=}\PYG{l+s}{\PYGZdq{}font\PYGZhy{}size: 16px; width: 90\PYGZpc{}; padding: 2.5\PYGZpc{}; margin: 2.5\PYGZpc{}; border: 1px solid blue;\PYGZdq{}}\PYG{p}{\PYGZgt{}}
    \PYG{p}{\PYGZlt{}}\PYG{n+nt}{div} \PYG{n+na}{style}\PYG{o}{=}\PYG{l+s}{\PYGZdq{}width: 5em;  background\PYGZhy{}color: red; color: white; margin: 1em 0;\PYGZdq{}}\PYG{p}{\PYGZgt{}}5em\PYG{p}{\PYGZlt{}}\PYG{p}{/}\PYG{n+nt}{div}\PYG{p}{\PYGZgt{}}
    \PYG{p}{\PYGZlt{}}\PYG{n+nt}{div} \PYG{n+na}{style}\PYG{o}{=}\PYG{l+s}{\PYGZdq{}width: 5rem; background\PYGZhy{}color: red; color: white; margin: 1em 0;\PYGZdq{}}\PYG{p}{\PYGZgt{}}5rem\PYG{p}{\PYGZlt{}}\PYG{p}{/}\PYG{n+nt}{div}\PYG{p}{\PYGZgt{}}
    \PYG{p}{\PYGZlt{}}\PYG{n+nt}{div} \PYG{n+na}{style}\PYG{o}{=}\PYG{l+s}{\PYGZdq{}width: 50px; background\PYGZhy{}color: red; color: white; margin: 1em 0;\PYGZdq{}}\PYG{p}{\PYGZgt{}}50px\PYG{p}{\PYGZlt{}}\PYG{p}{/}\PYG{n+nt}{div}\PYG{p}{\PYGZgt{}}
    \PYG{p}{\PYGZlt{}}\PYG{n+nt}{div} \PYG{n+na}{style}\PYG{o}{=}\PYG{l+s}{\PYGZdq{}width: 50\PYGZpc{};  background\PYGZhy{}color: red; color: white;\PYGZdq{}}\PYG{p}{\PYGZgt{}}50\PYGZpc{}\PYG{p}{\PYGZlt{}}\PYG{p}{/}\PYG{n+nt}{div}\PYG{p}{\PYGZgt{}}
\PYG{p}{\PYGZlt{}}\PYG{p}{/}\PYG{n+nt}{div}\PYG{p}{\PYGZgt{}}
\end{sphinxVerbatim}



El div azul que contiene a todos los otros divs establece el tamaño de la
fuente, notar que los divs interiores tienen el mismo estilo que los de arriba:

\fvset{hllines={, ,}}%
\begin{sphinxVerbatim}[commandchars=\\\{\}]
\PYG{p}{\PYGZlt{}}\PYG{n+nt}{div} \PYG{n+na}{style}\PYG{o}{=}\PYG{l+s}{\PYGZdq{}font\PYGZhy{}size: 32px; width: 90\PYGZpc{}; padding: 2.5\PYGZpc{}; margin: 2.5\PYGZpc{}; border: 1px solid blue;\PYGZdq{}}\PYG{p}{\PYGZgt{}}
    \PYG{p}{\PYGZlt{}}\PYG{n+nt}{div} \PYG{n+na}{style}\PYG{o}{=}\PYG{l+s}{\PYGZdq{}width: 5em;  background\PYGZhy{}color: red; color: white; margin: 1em 0;\PYGZdq{}}\PYG{p}{\PYGZgt{}}5em\PYG{p}{\PYGZlt{}}\PYG{p}{/}\PYG{n+nt}{div}\PYG{p}{\PYGZgt{}}
    \PYG{p}{\PYGZlt{}}\PYG{n+nt}{div} \PYG{n+na}{style}\PYG{o}{=}\PYG{l+s}{\PYGZdq{}width: 5rem; background\PYGZhy{}color: red; color: white; margin: 1em 0;\PYGZdq{}}\PYG{p}{\PYGZgt{}}5rem\PYG{p}{\PYGZlt{}}\PYG{p}{/}\PYG{n+nt}{div}\PYG{p}{\PYGZgt{}}
    \PYG{p}{\PYGZlt{}}\PYG{n+nt}{div} \PYG{n+na}{style}\PYG{o}{=}\PYG{l+s}{\PYGZdq{}width: 50px; background\PYGZhy{}color: red; color: white; margin: 1em 0;\PYGZdq{}}\PYG{p}{\PYGZgt{}}50px\PYG{p}{\PYGZlt{}}\PYG{p}{/}\PYG{n+nt}{div}\PYG{p}{\PYGZgt{}}
    \PYG{p}{\PYGZlt{}}\PYG{n+nt}{div} \PYG{n+na}{style}\PYG{o}{=}\PYG{l+s}{\PYGZdq{}width: 50\PYGZpc{};  background\PYGZhy{}color: red; color: white;\PYGZdq{}}\PYG{p}{\PYGZgt{}}50\PYGZpc{}\PYG{p}{\PYGZlt{}}\PYG{p}{/}\PYG{n+nt}{div}\PYG{p}{\PYGZgt{}}
\PYG{p}{\PYGZlt{}}\PYG{p}{/}\PYG{n+nt}{div}\PYG{p}{\PYGZgt{}}
\end{sphinxVerbatim}



El primer div tiene un ancho de 5em, como el div padre establece el tamaño de
la fuente a distintos valores en los dos ejemplos, el ancho resultante es
distinto.

El segundo div tiene un ancho de 5rem, como ambos están en el mismo documento
raíz, tienen el mismo ancho, aun cuando el texto interior cambia, ya que
"hereda" el tamaño del div padre.

El tercero esta en pixeles, así que va a ser igual, el cuarto esta en
porcentaje, y ya que ambos divs padres tienen el mismo ancho, su ancho es
igual.

Probemos algo un poco distinto:

\fvset{hllines={, ,}}%
\begin{sphinxVerbatim}[commandchars=\\\{\}]
\PYG{p}{\PYGZlt{}}\PYG{n+nt}{div} \PYG{n+na}{style}\PYG{o}{=}\PYG{l+s}{\PYGZdq{}font\PYGZhy{}size: 32px; width: 50\PYGZpc{}; padding: 2.5\PYGZpc{}; margin: 2.5\PYGZpc{}; border: 1px solid blue;\PYGZdq{}}\PYG{p}{\PYGZgt{}}
    \PYG{p}{\PYGZlt{}}\PYG{n+nt}{div} \PYG{n+na}{style}\PYG{o}{=}\PYG{l+s}{\PYGZdq{}width: 5em;  font\PYGZhy{}size: 1rem; background\PYGZhy{}color: red; color: white; margin: 1em 0;\PYGZdq{}}\PYG{p}{\PYGZgt{}}5em\PYG{p}{\PYGZlt{}}\PYG{p}{/}\PYG{n+nt}{div}\PYG{p}{\PYGZgt{}}
    \PYG{p}{\PYGZlt{}}\PYG{n+nt}{div} \PYG{n+na}{style}\PYG{o}{=}\PYG{l+s}{\PYGZdq{}width: 5rem; font\PYGZhy{}size: 1rem; background\PYGZhy{}color: red; color: white; margin: 1em 0;\PYGZdq{}}\PYG{p}{\PYGZgt{}}5rem\PYG{p}{\PYGZlt{}}\PYG{p}{/}\PYG{n+nt}{div}\PYG{p}{\PYGZgt{}}
    \PYG{p}{\PYGZlt{}}\PYG{n+nt}{div} \PYG{n+na}{style}\PYG{o}{=}\PYG{l+s}{\PYGZdq{}width: 50px; font\PYGZhy{}size: 1rem; background\PYGZhy{}color: red; color: white; margin: 1em 0;\PYGZdq{}}\PYG{p}{\PYGZgt{}}50px\PYG{p}{\PYGZlt{}}\PYG{p}{/}\PYG{n+nt}{div}\PYG{p}{\PYGZgt{}}
    \PYG{p}{\PYGZlt{}}\PYG{n+nt}{div} \PYG{n+na}{style}\PYG{o}{=}\PYG{l+s}{\PYGZdq{}width: 50\PYGZpc{};  font\PYGZhy{}size: 1rem; background\PYGZhy{}color: red; color: white;\PYGZdq{}}\PYG{p}{\PYGZgt{}}50\PYGZpc{}\PYG{p}{\PYGZlt{}}\PYG{p}{/}\PYG{n+nt}{div}\PYG{p}{\PYGZgt{}}
\PYG{p}{\PYGZlt{}}\PYG{p}{/}\PYG{n+nt}{div}\PYG{p}{\PYGZgt{}}
\end{sphinxVerbatim}



El div padre ahora ocupa el 50\% del ancho de la pagina, por lo que el ancho del
ultimo div hijo debería ser la mitad de los anteriores, para "estandarizar" em
y rem, seteo el tamaño de fuente de los divs hijos a 1rem.

Como no se en que pantalla estas viendo esto no te puedo decir mucho sobre que
ancho van a tener, lo único que se, es que como la fuente de ambos tiene 1rem
de tamaño, el ancho de los dos debería ser el mismo.

Si el tamaño de la fuente del documento resulta ser 10px, entonces el tercer
div tendrá el mismo ancho.

Como siempre, no hace falta que se entienda todo ahora, solo saber las unidades
mas usadas, las recomendadas y mas o menos como se comportan, el resto es
prueba y error.


\chapter{Si son datos, hay tabla}
\label{\detokenize{hay-tabla:si-son-datos-hay-tabla}}\label{\detokenize{hay-tabla::doc}}
Como presentamos muchos datos en una pagina? con tablas!


\section{Partes de una tabla}
\label{\detokenize{hay-tabla:partes-de-una-tabla}}

\subsection{Tabla de datos}
\label{\detokenize{hay-tabla:tabla-de-datos}}
Vamos de menos a mas, la tabla mas simple que podemos tener solo tiene una
parte, el cuerpo de la tabla. Es decir, los datos.

Los datos en una tabla son cero o mas filas, cada fila puede tener cero o mas
columnas:

Osea que las partes de una tabla por ahora son:
\begin{itemize}
\item {} 
Tabla
\begin{itemize}
\item {} 
Cuerpo (1)
\begin{itemize}
\item {} 
Filas (0 o mas)
\begin{itemize}
\item {} 
Columnas (0 o mas)

\end{itemize}

\end{itemize}

\end{itemize}

\end{itemize}

Si alguna vez usaste una planilla de cálculos, las tablas son una forma de
presentar información que presentarías en una planilla de cálculos.

Vamos a ver una tabla de ejemplo:

\fvset{hllines={, ,}}%
\begin{sphinxVerbatim}[commandchars=\\\{\}]
\PYG{p}{\PYGZlt{}}\PYG{n+nt}{table}\PYG{p}{\PYGZgt{}}
  \PYG{p}{\PYGZlt{}}\PYG{n+nt}{tr}\PYG{p}{\PYGZgt{}}
    \PYG{p}{\PYGZlt{}}\PYG{n+nt}{td}\PYG{p}{\PYGZgt{}}Arenita\PYG{p}{\PYGZlt{}}\PYG{p}{/}\PYG{n+nt}{td}\PYG{p}{\PYGZgt{}}
    \PYG{p}{\PYGZlt{}}\PYG{n+nt}{td}\PYG{p}{\PYGZgt{}}Ardilla\PYG{p}{\PYGZlt{}}\PYG{p}{/}\PYG{n+nt}{td}\PYG{p}{\PYGZgt{}}
    \PYG{p}{\PYGZlt{}}\PYG{n+nt}{td}\PYG{p}{\PYGZgt{}}Marron\PYG{p}{\PYGZlt{}}\PYG{p}{/}\PYG{n+nt}{td}\PYG{p}{\PYGZgt{}}
  \PYG{p}{\PYGZlt{}}\PYG{p}{/}\PYG{n+nt}{tr}\PYG{p}{\PYGZgt{}}
  \PYG{p}{\PYGZlt{}}\PYG{n+nt}{tr}\PYG{p}{\PYGZgt{}}
    \PYG{p}{\PYGZlt{}}\PYG{n+nt}{td}\PYG{p}{\PYGZgt{}}Bob\PYG{p}{\PYGZlt{}}\PYG{p}{/}\PYG{n+nt}{td}\PYG{p}{\PYGZgt{}}
    \PYG{p}{\PYGZlt{}}\PYG{n+nt}{td}\PYG{p}{\PYGZgt{}}Esponja\PYG{p}{\PYGZlt{}}\PYG{p}{/}\PYG{n+nt}{td}\PYG{p}{\PYGZgt{}}
    \PYG{p}{\PYGZlt{}}\PYG{n+nt}{td}\PYG{p}{\PYGZgt{}}Amarillo\PYG{p}{\PYGZlt{}}\PYG{p}{/}\PYG{n+nt}{td}\PYG{p}{\PYGZgt{}}
  \PYG{p}{\PYGZlt{}}\PYG{p}{/}\PYG{n+nt}{tr}\PYG{p}{\PYGZgt{}}
  \PYG{p}{\PYGZlt{}}\PYG{n+nt}{tr}\PYG{p}{\PYGZgt{}}
    \PYG{p}{\PYGZlt{}}\PYG{n+nt}{td}\PYG{p}{\PYGZgt{}}Patricio\PYG{p}{\PYGZlt{}}\PYG{p}{/}\PYG{n+nt}{td}\PYG{p}{\PYGZgt{}}
    \PYG{p}{\PYGZlt{}}\PYG{n+nt}{td}\PYG{p}{\PYGZgt{}}Estrella\PYG{p}{\PYGZlt{}}\PYG{p}{/}\PYG{n+nt}{td}\PYG{p}{\PYGZgt{}}
    \PYG{p}{\PYGZlt{}}\PYG{n+nt}{td}\PYG{p}{\PYGZgt{}}Rosa\PYG{p}{\PYGZlt{}}\PYG{p}{/}\PYG{n+nt}{td}\PYG{p}{\PYGZgt{}}
  \PYG{p}{\PYGZlt{}}\PYG{p}{/}\PYG{n+nt}{tr}\PYG{p}{\PYGZgt{}}
\PYG{p}{\PYGZlt{}}\PYG{p}{/}\PYG{n+nt}{table}\PYG{p}{\PYGZgt{}}
\end{sphinxVerbatim}

Que se ve algo así:



Como vemos en el código, el tag \sphinxhref{https://developer.mozilla.org/es/docs/Web/HTML/Elemento/table}{table} indica que vamos a usar una tabla, si
solo vamos a mostrar datos el tag que indica el cuerpo \sphinxhref{https://developer.mozilla.org/es/docs/Web/HTML/Elemento/tbody}{tbody} (table body,
cuerpo de tabla) es opcional, luego tenemos el tag que indica una fila \sphinxhref{https://developer.mozilla.org/es/docs/Web/HTML/Elemento/tr}{tr}
(table row: fila de tabla), dentro del tenemos el tag para una celda de datos
\sphinxhref{https://developer.mozilla.org/es/docs/Web/HTML/Elemento/td}{td} (table cell data: celda de datos de tabla), uno por cada celda.


\subsection{Cabecera}
\label{\detokenize{hay-tabla:cabecera}}
En la tabla anterior, como sabemos que significa cada columna? para eso
necesitamos una cabecera, esto se hace con el tag \sphinxhref{https://developer.mozilla.org/es/docs/Web/HTML/Elemento/thead}{thead} (table head, cabecera
de tabla)
\begin{itemize}
\item {} 
Tabla
\begin{itemize}
\item {} 
Cabecera (0 o 1)
\begin{itemize}
\item {} 
Filas (1 o mas)
\begin{itemize}
\item {} 
Columnas (1 o mas)

\end{itemize}

\end{itemize}

\item {} 
Cuerpo (1)
\begin{itemize}
\item {} 
Filas (0 o mas)
\begin{itemize}
\item {} 
Columnas (0 o mas)

\end{itemize}

\end{itemize}

\end{itemize}

\end{itemize}

\fvset{hllines={, ,}}%
\begin{sphinxVerbatim}[commandchars=\\\{\}]
\PYG{p}{\PYGZlt{}}\PYG{n+nt}{table}\PYG{p}{\PYGZgt{}}
  \PYG{p}{\PYGZlt{}}\PYG{n+nt}{thead}\PYG{p}{\PYGZgt{}}
    \PYG{p}{\PYGZlt{}}\PYG{n+nt}{tr}\PYG{p}{\PYGZgt{}}
      \PYG{p}{\PYGZlt{}}\PYG{n+nt}{th}\PYG{p}{\PYGZgt{}}Nombre\PYG{p}{\PYGZlt{}}\PYG{p}{/}\PYG{n+nt}{th}\PYG{p}{\PYGZgt{}}
      \PYG{p}{\PYGZlt{}}\PYG{n+nt}{th}\PYG{p}{\PYGZgt{}}Tipo\PYG{p}{\PYGZlt{}}\PYG{p}{/}\PYG{n+nt}{th}\PYG{p}{\PYGZgt{}}
      \PYG{p}{\PYGZlt{}}\PYG{n+nt}{th}\PYG{p}{\PYGZgt{}}Color\PYG{p}{\PYGZlt{}}\PYG{p}{/}\PYG{n+nt}{th}\PYG{p}{\PYGZgt{}}
    \PYG{p}{\PYGZlt{}}\PYG{p}{/}\PYG{n+nt}{tr}\PYG{p}{\PYGZgt{}}
  \PYG{p}{\PYGZlt{}}\PYG{p}{/}\PYG{n+nt}{thead}\PYG{p}{\PYGZgt{}}

  \PYG{p}{\PYGZlt{}}\PYG{n+nt}{tbody}\PYG{p}{\PYGZgt{}}
    \PYG{p}{\PYGZlt{}}\PYG{n+nt}{tr}\PYG{p}{\PYGZgt{}}
      \PYG{p}{\PYGZlt{}}\PYG{n+nt}{td}\PYG{p}{\PYGZgt{}}Arenita\PYG{p}{\PYGZlt{}}\PYG{p}{/}\PYG{n+nt}{td}\PYG{p}{\PYGZgt{}}
      \PYG{p}{\PYGZlt{}}\PYG{n+nt}{td}\PYG{p}{\PYGZgt{}}Ardilla\PYG{p}{\PYGZlt{}}\PYG{p}{/}\PYG{n+nt}{td}\PYG{p}{\PYGZgt{}}
      \PYG{p}{\PYGZlt{}}\PYG{n+nt}{td}\PYG{p}{\PYGZgt{}}Marron\PYG{p}{\PYGZlt{}}\PYG{p}{/}\PYG{n+nt}{td}\PYG{p}{\PYGZgt{}}
    \PYG{p}{\PYGZlt{}}\PYG{p}{/}\PYG{n+nt}{tr}\PYG{p}{\PYGZgt{}}
    \PYG{p}{\PYGZlt{}}\PYG{n+nt}{tr}\PYG{p}{\PYGZgt{}}
      \PYG{p}{\PYGZlt{}}\PYG{n+nt}{td}\PYG{p}{\PYGZgt{}}Bob\PYG{p}{\PYGZlt{}}\PYG{p}{/}\PYG{n+nt}{td}\PYG{p}{\PYGZgt{}}
      \PYG{p}{\PYGZlt{}}\PYG{n+nt}{td}\PYG{p}{\PYGZgt{}}Esponja\PYG{p}{\PYGZlt{}}\PYG{p}{/}\PYG{n+nt}{td}\PYG{p}{\PYGZgt{}}
      \PYG{p}{\PYGZlt{}}\PYG{n+nt}{td}\PYG{p}{\PYGZgt{}}Amarillo\PYG{p}{\PYGZlt{}}\PYG{p}{/}\PYG{n+nt}{td}\PYG{p}{\PYGZgt{}}
    \PYG{p}{\PYGZlt{}}\PYG{p}{/}\PYG{n+nt}{tr}\PYG{p}{\PYGZgt{}}
    \PYG{p}{\PYGZlt{}}\PYG{n+nt}{tr}\PYG{p}{\PYGZgt{}}
      \PYG{p}{\PYGZlt{}}\PYG{n+nt}{td}\PYG{p}{\PYGZgt{}}Patricio\PYG{p}{\PYGZlt{}}\PYG{p}{/}\PYG{n+nt}{td}\PYG{p}{\PYGZgt{}}
      \PYG{p}{\PYGZlt{}}\PYG{n+nt}{td}\PYG{p}{\PYGZgt{}}Estrella\PYG{p}{\PYGZlt{}}\PYG{p}{/}\PYG{n+nt}{td}\PYG{p}{\PYGZgt{}}
      \PYG{p}{\PYGZlt{}}\PYG{n+nt}{td}\PYG{p}{\PYGZgt{}}Rosa\PYG{p}{\PYGZlt{}}\PYG{p}{/}\PYG{n+nt}{td}\PYG{p}{\PYGZgt{}}
    \PYG{p}{\PYGZlt{}}\PYG{p}{/}\PYG{n+nt}{tr}\PYG{p}{\PYGZgt{}}
  \PYG{p}{\PYGZlt{}}\PYG{p}{/}\PYG{n+nt}{tbody}\PYG{p}{\PYGZgt{}}
\PYG{p}{\PYGZlt{}}\PYG{p}{/}\PYG{n+nt}{table}\PYG{p}{\PYGZgt{}}
\end{sphinxVerbatim}



Como podemos ver, en este caso tenemos que indicar explicitamente el cuerpo
de la tabla con el tag \sphinxtitleref{tbody} (table body, cuerpo de tabla).

Lo hacemos para poder tambien indicar la cabecera con el tag \sphinxtitleref{thead} (table
head, cabecera de tabla).

El cuerpo ya lo vimos antes, la cabecera es muy similar, solo que en lugar de
usar el tag \sphinxtitleref{td} para las columnas usamos el tag \sphinxhref{https://developer.mozilla.org/es/docs/Web/HTML/Elemento/th}{th} (table cell header, celda
de cabecera de tabla)


\subsection{Pie de Tabla}
\label{\detokenize{hay-tabla:pie-de-tabla}}
Algunas tablas que presentan datos suelen tener al final una fila de
sumarización que presenta valores resumidos para toda la columna.

Obviamente se puede usar para otras cosas, pero normalmente se usa para eso.

Vamos a ver un ejemplo que no aplica mucho a nuestro caso porque no estamos
presentando valores que se puedan sumarizar.
\begin{itemize}
\item {} 
Tabla
\begin{itemize}
\item {} 
Cabecera (0 o 1)
\begin{itemize}
\item {} 
Filas (1 o mas)
\begin{itemize}
\item {} 
Columnas (1 o mas)

\end{itemize}

\end{itemize}

\item {} 
Cuerpo (1)
\begin{itemize}
\item {} 
Filas (0 o mas)
\begin{itemize}
\item {} 
Columnas (0 o mas)

\end{itemize}

\end{itemize}

\item {} 
Pie (0 o 1)
\begin{itemize}
\item {} 
Filas (1 o mas)
\begin{itemize}
\item {} 
Columnas (1 o mas)

\end{itemize}

\end{itemize}

\end{itemize}

\end{itemize}

\fvset{hllines={, ,}}%
\begin{sphinxVerbatim}[commandchars=\\\{\}]
\PYG{p}{\PYGZlt{}}\PYG{n+nt}{table}\PYG{p}{\PYGZgt{}}
  \PYG{p}{\PYGZlt{}}\PYG{n+nt}{thead}\PYG{p}{\PYGZgt{}}
    \PYG{p}{\PYGZlt{}}\PYG{n+nt}{tr}\PYG{p}{\PYGZgt{}}
      \PYG{p}{\PYGZlt{}}\PYG{n+nt}{th}\PYG{p}{\PYGZgt{}}Nombre\PYG{p}{\PYGZlt{}}\PYG{p}{/}\PYG{n+nt}{th}\PYG{p}{\PYGZgt{}}
      \PYG{p}{\PYGZlt{}}\PYG{n+nt}{th}\PYG{p}{\PYGZgt{}}Tipo\PYG{p}{\PYGZlt{}}\PYG{p}{/}\PYG{n+nt}{th}\PYG{p}{\PYGZgt{}}
      \PYG{p}{\PYGZlt{}}\PYG{n+nt}{th}\PYG{p}{\PYGZgt{}}Color\PYG{p}{\PYGZlt{}}\PYG{p}{/}\PYG{n+nt}{th}\PYG{p}{\PYGZgt{}}
    \PYG{p}{\PYGZlt{}}\PYG{p}{/}\PYG{n+nt}{tr}\PYG{p}{\PYGZgt{}}
  \PYG{p}{\PYGZlt{}}\PYG{p}{/}\PYG{n+nt}{thead}\PYG{p}{\PYGZgt{}}

  \PYG{p}{\PYGZlt{}}\PYG{n+nt}{tbody}\PYG{p}{\PYGZgt{}}
    \PYG{p}{\PYGZlt{}}\PYG{n+nt}{tr}\PYG{p}{\PYGZgt{}}
      \PYG{p}{\PYGZlt{}}\PYG{n+nt}{td}\PYG{p}{\PYGZgt{}}Arenita\PYG{p}{\PYGZlt{}}\PYG{p}{/}\PYG{n+nt}{td}\PYG{p}{\PYGZgt{}}
      \PYG{p}{\PYGZlt{}}\PYG{n+nt}{td}\PYG{p}{\PYGZgt{}}Ardilla\PYG{p}{\PYGZlt{}}\PYG{p}{/}\PYG{n+nt}{td}\PYG{p}{\PYGZgt{}}
      \PYG{p}{\PYGZlt{}}\PYG{n+nt}{td}\PYG{p}{\PYGZgt{}}Marron\PYG{p}{\PYGZlt{}}\PYG{p}{/}\PYG{n+nt}{td}\PYG{p}{\PYGZgt{}}
    \PYG{p}{\PYGZlt{}}\PYG{p}{/}\PYG{n+nt}{tr}\PYG{p}{\PYGZgt{}}
    \PYG{p}{\PYGZlt{}}\PYG{n+nt}{tr}\PYG{p}{\PYGZgt{}}
      \PYG{p}{\PYGZlt{}}\PYG{n+nt}{td}\PYG{p}{\PYGZgt{}}Bob\PYG{p}{\PYGZlt{}}\PYG{p}{/}\PYG{n+nt}{td}\PYG{p}{\PYGZgt{}}
      \PYG{p}{\PYGZlt{}}\PYG{n+nt}{td}\PYG{p}{\PYGZgt{}}Esponja\PYG{p}{\PYGZlt{}}\PYG{p}{/}\PYG{n+nt}{td}\PYG{p}{\PYGZgt{}}
      \PYG{p}{\PYGZlt{}}\PYG{n+nt}{td}\PYG{p}{\PYGZgt{}}Amarillo\PYG{p}{\PYGZlt{}}\PYG{p}{/}\PYG{n+nt}{td}\PYG{p}{\PYGZgt{}}
    \PYG{p}{\PYGZlt{}}\PYG{p}{/}\PYG{n+nt}{tr}\PYG{p}{\PYGZgt{}}
    \PYG{p}{\PYGZlt{}}\PYG{n+nt}{tr}\PYG{p}{\PYGZgt{}}
      \PYG{p}{\PYGZlt{}}\PYG{n+nt}{td}\PYG{p}{\PYGZgt{}}Patricio\PYG{p}{\PYGZlt{}}\PYG{p}{/}\PYG{n+nt}{td}\PYG{p}{\PYGZgt{}}
      \PYG{p}{\PYGZlt{}}\PYG{n+nt}{td}\PYG{p}{\PYGZgt{}}Estrella\PYG{p}{\PYGZlt{}}\PYG{p}{/}\PYG{n+nt}{td}\PYG{p}{\PYGZgt{}}
      \PYG{p}{\PYGZlt{}}\PYG{n+nt}{td}\PYG{p}{\PYGZgt{}}Rosa\PYG{p}{\PYGZlt{}}\PYG{p}{/}\PYG{n+nt}{td}\PYG{p}{\PYGZgt{}}
    \PYG{p}{\PYGZlt{}}\PYG{p}{/}\PYG{n+nt}{tr}\PYG{p}{\PYGZgt{}}
  \PYG{p}{\PYGZlt{}}\PYG{p}{/}\PYG{n+nt}{tbody}\PYG{p}{\PYGZgt{}}

  \PYG{p}{\PYGZlt{}}\PYG{n+nt}{tfoot}\PYG{p}{\PYGZgt{}}
    \PYG{p}{\PYGZlt{}}\PYG{n+nt}{tr}\PYG{p}{\PYGZgt{}}
      \PYG{p}{\PYGZlt{}}\PYG{n+nt}{td}\PYG{p}{\PYGZgt{}}Nombre\PYG{p}{\PYGZlt{}}\PYG{p}{/}\PYG{n+nt}{td}\PYG{p}{\PYGZgt{}}
      \PYG{p}{\PYGZlt{}}\PYG{n+nt}{td}\PYG{p}{\PYGZgt{}}Tipo\PYG{p}{\PYGZlt{}}\PYG{p}{/}\PYG{n+nt}{td}\PYG{p}{\PYGZgt{}}
      \PYG{p}{\PYGZlt{}}\PYG{n+nt}{td}\PYG{p}{\PYGZgt{}}Color\PYG{p}{\PYGZlt{}}\PYG{p}{/}\PYG{n+nt}{td}\PYG{p}{\PYGZgt{}}
    \PYG{p}{\PYGZlt{}}\PYG{p}{/}\PYG{n+nt}{tr}\PYG{p}{\PYGZgt{}}
  \PYG{p}{\PYGZlt{}}\PYG{p}{/}\PYG{n+nt}{tfoot}\PYG{p}{\PYGZgt{}}
\PYG{p}{\PYGZlt{}}\PYG{p}{/}\PYG{n+nt}{table}\PYG{p}{\PYGZgt{}}
\end{sphinxVerbatim}



Como podemos ver, el pie de tabla se define con el tag \sphinxhref{https://developer.mozilla.org/es/docs/Web/HTML/Elemento/tfoot}{tfoot} (table footer,
pie de tabla), dentro usamos los mismos tags que en el cuerpo.


\section{Un poco de estilo}
\label{\detokenize{hay-tabla:un-poco-de-estilo}}
Los ejemplos que vimos hasta ahora tienen un aspecto simple pero agradable,
eso es porque estaba usando una clase de bootstrap para darle un aspecto
aceptable, la clase que estaba usando es la clase \sphinxtitleref{table}.

\fvset{hllines={, ,}}%
\begin{sphinxVerbatim}[commandchars=\\\{\}]
\PYG{p}{\PYGZlt{}}\PYG{n+nt}{table} \PYG{n+na}{class}\PYG{o}{=}\PYG{l+s}{\PYGZdq{}table\PYGZdq{}}\PYG{p}{\PYGZgt{}}
\end{sphinxVerbatim}

En esta sección vamos a explorar otras clases que podemos aplicarle a una
tabla.


\subsection{Con bordes}
\label{\detokenize{hay-tabla:con-bordes}}
\fvset{hllines={, ,}}%
\begin{sphinxVerbatim}[commandchars=\\\{\}]
\PYG{p}{\PYGZlt{}}\PYG{n+nt}{table} \PYG{n+na}{class}\PYG{o}{=}\PYG{l+s}{\PYGZdq{}table table\PYGZhy{}bordered\PYGZdq{}}\PYG{p}{\PYGZgt{}}
\end{sphinxVerbatim}




\subsection{Colores invertidos}
\label{\detokenize{hay-tabla:colores-invertidos}}
\fvset{hllines={, ,}}%
\begin{sphinxVerbatim}[commandchars=\\\{\}]
\PYG{p}{\PYGZlt{}}\PYG{n+nt}{table} \PYG{n+na}{class}\PYG{o}{=}\PYG{l+s}{\PYGZdq{}table table\PYGZhy{}dark\PYGZdq{}}\PYG{p}{\PYGZgt{}}
\end{sphinxVerbatim}




\subsection{Cabecera invertida}
\label{\detokenize{hay-tabla:cabecera-invertida}}
\fvset{hllines={, ,}}%
\begin{sphinxVerbatim}[commandchars=\\\{\}]
\PYG{p}{\PYGZlt{}}\PYG{n+nt}{table} \PYG{n+na}{class}\PYG{o}{=}\PYG{l+s}{\PYGZdq{}table\PYGZdq{}}\PYG{p}{\PYGZgt{}}
  \PYG{p}{\PYGZlt{}}\PYG{n+nt}{thead} \PYG{n+na}{class}\PYG{o}{=}\PYG{l+s}{\PYGZdq{}thead\PYGZhy{}dark\PYGZdq{}}\PYG{p}{\PYGZgt{}}
\end{sphinxVerbatim}




\subsection{Filas "rayadas como una zebra"}
\label{\detokenize{hay-tabla:filas-rayadas-como-una-zebra}}
Cambiar el color de filas adyacentes es útil para que el usuario pueda seguir
las columnas de una fila sin perderse o empezar a leer celdas de las filas
cercanas sin darse cuenta.

\fvset{hllines={, ,}}%
\begin{sphinxVerbatim}[commandchars=\\\{\}]
\PYG{p}{\PYGZlt{}}\PYG{n+nt}{table} \PYG{n+na}{class}\PYG{o}{=}\PYG{l+s}{\PYGZdq{}table table\PYGZhy{}striped\PYGZdq{}}\PYG{p}{\PYGZgt{}}
\end{sphinxVerbatim}




\subsection{Énfasis en fila con foco del mouse}
\label{\detokenize{hay-tabla:enfasis-en-fila-con-foco-del-mouse}}
Resaltar la fila que tiene el foco del mouse es útil para que el usuario pueda
seguir las columnas de una fila sin perderse o empezar a leer celdas de las
filas cercanas sin darse cuenta.

\fvset{hllines={, ,}}%
\begin{sphinxVerbatim}[commandchars=\\\{\}]
\PYG{p}{\PYGZlt{}}\PYG{n+nt}{table} \PYG{n+na}{class}\PYG{o}{=}\PYG{l+s}{\PYGZdq{}table table\PYGZhy{}hover\PYGZdq{}}\PYG{p}{\PYGZgt{}}
\end{sphinxVerbatim}




\subsection{Compacta}
\label{\detokenize{hay-tabla:compacta}}
Una tabla con menos espacios, si es necesario mostrar mas datos.

\fvset{hllines={, ,}}%
\begin{sphinxVerbatim}[commandchars=\\\{\}]
\PYG{p}{\PYGZlt{}}\PYG{n+nt}{table} \PYG{n+na}{class}\PYG{o}{=}\PYG{l+s}{\PYGZdq{}table table\PYGZhy{}sm\PYGZdq{}}\PYG{p}{\PYGZgt{}}
\end{sphinxVerbatim}




\subsection{Resaltando filas}
\label{\detokenize{hay-tabla:resaltando-filas}}
Si necesitamos indicar algo en una fila lo podemos hacer agregando una clase.

\fvset{hllines={, ,}}%
\begin{sphinxVerbatim}[commandchars=\\\{\}]
\PYG{p}{\PYGZlt{}}\PYG{n+nt}{tr} \PYG{n+na}{class}\PYG{o}{=}\PYG{l+s}{\PYGZdq{}table\PYGZhy{}active\PYGZdq{}}\PYG{p}{\PYGZgt{}}
  \PYG{p}{\PYGZlt{}}\PYG{n+nt}{td}\PYG{p}{\PYGZgt{}}Arenita\PYG{p}{\PYGZlt{}}\PYG{p}{/}\PYG{n+nt}{td}\PYG{p}{\PYGZgt{}}
  \PYG{p}{\PYGZlt{}}\PYG{n+nt}{td}\PYG{p}{\PYGZgt{}}Ardilla\PYG{p}{\PYGZlt{}}\PYG{p}{/}\PYG{n+nt}{td}\PYG{p}{\PYGZgt{}}
  \PYG{p}{\PYGZlt{}}\PYG{n+nt}{td}\PYG{p}{\PYGZgt{}}Marron\PYG{p}{\PYGZlt{}}\PYG{p}{/}\PYG{n+nt}{td}\PYG{p}{\PYGZgt{}}
\PYG{p}{\PYGZlt{}}\PYG{p}{/}\PYG{n+nt}{tr}\PYG{p}{\PYGZgt{}}
\PYG{p}{\PYGZlt{}}\PYG{n+nt}{tr} \PYG{n+na}{class}\PYG{o}{=}\PYG{l+s}{\PYGZdq{}table\PYGZhy{}primary\PYGZdq{}}\PYG{p}{\PYGZgt{}}
  \PYG{p}{\PYGZlt{}}\PYG{n+nt}{td}\PYG{p}{\PYGZgt{}}Bob\PYG{p}{\PYGZlt{}}\PYG{p}{/}\PYG{n+nt}{td}\PYG{p}{\PYGZgt{}}
  \PYG{p}{\PYGZlt{}}\PYG{n+nt}{td}\PYG{p}{\PYGZgt{}}Esponja\PYG{p}{\PYGZlt{}}\PYG{p}{/}\PYG{n+nt}{td}\PYG{p}{\PYGZgt{}}
  \PYG{p}{\PYGZlt{}}\PYG{n+nt}{td}\PYG{p}{\PYGZgt{}}Amarillo\PYG{p}{\PYGZlt{}}\PYG{p}{/}\PYG{n+nt}{td}\PYG{p}{\PYGZgt{}}
\PYG{p}{\PYGZlt{}}\PYG{p}{/}\PYG{n+nt}{tr}\PYG{p}{\PYGZgt{}}
\PYG{p}{\PYGZlt{}}\PYG{n+nt}{tr} \PYG{n+na}{class}\PYG{o}{=}\PYG{l+s}{\PYGZdq{}table\PYGZhy{}secondary\PYGZdq{}}\PYG{p}{\PYGZgt{}}
  \PYG{p}{\PYGZlt{}}\PYG{n+nt}{td}\PYG{p}{\PYGZgt{}}Patricio\PYG{p}{\PYGZlt{}}\PYG{p}{/}\PYG{n+nt}{td}\PYG{p}{\PYGZgt{}}
  \PYG{p}{\PYGZlt{}}\PYG{n+nt}{td}\PYG{p}{\PYGZgt{}}Estrella\PYG{p}{\PYGZlt{}}\PYG{p}{/}\PYG{n+nt}{td}\PYG{p}{\PYGZgt{}}
  \PYG{p}{\PYGZlt{}}\PYG{n+nt}{td}\PYG{p}{\PYGZgt{}}Rosa\PYG{p}{\PYGZlt{}}\PYG{p}{/}\PYG{n+nt}{td}\PYG{p}{\PYGZgt{}}
\PYG{p}{\PYGZlt{}}\PYG{p}{/}\PYG{n+nt}{tr}\PYG{p}{\PYGZgt{}}
\PYG{p}{\PYGZlt{}}\PYG{n+nt}{tr} \PYG{n+na}{class}\PYG{o}{=}\PYG{l+s}{\PYGZdq{}table\PYGZhy{}success\PYGZdq{}}\PYG{p}{\PYGZgt{}}
  \PYG{p}{\PYGZlt{}}\PYG{n+nt}{td}\PYG{p}{\PYGZgt{}}Arenita\PYG{p}{\PYGZlt{}}\PYG{p}{/}\PYG{n+nt}{td}\PYG{p}{\PYGZgt{}}
  \PYG{p}{\PYGZlt{}}\PYG{n+nt}{td}\PYG{p}{\PYGZgt{}}Ardilla\PYG{p}{\PYGZlt{}}\PYG{p}{/}\PYG{n+nt}{td}\PYG{p}{\PYGZgt{}}
  \PYG{p}{\PYGZlt{}}\PYG{n+nt}{td}\PYG{p}{\PYGZgt{}}Marron\PYG{p}{\PYGZlt{}}\PYG{p}{/}\PYG{n+nt}{td}\PYG{p}{\PYGZgt{}}
\PYG{p}{\PYGZlt{}}\PYG{p}{/}\PYG{n+nt}{tr}\PYG{p}{\PYGZgt{}}
\PYG{p}{\PYGZlt{}}\PYG{n+nt}{tr} \PYG{n+na}{class}\PYG{o}{=}\PYG{l+s}{\PYGZdq{}table\PYGZhy{}danger\PYGZdq{}}\PYG{p}{\PYGZgt{}}
  \PYG{p}{\PYGZlt{}}\PYG{n+nt}{td}\PYG{p}{\PYGZgt{}}Bob\PYG{p}{\PYGZlt{}}\PYG{p}{/}\PYG{n+nt}{td}\PYG{p}{\PYGZgt{}}
  \PYG{p}{\PYGZlt{}}\PYG{n+nt}{td}\PYG{p}{\PYGZgt{}}Esponja\PYG{p}{\PYGZlt{}}\PYG{p}{/}\PYG{n+nt}{td}\PYG{p}{\PYGZgt{}}
  \PYG{p}{\PYGZlt{}}\PYG{n+nt}{td}\PYG{p}{\PYGZgt{}}Amarillo\PYG{p}{\PYGZlt{}}\PYG{p}{/}\PYG{n+nt}{td}\PYG{p}{\PYGZgt{}}
\PYG{p}{\PYGZlt{}}\PYG{p}{/}\PYG{n+nt}{tr}\PYG{p}{\PYGZgt{}}
\PYG{p}{\PYGZlt{}}\PYG{n+nt}{tr} \PYG{n+na}{class}\PYG{o}{=}\PYG{l+s}{\PYGZdq{}table\PYGZhy{}warning\PYGZdq{}}\PYG{p}{\PYGZgt{}}
  \PYG{p}{\PYGZlt{}}\PYG{n+nt}{td}\PYG{p}{\PYGZgt{}}Patricio\PYG{p}{\PYGZlt{}}\PYG{p}{/}\PYG{n+nt}{td}\PYG{p}{\PYGZgt{}}
  \PYG{p}{\PYGZlt{}}\PYG{n+nt}{td}\PYG{p}{\PYGZgt{}}Estrella\PYG{p}{\PYGZlt{}}\PYG{p}{/}\PYG{n+nt}{td}\PYG{p}{\PYGZgt{}}
  \PYG{p}{\PYGZlt{}}\PYG{n+nt}{td}\PYG{p}{\PYGZgt{}}Rosa\PYG{p}{\PYGZlt{}}\PYG{p}{/}\PYG{n+nt}{td}\PYG{p}{\PYGZgt{}}
\PYG{p}{\PYGZlt{}}\PYG{p}{/}\PYG{n+nt}{tr}\PYG{p}{\PYGZgt{}}
\PYG{p}{\PYGZlt{}}\PYG{n+nt}{tr} \PYG{n+na}{class}\PYG{o}{=}\PYG{l+s}{\PYGZdq{}table\PYGZhy{}info\PYGZdq{}}\PYG{p}{\PYGZgt{}}
  \PYG{p}{\PYGZlt{}}\PYG{n+nt}{td}\PYG{p}{\PYGZgt{}}Arenita\PYG{p}{\PYGZlt{}}\PYG{p}{/}\PYG{n+nt}{td}\PYG{p}{\PYGZgt{}}
  \PYG{p}{\PYGZlt{}}\PYG{n+nt}{td}\PYG{p}{\PYGZgt{}}Ardilla\PYG{p}{\PYGZlt{}}\PYG{p}{/}\PYG{n+nt}{td}\PYG{p}{\PYGZgt{}}
  \PYG{p}{\PYGZlt{}}\PYG{n+nt}{td}\PYG{p}{\PYGZgt{}}Marron\PYG{p}{\PYGZlt{}}\PYG{p}{/}\PYG{n+nt}{td}\PYG{p}{\PYGZgt{}}
\PYG{p}{\PYGZlt{}}\PYG{p}{/}\PYG{n+nt}{tr}\PYG{p}{\PYGZgt{}}
\PYG{p}{\PYGZlt{}}\PYG{n+nt}{tr} \PYG{n+na}{class}\PYG{o}{=}\PYG{l+s}{\PYGZdq{}table\PYGZhy{}light\PYGZdq{}}\PYG{p}{\PYGZgt{}}
  \PYG{p}{\PYGZlt{}}\PYG{n+nt}{td}\PYG{p}{\PYGZgt{}}Bob\PYG{p}{\PYGZlt{}}\PYG{p}{/}\PYG{n+nt}{td}\PYG{p}{\PYGZgt{}}
  \PYG{p}{\PYGZlt{}}\PYG{n+nt}{td}\PYG{p}{\PYGZgt{}}Esponja\PYG{p}{\PYGZlt{}}\PYG{p}{/}\PYG{n+nt}{td}\PYG{p}{\PYGZgt{}}
  \PYG{p}{\PYGZlt{}}\PYG{n+nt}{td}\PYG{p}{\PYGZgt{}}Amarillo\PYG{p}{\PYGZlt{}}\PYG{p}{/}\PYG{n+nt}{td}\PYG{p}{\PYGZgt{}}
\PYG{p}{\PYGZlt{}}\PYG{p}{/}\PYG{n+nt}{tr}\PYG{p}{\PYGZgt{}}
\PYG{p}{\PYGZlt{}}\PYG{n+nt}{tr} \PYG{n+na}{class}\PYG{o}{=}\PYG{l+s}{\PYGZdq{}table\PYGZhy{}dark\PYGZdq{}}\PYG{p}{\PYGZgt{}}
  \PYG{p}{\PYGZlt{}}\PYG{n+nt}{td}\PYG{p}{\PYGZgt{}}Patricio\PYG{p}{\PYGZlt{}}\PYG{p}{/}\PYG{n+nt}{td}\PYG{p}{\PYGZgt{}}
  \PYG{p}{\PYGZlt{}}\PYG{n+nt}{td}\PYG{p}{\PYGZgt{}}Estrella\PYG{p}{\PYGZlt{}}\PYG{p}{/}\PYG{n+nt}{td}\PYG{p}{\PYGZgt{}}
  \PYG{p}{\PYGZlt{}}\PYG{n+nt}{td}\PYG{p}{\PYGZgt{}}Rosa\PYG{p}{\PYGZlt{}}\PYG{p}{/}\PYG{n+nt}{td}\PYG{p}{\PYGZgt{}}
\PYG{p}{\PYGZlt{}}\PYG{p}{/}\PYG{n+nt}{tr}\PYG{p}{\PYGZgt{}}
\end{sphinxVerbatim}




\subsection{Todo junto!}
\label{\detokenize{hay-tabla:todo-junto}}
\fvset{hllines={, ,}}%
\begin{sphinxVerbatim}[commandchars=\\\{\}]
\PYG{p}{\PYGZlt{}}\PYG{n+nt}{table} \PYG{n+na}{class}\PYG{o}{=}\PYG{l+s}{\PYGZdq{}table table\PYGZhy{}bordered table\PYGZhy{}striped table\PYGZhy{}sm table\PYGZhy{}dark table\PYGZhy{}hover\PYGZdq{}}\PYG{p}{\PYGZgt{}}
  \PYG{p}{\PYGZlt{}}\PYG{n+nt}{thead} \PYG{n+na}{class}\PYG{o}{=}\PYG{l+s}{\PYGZdq{}thead\PYGZhy{}light\PYGZdq{}}\PYG{p}{\PYGZgt{}}
    \PYG{p}{\PYGZlt{}}\PYG{n+nt}{tr}\PYG{p}{\PYGZgt{}}
      \PYG{p}{\PYGZlt{}}\PYG{n+nt}{th}\PYG{p}{\PYGZgt{}}Nombre\PYG{p}{\PYGZlt{}}\PYG{p}{/}\PYG{n+nt}{th}\PYG{p}{\PYGZgt{}}
      \PYG{p}{\PYGZlt{}}\PYG{n+nt}{th}\PYG{p}{\PYGZgt{}}Tipo\PYG{p}{\PYGZlt{}}\PYG{p}{/}\PYG{n+nt}{th}\PYG{p}{\PYGZgt{}}
      \PYG{p}{\PYGZlt{}}\PYG{n+nt}{th}\PYG{p}{\PYGZgt{}}Color\PYG{p}{\PYGZlt{}}\PYG{p}{/}\PYG{n+nt}{th}\PYG{p}{\PYGZgt{}}
    \PYG{p}{\PYGZlt{}}\PYG{p}{/}\PYG{n+nt}{tr}\PYG{p}{\PYGZgt{}}
  \PYG{p}{\PYGZlt{}}\PYG{p}{/}\PYG{n+nt}{thead}\PYG{p}{\PYGZgt{}}

  \PYG{p}{\PYGZlt{}}\PYG{n+nt}{tbody}\PYG{p}{\PYGZgt{}}
    \PYG{p}{\PYGZlt{}}\PYG{n+nt}{tr} \PYG{n+na}{class}\PYG{o}{=}\PYG{l+s}{\PYGZdq{}table\PYGZhy{}active\PYGZdq{}}\PYG{p}{\PYGZgt{}}
      \PYG{p}{\PYGZlt{}}\PYG{n+nt}{td}\PYG{p}{\PYGZgt{}}Arenita\PYG{p}{\PYGZlt{}}\PYG{p}{/}\PYG{n+nt}{td}\PYG{p}{\PYGZgt{}}
      \PYG{p}{\PYGZlt{}}\PYG{n+nt}{td}\PYG{p}{\PYGZgt{}}Ardilla\PYG{p}{\PYGZlt{}}\PYG{p}{/}\PYG{n+nt}{td}\PYG{p}{\PYGZgt{}}
      \PYG{p}{\PYGZlt{}}\PYG{n+nt}{td}\PYG{p}{\PYGZgt{}}Marron\PYG{p}{\PYGZlt{}}\PYG{p}{/}\PYG{n+nt}{td}\PYG{p}{\PYGZgt{}}
    \PYG{p}{\PYGZlt{}}\PYG{p}{/}\PYG{n+nt}{tr}\PYG{p}{\PYGZgt{}}
    \PYG{p}{\PYGZlt{}}\PYG{n+nt}{tr} \PYG{n+na}{class}\PYG{o}{=}\PYG{l+s}{\PYGZdq{}table\PYGZhy{}primary\PYGZdq{}}\PYG{p}{\PYGZgt{}}
      \PYG{p}{\PYGZlt{}}\PYG{n+nt}{td}\PYG{p}{\PYGZgt{}}Bob\PYG{p}{\PYGZlt{}}\PYG{p}{/}\PYG{n+nt}{td}\PYG{p}{\PYGZgt{}}
      \PYG{p}{\PYGZlt{}}\PYG{n+nt}{td}\PYG{p}{\PYGZgt{}}Esponja\PYG{p}{\PYGZlt{}}\PYG{p}{/}\PYG{n+nt}{td}\PYG{p}{\PYGZgt{}}
      \PYG{p}{\PYGZlt{}}\PYG{n+nt}{td}\PYG{p}{\PYGZgt{}}Amarillo\PYG{p}{\PYGZlt{}}\PYG{p}{/}\PYG{n+nt}{td}\PYG{p}{\PYGZgt{}}
    \PYG{p}{\PYGZlt{}}\PYG{p}{/}\PYG{n+nt}{tr}\PYG{p}{\PYGZgt{}}
    \PYG{p}{\PYGZlt{}}\PYG{n+nt}{tr} \PYG{n+na}{class}\PYG{o}{=}\PYG{l+s}{\PYGZdq{}table\PYGZhy{}secondary\PYGZdq{}}\PYG{p}{\PYGZgt{}}
      \PYG{p}{\PYGZlt{}}\PYG{n+nt}{td}\PYG{p}{\PYGZgt{}}Patricio\PYG{p}{\PYGZlt{}}\PYG{p}{/}\PYG{n+nt}{td}\PYG{p}{\PYGZgt{}}
      \PYG{p}{\PYGZlt{}}\PYG{n+nt}{td}\PYG{p}{\PYGZgt{}}Estrella\PYG{p}{\PYGZlt{}}\PYG{p}{/}\PYG{n+nt}{td}\PYG{p}{\PYGZgt{}}
      \PYG{p}{\PYGZlt{}}\PYG{n+nt}{td}\PYG{p}{\PYGZgt{}}Rosa\PYG{p}{\PYGZlt{}}\PYG{p}{/}\PYG{n+nt}{td}\PYG{p}{\PYGZgt{}}
    \PYG{p}{\PYGZlt{}}\PYG{p}{/}\PYG{n+nt}{tr}\PYG{p}{\PYGZgt{}}
    \PYG{p}{\PYGZlt{}}\PYG{n+nt}{tr} \PYG{n+na}{class}\PYG{o}{=}\PYG{l+s}{\PYGZdq{}table\PYGZhy{}success\PYGZdq{}}\PYG{p}{\PYGZgt{}}
      \PYG{p}{\PYGZlt{}}\PYG{n+nt}{td}\PYG{p}{\PYGZgt{}}Arenita\PYG{p}{\PYGZlt{}}\PYG{p}{/}\PYG{n+nt}{td}\PYG{p}{\PYGZgt{}}
      \PYG{p}{\PYGZlt{}}\PYG{n+nt}{td}\PYG{p}{\PYGZgt{}}Ardilla\PYG{p}{\PYGZlt{}}\PYG{p}{/}\PYG{n+nt}{td}\PYG{p}{\PYGZgt{}}
      \PYG{p}{\PYGZlt{}}\PYG{n+nt}{td}\PYG{p}{\PYGZgt{}}Marron\PYG{p}{\PYGZlt{}}\PYG{p}{/}\PYG{n+nt}{td}\PYG{p}{\PYGZgt{}}
    \PYG{p}{\PYGZlt{}}\PYG{p}{/}\PYG{n+nt}{tr}\PYG{p}{\PYGZgt{}}
    \PYG{p}{\PYGZlt{}}\PYG{n+nt}{tr} \PYG{n+na}{class}\PYG{o}{=}\PYG{l+s}{\PYGZdq{}table\PYGZhy{}danger\PYGZdq{}}\PYG{p}{\PYGZgt{}}
      \PYG{p}{\PYGZlt{}}\PYG{n+nt}{td}\PYG{p}{\PYGZgt{}}Bob\PYG{p}{\PYGZlt{}}\PYG{p}{/}\PYG{n+nt}{td}\PYG{p}{\PYGZgt{}}
      \PYG{p}{\PYGZlt{}}\PYG{n+nt}{td}\PYG{p}{\PYGZgt{}}Esponja\PYG{p}{\PYGZlt{}}\PYG{p}{/}\PYG{n+nt}{td}\PYG{p}{\PYGZgt{}}
      \PYG{p}{\PYGZlt{}}\PYG{n+nt}{td}\PYG{p}{\PYGZgt{}}Amarillo\PYG{p}{\PYGZlt{}}\PYG{p}{/}\PYG{n+nt}{td}\PYG{p}{\PYGZgt{}}
    \PYG{p}{\PYGZlt{}}\PYG{p}{/}\PYG{n+nt}{tr}\PYG{p}{\PYGZgt{}}
    \PYG{p}{\PYGZlt{}}\PYG{n+nt}{tr} \PYG{n+na}{class}\PYG{o}{=}\PYG{l+s}{\PYGZdq{}table\PYGZhy{}warning\PYGZdq{}}\PYG{p}{\PYGZgt{}}
      \PYG{p}{\PYGZlt{}}\PYG{n+nt}{td}\PYG{p}{\PYGZgt{}}Patricio\PYG{p}{\PYGZlt{}}\PYG{p}{/}\PYG{n+nt}{td}\PYG{p}{\PYGZgt{}}
      \PYG{p}{\PYGZlt{}}\PYG{n+nt}{td}\PYG{p}{\PYGZgt{}}Estrella\PYG{p}{\PYGZlt{}}\PYG{p}{/}\PYG{n+nt}{td}\PYG{p}{\PYGZgt{}}
      \PYG{p}{\PYGZlt{}}\PYG{n+nt}{td}\PYG{p}{\PYGZgt{}}Rosa\PYG{p}{\PYGZlt{}}\PYG{p}{/}\PYG{n+nt}{td}\PYG{p}{\PYGZgt{}}
    \PYG{p}{\PYGZlt{}}\PYG{p}{/}\PYG{n+nt}{tr}\PYG{p}{\PYGZgt{}}
    \PYG{p}{\PYGZlt{}}\PYG{n+nt}{tr} \PYG{n+na}{class}\PYG{o}{=}\PYG{l+s}{\PYGZdq{}table\PYGZhy{}info\PYGZdq{}}\PYG{p}{\PYGZgt{}}
      \PYG{p}{\PYGZlt{}}\PYG{n+nt}{td}\PYG{p}{\PYGZgt{}}Arenita\PYG{p}{\PYGZlt{}}\PYG{p}{/}\PYG{n+nt}{td}\PYG{p}{\PYGZgt{}}
      \PYG{p}{\PYGZlt{}}\PYG{n+nt}{td}\PYG{p}{\PYGZgt{}}Ardilla\PYG{p}{\PYGZlt{}}\PYG{p}{/}\PYG{n+nt}{td}\PYG{p}{\PYGZgt{}}
      \PYG{p}{\PYGZlt{}}\PYG{n+nt}{td}\PYG{p}{\PYGZgt{}}Marron\PYG{p}{\PYGZlt{}}\PYG{p}{/}\PYG{n+nt}{td}\PYG{p}{\PYGZgt{}}
    \PYG{p}{\PYGZlt{}}\PYG{p}{/}\PYG{n+nt}{tr}\PYG{p}{\PYGZgt{}}
    \PYG{p}{\PYGZlt{}}\PYG{n+nt}{tr} \PYG{n+na}{class}\PYG{o}{=}\PYG{l+s}{\PYGZdq{}table\PYGZhy{}light\PYGZdq{}}\PYG{p}{\PYGZgt{}}
      \PYG{p}{\PYGZlt{}}\PYG{n+nt}{td}\PYG{p}{\PYGZgt{}}Bob\PYG{p}{\PYGZlt{}}\PYG{p}{/}\PYG{n+nt}{td}\PYG{p}{\PYGZgt{}}
      \PYG{p}{\PYGZlt{}}\PYG{n+nt}{td}\PYG{p}{\PYGZgt{}}Esponja\PYG{p}{\PYGZlt{}}\PYG{p}{/}\PYG{n+nt}{td}\PYG{p}{\PYGZgt{}}
      \PYG{p}{\PYGZlt{}}\PYG{n+nt}{td}\PYG{p}{\PYGZgt{}}Amarillo\PYG{p}{\PYGZlt{}}\PYG{p}{/}\PYG{n+nt}{td}\PYG{p}{\PYGZgt{}}
    \PYG{p}{\PYGZlt{}}\PYG{p}{/}\PYG{n+nt}{tr}\PYG{p}{\PYGZgt{}}
    \PYG{p}{\PYGZlt{}}\PYG{n+nt}{tr} \PYG{n+na}{class}\PYG{o}{=}\PYG{l+s}{\PYGZdq{}table\PYGZhy{}dark\PYGZdq{}}\PYG{p}{\PYGZgt{}}
      \PYG{p}{\PYGZlt{}}\PYG{n+nt}{td}\PYG{p}{\PYGZgt{}}Patricio\PYG{p}{\PYGZlt{}}\PYG{p}{/}\PYG{n+nt}{td}\PYG{p}{\PYGZgt{}}
      \PYG{p}{\PYGZlt{}}\PYG{n+nt}{td}\PYG{p}{\PYGZgt{}}Estrella\PYG{p}{\PYGZlt{}}\PYG{p}{/}\PYG{n+nt}{td}\PYG{p}{\PYGZgt{}}
      \PYG{p}{\PYGZlt{}}\PYG{n+nt}{td}\PYG{p}{\PYGZgt{}}Rosa\PYG{p}{\PYGZlt{}}\PYG{p}{/}\PYG{n+nt}{td}\PYG{p}{\PYGZgt{}}
    \PYG{p}{\PYGZlt{}}\PYG{p}{/}\PYG{n+nt}{tr}\PYG{p}{\PYGZgt{}}
  \PYG{p}{\PYGZlt{}}\PYG{p}{/}\PYG{n+nt}{tbody}\PYG{p}{\PYGZgt{}}

  \PYG{p}{\PYGZlt{}}\PYG{n+nt}{tfoot}\PYG{p}{\PYGZgt{}}
    \PYG{p}{\PYGZlt{}}\PYG{n+nt}{tr}\PYG{p}{\PYGZgt{}}
      \PYG{p}{\PYGZlt{}}\PYG{n+nt}{td}\PYG{p}{\PYGZgt{}}Nombre\PYG{p}{\PYGZlt{}}\PYG{p}{/}\PYG{n+nt}{td}\PYG{p}{\PYGZgt{}}
      \PYG{p}{\PYGZlt{}}\PYG{n+nt}{td}\PYG{p}{\PYGZgt{}}Tipo\PYG{p}{\PYGZlt{}}\PYG{p}{/}\PYG{n+nt}{td}\PYG{p}{\PYGZgt{}}
      \PYG{p}{\PYGZlt{}}\PYG{n+nt}{td}\PYG{p}{\PYGZgt{}}Color\PYG{p}{\PYGZlt{}}\PYG{p}{/}\PYG{n+nt}{td}\PYG{p}{\PYGZgt{}}
    \PYG{p}{\PYGZlt{}}\PYG{p}{/}\PYG{n+nt}{tr}\PYG{p}{\PYGZgt{}}
  \PYG{p}{\PYGZlt{}}\PYG{p}{/}\PYG{n+nt}{tfoot}\PYG{p}{\PYGZgt{}}
\PYG{p}{\PYGZlt{}}\PYG{p}{/}\PYG{n+nt}{table}\PYG{p}{\PYGZgt{}}
\end{sphinxVerbatim}




\section{Subtítulo}
\label{\detokenize{hay-tabla:subtitulo}}
Es útil indicar que datos se están mostrando en la tabla, para eso hay un tag
llamado \sphinxhref{https://developer.mozilla.org/es/docs/Web/HTML/Elemento/caption}{caption} que puede ser usado para tal fin.

\fvset{hllines={, ,}}%
\begin{sphinxVerbatim}[commandchars=\\\{\}]
\PYG{p}{\PYGZlt{}}\PYG{n+nt}{table} \PYG{n+na}{class}\PYG{o}{=}\PYG{l+s}{\PYGZdq{}table\PYGZdq{}}\PYG{p}{\PYGZgt{}}
  \PYG{p}{\PYGZlt{}}\PYG{n+nt}{caption}\PYG{p}{\PYGZgt{}}Personajes de Bob Esponja\PYG{p}{\PYGZlt{}}\PYG{p}{/}\PYG{n+nt}{caption}\PYG{p}{\PYGZgt{}}

  \PYG{p}{\PYGZlt{}}\PYG{n+nt}{thead}\PYG{p}{\PYGZgt{}}
    \PYG{p}{\PYGZlt{}}\PYG{n+nt}{tr}\PYG{p}{\PYGZgt{}}
      \PYG{p}{\PYGZlt{}}\PYG{n+nt}{th}\PYG{p}{\PYGZgt{}}Nombre\PYG{p}{\PYGZlt{}}\PYG{p}{/}\PYG{n+nt}{th}\PYG{p}{\PYGZgt{}}
      \PYG{p}{\PYGZlt{}}\PYG{n+nt}{th}\PYG{p}{\PYGZgt{}}Tipo\PYG{p}{\PYGZlt{}}\PYG{p}{/}\PYG{n+nt}{th}\PYG{p}{\PYGZgt{}}
      \PYG{p}{\PYGZlt{}}\PYG{n+nt}{th}\PYG{p}{\PYGZgt{}}Color\PYG{p}{\PYGZlt{}}\PYG{p}{/}\PYG{n+nt}{th}\PYG{p}{\PYGZgt{}}
    \PYG{p}{\PYGZlt{}}\PYG{p}{/}\PYG{n+nt}{tr}\PYG{p}{\PYGZgt{}}
  \PYG{p}{\PYGZlt{}}\PYG{p}{/}\PYG{n+nt}{thead}\PYG{p}{\PYGZgt{}}

  \PYG{p}{\PYGZlt{}}\PYG{n+nt}{tbody}\PYG{p}{\PYGZgt{}}
    \PYG{p}{\PYGZlt{}}\PYG{n+nt}{tr}\PYG{p}{\PYGZgt{}}
      \PYG{p}{\PYGZlt{}}\PYG{n+nt}{td}\PYG{p}{\PYGZgt{}}Arenita\PYG{p}{\PYGZlt{}}\PYG{p}{/}\PYG{n+nt}{td}\PYG{p}{\PYGZgt{}}
      \PYG{p}{\PYGZlt{}}\PYG{n+nt}{td}\PYG{p}{\PYGZgt{}}Ardilla\PYG{p}{\PYGZlt{}}\PYG{p}{/}\PYG{n+nt}{td}\PYG{p}{\PYGZgt{}}
      \PYG{p}{\PYGZlt{}}\PYG{n+nt}{td}\PYG{p}{\PYGZgt{}}Marron\PYG{p}{\PYGZlt{}}\PYG{p}{/}\PYG{n+nt}{td}\PYG{p}{\PYGZgt{}}
    \PYG{p}{\PYGZlt{}}\PYG{p}{/}\PYG{n+nt}{tr}\PYG{p}{\PYGZgt{}}
    \PYG{p}{\PYGZlt{}}\PYG{n+nt}{tr}\PYG{p}{\PYGZgt{}}
      \PYG{p}{\PYGZlt{}}\PYG{n+nt}{td}\PYG{p}{\PYGZgt{}}Bob\PYG{p}{\PYGZlt{}}\PYG{p}{/}\PYG{n+nt}{td}\PYG{p}{\PYGZgt{}}
      \PYG{p}{\PYGZlt{}}\PYG{n+nt}{td}\PYG{p}{\PYGZgt{}}Esponja\PYG{p}{\PYGZlt{}}\PYG{p}{/}\PYG{n+nt}{td}\PYG{p}{\PYGZgt{}}
      \PYG{p}{\PYGZlt{}}\PYG{n+nt}{td}\PYG{p}{\PYGZgt{}}Amarillo\PYG{p}{\PYGZlt{}}\PYG{p}{/}\PYG{n+nt}{td}\PYG{p}{\PYGZgt{}}
    \PYG{p}{\PYGZlt{}}\PYG{p}{/}\PYG{n+nt}{tr}\PYG{p}{\PYGZgt{}}
    \PYG{p}{\PYGZlt{}}\PYG{n+nt}{tr}\PYG{p}{\PYGZgt{}}
      \PYG{p}{\PYGZlt{}}\PYG{n+nt}{td}\PYG{p}{\PYGZgt{}}Patricio\PYG{p}{\PYGZlt{}}\PYG{p}{/}\PYG{n+nt}{td}\PYG{p}{\PYGZgt{}}
      \PYG{p}{\PYGZlt{}}\PYG{n+nt}{td}\PYG{p}{\PYGZgt{}}Estrella\PYG{p}{\PYGZlt{}}\PYG{p}{/}\PYG{n+nt}{td}\PYG{p}{\PYGZgt{}}
      \PYG{p}{\PYGZlt{}}\PYG{n+nt}{td}\PYG{p}{\PYGZgt{}}Rosa\PYG{p}{\PYGZlt{}}\PYG{p}{/}\PYG{n+nt}{td}\PYG{p}{\PYGZgt{}}
    \PYG{p}{\PYGZlt{}}\PYG{p}{/}\PYG{n+nt}{tr}\PYG{p}{\PYGZgt{}}
  \PYG{p}{\PYGZlt{}}\PYG{p}{/}\PYG{n+nt}{tbody}\PYG{p}{\PYGZgt{}}

  \PYG{p}{\PYGZlt{}}\PYG{n+nt}{tfoot}\PYG{p}{\PYGZgt{}}
    \PYG{p}{\PYGZlt{}}\PYG{n+nt}{tr}\PYG{p}{\PYGZgt{}}
      \PYG{p}{\PYGZlt{}}\PYG{n+nt}{td}\PYG{p}{\PYGZgt{}}Nombre\PYG{p}{\PYGZlt{}}\PYG{p}{/}\PYG{n+nt}{td}\PYG{p}{\PYGZgt{}}
      \PYG{p}{\PYGZlt{}}\PYG{n+nt}{td}\PYG{p}{\PYGZgt{}}Tipo\PYG{p}{\PYGZlt{}}\PYG{p}{/}\PYG{n+nt}{td}\PYG{p}{\PYGZgt{}}
      \PYG{p}{\PYGZlt{}}\PYG{n+nt}{td}\PYG{p}{\PYGZgt{}}Color\PYG{p}{\PYGZlt{}}\PYG{p}{/}\PYG{n+nt}{td}\PYG{p}{\PYGZgt{}}
    \PYG{p}{\PYGZlt{}}\PYG{p}{/}\PYG{n+nt}{tr}\PYG{p}{\PYGZgt{}}
  \PYG{p}{\PYGZlt{}}\PYG{p}{/}\PYG{n+nt}{tfoot}\PYG{p}{\PYGZgt{}}
\PYG{p}{\PYGZlt{}}\PYG{p}{/}\PYG{n+nt}{table}\PYG{p}{\PYGZgt{}}
\end{sphinxVerbatim}




\section{Celdas vacías y celdas unidas}
\label{\detokenize{hay-tabla:celdas-vacias-y-celdas-unidas}}
En algunos casos podemos querer dejar algunas filas con menos celdas que otras,
en otros casos queremos que una celda ocupe mas de una columna.

Usar menos celdas es fácil, simplemente no las escribimos.

\fvset{hllines={, ,}}%
\begin{sphinxVerbatim}[commandchars=\\\{\}]
\PYG{p}{\PYGZlt{}}\PYG{n+nt}{table} \PYG{n+na}{class}\PYG{o}{=}\PYG{l+s}{\PYGZdq{}table table\PYGZhy{}bordered\PYGZdq{}}\PYG{p}{\PYGZgt{}}
  \PYG{p}{\PYGZlt{}}\PYG{n+nt}{tbody}\PYG{p}{\PYGZgt{}}
    \PYG{p}{\PYGZlt{}}\PYG{n+nt}{tr}\PYG{p}{\PYGZgt{}}
      \PYG{p}{\PYGZlt{}}\PYG{n+nt}{td}\PYG{p}{\PYGZgt{}}1.1\PYG{p}{\PYGZlt{}}\PYG{p}{/}\PYG{n+nt}{td}\PYG{p}{\PYGZgt{}}
    \PYG{p}{\PYGZlt{}}\PYG{p}{/}\PYG{n+nt}{tr}\PYG{p}{\PYGZgt{}}
    \PYG{p}{\PYGZlt{}}\PYG{n+nt}{tr}\PYG{p}{\PYGZgt{}}
      \PYG{p}{\PYGZlt{}}\PYG{n+nt}{td}\PYG{p}{\PYGZgt{}}2.1\PYG{p}{\PYGZlt{}}\PYG{p}{/}\PYG{n+nt}{td}\PYG{p}{\PYGZgt{}}
      \PYG{p}{\PYGZlt{}}\PYG{n+nt}{td}\PYG{p}{\PYGZgt{}}2.2\PYG{p}{\PYGZlt{}}\PYG{p}{/}\PYG{n+nt}{td}\PYG{p}{\PYGZgt{}}
    \PYG{p}{\PYGZlt{}}\PYG{p}{/}\PYG{n+nt}{tr}\PYG{p}{\PYGZgt{}}
    \PYG{p}{\PYGZlt{}}\PYG{n+nt}{tr}\PYG{p}{\PYGZgt{}}
      \PYG{p}{\PYGZlt{}}\PYG{n+nt}{td}\PYG{p}{\PYGZgt{}}3.1\PYG{p}{\PYGZlt{}}\PYG{p}{/}\PYG{n+nt}{td}\PYG{p}{\PYGZgt{}}
      \PYG{p}{\PYGZlt{}}\PYG{n+nt}{td}\PYG{p}{\PYGZgt{}}3.2\PYG{p}{\PYGZlt{}}\PYG{p}{/}\PYG{n+nt}{td}\PYG{p}{\PYGZgt{}}
      \PYG{p}{\PYGZlt{}}\PYG{n+nt}{td}\PYG{p}{\PYGZgt{}}3.3\PYG{p}{\PYGZlt{}}\PYG{p}{/}\PYG{n+nt}{td}\PYG{p}{\PYGZgt{}}
    \PYG{p}{\PYGZlt{}}\PYG{p}{/}\PYG{n+nt}{tr}\PYG{p}{\PYGZgt{}}
  \PYG{p}{\PYGZlt{}}\PYG{p}{/}\PYG{n+nt}{tbody}\PYG{p}{\PYGZgt{}}
\PYG{p}{\PYGZlt{}}\PYG{p}{/}\PYG{n+nt}{table}\PYG{p}{\PYGZgt{}}
\end{sphinxVerbatim}



Si queremos que una celda ocupe mas de una columna, tenemos que usar el
atributo \sphinxtitleref{colspan} (extensión de columnas) e indicarle cuantas columnas
queremos que ocupe, por defecto cada celda tiene un \sphinxtitleref{colspan} de 1.

\fvset{hllines={, ,}}%
\begin{sphinxVerbatim}[commandchars=\\\{\}]
\PYG{p}{\PYGZlt{}}\PYG{n+nt}{table} \PYG{n+na}{class}\PYG{o}{=}\PYG{l+s}{\PYGZdq{}table table\PYGZhy{}bordered\PYGZdq{}}\PYG{p}{\PYGZgt{}}
  \PYG{p}{\PYGZlt{}}\PYG{n+nt}{tbody}\PYG{p}{\PYGZgt{}}
    \PYG{p}{\PYGZlt{}}\PYG{n+nt}{tr}\PYG{p}{\PYGZgt{}}
      \PYG{p}{\PYGZlt{}}\PYG{n+nt}{td} \PYG{n+na}{colspan}\PYG{o}{=}\PYG{l+s}{3}\PYG{p}{\PYGZgt{}}1.1\PYG{p}{\PYGZlt{}}\PYG{p}{/}\PYG{n+nt}{td}\PYG{p}{\PYGZgt{}}
    \PYG{p}{\PYGZlt{}}\PYG{p}{/}\PYG{n+nt}{tr}\PYG{p}{\PYGZgt{}}
    \PYG{p}{\PYGZlt{}}\PYG{n+nt}{tr}\PYG{p}{\PYGZgt{}}
      \PYG{p}{\PYGZlt{}}\PYG{n+nt}{td}\PYG{p}{\PYGZgt{}}2.1\PYG{p}{\PYGZlt{}}\PYG{p}{/}\PYG{n+nt}{td}\PYG{p}{\PYGZgt{}}
      \PYG{p}{\PYGZlt{}}\PYG{n+nt}{td} \PYG{n+na}{colspan}\PYG{o}{=}\PYG{l+s}{2}\PYG{p}{\PYGZgt{}}2.2\PYG{p}{\PYGZlt{}}\PYG{p}{/}\PYG{n+nt}{td}\PYG{p}{\PYGZgt{}}
    \PYG{p}{\PYGZlt{}}\PYG{p}{/}\PYG{n+nt}{tr}\PYG{p}{\PYGZgt{}}
    \PYG{p}{\PYGZlt{}}\PYG{n+nt}{tr}\PYG{p}{\PYGZgt{}}
      \PYG{p}{\PYGZlt{}}\PYG{n+nt}{td}\PYG{p}{\PYGZgt{}}3.1\PYG{p}{\PYGZlt{}}\PYG{p}{/}\PYG{n+nt}{td}\PYG{p}{\PYGZgt{}}
      \PYG{p}{\PYGZlt{}}\PYG{n+nt}{td}\PYG{p}{\PYGZgt{}}3.2\PYG{p}{\PYGZlt{}}\PYG{p}{/}\PYG{n+nt}{td}\PYG{p}{\PYGZgt{}}
      \PYG{p}{\PYGZlt{}}\PYG{n+nt}{td}\PYG{p}{\PYGZgt{}}3.3\PYG{p}{\PYGZlt{}}\PYG{p}{/}\PYG{n+nt}{td}\PYG{p}{\PYGZgt{}}
    \PYG{p}{\PYGZlt{}}\PYG{p}{/}\PYG{n+nt}{tr}\PYG{p}{\PYGZgt{}}
  \PYG{p}{\PYGZlt{}}\PYG{p}{/}\PYG{n+nt}{tbody}\PYG{p}{\PYGZgt{}}
\PYG{p}{\PYGZlt{}}\PYG{p}{/}\PYG{n+nt}{table}\PYG{p}{\PYGZgt{}}
\end{sphinxVerbatim}



Esto lo podemos usar en la cabecera para crear cabeceras multinivel:

\fvset{hllines={, ,}}%
\begin{sphinxVerbatim}[commandchars=\\\{\}]
 \PYG{p}{\PYGZlt{}}\PYG{n+nt}{table} \PYG{n+na}{class}\PYG{o}{=}\PYG{l+s}{\PYGZdq{}table table\PYGZhy{}bordered\PYGZdq{}}\PYG{p}{\PYGZgt{}}
   \PYG{p}{\PYGZlt{}}\PYG{n+nt}{thead} \PYG{n+na}{class}\PYG{o}{=}\PYG{l+s}{\PYGZdq{}thead\PYGZhy{}dark\PYGZdq{}}\PYG{p}{\PYGZgt{}}
     \PYG{p}{\PYGZlt{}}\PYG{n+nt}{tr}\PYG{p}{\PYGZgt{}}
       \PYG{p}{\PYGZlt{}}\PYG{n+nt}{th} \PYG{n+na}{colspan}\PYG{o}{=}\PYG{l+s}{3}\PYG{p}{\PYGZgt{}}Lugar\PYG{p}{\PYGZlt{}}\PYG{p}{/}\PYG{n+nt}{th}\PYG{p}{\PYGZgt{}}
       \PYG{p}{\PYGZlt{}}\PYG{n+nt}{th} \PYG{n+na}{rowspan}\PYG{o}{=}\PYG{l+s}{2} \PYG{n+na}{style}\PYG{o}{=}\PYG{l+s}{\PYGZdq{}vertical\PYGZhy{}align: middle\PYGZdq{}}\PYG{p}{\PYGZgt{}}Población\PYG{p}{\PYGZlt{}}\PYG{p}{/}\PYG{n+nt}{th}\PYG{p}{\PYGZgt{}}
     \PYG{p}{\PYGZlt{}}\PYG{p}{/}\PYG{n+nt}{tr}\PYG{p}{\PYGZgt{}}
     \PYG{p}{\PYGZlt{}}\PYG{n+nt}{tr}\PYG{p}{\PYGZgt{}}
       \PYG{p}{\PYGZlt{}}\PYG{n+nt}{th}\PYG{p}{\PYGZgt{}}Continente\PYG{p}{\PYGZlt{}}\PYG{p}{/}\PYG{n+nt}{th}\PYG{p}{\PYGZgt{}}
       \PYG{p}{\PYGZlt{}}\PYG{n+nt}{th}\PYG{p}{\PYGZgt{}}Pais\PYG{p}{\PYGZlt{}}\PYG{p}{/}\PYG{n+nt}{th}\PYG{p}{\PYGZgt{}}
       \PYG{p}{\PYGZlt{}}\PYG{n+nt}{th}\PYG{p}{\PYGZgt{}}Ciudad\PYG{p}{\PYGZlt{}}\PYG{p}{/}\PYG{n+nt}{th}\PYG{p}{\PYGZgt{}}
     \PYG{p}{\PYGZlt{}}\PYG{p}{/}\PYG{n+nt}{tr}\PYG{p}{\PYGZgt{}}
   \PYG{p}{\PYGZlt{}}\PYG{p}{/}\PYG{n+nt}{thead}\PYG{p}{\PYGZgt{}}

  \PYG{p}{\PYGZlt{}}\PYG{n+nt}{tbody}\PYG{p}{\PYGZgt{}}
    \PYG{p}{\PYGZlt{}}\PYG{n+nt}{tr}\PYG{p}{\PYGZgt{}}
      \PYG{p}{\PYGZlt{}}\PYG{n+nt}{td}\PYG{p}{\PYGZgt{}}America\PYG{p}{\PYGZlt{}}\PYG{p}{/}\PYG{n+nt}{td}\PYG{p}{\PYGZgt{}}
      \PYG{p}{\PYGZlt{}}\PYG{n+nt}{td}\PYG{p}{\PYGZgt{}}México\PYG{p}{\PYGZlt{}}\PYG{p}{/}\PYG{n+nt}{td}\PYG{p}{\PYGZgt{}}
      \PYG{p}{\PYGZlt{}}\PYG{n+nt}{td}\PYG{p}{\PYGZgt{}}Ciudad de México\PYG{p}{\PYGZlt{}}\PYG{p}{/}\PYG{n+nt}{td}\PYG{p}{\PYGZgt{}}
      \PYG{p}{\PYGZlt{}}\PYG{n+nt}{td}\PYG{p}{\PYGZgt{}}8 918 653\PYG{p}{\PYGZlt{}}\PYG{p}{/}\PYG{n+nt}{td}\PYG{p}{\PYGZgt{}}
    \PYG{p}{\PYGZlt{}}\PYG{p}{/}\PYG{n+nt}{tr}\PYG{p}{\PYGZgt{}}
    \PYG{p}{\PYGZlt{}}\PYG{n+nt}{tr}\PYG{p}{\PYGZgt{}}
      \PYG{p}{\PYGZlt{}}\PYG{n+nt}{td}\PYG{p}{\PYGZgt{}}America\PYG{p}{\PYGZlt{}}\PYG{p}{/}\PYG{n+nt}{td}\PYG{p}{\PYGZgt{}}
      \PYG{p}{\PYGZlt{}}\PYG{n+nt}{td}\PYG{p}{\PYGZgt{}}Argentina\PYG{p}{\PYGZlt{}}\PYG{p}{/}\PYG{n+nt}{td}\PYG{p}{\PYGZgt{}}
      \PYG{p}{\PYGZlt{}}\PYG{n+nt}{td}\PYG{p}{\PYGZgt{}}Buenos Aires\PYG{p}{\PYGZlt{}}\PYG{p}{/}\PYG{n+nt}{td}\PYG{p}{\PYGZgt{}}
      \PYG{p}{\PYGZlt{}}\PYG{n+nt}{td}\PYG{p}{\PYGZgt{}}2 890 151\PYG{p}{\PYGZlt{}}\PYG{p}{/}\PYG{n+nt}{td}\PYG{p}{\PYGZgt{}}
    \PYG{p}{\PYGZlt{}}\PYG{p}{/}\PYG{n+nt}{tr}\PYG{p}{\PYGZgt{}}
    \PYG{p}{\PYGZlt{}}\PYG{n+nt}{tr}\PYG{p}{\PYGZgt{}}
      \PYG{p}{\PYGZlt{}}\PYG{n+nt}{td}\PYG{p}{\PYGZgt{}}America\PYG{p}{\PYGZlt{}}\PYG{p}{/}\PYG{n+nt}{td}\PYG{p}{\PYGZgt{}}
      \PYG{p}{\PYGZlt{}}\PYG{n+nt}{td}\PYG{p}{\PYGZgt{}}Brasil\PYG{p}{\PYGZlt{}}\PYG{p}{/}\PYG{n+nt}{td}\PYG{p}{\PYGZgt{}}
      \PYG{p}{\PYGZlt{}}\PYG{n+nt}{td}\PYG{p}{\PYGZgt{}}Brasilia\PYG{p}{\PYGZlt{}}\PYG{p}{/}\PYG{n+nt}{td}\PYG{p}{\PYGZgt{}}
      \PYG{p}{\PYGZlt{}}\PYG{n+nt}{td}\PYG{p}{\PYGZgt{}}2 789 761\PYG{p}{\PYGZlt{}}\PYG{p}{/}\PYG{n+nt}{td}\PYG{p}{\PYGZgt{}}
    \PYG{p}{\PYGZlt{}}\PYG{p}{/}\PYG{n+nt}{tr}\PYG{p}{\PYGZgt{}}
  \PYG{p}{\PYGZlt{}}\PYG{p}{/}\PYG{n+nt}{tbody}\PYG{p}{\PYGZgt{}}
\PYG{p}{\PYGZlt{}}\PYG{p}{/}\PYG{n+nt}{table}\PYG{p}{\PYGZgt{}}
\end{sphinxVerbatim}



Quizás notaste que use rowspan (extensión de fila) para "unir" las dos celdas de población verticalmente, también use el css \sphinxhref{https://developer.mozilla.org/es/docs/Web/CSS/vertical-align}{vertical-align} (alineación vertical) para que el texto este en el medio verticalmente.


\chapter{Filas, columnas y pantallas de todos los tamaños}
\label{\detokenize{filas-columnas::doc}}\label{\detokenize{filas-columnas:filas-columnas-y-pantallas-de-todos-los-tamanos}}
Hasta ahora hemos creado paginas con HTML donde el documento tiene una
estructura básica: una cosa debajo de la otra.

Si prestas atención a sitios que visitas notaras que la estructura de esas
paginas es mas complejas, la forma principal de organizarlas es con filas y
columnas.

El lenguaje HTML provee algunos tags para indicar la intención del contenido
dentro de esos tags pero no provee la estructura en si misma, para eso usamos
CSS, el tema con los atributos CSS para definir estructura es que son muy
específicos y flexibles, la idea es que con ellos podamos lograr cualquier tipo
de estructura que deseemos, pero con gran flexibilidad viene gran complejidad.

Por esta razón han surgido distintas librerías CSS que permiten describir la
estructura de un documento en términos mas generales (lo que los hace menos
flexibles) pero de una forma que sirven para la mayoría de los casos que
necesitamos.

Esta sección va a explorar como usar bootstrap para definir la estructura de
nuestra pagina.


\section{Vocabulario}
\label{\detokenize{filas-columnas:vocabulario}}
Cuando definimos la estructura de un documento con bootstrap las palabras que
aparecen son las siguientes:
\begin{description}
\item[{Contenedor (container en ingles)}] \leavevmode
La raiz de una estructura con filas y columnas, podemos tener contenedores
dentro de columnas pero no es algo común

\item[{Fila (row en ingles)}] \leavevmode
Una sección horizontal dentro de un contenedor o dentro de una columna

\item[{Columna (col/column en ingles)}] \leavevmode
Una sección vertical dentro de una fila

\item[{Puntos de corte (breakpoints en ingles)}] \leavevmode
Limites de resolución horizontal en la cual ciertas reglas cambian de
significado (si, suena vago, vamos a ver esto en detalle luego)

\end{description}


\section{Empezando}
\label{\detokenize{filas-columnas:empezando}}

\section{Estructura básica}
\label{\detokenize{filas-columnas:estructura-basica}}
Vamos a empezar con el ejemplo mas básico, un contenedor que tiene una fila
que tiene una columna.



\fvset{hllines={, ,}}%
\begin{sphinxVerbatim}[commandchars=\\\{\}]
\PYG{p}{\PYGZlt{}}\PYG{n+nt}{div} \PYG{n+na}{class}\PYG{o}{=}\PYG{l+s}{\PYGZdq{}container\PYGZhy{}fluid cew\PYGZhy{}9\PYGZdq{}}\PYG{p}{\PYGZgt{}}
 \PYG{p}{\PYGZlt{}}\PYG{n+nt}{div} \PYG{n+na}{class}\PYG{o}{=}\PYG{l+s}{\PYGZdq{}row\PYGZdq{}}\PYG{p}{\PYGZgt{}}
  \PYG{p}{\PYGZlt{}}\PYG{n+nt}{div} \PYG{n+na}{class}\PYG{o}{=}\PYG{l+s}{\PYGZdq{}col\PYGZdq{}}\PYG{p}{\PYGZgt{}}
    Columna 1
  \PYG{p}{\PYGZlt{}}\PYG{p}{/}\PYG{n+nt}{div}\PYG{p}{\PYGZgt{}}
 \PYG{p}{\PYGZlt{}}\PYG{p}{/}\PYG{n+nt}{div}\PYG{p}{\PYGZgt{}}
\PYG{p}{\PYGZlt{}}\PYG{p}{/}\PYG{n+nt}{div}\PYG{p}{\PYGZgt{}}
\end{sphinxVerbatim}

Nada raro, simplemente esta estructura definida con divs y clases:
\begin{itemize}
\item {} 
container
\begin{itemize}
\item {} 
row
\begin{itemize}
\item {} 
col

\end{itemize}

\end{itemize}

\end{itemize}

Agregue una clase cew-9 a container-fluid para poder resaltar los distintos tags con
css ya que si no lo hago es difícil percibir la estructura, los colores son los
siguientes:
\begin{itemize}
\item {} 
container-fluid: rojo

\item {} 
row: gris y blanco

\item {} 
col: verde

\end{itemize}



Usamos la clase \sphinxtitleref{container-fluid} para que el contenedor se estire el 100\% del
ancho del tag que lo contiene, la clase \sphinxtitleref{container} puede ser usada cuando
queremos mas control sobre el ancho del contenedor.


\section{Dos columnas}
\label{\detokenize{filas-columnas:dos-columnas}}
Un paso mas, dos columnas, que ya es algo que nos permite replicar bastantes
estructuras encontradas en la web.

\fvset{hllines={, ,}}%
\begin{sphinxVerbatim}[commandchars=\\\{\}]
\PYG{p}{\PYGZlt{}}\PYG{n+nt}{div} \PYG{n+na}{class}\PYG{o}{=}\PYG{l+s}{\PYGZdq{}container\PYGZhy{}fluid cew\PYGZhy{}9\PYGZdq{}}\PYG{p}{\PYGZgt{}}
 \PYG{p}{\PYGZlt{}}\PYG{n+nt}{div} \PYG{n+na}{class}\PYG{o}{=}\PYG{l+s}{\PYGZdq{}row\PYGZdq{}}\PYG{p}{\PYGZgt{}}
  \PYG{p}{\PYGZlt{}}\PYG{n+nt}{div} \PYG{n+na}{class}\PYG{o}{=}\PYG{l+s}{\PYGZdq{}col\PYGZdq{}}\PYG{p}{\PYGZgt{}}
    Columna 1
  \PYG{p}{\PYGZlt{}}\PYG{p}{/}\PYG{n+nt}{div}\PYG{p}{\PYGZgt{}}
  \PYG{p}{\PYGZlt{}}\PYG{n+nt}{div} \PYG{n+na}{class}\PYG{o}{=}\PYG{l+s}{\PYGZdq{}col\PYGZdq{}}\PYG{p}{\PYGZgt{}}
    Columna 2
  \PYG{p}{\PYGZlt{}}\PYG{p}{/}\PYG{n+nt}{div}\PYG{p}{\PYGZgt{}}
 \PYG{p}{\PYGZlt{}}\PYG{p}{/}\PYG{n+nt}{div}\PYG{p}{\PYGZgt{}}
\PYG{p}{\PYGZlt{}}\PYG{p}{/}\PYG{n+nt}{div}\PYG{p}{\PYGZgt{}}
\end{sphinxVerbatim}



La estructura es
\begin{itemize}
\item {} 
container
\begin{itemize}
\item {} 
row
\begin{itemize}
\item {} 
col 1

\item {} 
col 2

\end{itemize}

\end{itemize}

\end{itemize}

Normalmente en la web encontramos esta estructura de dos columnas pero donde
una de ellas es una especie de menú o contenido secundario y la otra es el
contenido principal, por lo cual el contenido principal usa mas espacio.

Para poder indicar esto de una forma que se adapte a todas las resoluciones de pantalla bootstrap define que una fila puede estar dividida en 12 "columnas", si no
lo indicamos cada columna toma una cantidad igual de esas 12 columnas, por lo
que si tenemos una columna tomara las 12, si tenemos 2 cada una tomara 6.

En nuestro caso queremos que la segunda sea la columna principal, por lo que le
vamos a indicar que queremos que tome 8 de esas 12 columnas.

\fvset{hllines={, ,}}%
\begin{sphinxVerbatim}[commandchars=\\\{\}]
\PYG{p}{\PYGZlt{}}\PYG{n+nt}{div} \PYG{n+na}{class}\PYG{o}{=}\PYG{l+s}{\PYGZdq{}container\PYGZhy{}fluid cew\PYGZhy{}9\PYGZdq{}}\PYG{p}{\PYGZgt{}}
 \PYG{p}{\PYGZlt{}}\PYG{n+nt}{div} \PYG{n+na}{class}\PYG{o}{=}\PYG{l+s}{\PYGZdq{}row\PYGZdq{}}\PYG{p}{\PYGZgt{}}
  \PYG{p}{\PYGZlt{}}\PYG{n+nt}{div} \PYG{n+na}{class}\PYG{o}{=}\PYG{l+s}{\PYGZdq{}col\PYGZdq{}}\PYG{p}{\PYGZgt{}}
    Columna 1
  \PYG{p}{\PYGZlt{}}\PYG{p}{/}\PYG{n+nt}{div}\PYG{p}{\PYGZgt{}}
  \PYG{p}{\PYGZlt{}}\PYG{n+nt}{div} \PYG{n+na}{class}\PYG{o}{=}\PYG{l+s}{\PYGZdq{}col\PYGZhy{}8\PYGZdq{}}\PYG{p}{\PYGZgt{}}
    Columna 2
  \PYG{p}{\PYGZlt{}}\PYG{p}{/}\PYG{n+nt}{div}\PYG{p}{\PYGZgt{}}
 \PYG{p}{\PYGZlt{}}\PYG{p}{/}\PYG{n+nt}{div}\PYG{p}{\PYGZgt{}}
\PYG{p}{\PYGZlt{}}\PYG{p}{/}\PYG{n+nt}{div}\PYG{p}{\PYGZgt{}}
\end{sphinxVerbatim}



Para hacerlo cambiamos la clase \sphinxtitleref{col} por la clase \sphinxtitleref{col-8} que indica que
queremos que tome 8 de las 12 columnas disponibles.


\section{Dos columnas con cabecera y pie de pagina}
\label{\detokenize{filas-columnas:dos-columnas-con-cabecera-y-pie-de-pagina}}
Esta estructura es la mas común para blogs o artículos, arriba tenemos una
cabecera que ocupa todo el ancho con logo, titulo, navegación y algunas otras
cosas, luego el contenido en si con dos columnas, luego un pie de pagina con
información extra.

\fvset{hllines={, ,}}%
\begin{sphinxVerbatim}[commandchars=\\\{\}]
\PYG{p}{\PYGZlt{}}\PYG{n+nt}{div} \PYG{n+na}{class}\PYG{o}{=}\PYG{l+s}{\PYGZdq{}container\PYGZhy{}fluid cew\PYGZhy{}9\PYGZdq{}}\PYG{p}{\PYGZgt{}}
 \PYG{p}{\PYGZlt{}}\PYG{n+nt}{div} \PYG{n+na}{class}\PYG{o}{=}\PYG{l+s}{\PYGZdq{}row\PYGZdq{}}\PYG{p}{\PYGZgt{}}
  \PYG{p}{\PYGZlt{}}\PYG{n+nt}{div} \PYG{n+na}{class}\PYG{o}{=}\PYG{l+s}{\PYGZdq{}col\PYGZdq{}}\PYG{p}{\PYGZgt{}}
   Cabecera
  \PYG{p}{\PYGZlt{}}\PYG{p}{/}\PYG{n+nt}{div}\PYG{p}{\PYGZgt{}}
 \PYG{p}{\PYGZlt{}}\PYG{p}{/}\PYG{n+nt}{div}\PYG{p}{\PYGZgt{}}

 \PYG{p}{\PYGZlt{}}\PYG{n+nt}{div} \PYG{n+na}{class}\PYG{o}{=}\PYG{l+s}{\PYGZdq{}row\PYGZdq{}}\PYG{p}{\PYGZgt{}}
  \PYG{p}{\PYGZlt{}}\PYG{n+nt}{div} \PYG{n+na}{class}\PYG{o}{=}\PYG{l+s}{\PYGZdq{}col\PYGZdq{}}\PYG{p}{\PYGZgt{}}
    Columna 1
  \PYG{p}{\PYGZlt{}}\PYG{p}{/}\PYG{n+nt}{div}\PYG{p}{\PYGZgt{}}
  \PYG{p}{\PYGZlt{}}\PYG{n+nt}{div} \PYG{n+na}{class}\PYG{o}{=}\PYG{l+s}{\PYGZdq{}col\PYGZhy{}8\PYGZdq{}}\PYG{p}{\PYGZgt{}}
    Columna 2
  \PYG{p}{\PYGZlt{}}\PYG{p}{/}\PYG{n+nt}{div}\PYG{p}{\PYGZgt{}}
 \PYG{p}{\PYGZlt{}}\PYG{p}{/}\PYG{n+nt}{div}\PYG{p}{\PYGZgt{}}

 \PYG{p}{\PYGZlt{}}\PYG{n+nt}{div} \PYG{n+na}{class}\PYG{o}{=}\PYG{l+s}{\PYGZdq{}row\PYGZdq{}}\PYG{p}{\PYGZgt{}}
  \PYG{p}{\PYGZlt{}}\PYG{n+nt}{div} \PYG{n+na}{class}\PYG{o}{=}\PYG{l+s}{\PYGZdq{}col\PYGZdq{}}\PYG{p}{\PYGZgt{}}
   Pie de pagina
  \PYG{p}{\PYGZlt{}}\PYG{p}{/}\PYG{n+nt}{div}\PYG{p}{\PYGZgt{}}
 \PYG{p}{\PYGZlt{}}\PYG{p}{/}\PYG{n+nt}{div}\PYG{p}{\PYGZgt{}}
\PYG{p}{\PYGZlt{}}\PYG{p}{/}\PYG{n+nt}{div}\PYG{p}{\PYGZgt{}}
\end{sphinxVerbatim}



La estructura queda así:
\begin{itemize}
\item {} 
container
\begin{itemize}
\item {} 
row (cabecera)
\begin{itemize}
\item {} 
col (contenido de cabecera)

\end{itemize}

\item {} 
row (cuerpo)
\begin{itemize}
\item {} 
col 1 (contenido secundario)

\item {} 
col 2 (contenido principal)

\end{itemize}

\item {} 
row (pie de pagina)
\begin{itemize}
\item {} 
col (contenido de pie de pagina)

\end{itemize}

\end{itemize}

\end{itemize}


\section{Tortugas hasta el fondo}
\label{\detokenize{filas-columnas:tortugas-hasta-el-fondo}}
\fvset{hllines={, ,}}%
\begin{sphinxVerbatim}[commandchars=\\\{\}]
Un célebre científico dio una conferencia sobre astronomía.
Describió cómo la Tierra gira alrededor del Sol y cómo éste, a su vez,
gira alrededor de un inmenso conjunto de estrellas al que llamamos nuestra galaxia.

Al final de la conferencia, una vieja señora se levantó del fondo de la sala y dijo:

\PYGZhy{} Todo lo que nos ha contado son disparates.
  En realidad, el mundo es una placa plana que se sostiene sobre el caparazón
  de una tortuga gigante

El científico sonrió con suficiencia antes de replicar:

\PYGZhy{} ¿Y sobre qué se sostiene la tortuga?
\PYGZhy{} Sobre el caparazón de otra torguta gigante. \PYGZhy{}respondió la señora
\PYGZhy{} ¿Y qué sostiene a esa otra tortuga? volvió a preguntar el científico.
\PYGZhy{} Se cree usted muy agudo, joven, dijo la anciana,
  pero hay tortugas hasta el fondo.
\end{sphinxVerbatim}

Como hacemos si queremos tener una columna que a su vez tiene su propia estructura?

Podemos tener filas dentro de columnas.

\fvset{hllines={, ,}}%
\begin{sphinxVerbatim}[commandchars=\\\{\}]
\PYG{p}{\PYGZlt{}}\PYG{n+nt}{div} \PYG{n+na}{class}\PYG{o}{=}\PYG{l+s}{\PYGZdq{}container\PYGZhy{}fluid cew\PYGZhy{}9\PYGZdq{}}\PYG{p}{\PYGZgt{}}
 \PYG{p}{\PYGZlt{}}\PYG{n+nt}{div} \PYG{n+na}{class}\PYG{o}{=}\PYG{l+s}{\PYGZdq{}row\PYGZdq{}}\PYG{p}{\PYGZgt{}}
  \PYG{p}{\PYGZlt{}}\PYG{n+nt}{div} \PYG{n+na}{class}\PYG{o}{=}\PYG{l+s}{\PYGZdq{}col\PYGZdq{}}\PYG{p}{\PYGZgt{}}
   Cabecera
  \PYG{p}{\PYGZlt{}}\PYG{p}{/}\PYG{n+nt}{div}\PYG{p}{\PYGZgt{}}
 \PYG{p}{\PYGZlt{}}\PYG{p}{/}\PYG{n+nt}{div}\PYG{p}{\PYGZgt{}}

 \PYG{p}{\PYGZlt{}}\PYG{n+nt}{div} \PYG{n+na}{class}\PYG{o}{=}\PYG{l+s}{\PYGZdq{}row\PYGZdq{}}\PYG{p}{\PYGZgt{}}
  \PYG{p}{\PYGZlt{}}\PYG{n+nt}{div} \PYG{n+na}{class}\PYG{o}{=}\PYG{l+s}{\PYGZdq{}col\PYGZdq{}}\PYG{p}{\PYGZgt{}}
    Columna 1
  \PYG{p}{\PYGZlt{}}\PYG{p}{/}\PYG{n+nt}{div}\PYG{p}{\PYGZgt{}}
  \PYG{p}{\PYGZlt{}}\PYG{n+nt}{div} \PYG{n+na}{class}\PYG{o}{=}\PYG{l+s}{\PYGZdq{}col\PYGZhy{}8\PYGZdq{}}\PYG{p}{\PYGZgt{}}
   \PYG{p}{\PYGZlt{}}\PYG{n+nt}{div} \PYG{n+na}{class}\PYG{o}{=}\PYG{l+s}{\PYGZdq{}row\PYGZdq{}}\PYG{p}{\PYGZgt{}}
    \PYG{p}{\PYGZlt{}}\PYG{n+nt}{div} \PYG{n+na}{class}\PYG{o}{=}\PYG{l+s}{\PYGZdq{}col\PYGZdq{}}\PYG{p}{\PYGZgt{}}
     Columna 2.1.1
    \PYG{p}{\PYGZlt{}}\PYG{p}{/}\PYG{n+nt}{div}\PYG{p}{\PYGZgt{}}
    \PYG{p}{\PYGZlt{}}\PYG{n+nt}{div} \PYG{n+na}{class}\PYG{o}{=}\PYG{l+s}{\PYGZdq{}col\PYGZdq{}}\PYG{p}{\PYGZgt{}}
     Columna 2.1.2
    \PYG{p}{\PYGZlt{}}\PYG{p}{/}\PYG{n+nt}{div}\PYG{p}{\PYGZgt{}}
   \PYG{p}{\PYGZlt{}}\PYG{p}{/}\PYG{n+nt}{div}\PYG{p}{\PYGZgt{}}

   \PYG{p}{\PYGZlt{}}\PYG{n+nt}{div} \PYG{n+na}{class}\PYG{o}{=}\PYG{l+s}{\PYGZdq{}row\PYGZdq{}}\PYG{p}{\PYGZgt{}}
    \PYG{p}{\PYGZlt{}}\PYG{n+nt}{div} \PYG{n+na}{class}\PYG{o}{=}\PYG{l+s}{\PYGZdq{}col\PYGZdq{}}\PYG{p}{\PYGZgt{}}
     Columna 2.2.1
    \PYG{p}{\PYGZlt{}}\PYG{p}{/}\PYG{n+nt}{div}\PYG{p}{\PYGZgt{}}
    \PYG{p}{\PYGZlt{}}\PYG{n+nt}{div} \PYG{n+na}{class}\PYG{o}{=}\PYG{l+s}{\PYGZdq{}col\PYGZdq{}}\PYG{p}{\PYGZgt{}}
     Columna 2.2.2
    \PYG{p}{\PYGZlt{}}\PYG{p}{/}\PYG{n+nt}{div}\PYG{p}{\PYGZgt{}}
    \PYG{p}{\PYGZlt{}}\PYG{n+nt}{div} \PYG{n+na}{class}\PYG{o}{=}\PYG{l+s}{\PYGZdq{}col\PYGZdq{}}\PYG{p}{\PYGZgt{}}
     Columna 2.2.3
    \PYG{p}{\PYGZlt{}}\PYG{p}{/}\PYG{n+nt}{div}\PYG{p}{\PYGZgt{}}
   \PYG{p}{\PYGZlt{}}\PYG{p}{/}\PYG{n+nt}{div}\PYG{p}{\PYGZgt{}}
  \PYG{p}{\PYGZlt{}}\PYG{p}{/}\PYG{n+nt}{div}\PYG{p}{\PYGZgt{}}
 \PYG{p}{\PYGZlt{}}\PYG{p}{/}\PYG{n+nt}{div}\PYG{p}{\PYGZgt{}}

 \PYG{p}{\PYGZlt{}}\PYG{n+nt}{div} \PYG{n+na}{class}\PYG{o}{=}\PYG{l+s}{\PYGZdq{}row\PYGZdq{}}\PYG{p}{\PYGZgt{}}
  \PYG{p}{\PYGZlt{}}\PYG{n+nt}{div} \PYG{n+na}{class}\PYG{o}{=}\PYG{l+s}{\PYGZdq{}col\PYGZdq{}}\PYG{p}{\PYGZgt{}}
   Pie de pagina
  \PYG{p}{\PYGZlt{}}\PYG{p}{/}\PYG{n+nt}{div}\PYG{p}{\PYGZgt{}}
 \PYG{p}{\PYGZlt{}}\PYG{p}{/}\PYG{n+nt}{div}\PYG{p}{\PYGZgt{}}
\PYG{p}{\PYGZlt{}}\PYG{p}{/}\PYG{n+nt}{div}\PYG{p}{\PYGZgt{}}
\end{sphinxVerbatim}



La estructura queda así:
\begin{itemize}
\item {} 
container
\begin{itemize}
\item {} 
row (cabecera)
\begin{itemize}
\item {} 
col (contenido de cabecera)

\end{itemize}

\item {} 
row (cuerpo)
\begin{itemize}
\item {} 
col 1 (contenido secundario)

\item {} 
col 2 (contenido principal)

\end{itemize}
\begin{itemize}
\item {} 
row 2.1
\begin{itemize}
\item {} 
col 2.1.1

\item {} 
col 2.1.2

\end{itemize}

\item {} 
row 2.2
\begin{itemize}
\item {} 
col 2.2.1

\item {} 
col 2.2.2

\item {} 
col 2.2.3

\end{itemize}

\end{itemize}

\item {} 
row (pie de pagina)
\begin{itemize}
\item {} 
col (contenido de pie de pagina)

\end{itemize}

\end{itemize}

\end{itemize}


\section{Resoluciones de pantalla}
\label{\detokenize{filas-columnas:resoluciones-de-pantalla}}
Con este nuevo conocimiento creamos una pagina con una estructura perfecta
para la pantalla que estamos usando y orgullosamente la compartimos con gente
para que la vean, el primer mensaje que recibimos es:

\fvset{hllines={, ,}}%
\begin{sphinxVerbatim}[commandchars=\\\{\}]
\PYG{o}{\PYGZhy{}} \PYG{n}{No} \PYG{n}{se} \PYG{n}{ve} \PYG{n}{bien} \PYG{n}{en} \PYG{n}{mi} \PYG{n}{celular}\PYG{p}{,} \PYG{n}{todo} \PYG{n}{es} \PYG{n}{muy} \PYG{n}{chico} \PYG{n}{y} \PYG{n}{con} \PYG{n}{poco} \PYG{n}{espacio}
\end{sphinxVerbatim}

No pensamos en que la pagina va a ser vista por personas usando un smartphone
viejo, uno de ultima generación, una tablet, una laptop chica, una grande,
una PC y la pantalla de un diseñador con mas pixeles de los que podemos contar.

Como hacemos para que nuestra pagina se adapte a la resolución de cualquier
dispositivo que quiera visitar nuestra pagina?

Una idea seria ver cuales son las resoluciones mas comunes en pixeles y aplicar
reglas para esos, si bien eso funcionaba en la prehistoria de la web (esto es,
hace 10 años), ya no es así, veamos algunas de las resoluciones mas comunes
disponibles actualmente:

\begin{figure}[htbp]
\centering

\noindent\sphinxincludegraphics{{resoluciones}.png}
\end{figure}

Intentando hacer esto manejable entran en juego los \sphinxtitleref{puntos de corte} que
mencionamos al principio del articulo.

Los puntos de corte son limites de resolución que agrupan a la resolución de los
dispositivos en 5 grandes grupos, similares a los de la ropa:
\begin{itemize}
\item {} 
xs: Extra Small
\begin{itemize}
\item {} 
Extra pequeño, dispositivos con menos de 576 pixeles de ancho

\end{itemize}

\item {} 
sm: Small
\begin{itemize}
\item {} 
Pequeño, dispositivos con menos de 768 pixeles de ancho

\end{itemize}

\item {} 
md: Medium
\begin{itemize}
\item {} 
Medio, dispositivos con menos de 992 pixeles de ancho

\end{itemize}

\item {} 
lg: Large
\begin{itemize}
\item {} 
Grande, dispositivos con menos de 1200 pixeles de ancho

\end{itemize}

\item {} 
xl: Extra Large
\begin{itemize}
\item {} 
Extra Grande, dispositivos con 1200 pixeles de ancho o mas

\end{itemize}

\end{itemize}

Como los usamos? indicando el ancho de la columna con el tipo de dispositivo
mínimo para el cual el tamaño aplica, entonces podemos decir algo como:

"Esta columna ocupa 12 columnas si es una resolución xs y 6 si no"

lo expresamos en clases: \sphinxtitleref{.col-12 .col-sm-6}

O mas complejas como

"Esta columna ocupa 12 columnas si es una resolución xs, 8 si es una resolución sm y 6 si no"

lo expresamos en clases: \sphinxtitleref{.col-12 .col-sm-8 .col-md-6}

Control completo si especificamos todas:

"Esta columna ocupa 12 columnas si es una resolución xs, 8 si es una resolución sm, 6 si es una resolución md, 4 si es lg y 2 si es xl"

lo expresamos en clases: \sphinxtitleref{.col-12 .col-sm-8 .col-md-6 .col-lg-4 .col-xl-2}

Bootstrap va a buscar el grupo mas cercano a la resolución actual y aplicar esa
regla, si nuestro dispositivo tiene una resolución de 1000 pixeles y hay una
regla para md (\textless{} 992px) va a aplicar esa, sino va a buscar la regla sm y sino
la xs.

Probemos un ejemplo:

\fvset{hllines={, ,}}%
\begin{sphinxVerbatim}[commandchars=\\\{\}]
\PYG{p}{\PYGZlt{}}\PYG{n+nt}{div} \PYG{n+na}{class}\PYG{o}{=}\PYG{l+s}{\PYGZdq{}container\PYGZhy{}fluid cew\PYGZhy{}9\PYGZdq{}}\PYG{p}{\PYGZgt{}}
 \PYG{p}{\PYGZlt{}}\PYG{n+nt}{div} \PYG{n+na}{class}\PYG{o}{=}\PYG{l+s}{\PYGZdq{}row\PYGZdq{}}\PYG{p}{\PYGZgt{}}
  \PYG{p}{\PYGZlt{}}\PYG{n+nt}{div} \PYG{n+na}{class}\PYG{o}{=}\PYG{l+s}{\PYGZdq{}col\PYGZhy{}12 col\PYGZhy{}sm\PYGZhy{}6 col\PYGZhy{}md\PYGZhy{}3\PYGZdq{}}\PYG{p}{\PYGZgt{}}
       Reglas: col\PYGZhy{}12 col\PYGZhy{}sm\PYGZhy{}6 col\PYGZhy{}md\PYGZhy{}3
  \PYG{p}{\PYGZlt{}}\PYG{p}{/}\PYG{n+nt}{div}\PYG{p}{\PYGZgt{}}
  \PYG{p}{\PYGZlt{}}\PYG{n+nt}{div} \PYG{n+na}{class}\PYG{o}{=}\PYG{l+s}{\PYGZdq{}col\PYGZhy{}12 col\PYGZhy{}sm\PYGZhy{}6 col\PYGZhy{}md\PYGZhy{}6\PYGZdq{}}\PYG{p}{\PYGZgt{}}
       Reglas: col\PYGZhy{}12 col\PYGZhy{}sm\PYGZhy{}6 col\PYGZhy{}md\PYGZhy{}6
  \PYG{p}{\PYGZlt{}}\PYG{p}{/}\PYG{n+nt}{div}\PYG{p}{\PYGZgt{}}
  \PYG{p}{\PYGZlt{}}\PYG{n+nt}{div} \PYG{n+na}{class}\PYG{o}{=}\PYG{l+s}{\PYGZdq{}col\PYGZhy{}12 col\PYGZhy{}sm\PYGZhy{}12 col\PYGZhy{}md\PYGZhy{}3\PYGZdq{}}\PYG{p}{\PYGZgt{}}
        Reglas: col\PYGZhy{}12 col\PYGZhy{}sm\PYGZhy{}12 col\PYGZhy{}md\PYGZhy{}3
  \PYG{p}{\PYGZlt{}}\PYG{p}{/}\PYG{n+nt}{div}\PYG{p}{\PYGZgt{}}
 \PYG{p}{\PYGZlt{}}\PYG{p}{/}\PYG{n+nt}{div}\PYG{p}{\PYGZgt{}}
\PYG{p}{\PYGZlt{}}\PYG{p}{/}\PYG{n+nt}{div}\PYG{p}{\PYGZgt{}}
\end{sphinxVerbatim}



Si estas viendo esto en una PC o una laptop probablemente la regla que aplique
sea md, donde vas a ver 3 columnas, la del medio el doble de ancho que las
laterales.

Pero como probamos para distintas resoluciones sin tener disponibles dispositivos
para cada punto de corte?

En Firefox en el menú \sphinxtitleref{Herramientas \textgreater{} Desarrollador web \textgreater{} Vista de diseño adaptable} o el atajo de teclado \sphinxtitleref{Ctrl+Shift+M}

\begin{figure}[htbp]
\centering

\noindent\sphinxincludegraphics{{firefox-ctrl-shift-m}.png}
\end{figure}

En Chrome en el menú \sphinxtitleref{Menú \textgreater{} Mas Herramientas \textgreater{} Herramientas para desarrolladores} y al abrirse seleccionamos el segundo icono de arriba a la derecha o el atajo de teclado \sphinxtitleref{Ctrl+Shift+M}

\begin{figure}[htbp]
\centering

\noindent\sphinxincludegraphics{{chrome-ctrl-shift-m}.png}
\end{figure}

Esto va a abrir una herramienta que nos permite simular distintas resoluciones
y ver como la pagina se adapta a los cambios, lo único que vamos a usar ahora
es cambiar la resolución manualmente o elegir un dispositivo predeterminado.

Así es como se ve en mi computadora en firefox:

\begin{figure}[htbp]
\centering
\capstart

\noindent\sphinxincludegraphics{{ff-xs}.png}
\caption{Resolución xs: 320x480}\label{\detokenize{filas-columnas:id1}}\end{figure}

\begin{figure}[htbp]
\centering
\capstart

\noindent\sphinxincludegraphics{{ff-sm}.png}
\caption{Resolución sm: 760x480}\label{\detokenize{filas-columnas:id2}}\end{figure}

\begin{figure}[htbp]
\centering
\capstart

\noindent\sphinxincludegraphics{{ff-md}.png}
\caption{Resolución md: 800x480}\label{\detokenize{filas-columnas:id3}}\end{figure}

Te recomiendo que lo pruebes vos, actividad extra, navega por paginas que visites
frecuentemente con esta herramienta abierta, fijate como se adapta (o no) a
la resolución que elegiste.


\chapter{Herramientas de Desarrollo en Firefox y Chrome: Inspector HTML y CSS}
\label{\detokenize{herramientas-desarrollador::doc}}\label{\detokenize{herramientas-desarrollador:herramientas-de-desarrollo-en-firefox-y-chrome-inspector-html-y-css}}
Esta sección es mas simple con un video así que acá esta:



Link: \sphinxurl{https://www.youtube.com/watch?v=aEndBQ3ZikY}


\chapter{Mas componentes}
\label{\detokenize{mas-componentes:mas-componentes}}\label{\detokenize{mas-componentes::doc}}
Ahora que sabemos usar las clases de layout de bootstrap exploremos algunos
componentes mas.




\section{Breadcrumbs (migas de pan)}
\label{\detokenize{mas-componentes:breadcrumbs-migas-de-pan}}
\fvset{hllines={, ,}}%
\begin{sphinxVerbatim}[commandchars=\\\{\}]
\PYG{p}{\PYGZlt{}}\PYG{n+nt}{nav} \PYG{n+na}{aria\PYGZhy{}label}\PYG{o}{=}\PYG{l+s}{\PYGZdq{}breadcrumb\PYGZdq{}}\PYG{p}{\PYGZgt{}}
  \PYG{p}{\PYGZlt{}}\PYG{n+nt}{ol} \PYG{n+na}{class}\PYG{o}{=}\PYG{l+s}{\PYGZdq{}breadcrumb\PYGZdq{}}\PYG{p}{\PYGZgt{}}
    \PYG{p}{\PYGZlt{}}\PYG{n+nt}{li} \PYG{n+na}{class}\PYG{o}{=}\PYG{l+s}{\PYGZdq{}breadcrumb\PYGZhy{}item active\PYGZdq{}} \PYG{n+na}{aria\PYGZhy{}current}\PYG{o}{=}\PYG{l+s}{\PYGZdq{}page\PYGZdq{}}\PYG{p}{\PYGZgt{}}Principal\PYG{p}{\PYGZlt{}}\PYG{p}{/}\PYG{n+nt}{li}\PYG{p}{\PYGZgt{}}
  \PYG{p}{\PYGZlt{}}\PYG{p}{/}\PYG{n+nt}{ol}\PYG{p}{\PYGZgt{}}
\PYG{p}{\PYGZlt{}}\PYG{p}{/}\PYG{n+nt}{nav}\PYG{p}{\PYGZgt{}}
\end{sphinxVerbatim}



\fvset{hllines={, ,}}%
\begin{sphinxVerbatim}[commandchars=\\\{\}]
\PYG{p}{\PYGZlt{}}\PYG{n+nt}{nav} \PYG{n+na}{aria\PYGZhy{}label}\PYG{o}{=}\PYG{l+s}{\PYGZdq{}breadcrumb\PYGZdq{}}\PYG{p}{\PYGZgt{}}
  \PYG{p}{\PYGZlt{}}\PYG{n+nt}{ol} \PYG{n+na}{class}\PYG{o}{=}\PYG{l+s}{\PYGZdq{}breadcrumb\PYGZdq{}}\PYG{p}{\PYGZgt{}}
    \PYG{p}{\PYGZlt{}}\PYG{n+nt}{li} \PYG{n+na}{class}\PYG{o}{=}\PYG{l+s}{\PYGZdq{}breadcrumb\PYGZhy{}item\PYGZdq{}}\PYG{p}{\PYGZgt{}}\PYG{p}{\PYGZlt{}}\PYG{n+nt}{a} \PYG{n+na}{href}\PYG{o}{=}\PYG{l+s}{\PYGZdq{}\PYGZsh{}\PYGZdq{}}\PYG{p}{\PYGZgt{}}Principal\PYG{p}{\PYGZlt{}}\PYG{p}{/}\PYG{n+nt}{a}\PYG{p}{\PYGZgt{}}\PYG{p}{\PYGZlt{}}\PYG{p}{/}\PYG{n+nt}{li}\PYG{p}{\PYGZgt{}}
    \PYG{p}{\PYGZlt{}}\PYG{n+nt}{li} \PYG{n+na}{class}\PYG{o}{=}\PYG{l+s}{\PYGZdq{}breadcrumb\PYGZhy{}item active\PYGZdq{}} \PYG{n+na}{aria\PYGZhy{}current}\PYG{o}{=}\PYG{l+s}{\PYGZdq{}page\PYGZdq{}}\PYG{p}{\PYGZgt{}}Articulos\PYG{p}{\PYGZlt{}}\PYG{p}{/}\PYG{n+nt}{li}\PYG{p}{\PYGZgt{}}
  \PYG{p}{\PYGZlt{}}\PYG{p}{/}\PYG{n+nt}{ol}\PYG{p}{\PYGZgt{}}
\PYG{p}{\PYGZlt{}}\PYG{p}{/}\PYG{n+nt}{nav}\PYG{p}{\PYGZgt{}}
\end{sphinxVerbatim}



\fvset{hllines={, ,}}%
\begin{sphinxVerbatim}[commandchars=\\\{\}]
\PYG{p}{\PYGZlt{}}\PYG{n+nt}{nav} \PYG{n+na}{aria\PYGZhy{}label}\PYG{o}{=}\PYG{l+s}{\PYGZdq{}breadcrumb\PYGZdq{}}\PYG{p}{\PYGZgt{}}
  \PYG{p}{\PYGZlt{}}\PYG{n+nt}{ol} \PYG{n+na}{class}\PYG{o}{=}\PYG{l+s}{\PYGZdq{}breadcrumb\PYGZdq{}}\PYG{p}{\PYGZgt{}}
    \PYG{p}{\PYGZlt{}}\PYG{n+nt}{li} \PYG{n+na}{class}\PYG{o}{=}\PYG{l+s}{\PYGZdq{}breadcrumb\PYGZhy{}item\PYGZdq{}}\PYG{p}{\PYGZgt{}}\PYG{p}{\PYGZlt{}}\PYG{n+nt}{a} \PYG{n+na}{href}\PYG{o}{=}\PYG{l+s}{\PYGZdq{}\PYGZsh{}\PYGZdq{}}\PYG{p}{\PYGZgt{}}Principal\PYG{p}{\PYGZlt{}}\PYG{p}{/}\PYG{n+nt}{a}\PYG{p}{\PYGZgt{}}\PYG{p}{\PYGZlt{}}\PYG{p}{/}\PYG{n+nt}{li}\PYG{p}{\PYGZgt{}}
    \PYG{p}{\PYGZlt{}}\PYG{n+nt}{li} \PYG{n+na}{class}\PYG{o}{=}\PYG{l+s}{\PYGZdq{}breadcrumb\PYGZhy{}item\PYGZdq{}}\PYG{p}{\PYGZgt{}}\PYG{p}{\PYGZlt{}}\PYG{n+nt}{a} \PYG{n+na}{href}\PYG{o}{=}\PYG{l+s}{\PYGZdq{}\PYGZsh{}\PYGZdq{}}\PYG{p}{\PYGZgt{}}Articulos\PYG{p}{\PYGZlt{}}\PYG{p}{/}\PYG{n+nt}{a}\PYG{p}{\PYGZgt{}}\PYG{p}{\PYGZlt{}}\PYG{p}{/}\PYG{n+nt}{li}\PYG{p}{\PYGZgt{}}
    \PYG{p}{\PYGZlt{}}\PYG{n+nt}{li} \PYG{n+na}{class}\PYG{o}{=}\PYG{l+s}{\PYGZdq{}breadcrumb\PYGZhy{}item active\PYGZdq{}} \PYG{n+na}{aria\PYGZhy{}current}\PYG{o}{=}\PYG{l+s}{\PYGZdq{}page\PYGZdq{}}\PYG{p}{\PYGZgt{}}Articulo I\PYG{p}{\PYGZlt{}}\PYG{p}{/}\PYG{n+nt}{li}\PYG{p}{\PYGZgt{}}
  \PYG{p}{\PYGZlt{}}\PYG{p}{/}\PYG{n+nt}{ol}\PYG{p}{\PYGZgt{}}
\PYG{p}{\PYGZlt{}}\PYG{p}{/}\PYG{n+nt}{nav}\PYG{p}{\PYGZgt{}}
\end{sphinxVerbatim}




\section{Botones}
\label{\detokenize{mas-componentes:botones}}

\subsection{Estándar}
\label{\detokenize{mas-componentes:estandar}}
\fvset{hllines={, ,}}%
\begin{sphinxVerbatim}[commandchars=\\\{\}]
\PYG{p}{\PYGZlt{}}\PYG{n+nt}{button} \PYG{n+na}{type}\PYG{o}{=}\PYG{l+s}{\PYGZdq{}button\PYGZdq{}} \PYG{n+na}{class}\PYG{o}{=}\PYG{l+s}{\PYGZdq{}btn btn\PYGZhy{}primary\PYGZdq{}}\PYG{p}{\PYGZgt{}}Primario\PYG{p}{\PYGZlt{}}\PYG{p}{/}\PYG{n+nt}{button}\PYG{p}{\PYGZgt{}}
\PYG{p}{\PYGZlt{}}\PYG{n+nt}{button} \PYG{n+na}{type}\PYG{o}{=}\PYG{l+s}{\PYGZdq{}button\PYGZdq{}} \PYG{n+na}{class}\PYG{o}{=}\PYG{l+s}{\PYGZdq{}btn btn\PYGZhy{}secondary\PYGZdq{}}\PYG{p}{\PYGZgt{}}Secundario\PYG{p}{\PYGZlt{}}\PYG{p}{/}\PYG{n+nt}{button}\PYG{p}{\PYGZgt{}}
\PYG{p}{\PYGZlt{}}\PYG{n+nt}{button} \PYG{n+na}{type}\PYG{o}{=}\PYG{l+s}{\PYGZdq{}button\PYGZdq{}} \PYG{n+na}{class}\PYG{o}{=}\PYG{l+s}{\PYGZdq{}btn btn\PYGZhy{}success\PYGZdq{}}\PYG{p}{\PYGZgt{}}Exito\PYG{p}{\PYGZlt{}}\PYG{p}{/}\PYG{n+nt}{button}\PYG{p}{\PYGZgt{}}
\PYG{p}{\PYGZlt{}}\PYG{n+nt}{button} \PYG{n+na}{type}\PYG{o}{=}\PYG{l+s}{\PYGZdq{}button\PYGZdq{}} \PYG{n+na}{class}\PYG{o}{=}\PYG{l+s}{\PYGZdq{}btn btn\PYGZhy{}danger\PYGZdq{}}\PYG{p}{\PYGZgt{}}Peligro\PYG{p}{\PYGZlt{}}\PYG{p}{/}\PYG{n+nt}{button}\PYG{p}{\PYGZgt{}}
\PYG{p}{\PYGZlt{}}\PYG{n+nt}{button} \PYG{n+na}{type}\PYG{o}{=}\PYG{l+s}{\PYGZdq{}button\PYGZdq{}} \PYG{n+na}{class}\PYG{o}{=}\PYG{l+s}{\PYGZdq{}btn btn\PYGZhy{}warning\PYGZdq{}}\PYG{p}{\PYGZgt{}}Advertencia\PYG{p}{\PYGZlt{}}\PYG{p}{/}\PYG{n+nt}{button}\PYG{p}{\PYGZgt{}}
\PYG{p}{\PYGZlt{}}\PYG{n+nt}{button} \PYG{n+na}{type}\PYG{o}{=}\PYG{l+s}{\PYGZdq{}button\PYGZdq{}} \PYG{n+na}{class}\PYG{o}{=}\PYG{l+s}{\PYGZdq{}btn btn\PYGZhy{}info\PYGZdq{}}\PYG{p}{\PYGZgt{}}Informacion\PYG{p}{\PYGZlt{}}\PYG{p}{/}\PYG{n+nt}{button}\PYG{p}{\PYGZgt{}}
\PYG{p}{\PYGZlt{}}\PYG{n+nt}{button} \PYG{n+na}{type}\PYG{o}{=}\PYG{l+s}{\PYGZdq{}button\PYGZdq{}} \PYG{n+na}{class}\PYG{o}{=}\PYG{l+s}{\PYGZdq{}btn btn\PYGZhy{}light\PYGZdq{}}\PYG{p}{\PYGZgt{}}Claro\PYG{p}{\PYGZlt{}}\PYG{p}{/}\PYG{n+nt}{button}\PYG{p}{\PYGZgt{}}
\PYG{p}{\PYGZlt{}}\PYG{n+nt}{button} \PYG{n+na}{type}\PYG{o}{=}\PYG{l+s}{\PYGZdq{}button\PYGZdq{}} \PYG{n+na}{class}\PYG{o}{=}\PYG{l+s}{\PYGZdq{}btn btn\PYGZhy{}dark\PYGZdq{}}\PYG{p}{\PYGZgt{}}Oscuro\PYG{p}{\PYGZlt{}}\PYG{p}{/}\PYG{n+nt}{button}\PYG{p}{\PYGZgt{}}

\PYG{p}{\PYGZlt{}}\PYG{n+nt}{button} \PYG{n+na}{type}\PYG{o}{=}\PYG{l+s}{\PYGZdq{}button\PYGZdq{}} \PYG{n+na}{class}\PYG{o}{=}\PYG{l+s}{\PYGZdq{}btn btn\PYGZhy{}link\PYGZdq{}}\PYG{p}{\PYGZgt{}}Link\PYG{p}{\PYGZlt{}}\PYG{p}{/}\PYG{n+nt}{button}\PYG{p}{\PYGZgt{}}
\end{sphinxVerbatim}




\subsection{Bordes}
\label{\detokenize{mas-componentes:bordes}}
\fvset{hllines={, ,}}%
\begin{sphinxVerbatim}[commandchars=\\\{\}]
\PYG{p}{\PYGZlt{}}\PYG{n+nt}{button} \PYG{n+na}{type}\PYG{o}{=}\PYG{l+s}{\PYGZdq{}button\PYGZdq{}} \PYG{n+na}{class}\PYG{o}{=}\PYG{l+s}{\PYGZdq{}btn btn\PYGZhy{}outline\PYGZhy{}primary\PYGZdq{}}\PYG{p}{\PYGZgt{}}Primario\PYG{p}{\PYGZlt{}}\PYG{p}{/}\PYG{n+nt}{button}\PYG{p}{\PYGZgt{}}
\PYG{p}{\PYGZlt{}}\PYG{n+nt}{button} \PYG{n+na}{type}\PYG{o}{=}\PYG{l+s}{\PYGZdq{}button\PYGZdq{}} \PYG{n+na}{class}\PYG{o}{=}\PYG{l+s}{\PYGZdq{}btn btn\PYGZhy{}outline\PYGZhy{}secondary\PYGZdq{}}\PYG{p}{\PYGZgt{}}Secundario\PYG{p}{\PYGZlt{}}\PYG{p}{/}\PYG{n+nt}{button}\PYG{p}{\PYGZgt{}}
\PYG{p}{\PYGZlt{}}\PYG{n+nt}{button} \PYG{n+na}{type}\PYG{o}{=}\PYG{l+s}{\PYGZdq{}button\PYGZdq{}} \PYG{n+na}{class}\PYG{o}{=}\PYG{l+s}{\PYGZdq{}btn btn\PYGZhy{}outline\PYGZhy{}success\PYGZdq{}}\PYG{p}{\PYGZgt{}}Exito\PYG{p}{\PYGZlt{}}\PYG{p}{/}\PYG{n+nt}{button}\PYG{p}{\PYGZgt{}}
\PYG{p}{\PYGZlt{}}\PYG{n+nt}{button} \PYG{n+na}{type}\PYG{o}{=}\PYG{l+s}{\PYGZdq{}button\PYGZdq{}} \PYG{n+na}{class}\PYG{o}{=}\PYG{l+s}{\PYGZdq{}btn btn\PYGZhy{}outline\PYGZhy{}danger\PYGZdq{}}\PYG{p}{\PYGZgt{}}Peligro\PYG{p}{\PYGZlt{}}\PYG{p}{/}\PYG{n+nt}{button}\PYG{p}{\PYGZgt{}}
\PYG{p}{\PYGZlt{}}\PYG{n+nt}{button} \PYG{n+na}{type}\PYG{o}{=}\PYG{l+s}{\PYGZdq{}button\PYGZdq{}} \PYG{n+na}{class}\PYG{o}{=}\PYG{l+s}{\PYGZdq{}btn btn\PYGZhy{}outline\PYGZhy{}warning\PYGZdq{}}\PYG{p}{\PYGZgt{}}Advertencia\PYG{p}{\PYGZlt{}}\PYG{p}{/}\PYG{n+nt}{button}\PYG{p}{\PYGZgt{}}
\PYG{p}{\PYGZlt{}}\PYG{n+nt}{button} \PYG{n+na}{type}\PYG{o}{=}\PYG{l+s}{\PYGZdq{}button\PYGZdq{}} \PYG{n+na}{class}\PYG{o}{=}\PYG{l+s}{\PYGZdq{}btn btn\PYGZhy{}outline\PYGZhy{}info\PYGZdq{}}\PYG{p}{\PYGZgt{}}Informacion\PYG{p}{\PYGZlt{}}\PYG{p}{/}\PYG{n+nt}{button}\PYG{p}{\PYGZgt{}}
\PYG{p}{\PYGZlt{}}\PYG{n+nt}{button} \PYG{n+na}{type}\PYG{o}{=}\PYG{l+s}{\PYGZdq{}button\PYGZdq{}} \PYG{n+na}{class}\PYG{o}{=}\PYG{l+s}{\PYGZdq{}btn btn\PYGZhy{}outline\PYGZhy{}light\PYGZdq{}}\PYG{p}{\PYGZgt{}}Claro\PYG{p}{\PYGZlt{}}\PYG{p}{/}\PYG{n+nt}{button}\PYG{p}{\PYGZgt{}}
\PYG{p}{\PYGZlt{}}\PYG{n+nt}{button} \PYG{n+na}{type}\PYG{o}{=}\PYG{l+s}{\PYGZdq{}button\PYGZdq{}} \PYG{n+na}{class}\PYG{o}{=}\PYG{l+s}{\PYGZdq{}btn btn\PYGZhy{}outline\PYGZhy{}dark\PYGZdq{}}\PYG{p}{\PYGZgt{}}Oscuro\PYG{p}{\PYGZlt{}}\PYG{p}{/}\PYG{n+nt}{button}\PYG{p}{\PYGZgt{}}
\end{sphinxVerbatim}




\subsection{Tamaños}
\label{\detokenize{mas-componentes:tamanos}}
\fvset{hllines={, ,}}%
\begin{sphinxVerbatim}[commandchars=\\\{\}]
\PYG{p}{\PYGZlt{}}\PYG{n+nt}{button} \PYG{n+na}{type}\PYG{o}{=}\PYG{l+s}{\PYGZdq{}button\PYGZdq{}} \PYG{n+na}{class}\PYG{o}{=}\PYG{l+s}{\PYGZdq{}btn btn\PYGZhy{}primary btn\PYGZhy{}lg\PYGZdq{}}\PYG{p}{\PYGZgt{}}Grande\PYG{p}{\PYGZlt{}}\PYG{p}{/}\PYG{n+nt}{button}\PYG{p}{\PYGZgt{}}
\PYG{p}{\PYGZlt{}}\PYG{n+nt}{button} \PYG{n+na}{type}\PYG{o}{=}\PYG{l+s}{\PYGZdq{}button\PYGZdq{}} \PYG{n+na}{class}\PYG{o}{=}\PYG{l+s}{\PYGZdq{}btn btn\PYGZhy{}secondary btn\PYGZhy{}lg\PYGZdq{}}\PYG{p}{\PYGZgt{}}Grande\PYG{p}{\PYGZlt{}}\PYG{p}{/}\PYG{n+nt}{button}\PYG{p}{\PYGZgt{}}

\PYG{p}{\PYGZlt{}}\PYG{n+nt}{button} \PYG{n+na}{type}\PYG{o}{=}\PYG{l+s}{\PYGZdq{}button\PYGZdq{}} \PYG{n+na}{class}\PYG{o}{=}\PYG{l+s}{\PYGZdq{}btn btn\PYGZhy{}primary btn\PYGZhy{}sm\PYGZdq{}}\PYG{p}{\PYGZgt{}}Pequeño\PYG{p}{\PYGZlt{}}\PYG{p}{/}\PYG{n+nt}{button}\PYG{p}{\PYGZgt{}}
\PYG{p}{\PYGZlt{}}\PYG{n+nt}{button} \PYG{n+na}{type}\PYG{o}{=}\PYG{l+s}{\PYGZdq{}button\PYGZdq{}} \PYG{n+na}{class}\PYG{o}{=}\PYG{l+s}{\PYGZdq{}btn btn\PYGZhy{}secondary btn\PYGZhy{}sm\PYGZdq{}}\PYG{p}{\PYGZgt{}}Pequeño\PYG{p}{\PYGZlt{}}\PYG{p}{/}\PYG{n+nt}{button}\PYG{p}{\PYGZgt{}}
\end{sphinxVerbatim}




\subsection{Estados}
\label{\detokenize{mas-componentes:estados}}

\subsubsection{Activo}
\label{\detokenize{mas-componentes:activo}}
\fvset{hllines={, ,}}%
\begin{sphinxVerbatim}[commandchars=\\\{\}]
\PYG{p}{\PYGZlt{}}\PYG{n+nt}{a} \PYG{n+na}{href}\PYG{o}{=}\PYG{l+s}{\PYGZdq{}\PYGZsh{}\PYGZdq{}} \PYG{n+na}{class}\PYG{o}{=}\PYG{l+s}{\PYGZdq{}btn btn\PYGZhy{}primary btn\PYGZhy{}lg active\PYGZdq{}} \PYG{n+na}{role}\PYG{o}{=}\PYG{l+s}{\PYGZdq{}button\PYGZdq{}} \PYG{n+na}{aria\PYGZhy{}pressed}\PYG{o}{=}\PYG{l+s}{\PYGZdq{}true\PYGZdq{}}\PYG{p}{\PYGZgt{}}Link Primario\PYG{p}{\PYGZlt{}}\PYG{p}{/}\PYG{n+nt}{a}\PYG{p}{\PYGZgt{}}
\PYG{p}{\PYGZlt{}}\PYG{n+nt}{a} \PYG{n+na}{href}\PYG{o}{=}\PYG{l+s}{\PYGZdq{}\PYGZsh{}\PYGZdq{}} \PYG{n+na}{class}\PYG{o}{=}\PYG{l+s}{\PYGZdq{}btn btn\PYGZhy{}secondary btn\PYGZhy{}lg active\PYGZdq{}} \PYG{n+na}{role}\PYG{o}{=}\PYG{l+s}{\PYGZdq{}button\PYGZdq{}} \PYG{n+na}{aria\PYGZhy{}pressed}\PYG{o}{=}\PYG{l+s}{\PYGZdq{}true\PYGZdq{}}\PYG{p}{\PYGZgt{}}Link Secundario\PYG{p}{\PYGZlt{}}\PYG{p}{/}\PYG{n+nt}{a}\PYG{p}{\PYGZgt{}}
\end{sphinxVerbatim}




\subsubsection{Inactivo}
\label{\detokenize{mas-componentes:inactivo}}
\fvset{hllines={, ,}}%
\begin{sphinxVerbatim}[commandchars=\\\{\}]
\PYG{p}{\PYGZlt{}}\PYG{n+nt}{button} \PYG{n+na}{type}\PYG{o}{=}\PYG{l+s}{\PYGZdq{}button\PYGZdq{}} \PYG{n+na}{class}\PYG{o}{=}\PYG{l+s}{\PYGZdq{}btn btn\PYGZhy{}lg btn\PYGZhy{}primary\PYGZdq{}} \PYG{n+na}{disabled}\PYG{p}{\PYGZgt{}}Primario\PYG{p}{\PYGZlt{}}\PYG{p}{/}\PYG{n+nt}{button}\PYG{p}{\PYGZgt{}}
\PYG{p}{\PYGZlt{}}\PYG{n+nt}{button} \PYG{n+na}{type}\PYG{o}{=}\PYG{l+s}{\PYGZdq{}button\PYGZdq{}} \PYG{n+na}{class}\PYG{o}{=}\PYG{l+s}{\PYGZdq{}btn btn\PYGZhy{}secondary btn\PYGZhy{}lg\PYGZdq{}} \PYG{n+na}{disabled}\PYG{p}{\PYGZgt{}}Secundario\PYG{p}{\PYGZlt{}}\PYG{p}{/}\PYG{n+nt}{button}\PYG{p}{\PYGZgt{}}
\end{sphinxVerbatim}




\section{Grupos de botones}
\label{\detokenize{mas-componentes:grupos-de-botones}}
\fvset{hllines={, ,}}%
\begin{sphinxVerbatim}[commandchars=\\\{\}]
\PYG{p}{\PYGZlt{}}\PYG{n+nt}{div} \PYG{n+na}{class}\PYG{o}{=}\PYG{l+s}{\PYGZdq{}btn\PYGZhy{}group\PYGZdq{}} \PYG{n+na}{role}\PYG{o}{=}\PYG{l+s}{\PYGZdq{}group\PYGZdq{}} \PYG{n+na}{aria\PYGZhy{}label}\PYG{o}{=}\PYG{l+s}{\PYGZdq{}Basic example\PYGZdq{}}\PYG{p}{\PYGZgt{}}
  \PYG{p}{\PYGZlt{}}\PYG{n+nt}{button} \PYG{n+na}{type}\PYG{o}{=}\PYG{l+s}{\PYGZdq{}button\PYGZdq{}} \PYG{n+na}{class}\PYG{o}{=}\PYG{l+s}{\PYGZdq{}btn btn\PYGZhy{}secondary\PYGZdq{}}\PYG{p}{\PYGZgt{}}Izquierda\PYG{p}{\PYGZlt{}}\PYG{p}{/}\PYG{n+nt}{button}\PYG{p}{\PYGZgt{}}
  \PYG{p}{\PYGZlt{}}\PYG{n+nt}{button} \PYG{n+na}{type}\PYG{o}{=}\PYG{l+s}{\PYGZdq{}button\PYGZdq{}} \PYG{n+na}{class}\PYG{o}{=}\PYG{l+s}{\PYGZdq{}btn btn\PYGZhy{}secondary\PYGZdq{}}\PYG{p}{\PYGZgt{}}Medio\PYG{p}{\PYGZlt{}}\PYG{p}{/}\PYG{n+nt}{button}\PYG{p}{\PYGZgt{}}
  \PYG{p}{\PYGZlt{}}\PYG{n+nt}{button} \PYG{n+na}{type}\PYG{o}{=}\PYG{l+s}{\PYGZdq{}button\PYGZdq{}} \PYG{n+na}{class}\PYG{o}{=}\PYG{l+s}{\PYGZdq{}btn btn\PYGZhy{}secondary\PYGZdq{}}\PYG{p}{\PYGZgt{}}Derecha\PYG{p}{\PYGZlt{}}\PYG{p}{/}\PYG{n+nt}{button}\PYG{p}{\PYGZgt{}}
\PYG{p}{\PYGZlt{}}\PYG{p}{/}\PYG{n+nt}{div}\PYG{p}{\PYGZgt{}}
\end{sphinxVerbatim}




\section{Grupos de grupos de botones}
\label{\detokenize{mas-componentes:grupos-de-grupos-de-botones}}
\fvset{hllines={, ,}}%
\begin{sphinxVerbatim}[commandchars=\\\{\}]
\PYG{p}{\PYGZlt{}}\PYG{n+nt}{div} \PYG{n+na}{class}\PYG{o}{=}\PYG{l+s}{\PYGZdq{}btn\PYGZhy{}toolbar\PYGZdq{}} \PYG{n+na}{role}\PYG{o}{=}\PYG{l+s}{\PYGZdq{}toolbar\PYGZdq{}} \PYG{n+na}{aria\PYGZhy{}label}\PYG{o}{=}\PYG{l+s}{\PYGZdq{}Toolbar with button groups\PYGZdq{}}\PYG{p}{\PYGZgt{}}
  \PYG{p}{\PYGZlt{}}\PYG{n+nt}{div} \PYG{n+na}{class}\PYG{o}{=}\PYG{l+s}{\PYGZdq{}btn\PYGZhy{}group mr\PYGZhy{}2\PYGZdq{}} \PYG{n+na}{role}\PYG{o}{=}\PYG{l+s}{\PYGZdq{}group\PYGZdq{}} \PYG{n+na}{aria\PYGZhy{}label}\PYG{o}{=}\PYG{l+s}{\PYGZdq{}First group\PYGZdq{}}\PYG{p}{\PYGZgt{}}
    \PYG{p}{\PYGZlt{}}\PYG{n+nt}{button} \PYG{n+na}{type}\PYG{o}{=}\PYG{l+s}{\PYGZdq{}button\PYGZdq{}} \PYG{n+na}{class}\PYG{o}{=}\PYG{l+s}{\PYGZdq{}btn btn\PYGZhy{}secondary\PYGZdq{}}\PYG{p}{\PYGZgt{}}1\PYG{p}{\PYGZlt{}}\PYG{p}{/}\PYG{n+nt}{button}\PYG{p}{\PYGZgt{}}
    \PYG{p}{\PYGZlt{}}\PYG{n+nt}{button} \PYG{n+na}{type}\PYG{o}{=}\PYG{l+s}{\PYGZdq{}button\PYGZdq{}} \PYG{n+na}{class}\PYG{o}{=}\PYG{l+s}{\PYGZdq{}btn btn\PYGZhy{}secondary\PYGZdq{}}\PYG{p}{\PYGZgt{}}2\PYG{p}{\PYGZlt{}}\PYG{p}{/}\PYG{n+nt}{button}\PYG{p}{\PYGZgt{}}
    \PYG{p}{\PYGZlt{}}\PYG{n+nt}{button} \PYG{n+na}{type}\PYG{o}{=}\PYG{l+s}{\PYGZdq{}button\PYGZdq{}} \PYG{n+na}{class}\PYG{o}{=}\PYG{l+s}{\PYGZdq{}btn btn\PYGZhy{}secondary\PYGZdq{}}\PYG{p}{\PYGZgt{}}3\PYG{p}{\PYGZlt{}}\PYG{p}{/}\PYG{n+nt}{button}\PYG{p}{\PYGZgt{}}
    \PYG{p}{\PYGZlt{}}\PYG{n+nt}{button} \PYG{n+na}{type}\PYG{o}{=}\PYG{l+s}{\PYGZdq{}button\PYGZdq{}} \PYG{n+na}{class}\PYG{o}{=}\PYG{l+s}{\PYGZdq{}btn btn\PYGZhy{}secondary\PYGZdq{}}\PYG{p}{\PYGZgt{}}4\PYG{p}{\PYGZlt{}}\PYG{p}{/}\PYG{n+nt}{button}\PYG{p}{\PYGZgt{}}
  \PYG{p}{\PYGZlt{}}\PYG{p}{/}\PYG{n+nt}{div}\PYG{p}{\PYGZgt{}}
  \PYG{p}{\PYGZlt{}}\PYG{n+nt}{div} \PYG{n+na}{class}\PYG{o}{=}\PYG{l+s}{\PYGZdq{}btn\PYGZhy{}group mr\PYGZhy{}2\PYGZdq{}} \PYG{n+na}{role}\PYG{o}{=}\PYG{l+s}{\PYGZdq{}group\PYGZdq{}} \PYG{n+na}{aria\PYGZhy{}label}\PYG{o}{=}\PYG{l+s}{\PYGZdq{}Second group\PYGZdq{}}\PYG{p}{\PYGZgt{}}
    \PYG{p}{\PYGZlt{}}\PYG{n+nt}{button} \PYG{n+na}{type}\PYG{o}{=}\PYG{l+s}{\PYGZdq{}button\PYGZdq{}} \PYG{n+na}{class}\PYG{o}{=}\PYG{l+s}{\PYGZdq{}btn btn\PYGZhy{}secondary\PYGZdq{}}\PYG{p}{\PYGZgt{}}5\PYG{p}{\PYGZlt{}}\PYG{p}{/}\PYG{n+nt}{button}\PYG{p}{\PYGZgt{}}
    \PYG{p}{\PYGZlt{}}\PYG{n+nt}{button} \PYG{n+na}{type}\PYG{o}{=}\PYG{l+s}{\PYGZdq{}button\PYGZdq{}} \PYG{n+na}{class}\PYG{o}{=}\PYG{l+s}{\PYGZdq{}btn btn\PYGZhy{}secondary\PYGZdq{}}\PYG{p}{\PYGZgt{}}6\PYG{p}{\PYGZlt{}}\PYG{p}{/}\PYG{n+nt}{button}\PYG{p}{\PYGZgt{}}
    \PYG{p}{\PYGZlt{}}\PYG{n+nt}{button} \PYG{n+na}{type}\PYG{o}{=}\PYG{l+s}{\PYGZdq{}button\PYGZdq{}} \PYG{n+na}{class}\PYG{o}{=}\PYG{l+s}{\PYGZdq{}btn btn\PYGZhy{}secondary\PYGZdq{}}\PYG{p}{\PYGZgt{}}7\PYG{p}{\PYGZlt{}}\PYG{p}{/}\PYG{n+nt}{button}\PYG{p}{\PYGZgt{}}
  \PYG{p}{\PYGZlt{}}\PYG{p}{/}\PYG{n+nt}{div}\PYG{p}{\PYGZgt{}}
  \PYG{p}{\PYGZlt{}}\PYG{n+nt}{div} \PYG{n+na}{class}\PYG{o}{=}\PYG{l+s}{\PYGZdq{}btn\PYGZhy{}group\PYGZdq{}} \PYG{n+na}{role}\PYG{o}{=}\PYG{l+s}{\PYGZdq{}group\PYGZdq{}} \PYG{n+na}{aria\PYGZhy{}label}\PYG{o}{=}\PYG{l+s}{\PYGZdq{}Third group\PYGZdq{}}\PYG{p}{\PYGZgt{}}
    \PYG{p}{\PYGZlt{}}\PYG{n+nt}{button} \PYG{n+na}{type}\PYG{o}{=}\PYG{l+s}{\PYGZdq{}button\PYGZdq{}} \PYG{n+na}{class}\PYG{o}{=}\PYG{l+s}{\PYGZdq{}btn btn\PYGZhy{}secondary\PYGZdq{}}\PYG{p}{\PYGZgt{}}8\PYG{p}{\PYGZlt{}}\PYG{p}{/}\PYG{n+nt}{button}\PYG{p}{\PYGZgt{}}
  \PYG{p}{\PYGZlt{}}\PYG{p}{/}\PYG{n+nt}{div}\PYG{p}{\PYGZgt{}}
\PYG{p}{\PYGZlt{}}\PYG{p}{/}\PYG{n+nt}{div}\PYG{p}{\PYGZgt{}}
\end{sphinxVerbatim}




\subsection{Tamaños}
\label{\detokenize{mas-componentes:id1}}
\fvset{hllines={, ,}}%
\begin{sphinxVerbatim}[commandchars=\\\{\}]
\PYG{p}{\PYGZlt{}}\PYG{n+nt}{div} \PYG{n+na}{class}\PYG{o}{=}\PYG{l+s}{\PYGZdq{}btn\PYGZhy{}group btn\PYGZhy{}group\PYGZhy{}lg\PYGZdq{}} \PYG{n+na}{role}\PYG{o}{=}\PYG{l+s}{\PYGZdq{}group\PYGZdq{}} \PYG{n+na}{aria\PYGZhy{}label}\PYG{o}{=}\PYG{l+s}{\PYGZdq{}Large button group\PYGZdq{}}\PYG{p}{\PYGZgt{}}
  \PYG{p}{\PYGZlt{}}\PYG{n+nt}{button} \PYG{n+na}{type}\PYG{o}{=}\PYG{l+s}{\PYGZdq{}button\PYGZdq{}} \PYG{n+na}{class}\PYG{o}{=}\PYG{l+s}{\PYGZdq{}btn btn\PYGZhy{}secondary\PYGZdq{}}\PYG{p}{\PYGZgt{}}Izquierda\PYG{p}{\PYGZlt{}}\PYG{p}{/}\PYG{n+nt}{button}\PYG{p}{\PYGZgt{}}
  \PYG{p}{\PYGZlt{}}\PYG{n+nt}{button} \PYG{n+na}{type}\PYG{o}{=}\PYG{l+s}{\PYGZdq{}button\PYGZdq{}} \PYG{n+na}{class}\PYG{o}{=}\PYG{l+s}{\PYGZdq{}btn btn\PYGZhy{}secondary\PYGZdq{}}\PYG{p}{\PYGZgt{}}Medio\PYG{p}{\PYGZlt{}}\PYG{p}{/}\PYG{n+nt}{button}\PYG{p}{\PYGZgt{}}
  \PYG{p}{\PYGZlt{}}\PYG{n+nt}{button} \PYG{n+na}{type}\PYG{o}{=}\PYG{l+s}{\PYGZdq{}button\PYGZdq{}} \PYG{n+na}{class}\PYG{o}{=}\PYG{l+s}{\PYGZdq{}btn btn\PYGZhy{}secondary\PYGZdq{}}\PYG{p}{\PYGZgt{}}Derecha\PYG{p}{\PYGZlt{}}\PYG{p}{/}\PYG{n+nt}{button}\PYG{p}{\PYGZgt{}}
\PYG{p}{\PYGZlt{}}\PYG{p}{/}\PYG{n+nt}{div}\PYG{p}{\PYGZgt{}}
\end{sphinxVerbatim}



\fvset{hllines={, ,}}%
\begin{sphinxVerbatim}[commandchars=\\\{\}]
\PYG{p}{\PYGZlt{}}\PYG{n+nt}{div} \PYG{n+na}{class}\PYG{o}{=}\PYG{l+s}{\PYGZdq{}btn\PYGZhy{}group\PYGZdq{}} \PYG{n+na}{role}\PYG{o}{=}\PYG{l+s}{\PYGZdq{}group\PYGZdq{}} \PYG{n+na}{aria\PYGZhy{}label}\PYG{o}{=}\PYG{l+s}{\PYGZdq{}Default button group\PYGZdq{}}\PYG{p}{\PYGZgt{}}
  \PYG{p}{\PYGZlt{}}\PYG{n+nt}{button} \PYG{n+na}{type}\PYG{o}{=}\PYG{l+s}{\PYGZdq{}button\PYGZdq{}} \PYG{n+na}{class}\PYG{o}{=}\PYG{l+s}{\PYGZdq{}btn btn\PYGZhy{}secondary\PYGZdq{}}\PYG{p}{\PYGZgt{}}Izquierda\PYG{p}{\PYGZlt{}}\PYG{p}{/}\PYG{n+nt}{button}\PYG{p}{\PYGZgt{}}
  \PYG{p}{\PYGZlt{}}\PYG{n+nt}{button} \PYG{n+na}{type}\PYG{o}{=}\PYG{l+s}{\PYGZdq{}button\PYGZdq{}} \PYG{n+na}{class}\PYG{o}{=}\PYG{l+s}{\PYGZdq{}btn btn\PYGZhy{}secondary\PYGZdq{}}\PYG{p}{\PYGZgt{}}Medio\PYG{p}{\PYGZlt{}}\PYG{p}{/}\PYG{n+nt}{button}\PYG{p}{\PYGZgt{}}
  \PYG{p}{\PYGZlt{}}\PYG{n+nt}{button} \PYG{n+na}{type}\PYG{o}{=}\PYG{l+s}{\PYGZdq{}button\PYGZdq{}} \PYG{n+na}{class}\PYG{o}{=}\PYG{l+s}{\PYGZdq{}btn btn\PYGZhy{}secondary\PYGZdq{}}\PYG{p}{\PYGZgt{}}Derecha\PYG{p}{\PYGZlt{}}\PYG{p}{/}\PYG{n+nt}{button}\PYG{p}{\PYGZgt{}}
\PYG{p}{\PYGZlt{}}\PYG{p}{/}\PYG{n+nt}{div}\PYG{p}{\PYGZgt{}}
\end{sphinxVerbatim}



\fvset{hllines={, ,}}%
\begin{sphinxVerbatim}[commandchars=\\\{\}]
\PYG{p}{\PYGZlt{}}\PYG{n+nt}{div} \PYG{n+na}{class}\PYG{o}{=}\PYG{l+s}{\PYGZdq{}btn\PYGZhy{}group btn\PYGZhy{}group\PYGZhy{}sm\PYGZdq{}} \PYG{n+na}{role}\PYG{o}{=}\PYG{l+s}{\PYGZdq{}group\PYGZdq{}} \PYG{n+na}{aria\PYGZhy{}label}\PYG{o}{=}\PYG{l+s}{\PYGZdq{}Small button group\PYGZdq{}}\PYG{p}{\PYGZgt{}}
  \PYG{p}{\PYGZlt{}}\PYG{n+nt}{button} \PYG{n+na}{type}\PYG{o}{=}\PYG{l+s}{\PYGZdq{}button\PYGZdq{}} \PYG{n+na}{class}\PYG{o}{=}\PYG{l+s}{\PYGZdq{}btn btn\PYGZhy{}secondary\PYGZdq{}}\PYG{p}{\PYGZgt{}}Izquierda\PYG{p}{\PYGZlt{}}\PYG{p}{/}\PYG{n+nt}{button}\PYG{p}{\PYGZgt{}}
  \PYG{p}{\PYGZlt{}}\PYG{n+nt}{button} \PYG{n+na}{type}\PYG{o}{=}\PYG{l+s}{\PYGZdq{}button\PYGZdq{}} \PYG{n+na}{class}\PYG{o}{=}\PYG{l+s}{\PYGZdq{}btn btn\PYGZhy{}secondary\PYGZdq{}}\PYG{p}{\PYGZgt{}}Medio\PYG{p}{\PYGZlt{}}\PYG{p}{/}\PYG{n+nt}{button}\PYG{p}{\PYGZgt{}}
  \PYG{p}{\PYGZlt{}}\PYG{n+nt}{button} \PYG{n+na}{type}\PYG{o}{=}\PYG{l+s}{\PYGZdq{}button\PYGZdq{}} \PYG{n+na}{class}\PYG{o}{=}\PYG{l+s}{\PYGZdq{}btn btn\PYGZhy{}secondary\PYGZdq{}}\PYG{p}{\PYGZgt{}}Derecha\PYG{p}{\PYGZlt{}}\PYG{p}{/}\PYG{n+nt}{button}\PYG{p}{\PYGZgt{}}
\PYG{p}{\PYGZlt{}}\PYG{p}{/}\PYG{n+nt}{div}\PYG{p}{\PYGZgt{}}
\end{sphinxVerbatim}




\section{Listas}
\label{\detokenize{mas-componentes:listas}}
\fvset{hllines={, ,}}%
\begin{sphinxVerbatim}[commandchars=\\\{\}]
\PYG{p}{\PYGZlt{}}\PYG{n+nt}{ul} \PYG{n+na}{class}\PYG{o}{=}\PYG{l+s}{\PYGZdq{}list\PYGZhy{}group\PYGZdq{}}\PYG{p}{\PYGZgt{}}
  \PYG{p}{\PYGZlt{}}\PYG{n+nt}{li} \PYG{n+na}{class}\PYG{o}{=}\PYG{l+s}{\PYGZdq{}list\PYGZhy{}group\PYGZhy{}item\PYGZdq{}}\PYG{p}{\PYGZgt{}}Cras justo odio\PYG{p}{\PYGZlt{}}\PYG{p}{/}\PYG{n+nt}{li}\PYG{p}{\PYGZgt{}}
  \PYG{p}{\PYGZlt{}}\PYG{n+nt}{li} \PYG{n+na}{class}\PYG{o}{=}\PYG{l+s}{\PYGZdq{}list\PYGZhy{}group\PYGZhy{}item\PYGZdq{}}\PYG{p}{\PYGZgt{}}Dapibus ac facilisis in\PYG{p}{\PYGZlt{}}\PYG{p}{/}\PYG{n+nt}{li}\PYG{p}{\PYGZgt{}}
  \PYG{p}{\PYGZlt{}}\PYG{n+nt}{li} \PYG{n+na}{class}\PYG{o}{=}\PYG{l+s}{\PYGZdq{}list\PYGZhy{}group\PYGZhy{}item\PYGZdq{}}\PYG{p}{\PYGZgt{}}Morbi leo risus\PYG{p}{\PYGZlt{}}\PYG{p}{/}\PYG{n+nt}{li}\PYG{p}{\PYGZgt{}}
  \PYG{p}{\PYGZlt{}}\PYG{n+nt}{li} \PYG{n+na}{class}\PYG{o}{=}\PYG{l+s}{\PYGZdq{}list\PYGZhy{}group\PYGZhy{}item\PYGZdq{}}\PYG{p}{\PYGZgt{}}Porta ac consectetur ac\PYG{p}{\PYGZlt{}}\PYG{p}{/}\PYG{n+nt}{li}\PYG{p}{\PYGZgt{}}
  \PYG{p}{\PYGZlt{}}\PYG{n+nt}{li} \PYG{n+na}{class}\PYG{o}{=}\PYG{l+s}{\PYGZdq{}list\PYGZhy{}group\PYGZhy{}item\PYGZdq{}}\PYG{p}{\PYGZgt{}}Vestibulum at eros\PYG{p}{\PYGZlt{}}\PYG{p}{/}\PYG{n+nt}{li}\PYG{p}{\PYGZgt{}}
\PYG{p}{\PYGZlt{}}\PYG{p}{/}\PYG{n+nt}{ul}\PYG{p}{\PYGZgt{}}
\end{sphinxVerbatim}




\subsection{Elemento activo}
\label{\detokenize{mas-componentes:elemento-activo}}
\fvset{hllines={, ,}}%
\begin{sphinxVerbatim}[commandchars=\\\{\}]
\PYG{p}{\PYGZlt{}}\PYG{n+nt}{ul} \PYG{n+na}{class}\PYG{o}{=}\PYG{l+s}{\PYGZdq{}list\PYGZhy{}group\PYGZdq{}}\PYG{p}{\PYGZgt{}}
  \PYG{p}{\PYGZlt{}}\PYG{n+nt}{li} \PYG{n+na}{class}\PYG{o}{=}\PYG{l+s}{\PYGZdq{}list\PYGZhy{}group\PYGZhy{}item active\PYGZdq{}}\PYG{p}{\PYGZgt{}}Cras justo odio\PYG{p}{\PYGZlt{}}\PYG{p}{/}\PYG{n+nt}{li}\PYG{p}{\PYGZgt{}}
  \PYG{p}{\PYGZlt{}}\PYG{n+nt}{li} \PYG{n+na}{class}\PYG{o}{=}\PYG{l+s}{\PYGZdq{}list\PYGZhy{}group\PYGZhy{}item\PYGZdq{}}\PYG{p}{\PYGZgt{}}Dapibus ac facilisis in\PYG{p}{\PYGZlt{}}\PYG{p}{/}\PYG{n+nt}{li}\PYG{p}{\PYGZgt{}}
  \PYG{p}{\PYGZlt{}}\PYG{n+nt}{li} \PYG{n+na}{class}\PYG{o}{=}\PYG{l+s}{\PYGZdq{}list\PYGZhy{}group\PYGZhy{}item\PYGZdq{}}\PYG{p}{\PYGZgt{}}Morbi leo risus\PYG{p}{\PYGZlt{}}\PYG{p}{/}\PYG{n+nt}{li}\PYG{p}{\PYGZgt{}}
  \PYG{p}{\PYGZlt{}}\PYG{n+nt}{li} \PYG{n+na}{class}\PYG{o}{=}\PYG{l+s}{\PYGZdq{}list\PYGZhy{}group\PYGZhy{}item\PYGZdq{}}\PYG{p}{\PYGZgt{}}Porta ac consectetur ac\PYG{p}{\PYGZlt{}}\PYG{p}{/}\PYG{n+nt}{li}\PYG{p}{\PYGZgt{}}
  \PYG{p}{\PYGZlt{}}\PYG{n+nt}{li} \PYG{n+na}{class}\PYG{o}{=}\PYG{l+s}{\PYGZdq{}list\PYGZhy{}group\PYGZhy{}item\PYGZdq{}}\PYG{p}{\PYGZgt{}}Vestibulum at eros\PYG{p}{\PYGZlt{}}\PYG{p}{/}\PYG{n+nt}{li}\PYG{p}{\PYGZgt{}}
\PYG{p}{\PYGZlt{}}\PYG{p}{/}\PYG{n+nt}{ul}\PYG{p}{\PYGZgt{}}
\end{sphinxVerbatim}




\subsection{Elemento inactivo}
\label{\detokenize{mas-componentes:elemento-inactivo}}
\fvset{hllines={, ,}}%
\begin{sphinxVerbatim}[commandchars=\\\{\}]
\PYG{p}{\PYGZlt{}}\PYG{n+nt}{ul} \PYG{n+na}{class}\PYG{o}{=}\PYG{l+s}{\PYGZdq{}list\PYGZhy{}group\PYGZdq{}}\PYG{p}{\PYGZgt{}}
  \PYG{p}{\PYGZlt{}}\PYG{n+nt}{li} \PYG{n+na}{class}\PYG{o}{=}\PYG{l+s}{\PYGZdq{}list\PYGZhy{}group\PYGZhy{}item disabled\PYGZdq{}}\PYG{p}{\PYGZgt{}}Cras justo odio\PYG{p}{\PYGZlt{}}\PYG{p}{/}\PYG{n+nt}{li}\PYG{p}{\PYGZgt{}}
  \PYG{p}{\PYGZlt{}}\PYG{n+nt}{li} \PYG{n+na}{class}\PYG{o}{=}\PYG{l+s}{\PYGZdq{}list\PYGZhy{}group\PYGZhy{}item\PYGZdq{}}\PYG{p}{\PYGZgt{}}Dapibus ac facilisis in\PYG{p}{\PYGZlt{}}\PYG{p}{/}\PYG{n+nt}{li}\PYG{p}{\PYGZgt{}}
  \PYG{p}{\PYGZlt{}}\PYG{n+nt}{li} \PYG{n+na}{class}\PYG{o}{=}\PYG{l+s}{\PYGZdq{}list\PYGZhy{}group\PYGZhy{}item\PYGZdq{}}\PYG{p}{\PYGZgt{}}Morbi leo risus\PYG{p}{\PYGZlt{}}\PYG{p}{/}\PYG{n+nt}{li}\PYG{p}{\PYGZgt{}}
  \PYG{p}{\PYGZlt{}}\PYG{n+nt}{li} \PYG{n+na}{class}\PYG{o}{=}\PYG{l+s}{\PYGZdq{}list\PYGZhy{}group\PYGZhy{}item\PYGZdq{}}\PYG{p}{\PYGZgt{}}Porta ac consectetur ac\PYG{p}{\PYGZlt{}}\PYG{p}{/}\PYG{n+nt}{li}\PYG{p}{\PYGZgt{}}
  \PYG{p}{\PYGZlt{}}\PYG{n+nt}{li} \PYG{n+na}{class}\PYG{o}{=}\PYG{l+s}{\PYGZdq{}list\PYGZhy{}group\PYGZhy{}item\PYGZdq{}}\PYG{p}{\PYGZgt{}}Vestibulum at eros\PYG{p}{\PYGZlt{}}\PYG{p}{/}\PYG{n+nt}{li}\PYG{p}{\PYGZgt{}}
\PYG{p}{\PYGZlt{}}\PYG{p}{/}\PYG{n+nt}{ul}\PYG{p}{\PYGZgt{}}
\end{sphinxVerbatim}




\subsection{Links}
\label{\detokenize{mas-componentes:links}}
\fvset{hllines={, ,}}%
\begin{sphinxVerbatim}[commandchars=\\\{\}]
\PYG{p}{\PYGZlt{}}\PYG{n+nt}{div} \PYG{n+na}{class}\PYG{o}{=}\PYG{l+s}{\PYGZdq{}list\PYGZhy{}group\PYGZdq{}}\PYG{p}{\PYGZgt{}}
  \PYG{p}{\PYGZlt{}}\PYG{n+nt}{a} \PYG{n+na}{href}\PYG{o}{=}\PYG{l+s}{\PYGZdq{}\PYGZsh{}\PYGZdq{}} \PYG{n+na}{class}\PYG{o}{=}\PYG{l+s}{\PYGZdq{}list\PYGZhy{}group\PYGZhy{}item list\PYGZhy{}group\PYGZhy{}item\PYGZhy{}action active\PYGZdq{}}\PYG{p}{\PYGZgt{}}
    Cras justo odio
  \PYG{p}{\PYGZlt{}}\PYG{p}{/}\PYG{n+nt}{a}\PYG{p}{\PYGZgt{}}
  \PYG{p}{\PYGZlt{}}\PYG{n+nt}{a} \PYG{n+na}{href}\PYG{o}{=}\PYG{l+s}{\PYGZdq{}\PYGZsh{}\PYGZdq{}} \PYG{n+na}{class}\PYG{o}{=}\PYG{l+s}{\PYGZdq{}list\PYGZhy{}group\PYGZhy{}item list\PYGZhy{}group\PYGZhy{}item\PYGZhy{}action\PYGZdq{}}\PYG{p}{\PYGZgt{}}Dapibus ac facilisis in\PYG{p}{\PYGZlt{}}\PYG{p}{/}\PYG{n+nt}{a}\PYG{p}{\PYGZgt{}}
  \PYG{p}{\PYGZlt{}}\PYG{n+nt}{a} \PYG{n+na}{href}\PYG{o}{=}\PYG{l+s}{\PYGZdq{}\PYGZsh{}\PYGZdq{}} \PYG{n+na}{class}\PYG{o}{=}\PYG{l+s}{\PYGZdq{}list\PYGZhy{}group\PYGZhy{}item list\PYGZhy{}group\PYGZhy{}item\PYGZhy{}action\PYGZdq{}}\PYG{p}{\PYGZgt{}}Morbi leo risus\PYG{p}{\PYGZlt{}}\PYG{p}{/}\PYG{n+nt}{a}\PYG{p}{\PYGZgt{}}
  \PYG{p}{\PYGZlt{}}\PYG{n+nt}{a} \PYG{n+na}{href}\PYG{o}{=}\PYG{l+s}{\PYGZdq{}\PYGZsh{}\PYGZdq{}} \PYG{n+na}{class}\PYG{o}{=}\PYG{l+s}{\PYGZdq{}list\PYGZhy{}group\PYGZhy{}item list\PYGZhy{}group\PYGZhy{}item\PYGZhy{}action\PYGZdq{}}\PYG{p}{\PYGZgt{}}Porta ac consectetur ac\PYG{p}{\PYGZlt{}}\PYG{p}{/}\PYG{n+nt}{a}\PYG{p}{\PYGZgt{}}
  \PYG{p}{\PYGZlt{}}\PYG{n+nt}{a} \PYG{n+na}{href}\PYG{o}{=}\PYG{l+s}{\PYGZdq{}\PYGZsh{}\PYGZdq{}} \PYG{n+na}{class}\PYG{o}{=}\PYG{l+s}{\PYGZdq{}list\PYGZhy{}group\PYGZhy{}item list\PYGZhy{}group\PYGZhy{}item\PYGZhy{}action disabled\PYGZdq{}}\PYG{p}{\PYGZgt{}}Vestibulum at eros\PYG{p}{\PYGZlt{}}\PYG{p}{/}\PYG{n+nt}{a}\PYG{p}{\PYGZgt{}}
\PYG{p}{\PYGZlt{}}\PYG{p}{/}\PYG{n+nt}{div}\PYG{p}{\PYGZgt{}}
\end{sphinxVerbatim}




\subsection{Botones}
\label{\detokenize{mas-componentes:id2}}
\fvset{hllines={, ,}}%
\begin{sphinxVerbatim}[commandchars=\\\{\}]
\PYG{p}{\PYGZlt{}}\PYG{n+nt}{div} \PYG{n+na}{class}\PYG{o}{=}\PYG{l+s}{\PYGZdq{}list\PYGZhy{}group\PYGZdq{}}\PYG{p}{\PYGZgt{}}
  \PYG{p}{\PYGZlt{}}\PYG{n+nt}{button} \PYG{n+na}{type}\PYG{o}{=}\PYG{l+s}{\PYGZdq{}button\PYGZdq{}} \PYG{n+na}{class}\PYG{o}{=}\PYG{l+s}{\PYGZdq{}list\PYGZhy{}group\PYGZhy{}item list\PYGZhy{}group\PYGZhy{}item\PYGZhy{}action active\PYGZdq{}}\PYG{p}{\PYGZgt{}}
    Cras justo odio
  \PYG{p}{\PYGZlt{}}\PYG{p}{/}\PYG{n+nt}{button}\PYG{p}{\PYGZgt{}}
  \PYG{p}{\PYGZlt{}}\PYG{n+nt}{button} \PYG{n+na}{type}\PYG{o}{=}\PYG{l+s}{\PYGZdq{}button\PYGZdq{}} \PYG{n+na}{class}\PYG{o}{=}\PYG{l+s}{\PYGZdq{}list\PYGZhy{}group\PYGZhy{}item list\PYGZhy{}group\PYGZhy{}item\PYGZhy{}action\PYGZdq{}}\PYG{p}{\PYGZgt{}}Dapibus ac facilisis in\PYG{p}{\PYGZlt{}}\PYG{p}{/}\PYG{n+nt}{button}\PYG{p}{\PYGZgt{}}
  \PYG{p}{\PYGZlt{}}\PYG{n+nt}{button} \PYG{n+na}{type}\PYG{o}{=}\PYG{l+s}{\PYGZdq{}button\PYGZdq{}} \PYG{n+na}{class}\PYG{o}{=}\PYG{l+s}{\PYGZdq{}list\PYGZhy{}group\PYGZhy{}item list\PYGZhy{}group\PYGZhy{}item\PYGZhy{}action\PYGZdq{}}\PYG{p}{\PYGZgt{}}Morbi leo risus\PYG{p}{\PYGZlt{}}\PYG{p}{/}\PYG{n+nt}{button}\PYG{p}{\PYGZgt{}}
  \PYG{p}{\PYGZlt{}}\PYG{n+nt}{button} \PYG{n+na}{type}\PYG{o}{=}\PYG{l+s}{\PYGZdq{}button\PYGZdq{}} \PYG{n+na}{class}\PYG{o}{=}\PYG{l+s}{\PYGZdq{}list\PYGZhy{}group\PYGZhy{}item list\PYGZhy{}group\PYGZhy{}item\PYGZhy{}action\PYGZdq{}}\PYG{p}{\PYGZgt{}}Porta ac consectetur ac\PYG{p}{\PYGZlt{}}\PYG{p}{/}\PYG{n+nt}{button}\PYG{p}{\PYGZgt{}}
  \PYG{p}{\PYGZlt{}}\PYG{n+nt}{button} \PYG{n+na}{type}\PYG{o}{=}\PYG{l+s}{\PYGZdq{}button\PYGZdq{}} \PYG{n+na}{class}\PYG{o}{=}\PYG{l+s}{\PYGZdq{}list\PYGZhy{}group\PYGZhy{}item list\PYGZhy{}group\PYGZhy{}item\PYGZhy{}action\PYGZdq{}} \PYG{n+na}{disabled}\PYG{p}{\PYGZgt{}}Vestibulum at eros\PYG{p}{\PYGZlt{}}\PYG{p}{/}\PYG{n+nt}{button}\PYG{p}{\PYGZgt{}}
\PYG{p}{\PYGZlt{}}\PYG{p}{/}\PYG{n+nt}{div}\PYG{p}{\PYGZgt{}}
\end{sphinxVerbatim}




\subsection{Sin bordes}
\label{\detokenize{mas-componentes:sin-bordes}}
\fvset{hllines={, ,}}%
\begin{sphinxVerbatim}[commandchars=\\\{\}]
\PYG{p}{\PYGZlt{}}\PYG{n+nt}{ul} \PYG{n+na}{class}\PYG{o}{=}\PYG{l+s}{\PYGZdq{}list\PYGZhy{}group list\PYGZhy{}group\PYGZhy{}flush\PYGZdq{}}\PYG{p}{\PYGZgt{}}
  \PYG{p}{\PYGZlt{}}\PYG{n+nt}{li} \PYG{n+na}{class}\PYG{o}{=}\PYG{l+s}{\PYGZdq{}list\PYGZhy{}group\PYGZhy{}item\PYGZdq{}}\PYG{p}{\PYGZgt{}}Cras justo odio\PYG{p}{\PYGZlt{}}\PYG{p}{/}\PYG{n+nt}{li}\PYG{p}{\PYGZgt{}}
  \PYG{p}{\PYGZlt{}}\PYG{n+nt}{li} \PYG{n+na}{class}\PYG{o}{=}\PYG{l+s}{\PYGZdq{}list\PYGZhy{}group\PYGZhy{}item\PYGZdq{}}\PYG{p}{\PYGZgt{}}Dapibus ac facilisis in\PYG{p}{\PYGZlt{}}\PYG{p}{/}\PYG{n+nt}{li}\PYG{p}{\PYGZgt{}}
  \PYG{p}{\PYGZlt{}}\PYG{n+nt}{li} \PYG{n+na}{class}\PYG{o}{=}\PYG{l+s}{\PYGZdq{}list\PYGZhy{}group\PYGZhy{}item\PYGZdq{}}\PYG{p}{\PYGZgt{}}Morbi leo risus\PYG{p}{\PYGZlt{}}\PYG{p}{/}\PYG{n+nt}{li}\PYG{p}{\PYGZgt{}}
  \PYG{p}{\PYGZlt{}}\PYG{n+nt}{li} \PYG{n+na}{class}\PYG{o}{=}\PYG{l+s}{\PYGZdq{}list\PYGZhy{}group\PYGZhy{}item\PYGZdq{}}\PYG{p}{\PYGZgt{}}Porta ac consectetur ac\PYG{p}{\PYGZlt{}}\PYG{p}{/}\PYG{n+nt}{li}\PYG{p}{\PYGZgt{}}
  \PYG{p}{\PYGZlt{}}\PYG{n+nt}{li} \PYG{n+na}{class}\PYG{o}{=}\PYG{l+s}{\PYGZdq{}list\PYGZhy{}group\PYGZhy{}item\PYGZdq{}}\PYG{p}{\PYGZgt{}}Vestibulum at eros\PYG{p}{\PYGZlt{}}\PYG{p}{/}\PYG{n+nt}{li}\PYG{p}{\PYGZgt{}}
\PYG{p}{\PYGZlt{}}\PYG{p}{/}\PYG{n+nt}{ul}\PYG{p}{\PYGZgt{}}
\end{sphinxVerbatim}




\subsection{Clases útiles}
\label{\detokenize{mas-componentes:clases-utiles}}
\fvset{hllines={, ,}}%
\begin{sphinxVerbatim}[commandchars=\\\{\}]
\PYG{p}{\PYGZlt{}}\PYG{n+nt}{ul} \PYG{n+na}{class}\PYG{o}{=}\PYG{l+s}{\PYGZdq{}list\PYGZhy{}group\PYGZdq{}}\PYG{p}{\PYGZgt{}}
  \PYG{p}{\PYGZlt{}}\PYG{n+nt}{li} \PYG{n+na}{class}\PYG{o}{=}\PYG{l+s}{\PYGZdq{}list\PYGZhy{}group\PYGZhy{}item\PYGZdq{}}\PYG{p}{\PYGZgt{}}Dapibus ac facilisis in\PYG{p}{\PYGZlt{}}\PYG{p}{/}\PYG{n+nt}{li}\PYG{p}{\PYGZgt{}}

  \PYG{p}{\PYGZlt{}}\PYG{n+nt}{li} \PYG{n+na}{class}\PYG{o}{=}\PYG{l+s}{\PYGZdq{}list\PYGZhy{}group\PYGZhy{}item list\PYGZhy{}group\PYGZhy{}item\PYGZhy{}primary\PYGZdq{}}\PYG{p}{\PYGZgt{}}Primario\PYG{p}{\PYGZlt{}}\PYG{p}{/}\PYG{n+nt}{li}\PYG{p}{\PYGZgt{}}
  \PYG{p}{\PYGZlt{}}\PYG{n+nt}{li} \PYG{n+na}{class}\PYG{o}{=}\PYG{l+s}{\PYGZdq{}list\PYGZhy{}group\PYGZhy{}item list\PYGZhy{}group\PYGZhy{}item\PYGZhy{}secondary\PYGZdq{}}\PYG{p}{\PYGZgt{}}Secundario\PYG{p}{\PYGZlt{}}\PYG{p}{/}\PYG{n+nt}{li}\PYG{p}{\PYGZgt{}}
  \PYG{p}{\PYGZlt{}}\PYG{n+nt}{li} \PYG{n+na}{class}\PYG{o}{=}\PYG{l+s}{\PYGZdq{}list\PYGZhy{}group\PYGZhy{}item list\PYGZhy{}group\PYGZhy{}item\PYGZhy{}success\PYGZdq{}}\PYG{p}{\PYGZgt{}}Exito\PYG{p}{\PYGZlt{}}\PYG{p}{/}\PYG{n+nt}{li}\PYG{p}{\PYGZgt{}}
  \PYG{p}{\PYGZlt{}}\PYG{n+nt}{li} \PYG{n+na}{class}\PYG{o}{=}\PYG{l+s}{\PYGZdq{}list\PYGZhy{}group\PYGZhy{}item list\PYGZhy{}group\PYGZhy{}item\PYGZhy{}danger\PYGZdq{}}\PYG{p}{\PYGZgt{}}Peligro\PYG{p}{\PYGZlt{}}\PYG{p}{/}\PYG{n+nt}{li}\PYG{p}{\PYGZgt{}}
  \PYG{p}{\PYGZlt{}}\PYG{n+nt}{li} \PYG{n+na}{class}\PYG{o}{=}\PYG{l+s}{\PYGZdq{}list\PYGZhy{}group\PYGZhy{}item list\PYGZhy{}group\PYGZhy{}item\PYGZhy{}warning\PYGZdq{}}\PYG{p}{\PYGZgt{}}Advertencia\PYG{p}{\PYGZlt{}}\PYG{p}{/}\PYG{n+nt}{li}\PYG{p}{\PYGZgt{}}
  \PYG{p}{\PYGZlt{}}\PYG{n+nt}{li} \PYG{n+na}{class}\PYG{o}{=}\PYG{l+s}{\PYGZdq{}list\PYGZhy{}group\PYGZhy{}item list\PYGZhy{}group\PYGZhy{}item\PYGZhy{}info\PYGZdq{}}\PYG{p}{\PYGZgt{}}Informacion\PYG{p}{\PYGZlt{}}\PYG{p}{/}\PYG{n+nt}{li}\PYG{p}{\PYGZgt{}}
  \PYG{p}{\PYGZlt{}}\PYG{n+nt}{li} \PYG{n+na}{class}\PYG{o}{=}\PYG{l+s}{\PYGZdq{}list\PYGZhy{}group\PYGZhy{}item list\PYGZhy{}group\PYGZhy{}item\PYGZhy{}light\PYGZdq{}}\PYG{p}{\PYGZgt{}}Claro\PYG{p}{\PYGZlt{}}\PYG{p}{/}\PYG{n+nt}{li}\PYG{p}{\PYGZgt{}}
  \PYG{p}{\PYGZlt{}}\PYG{n+nt}{li} \PYG{n+na}{class}\PYG{o}{=}\PYG{l+s}{\PYGZdq{}list\PYGZhy{}group\PYGZhy{}item list\PYGZhy{}group\PYGZhy{}item\PYGZhy{}dark\PYGZdq{}}\PYG{p}{\PYGZgt{}}Oscuro\PYG{p}{\PYGZlt{}}\PYG{p}{/}\PYG{n+nt}{li}\PYG{p}{\PYGZgt{}}
\PYG{p}{\PYGZlt{}}\PYG{p}{/}\PYG{n+nt}{ul}\PYG{p}{\PYGZgt{}}
\end{sphinxVerbatim}



\fvset{hllines={, ,}}%
\begin{sphinxVerbatim}[commandchars=\\\{\}]
\PYG{p}{\PYGZlt{}}\PYG{n+nt}{div} \PYG{n+na}{class}\PYG{o}{=}\PYG{l+s}{\PYGZdq{}list\PYGZhy{}group\PYGZdq{}}\PYG{p}{\PYGZgt{}}
  \PYG{p}{\PYGZlt{}}\PYG{n+nt}{a} \PYG{n+na}{href}\PYG{o}{=}\PYG{l+s}{\PYGZdq{}\PYGZsh{}\PYGZdq{}} \PYG{n+na}{class}\PYG{o}{=}\PYG{l+s}{\PYGZdq{}list\PYGZhy{}group\PYGZhy{}item list\PYGZhy{}group\PYGZhy{}item\PYGZhy{}action\PYGZdq{}}\PYG{p}{\PYGZgt{}}Dapibus ac facilisis in\PYG{p}{\PYGZlt{}}\PYG{p}{/}\PYG{n+nt}{a}\PYG{p}{\PYGZgt{}}

  \PYG{p}{\PYGZlt{}}\PYG{n+nt}{a} \PYG{n+na}{href}\PYG{o}{=}\PYG{l+s}{\PYGZdq{}\PYGZsh{}\PYGZdq{}} \PYG{n+na}{class}\PYG{o}{=}\PYG{l+s}{\PYGZdq{}list\PYGZhy{}group\PYGZhy{}item list\PYGZhy{}group\PYGZhy{}item\PYGZhy{}action list\PYGZhy{}group\PYGZhy{}item\PYGZhy{}primary\PYGZdq{}}\PYG{p}{\PYGZgt{}}Primario\PYG{p}{\PYGZlt{}}\PYG{p}{/}\PYG{n+nt}{a}\PYG{p}{\PYGZgt{}}
  \PYG{p}{\PYGZlt{}}\PYG{n+nt}{a} \PYG{n+na}{href}\PYG{o}{=}\PYG{l+s}{\PYGZdq{}\PYGZsh{}\PYGZdq{}} \PYG{n+na}{class}\PYG{o}{=}\PYG{l+s}{\PYGZdq{}list\PYGZhy{}group\PYGZhy{}item list\PYGZhy{}group\PYGZhy{}item\PYGZhy{}action list\PYGZhy{}group\PYGZhy{}item\PYGZhy{}secondary\PYGZdq{}}\PYG{p}{\PYGZgt{}}Secundario\PYG{p}{\PYGZlt{}}\PYG{p}{/}\PYG{n+nt}{a}\PYG{p}{\PYGZgt{}}
  \PYG{p}{\PYGZlt{}}\PYG{n+nt}{a} \PYG{n+na}{href}\PYG{o}{=}\PYG{l+s}{\PYGZdq{}\PYGZsh{}\PYGZdq{}} \PYG{n+na}{class}\PYG{o}{=}\PYG{l+s}{\PYGZdq{}list\PYGZhy{}group\PYGZhy{}item list\PYGZhy{}group\PYGZhy{}item\PYGZhy{}action list\PYGZhy{}group\PYGZhy{}item\PYGZhy{}success\PYGZdq{}}\PYG{p}{\PYGZgt{}}Exito\PYG{p}{\PYGZlt{}}\PYG{p}{/}\PYG{n+nt}{a}\PYG{p}{\PYGZgt{}}
  \PYG{p}{\PYGZlt{}}\PYG{n+nt}{a} \PYG{n+na}{href}\PYG{o}{=}\PYG{l+s}{\PYGZdq{}\PYGZsh{}\PYGZdq{}} \PYG{n+na}{class}\PYG{o}{=}\PYG{l+s}{\PYGZdq{}list\PYGZhy{}group\PYGZhy{}item list\PYGZhy{}group\PYGZhy{}item\PYGZhy{}action list\PYGZhy{}group\PYGZhy{}item\PYGZhy{}danger\PYGZdq{}}\PYG{p}{\PYGZgt{}}Peligro\PYG{p}{\PYGZlt{}}\PYG{p}{/}\PYG{n+nt}{a}\PYG{p}{\PYGZgt{}}
  \PYG{p}{\PYGZlt{}}\PYG{n+nt}{a} \PYG{n+na}{href}\PYG{o}{=}\PYG{l+s}{\PYGZdq{}\PYGZsh{}\PYGZdq{}} \PYG{n+na}{class}\PYG{o}{=}\PYG{l+s}{\PYGZdq{}list\PYGZhy{}group\PYGZhy{}item list\PYGZhy{}group\PYGZhy{}item\PYGZhy{}action list\PYGZhy{}group\PYGZhy{}item\PYGZhy{}warning\PYGZdq{}}\PYG{p}{\PYGZgt{}}Advertencia\PYG{p}{\PYGZlt{}}\PYG{p}{/}\PYG{n+nt}{a}\PYG{p}{\PYGZgt{}}
  \PYG{p}{\PYGZlt{}}\PYG{n+nt}{a} \PYG{n+na}{href}\PYG{o}{=}\PYG{l+s}{\PYGZdq{}\PYGZsh{}\PYGZdq{}} \PYG{n+na}{class}\PYG{o}{=}\PYG{l+s}{\PYGZdq{}list\PYGZhy{}group\PYGZhy{}item list\PYGZhy{}group\PYGZhy{}item\PYGZhy{}action list\PYGZhy{}group\PYGZhy{}item\PYGZhy{}info\PYGZdq{}}\PYG{p}{\PYGZgt{}}Informacion\PYG{p}{\PYGZlt{}}\PYG{p}{/}\PYG{n+nt}{a}\PYG{p}{\PYGZgt{}}
  \PYG{p}{\PYGZlt{}}\PYG{n+nt}{a} \PYG{n+na}{href}\PYG{o}{=}\PYG{l+s}{\PYGZdq{}\PYGZsh{}\PYGZdq{}} \PYG{n+na}{class}\PYG{o}{=}\PYG{l+s}{\PYGZdq{}list\PYGZhy{}group\PYGZhy{}item list\PYGZhy{}group\PYGZhy{}item\PYGZhy{}action list\PYGZhy{}group\PYGZhy{}item\PYGZhy{}light\PYGZdq{}}\PYG{p}{\PYGZgt{}}Claro\PYG{p}{\PYGZlt{}}\PYG{p}{/}\PYG{n+nt}{a}\PYG{p}{\PYGZgt{}}
  \PYG{p}{\PYGZlt{}}\PYG{n+nt}{a} \PYG{n+na}{href}\PYG{o}{=}\PYG{l+s}{\PYGZdq{}\PYGZsh{}\PYGZdq{}} \PYG{n+na}{class}\PYG{o}{=}\PYG{l+s}{\PYGZdq{}list\PYGZhy{}group\PYGZhy{}item list\PYGZhy{}group\PYGZhy{}item\PYGZhy{}action list\PYGZhy{}group\PYGZhy{}item\PYGZhy{}dark\PYGZdq{}}\PYG{p}{\PYGZgt{}}Oscuro\PYG{p}{\PYGZlt{}}\PYG{p}{/}\PYG{n+nt}{a}\PYG{p}{\PYGZgt{}}
\PYG{p}{\PYGZlt{}}\PYG{p}{/}\PYG{n+nt}{div}\PYG{p}{\PYGZgt{}}
\end{sphinxVerbatim}




\subsection{Con badges}
\label{\detokenize{mas-componentes:con-badges}}
\fvset{hllines={, ,}}%
\begin{sphinxVerbatim}[commandchars=\\\{\}]
\PYG{p}{\PYGZlt{}}\PYG{n+nt}{ul} \PYG{n+na}{class}\PYG{o}{=}\PYG{l+s}{\PYGZdq{}list\PYGZhy{}group\PYGZdq{}}\PYG{p}{\PYGZgt{}}
  \PYG{p}{\PYGZlt{}}\PYG{n+nt}{li} \PYG{n+na}{class}\PYG{o}{=}\PYG{l+s}{\PYGZdq{}list\PYGZhy{}group\PYGZhy{}item d\PYGZhy{}flex justify\PYGZhy{}content\PYGZhy{}between align\PYGZhy{}items\PYGZhy{}center\PYGZdq{}}\PYG{p}{\PYGZgt{}}
    Cras justo odio
    \PYG{p}{\PYGZlt{}}\PYG{n+nt}{span} \PYG{n+na}{class}\PYG{o}{=}\PYG{l+s}{\PYGZdq{}badge badge\PYGZhy{}primary badge\PYGZhy{}pill\PYGZdq{}}\PYG{p}{\PYGZgt{}}14\PYG{p}{\PYGZlt{}}\PYG{p}{/}\PYG{n+nt}{span}\PYG{p}{\PYGZgt{}}
  \PYG{p}{\PYGZlt{}}\PYG{p}{/}\PYG{n+nt}{li}\PYG{p}{\PYGZgt{}}
  \PYG{p}{\PYGZlt{}}\PYG{n+nt}{li} \PYG{n+na}{class}\PYG{o}{=}\PYG{l+s}{\PYGZdq{}list\PYGZhy{}group\PYGZhy{}item d\PYGZhy{}flex justify\PYGZhy{}content\PYGZhy{}between align\PYGZhy{}items\PYGZhy{}center\PYGZdq{}}\PYG{p}{\PYGZgt{}}
    Dapibus ac facilisis in
    \PYG{p}{\PYGZlt{}}\PYG{n+nt}{span} \PYG{n+na}{class}\PYG{o}{=}\PYG{l+s}{\PYGZdq{}badge badge\PYGZhy{}primary badge\PYGZhy{}pill\PYGZdq{}}\PYG{p}{\PYGZgt{}}2\PYG{p}{\PYGZlt{}}\PYG{p}{/}\PYG{n+nt}{span}\PYG{p}{\PYGZgt{}}
  \PYG{p}{\PYGZlt{}}\PYG{p}{/}\PYG{n+nt}{li}\PYG{p}{\PYGZgt{}}
  \PYG{p}{\PYGZlt{}}\PYG{n+nt}{li} \PYG{n+na}{class}\PYG{o}{=}\PYG{l+s}{\PYGZdq{}list\PYGZhy{}group\PYGZhy{}item d\PYGZhy{}flex justify\PYGZhy{}content\PYGZhy{}between align\PYGZhy{}items\PYGZhy{}center\PYGZdq{}}\PYG{p}{\PYGZgt{}}
    Morbi leo risus
    \PYG{p}{\PYGZlt{}}\PYG{n+nt}{span} \PYG{n+na}{class}\PYG{o}{=}\PYG{l+s}{\PYGZdq{}badge badge\PYGZhy{}primary badge\PYGZhy{}pill\PYGZdq{}}\PYG{p}{\PYGZgt{}}1\PYG{p}{\PYGZlt{}}\PYG{p}{/}\PYG{n+nt}{span}\PYG{p}{\PYGZgt{}}
  \PYG{p}{\PYGZlt{}}\PYG{p}{/}\PYG{n+nt}{li}\PYG{p}{\PYGZgt{}}
\PYG{p}{\PYGZlt{}}\PYG{p}{/}\PYG{n+nt}{ul}\PYG{p}{\PYGZgt{}}
\end{sphinxVerbatim}




\subsection{Contenido propio}
\label{\detokenize{mas-componentes:contenido-propio}}
\fvset{hllines={, ,}}%
\begin{sphinxVerbatim}[commandchars=\\\{\}]
\PYG{p}{\PYGZlt{}}\PYG{n+nt}{div} \PYG{n+na}{class}\PYG{o}{=}\PYG{l+s}{\PYGZdq{}list\PYGZhy{}group\PYGZdq{}}\PYG{p}{\PYGZgt{}}
  \PYG{p}{\PYGZlt{}}\PYG{n+nt}{a} \PYG{n+na}{href}\PYG{o}{=}\PYG{l+s}{\PYGZdq{}\PYGZsh{}\PYGZdq{}} \PYG{n+na}{class}\PYG{o}{=}\PYG{l+s}{\PYGZdq{}list\PYGZhy{}group\PYGZhy{}item list\PYGZhy{}group\PYGZhy{}item\PYGZhy{}action flex\PYGZhy{}column align\PYGZhy{}items\PYGZhy{}start active\PYGZdq{}}\PYG{p}{\PYGZgt{}}
    \PYG{p}{\PYGZlt{}}\PYG{n+nt}{div} \PYG{n+na}{class}\PYG{o}{=}\PYG{l+s}{\PYGZdq{}d\PYGZhy{}flex w\PYGZhy{}100 justify\PYGZhy{}content\PYGZhy{}between\PYGZdq{}}\PYG{p}{\PYGZgt{}}
        \PYG{p}{\PYGZlt{}}\PYG{n+nt}{h5} \PYG{n+na}{class}\PYG{o}{=}\PYG{l+s}{\PYGZdq{}mb\PYGZhy{}1\PYGZdq{}}\PYG{p}{\PYGZgt{}}Cabecera de item\PYG{p}{\PYGZlt{}}\PYG{p}{/}\PYG{n+nt}{h5}\PYG{p}{\PYGZgt{}}
        \PYG{p}{\PYGZlt{}}\PYG{n+nt}{small}\PYG{p}{\PYGZgt{}}hace 3 dias\PYG{p}{\PYGZlt{}}\PYG{p}{/}\PYG{n+nt}{small}\PYG{p}{\PYGZgt{}}
      \PYG{p}{\PYGZlt{}}\PYG{p}{/}\PYG{n+nt}{div}\PYG{p}{\PYGZgt{}}
      \PYG{p}{\PYGZlt{}}\PYG{n+nt}{p} \PYG{n+na}{class}\PYG{o}{=}\PYG{l+s}{\PYGZdq{}mb\PYGZhy{}1\PYGZdq{}}\PYG{p}{\PYGZgt{}}Donec id elit non mi porta gravida at eget metus. Maecenas sed diam eget risus varius blandit.\PYG{p}{\PYGZlt{}}\PYG{p}{/}\PYG{n+nt}{p}\PYG{p}{\PYGZgt{}}
      \PYG{p}{\PYGZlt{}}\PYG{n+nt}{small}\PYG{p}{\PYGZgt{}}Donec id elit non mi porta.\PYG{p}{\PYGZlt{}}\PYG{p}{/}\PYG{n+nt}{small}\PYG{p}{\PYGZgt{}}
    \PYG{p}{\PYGZlt{}}\PYG{p}{/}\PYG{n+nt}{a}\PYG{p}{\PYGZgt{}}
    \PYG{p}{\PYGZlt{}}\PYG{n+nt}{a} \PYG{n+na}{href}\PYG{o}{=}\PYG{l+s}{\PYGZdq{}\PYGZsh{}\PYGZdq{}} \PYG{n+na}{class}\PYG{o}{=}\PYG{l+s}{\PYGZdq{}list\PYGZhy{}group\PYGZhy{}item list\PYGZhy{}group\PYGZhy{}item\PYGZhy{}action flex\PYGZhy{}column align\PYGZhy{}items\PYGZhy{}start\PYGZdq{}}\PYG{p}{\PYGZgt{}}
    \PYG{p}{\PYGZlt{}}\PYG{n+nt}{div} \PYG{n+na}{class}\PYG{o}{=}\PYG{l+s}{\PYGZdq{}d\PYGZhy{}flex w\PYGZhy{}100 justify\PYGZhy{}content\PYGZhy{}between\PYGZdq{}}\PYG{p}{\PYGZgt{}}
      \PYG{p}{\PYGZlt{}}\PYG{n+nt}{h5} \PYG{n+na}{class}\PYG{o}{=}\PYG{l+s}{\PYGZdq{}mb\PYGZhy{}1\PYGZdq{}}\PYG{p}{\PYGZgt{}}Cabecera de item\PYG{p}{\PYGZlt{}}\PYG{p}{/}\PYG{n+nt}{h5}\PYG{p}{\PYGZgt{}}
      \PYG{p}{\PYGZlt{}}\PYG{n+nt}{small} \PYG{n+na}{class}\PYG{o}{=}\PYG{l+s}{\PYGZdq{}text\PYGZhy{}muted\PYGZdq{}}\PYG{p}{\PYGZgt{}}hace 3 dias\PYG{p}{\PYGZlt{}}\PYG{p}{/}\PYG{n+nt}{small}\PYG{p}{\PYGZgt{}}
    \PYG{p}{\PYGZlt{}}\PYG{p}{/}\PYG{n+nt}{div}\PYG{p}{\PYGZgt{}}
    \PYG{p}{\PYGZlt{}}\PYG{n+nt}{p} \PYG{n+na}{class}\PYG{o}{=}\PYG{l+s}{\PYGZdq{}mb\PYGZhy{}1\PYGZdq{}}\PYG{p}{\PYGZgt{}}Donec id elit non mi porta gravida at eget metus. Maecenas sed diam eget risus varius blandit.\PYG{p}{\PYGZlt{}}\PYG{p}{/}\PYG{n+nt}{p}\PYG{p}{\PYGZgt{}}
    \PYG{p}{\PYGZlt{}}\PYG{n+nt}{small} \PYG{n+na}{class}\PYG{o}{=}\PYG{l+s}{\PYGZdq{}text\PYGZhy{}muted\PYGZdq{}}\PYG{p}{\PYGZgt{}}Donec id elit non mi porta.\PYG{p}{\PYGZlt{}}\PYG{p}{/}\PYG{n+nt}{small}\PYG{p}{\PYGZgt{}}
  \PYG{p}{\PYGZlt{}}\PYG{p}{/}\PYG{n+nt}{a}\PYG{p}{\PYGZgt{}}
  \PYG{p}{\PYGZlt{}}\PYG{n+nt}{a} \PYG{n+na}{href}\PYG{o}{=}\PYG{l+s}{\PYGZdq{}\PYGZsh{}\PYGZdq{}} \PYG{n+na}{class}\PYG{o}{=}\PYG{l+s}{\PYGZdq{}list\PYGZhy{}group\PYGZhy{}item list\PYGZhy{}group\PYGZhy{}item\PYGZhy{}action flex\PYGZhy{}column align\PYGZhy{}items\PYGZhy{}start\PYGZdq{}}\PYG{p}{\PYGZgt{}}
    \PYG{p}{\PYGZlt{}}\PYG{n+nt}{div} \PYG{n+na}{class}\PYG{o}{=}\PYG{l+s}{\PYGZdq{}d\PYGZhy{}flex w\PYGZhy{}100 justify\PYGZhy{}content\PYGZhy{}between\PYGZdq{}}\PYG{p}{\PYGZgt{}}
      \PYG{p}{\PYGZlt{}}\PYG{n+nt}{h5} \PYG{n+na}{class}\PYG{o}{=}\PYG{l+s}{\PYGZdq{}mb\PYGZhy{}1\PYGZdq{}}\PYG{p}{\PYGZgt{}}Cabecera de item\PYG{p}{\PYGZlt{}}\PYG{p}{/}\PYG{n+nt}{h5}\PYG{p}{\PYGZgt{}}
      \PYG{p}{\PYGZlt{}}\PYG{n+nt}{small} \PYG{n+na}{class}\PYG{o}{=}\PYG{l+s}{\PYGZdq{}text\PYGZhy{}muted\PYGZdq{}}\PYG{p}{\PYGZgt{}}hace 3 dias\PYG{p}{\PYGZlt{}}\PYG{p}{/}\PYG{n+nt}{small}\PYG{p}{\PYGZgt{}}
    \PYG{p}{\PYGZlt{}}\PYG{p}{/}\PYG{n+nt}{div}\PYG{p}{\PYGZgt{}}
    \PYG{p}{\PYGZlt{}}\PYG{n+nt}{p} \PYG{n+na}{class}\PYG{o}{=}\PYG{l+s}{\PYGZdq{}mb\PYGZhy{}1\PYGZdq{}}\PYG{p}{\PYGZgt{}}Donec id elit non mi porta gravida at eget metus. Maecenas sed diam eget risus varius blandit.\PYG{p}{\PYGZlt{}}\PYG{p}{/}\PYG{n+nt}{p}\PYG{p}{\PYGZgt{}}
    \PYG{p}{\PYGZlt{}}\PYG{n+nt}{small} \PYG{n+na}{class}\PYG{o}{=}\PYG{l+s}{\PYGZdq{}text\PYGZhy{}muted\PYGZdq{}}\PYG{p}{\PYGZgt{}}Donec id elit non mi porta.\PYG{p}{\PYGZlt{}}\PYG{p}{/}\PYG{n+nt}{small}\PYG{p}{\PYGZgt{}}
  \PYG{p}{\PYGZlt{}}\PYG{p}{/}\PYG{n+nt}{a}\PYG{p}{\PYGZgt{}}
\PYG{p}{\PYGZlt{}}\PYG{p}{/}\PYG{n+nt}{div}\PYG{p}{\PYGZgt{}}
\end{sphinxVerbatim}




\section{Cartas}
\label{\detokenize{mas-componentes:cartas}}
\fvset{hllines={, ,}}%
\begin{sphinxVerbatim}[commandchars=\\\{\}]
\PYG{p}{\PYGZlt{}}\PYG{n+nt}{div} \PYG{n+na}{class}\PYG{o}{=}\PYG{l+s}{\PYGZdq{}card\PYGZdq{}} \PYG{n+na}{style}\PYG{o}{=}\PYG{l+s}{\PYGZdq{}width: 18rem;\PYGZdq{}}\PYG{p}{\PYGZgt{}}
  \PYG{p}{\PYGZlt{}}\PYG{n+nt}{img} \PYG{n+na}{class}\PYG{o}{=}\PYG{l+s}{\PYGZdq{}card\PYGZhy{}img\PYGZhy{}top\PYGZdq{}} \PYG{n+na}{src}\PYG{o}{=}\PYG{l+s}{\PYGZdq{}galleries/cew/500\PYGZhy{}500\PYGZhy{}1.jpeg\PYGZdq{}} \PYG{n+na}{alt}\PYG{o}{=}\PYG{l+s}{\PYGZdq{}Texto alternativo de imagen\PYGZdq{}}\PYG{p}{\PYGZgt{}}
  \PYG{p}{\PYGZlt{}}\PYG{n+nt}{div} \PYG{n+na}{class}\PYG{o}{=}\PYG{l+s}{\PYGZdq{}card\PYGZhy{}body\PYGZdq{}}\PYG{p}{\PYGZgt{}}
    \PYG{p}{\PYGZlt{}}\PYG{n+nt}{h5} \PYG{n+na}{class}\PYG{o}{=}\PYG{l+s}{\PYGZdq{}card\PYGZhy{}title\PYGZdq{}}\PYG{p}{\PYGZgt{}}Título\PYG{p}{\PYGZlt{}}\PYG{p}{/}\PYG{n+nt}{h5}\PYG{p}{\PYGZgt{}}
    \PYG{p}{\PYGZlt{}}\PYG{n+nt}{p} \PYG{n+na}{class}\PYG{o}{=}\PYG{l+s}{\PYGZdq{}card\PYGZhy{}text\PYGZdq{}}\PYG{p}{\PYGZgt{}}Contenido principal.\PYG{p}{\PYGZlt{}}\PYG{p}{/}\PYG{n+nt}{p}\PYG{p}{\PYGZgt{}}
    \PYG{p}{\PYGZlt{}}\PYG{n+nt}{a} \PYG{n+na}{href}\PYG{o}{=}\PYG{l+s}{\PYGZdq{}\PYGZsh{}\PYGZdq{}} \PYG{n+na}{class}\PYG{o}{=}\PYG{l+s}{\PYGZdq{}btn btn\PYGZhy{}primary\PYGZdq{}}\PYG{p}{\PYGZgt{}}Ir a algun lado\PYG{p}{\PYGZlt{}}\PYG{p}{/}\PYG{n+nt}{a}\PYG{p}{\PYGZgt{}}
  \PYG{p}{\PYGZlt{}}\PYG{p}{/}\PYG{n+nt}{div}\PYG{p}{\PYGZgt{}}
\PYG{p}{\PYGZlt{}}\PYG{p}{/}\PYG{n+nt}{div}\PYG{p}{\PYGZgt{}}
\end{sphinxVerbatim}



\fvset{hllines={, ,}}%
\begin{sphinxVerbatim}[commandchars=\\\{\}]
\PYG{p}{\PYGZlt{}}\PYG{n+nt}{div} \PYG{n+na}{class}\PYG{o}{=}\PYG{l+s}{\PYGZdq{}card\PYGZdq{}} \PYG{n+na}{style}\PYG{o}{=}\PYG{l+s}{\PYGZdq{}width: 18rem;\PYGZdq{}}\PYG{p}{\PYGZgt{}}
  \PYG{p}{\PYGZlt{}}\PYG{n+nt}{div} \PYG{n+na}{class}\PYG{o}{=}\PYG{l+s}{\PYGZdq{}card\PYGZhy{}body\PYGZdq{}}\PYG{p}{\PYGZgt{}}
    \PYG{p}{\PYGZlt{}}\PYG{n+nt}{h5} \PYG{n+na}{class}\PYG{o}{=}\PYG{l+s}{\PYGZdq{}card\PYGZhy{}title\PYGZdq{}}\PYG{p}{\PYGZgt{}}Título\PYG{p}{\PYGZlt{}}\PYG{p}{/}\PYG{n+nt}{h5}\PYG{p}{\PYGZgt{}}
    \PYG{p}{\PYGZlt{}}\PYG{n+nt}{h6} \PYG{n+na}{class}\PYG{o}{=}\PYG{l+s}{\PYGZdq{}card\PYGZhy{}subtitle mb\PYGZhy{}2 text\PYGZhy{}muted\PYGZdq{}}\PYG{p}{\PYGZgt{}}Subtitulo\PYG{p}{\PYGZlt{}}\PYG{p}{/}\PYG{n+nt}{h6}\PYG{p}{\PYGZgt{}}
    \PYG{p}{\PYGZlt{}}\PYG{n+nt}{p} \PYG{n+na}{class}\PYG{o}{=}\PYG{l+s}{\PYGZdq{}card\PYGZhy{}text\PYGZdq{}}\PYG{p}{\PYGZgt{}}Contenido principal\PYG{p}{\PYGZlt{}}\PYG{p}{/}\PYG{n+nt}{p}\PYG{p}{\PYGZgt{}}
    \PYG{p}{\PYGZlt{}}\PYG{n+nt}{a} \PYG{n+na}{href}\PYG{o}{=}\PYG{l+s}{\PYGZdq{}\PYGZsh{}\PYGZdq{}} \PYG{n+na}{class}\PYG{o}{=}\PYG{l+s}{\PYGZdq{}card\PYGZhy{}link\PYGZdq{}}\PYG{p}{\PYGZgt{}}Link\PYG{p}{\PYGZlt{}}\PYG{p}{/}\PYG{n+nt}{a}\PYG{p}{\PYGZgt{}}
    \PYG{p}{\PYGZlt{}}\PYG{n+nt}{a} \PYG{n+na}{href}\PYG{o}{=}\PYG{l+s}{\PYGZdq{}\PYGZsh{}\PYGZdq{}} \PYG{n+na}{class}\PYG{o}{=}\PYG{l+s}{\PYGZdq{}card\PYGZhy{}link\PYGZdq{}}\PYG{p}{\PYGZgt{}}Otro link\PYG{p}{\PYGZlt{}}\PYG{p}{/}\PYG{n+nt}{a}\PYG{p}{\PYGZgt{}}
  \PYG{p}{\PYGZlt{}}\PYG{p}{/}\PYG{n+nt}{div}\PYG{p}{\PYGZgt{}}
\PYG{p}{\PYGZlt{}}\PYG{p}{/}\PYG{n+nt}{div}\PYG{p}{\PYGZgt{}}
\end{sphinxVerbatim}




\subsection{Carta con lista}
\label{\detokenize{mas-componentes:carta-con-lista}}
\fvset{hllines={, ,}}%
\begin{sphinxVerbatim}[commandchars=\\\{\}]
\PYG{p}{\PYGZlt{}}\PYG{n+nt}{div} \PYG{n+na}{class}\PYG{o}{=}\PYG{l+s}{\PYGZdq{}card\PYGZdq{}} \PYG{n+na}{style}\PYG{o}{=}\PYG{l+s}{\PYGZdq{}width: 18rem;\PYGZdq{}}\PYG{p}{\PYGZgt{}}
  \PYG{p}{\PYGZlt{}}\PYG{n+nt}{ul} \PYG{n+na}{class}\PYG{o}{=}\PYG{l+s}{\PYGZdq{}list\PYGZhy{}group list\PYGZhy{}group\PYGZhy{}flush\PYGZdq{}}\PYG{p}{\PYGZgt{}}
    \PYG{p}{\PYGZlt{}}\PYG{n+nt}{li} \PYG{n+na}{class}\PYG{o}{=}\PYG{l+s}{\PYGZdq{}list\PYGZhy{}group\PYGZhy{}item\PYGZdq{}}\PYG{p}{\PYGZgt{}}Cras justo odio\PYG{p}{\PYGZlt{}}\PYG{p}{/}\PYG{n+nt}{li}\PYG{p}{\PYGZgt{}}
    \PYG{p}{\PYGZlt{}}\PYG{n+nt}{li} \PYG{n+na}{class}\PYG{o}{=}\PYG{l+s}{\PYGZdq{}list\PYGZhy{}group\PYGZhy{}item\PYGZdq{}}\PYG{p}{\PYGZgt{}}Dapibus ac facilisis in\PYG{p}{\PYGZlt{}}\PYG{p}{/}\PYG{n+nt}{li}\PYG{p}{\PYGZgt{}}
    \PYG{p}{\PYGZlt{}}\PYG{n+nt}{li} \PYG{n+na}{class}\PYG{o}{=}\PYG{l+s}{\PYGZdq{}list\PYGZhy{}group\PYGZhy{}item\PYGZdq{}}\PYG{p}{\PYGZgt{}}Vestibulum at eros\PYG{p}{\PYGZlt{}}\PYG{p}{/}\PYG{n+nt}{li}\PYG{p}{\PYGZgt{}}
  \PYG{p}{\PYGZlt{}}\PYG{p}{/}\PYG{n+nt}{ul}\PYG{p}{\PYGZgt{}}
\PYG{p}{\PYGZlt{}}\PYG{p}{/}\PYG{n+nt}{div}\PYG{p}{\PYGZgt{}}
\end{sphinxVerbatim}



\fvset{hllines={, ,}}%
\begin{sphinxVerbatim}[commandchars=\\\{\}]
\PYG{p}{\PYGZlt{}}\PYG{n+nt}{div} \PYG{n+na}{class}\PYG{o}{=}\PYG{l+s}{\PYGZdq{}card\PYGZdq{}} \PYG{n+na}{style}\PYG{o}{=}\PYG{l+s}{\PYGZdq{}width: 18rem;\PYGZdq{}}\PYG{p}{\PYGZgt{}}
  \PYG{p}{\PYGZlt{}}\PYG{n+nt}{div} \PYG{n+na}{class}\PYG{o}{=}\PYG{l+s}{\PYGZdq{}card\PYGZhy{}header\PYGZdq{}}\PYG{p}{\PYGZgt{}}
    Cabecera
  \PYG{p}{\PYGZlt{}}\PYG{p}{/}\PYG{n+nt}{div}\PYG{p}{\PYGZgt{}}
  \PYG{p}{\PYGZlt{}}\PYG{n+nt}{ul} \PYG{n+na}{class}\PYG{o}{=}\PYG{l+s}{\PYGZdq{}list\PYGZhy{}group list\PYGZhy{}group\PYGZhy{}flush\PYGZdq{}}\PYG{p}{\PYGZgt{}}
    \PYG{p}{\PYGZlt{}}\PYG{n+nt}{li} \PYG{n+na}{class}\PYG{o}{=}\PYG{l+s}{\PYGZdq{}list\PYGZhy{}group\PYGZhy{}item\PYGZdq{}}\PYG{p}{\PYGZgt{}}Cras justo odio\PYG{p}{\PYGZlt{}}\PYG{p}{/}\PYG{n+nt}{li}\PYG{p}{\PYGZgt{}}
    \PYG{p}{\PYGZlt{}}\PYG{n+nt}{li} \PYG{n+na}{class}\PYG{o}{=}\PYG{l+s}{\PYGZdq{}list\PYGZhy{}group\PYGZhy{}item\PYGZdq{}}\PYG{p}{\PYGZgt{}}Dapibus ac facilisis in\PYG{p}{\PYGZlt{}}\PYG{p}{/}\PYG{n+nt}{li}\PYG{p}{\PYGZgt{}}
    \PYG{p}{\PYGZlt{}}\PYG{n+nt}{li} \PYG{n+na}{class}\PYG{o}{=}\PYG{l+s}{\PYGZdq{}list\PYGZhy{}group\PYGZhy{}item\PYGZdq{}}\PYG{p}{\PYGZgt{}}Vestibulum at eros\PYG{p}{\PYGZlt{}}\PYG{p}{/}\PYG{n+nt}{li}\PYG{p}{\PYGZgt{}}
  \PYG{p}{\PYGZlt{}}\PYG{p}{/}\PYG{n+nt}{ul}\PYG{p}{\PYGZgt{}}
\PYG{p}{\PYGZlt{}}\PYG{p}{/}\PYG{n+nt}{div}\PYG{p}{\PYGZgt{}}
\end{sphinxVerbatim}




\subsection{Cambalache}
\label{\detokenize{mas-componentes:cambalache}}
\fvset{hllines={, ,}}%
\begin{sphinxVerbatim}[commandchars=\\\{\}]
\PYG{p}{\PYGZlt{}}\PYG{n+nt}{div} \PYG{n+na}{class}\PYG{o}{=}\PYG{l+s}{\PYGZdq{}card\PYGZdq{}} \PYG{n+na}{style}\PYG{o}{=}\PYG{l+s}{\PYGZdq{}width: 18rem;\PYGZdq{}}\PYG{p}{\PYGZgt{}}
  \PYG{p}{\PYGZlt{}}\PYG{n+nt}{img} \PYG{n+na}{class}\PYG{o}{=}\PYG{l+s}{\PYGZdq{}card\PYGZhy{}img\PYGZhy{}top\PYGZdq{}} \PYG{n+na}{src}\PYG{o}{=}\PYG{l+s}{\PYGZdq{}galleries/cew/500\PYGZhy{}500\PYGZhy{}1.jpeg\PYGZdq{}} \PYG{n+na}{alt}\PYG{o}{=}\PYG{l+s}{\PYGZdq{}Text alternativo\PYGZdq{}}\PYG{p}{\PYGZgt{}}
  \PYG{p}{\PYGZlt{}}\PYG{n+nt}{div} \PYG{n+na}{class}\PYG{o}{=}\PYG{l+s}{\PYGZdq{}card\PYGZhy{}body\PYGZdq{}}\PYG{p}{\PYGZgt{}}
    \PYG{p}{\PYGZlt{}}\PYG{n+nt}{h5} \PYG{n+na}{class}\PYG{o}{=}\PYG{l+s}{\PYGZdq{}card\PYGZhy{}title\PYGZdq{}}\PYG{p}{\PYGZgt{}}Título\PYG{p}{\PYGZlt{}}\PYG{p}{/}\PYG{n+nt}{h5}\PYG{p}{\PYGZgt{}}
    \PYG{p}{\PYGZlt{}}\PYG{n+nt}{p} \PYG{n+na}{class}\PYG{o}{=}\PYG{l+s}{\PYGZdq{}card\PYGZhy{}text\PYGZdq{}}\PYG{p}{\PYGZgt{}}Contenido principal\PYG{p}{\PYGZlt{}}\PYG{p}{/}\PYG{n+nt}{p}\PYG{p}{\PYGZgt{}}
  \PYG{p}{\PYGZlt{}}\PYG{p}{/}\PYG{n+nt}{div}\PYG{p}{\PYGZgt{}}
  \PYG{p}{\PYGZlt{}}\PYG{n+nt}{ul} \PYG{n+na}{class}\PYG{o}{=}\PYG{l+s}{\PYGZdq{}list\PYGZhy{}group list\PYGZhy{}group\PYGZhy{}flush\PYGZdq{}}\PYG{p}{\PYGZgt{}}
    \PYG{p}{\PYGZlt{}}\PYG{n+nt}{li} \PYG{n+na}{class}\PYG{o}{=}\PYG{l+s}{\PYGZdq{}list\PYGZhy{}group\PYGZhy{}item\PYGZdq{}}\PYG{p}{\PYGZgt{}}Cras justo odio\PYG{p}{\PYGZlt{}}\PYG{p}{/}\PYG{n+nt}{li}\PYG{p}{\PYGZgt{}}
    \PYG{p}{\PYGZlt{}}\PYG{n+nt}{li} \PYG{n+na}{class}\PYG{o}{=}\PYG{l+s}{\PYGZdq{}list\PYGZhy{}group\PYGZhy{}item\PYGZdq{}}\PYG{p}{\PYGZgt{}}Dapibus ac facilisis in\PYG{p}{\PYGZlt{}}\PYG{p}{/}\PYG{n+nt}{li}\PYG{p}{\PYGZgt{}}
    \PYG{p}{\PYGZlt{}}\PYG{n+nt}{li} \PYG{n+na}{class}\PYG{o}{=}\PYG{l+s}{\PYGZdq{}list\PYGZhy{}group\PYGZhy{}item\PYGZdq{}}\PYG{p}{\PYGZgt{}}Vestibulum at eros\PYG{p}{\PYGZlt{}}\PYG{p}{/}\PYG{n+nt}{li}\PYG{p}{\PYGZgt{}}
  \PYG{p}{\PYGZlt{}}\PYG{p}{/}\PYG{n+nt}{ul}\PYG{p}{\PYGZgt{}}
  \PYG{p}{\PYGZlt{}}\PYG{n+nt}{div} \PYG{n+na}{class}\PYG{o}{=}\PYG{l+s}{\PYGZdq{}card\PYGZhy{}body\PYGZdq{}}\PYG{p}{\PYGZgt{}}
    \PYG{p}{\PYGZlt{}}\PYG{n+nt}{a} \PYG{n+na}{href}\PYG{o}{=}\PYG{l+s}{\PYGZdq{}\PYGZsh{}\PYGZdq{}} \PYG{n+na}{class}\PYG{o}{=}\PYG{l+s}{\PYGZdq{}card\PYGZhy{}link\PYGZdq{}}\PYG{p}{\PYGZgt{}}Link\PYG{p}{\PYGZlt{}}\PYG{p}{/}\PYG{n+nt}{a}\PYG{p}{\PYGZgt{}}
    \PYG{p}{\PYGZlt{}}\PYG{n+nt}{a} \PYG{n+na}{href}\PYG{o}{=}\PYG{l+s}{\PYGZdq{}\PYGZsh{}\PYGZdq{}} \PYG{n+na}{class}\PYG{o}{=}\PYG{l+s}{\PYGZdq{}card\PYGZhy{}link\PYGZdq{}}\PYG{p}{\PYGZgt{}}Otro link\PYG{p}{\PYGZlt{}}\PYG{p}{/}\PYG{n+nt}{a}\PYG{p}{\PYGZgt{}}
  \PYG{p}{\PYGZlt{}}\PYG{p}{/}\PYG{n+nt}{div}\PYG{p}{\PYGZgt{}}
\PYG{p}{\PYGZlt{}}\PYG{p}{/}\PYG{n+nt}{div}\PYG{p}{\PYGZgt{}}
\end{sphinxVerbatim}




\subsection{Cabecera y pie}
\label{\detokenize{mas-componentes:cabecera-y-pie}}
\fvset{hllines={, ,}}%
\begin{sphinxVerbatim}[commandchars=\\\{\}]
\PYG{p}{\PYGZlt{}}\PYG{n+nt}{div} \PYG{n+na}{class}\PYG{o}{=}\PYG{l+s}{\PYGZdq{}card text\PYGZhy{}center\PYGZdq{}} \PYG{n+na}{style}\PYG{o}{=}\PYG{l+s}{\PYGZdq{}width: 18rem\PYGZdq{}}\PYG{p}{\PYGZgt{}}
  \PYG{p}{\PYGZlt{}}\PYG{n+nt}{div} \PYG{n+na}{class}\PYG{o}{=}\PYG{l+s}{\PYGZdq{}card\PYGZhy{}header\PYGZdq{}}\PYG{p}{\PYGZgt{}}
    Cabecera
  \PYG{p}{\PYGZlt{}}\PYG{p}{/}\PYG{n+nt}{div}\PYG{p}{\PYGZgt{}}
  \PYG{p}{\PYGZlt{}}\PYG{n+nt}{div} \PYG{n+na}{class}\PYG{o}{=}\PYG{l+s}{\PYGZdq{}card\PYGZhy{}body\PYGZdq{}}\PYG{p}{\PYGZgt{}}
    \PYG{p}{\PYGZlt{}}\PYG{n+nt}{h5} \PYG{n+na}{class}\PYG{o}{=}\PYG{l+s}{\PYGZdq{}card\PYGZhy{}title\PYGZdq{}}\PYG{p}{\PYGZgt{}}Título\PYG{p}{\PYGZlt{}}\PYG{p}{/}\PYG{n+nt}{h5}\PYG{p}{\PYGZgt{}}
    \PYG{p}{\PYGZlt{}}\PYG{n+nt}{p} \PYG{n+na}{class}\PYG{o}{=}\PYG{l+s}{\PYGZdq{}card\PYGZhy{}text\PYGZdq{}}\PYG{p}{\PYGZgt{}}Contenido\PYG{p}{\PYGZlt{}}\PYG{p}{/}\PYG{n+nt}{p}\PYG{p}{\PYGZgt{}}
    \PYG{p}{\PYGZlt{}}\PYG{n+nt}{a} \PYG{n+na}{href}\PYG{o}{=}\PYG{l+s}{\PYGZdq{}\PYGZsh{}\PYGZdq{}} \PYG{n+na}{class}\PYG{o}{=}\PYG{l+s}{\PYGZdq{}btn btn\PYGZhy{}primary\PYGZdq{}}\PYG{p}{\PYGZgt{}}Botón\PYG{p}{\PYGZlt{}}\PYG{p}{/}\PYG{n+nt}{a}\PYG{p}{\PYGZgt{}}
  \PYG{p}{\PYGZlt{}}\PYG{p}{/}\PYG{n+nt}{div}\PYG{p}{\PYGZgt{}}
  \PYG{p}{\PYGZlt{}}\PYG{n+nt}{div} \PYG{n+na}{class}\PYG{o}{=}\PYG{l+s}{\PYGZdq{}card\PYGZhy{}footer text\PYGZhy{}muted\PYGZdq{}}\PYG{p}{\PYGZgt{}}
    Pie
  \PYG{p}{\PYGZlt{}}\PYG{p}{/}\PYG{n+nt}{div}\PYG{p}{\PYGZgt{}}
\PYG{p}{\PYGZlt{}}\PYG{p}{/}\PYG{n+nt}{div}\PYG{p}{\PYGZgt{}}
\end{sphinxVerbatim}




\subsection{Alineación de texto}
\label{\detokenize{mas-componentes:alineacion-de-texto}}
\fvset{hllines={, ,}}%
\begin{sphinxVerbatim}[commandchars=\\\{\}]
\PYG{p}{\PYGZlt{}}\PYG{n+nt}{div} \PYG{n+na}{class}\PYG{o}{=}\PYG{l+s}{\PYGZdq{}card\PYGZdq{}} \PYG{n+na}{style}\PYG{o}{=}\PYG{l+s}{\PYGZdq{}width: 18rem;\PYGZdq{}}\PYG{p}{\PYGZgt{}}
  \PYG{p}{\PYGZlt{}}\PYG{n+nt}{div} \PYG{n+na}{class}\PYG{o}{=}\PYG{l+s}{\PYGZdq{}card\PYGZhy{}body\PYGZdq{}}\PYG{p}{\PYGZgt{}}
    \PYG{p}{\PYGZlt{}}\PYG{n+nt}{h5} \PYG{n+na}{class}\PYG{o}{=}\PYG{l+s}{\PYGZdq{}card\PYGZhy{}title\PYGZdq{}}\PYG{p}{\PYGZgt{}}Título\PYG{p}{\PYGZlt{}}\PYG{p}{/}\PYG{n+nt}{h5}\PYG{p}{\PYGZgt{}}
    \PYG{p}{\PYGZlt{}}\PYG{n+nt}{p} \PYG{n+na}{class}\PYG{o}{=}\PYG{l+s}{\PYGZdq{}card\PYGZhy{}text\PYGZdq{}}\PYG{p}{\PYGZgt{}}Contenido\PYG{p}{\PYGZlt{}}\PYG{p}{/}\PYG{n+nt}{p}\PYG{p}{\PYGZgt{}}
    \PYG{p}{\PYGZlt{}}\PYG{n+nt}{a} \PYG{n+na}{href}\PYG{o}{=}\PYG{l+s}{\PYGZdq{}\PYGZsh{}\PYGZdq{}} \PYG{n+na}{class}\PYG{o}{=}\PYG{l+s}{\PYGZdq{}btn btn\PYGZhy{}primary\PYGZdq{}}\PYG{p}{\PYGZgt{}}Botón\PYG{p}{\PYGZlt{}}\PYG{p}{/}\PYG{n+nt}{a}\PYG{p}{\PYGZgt{}}
  \PYG{p}{\PYGZlt{}}\PYG{p}{/}\PYG{n+nt}{div}\PYG{p}{\PYGZgt{}}
\PYG{p}{\PYGZlt{}}\PYG{p}{/}\PYG{n+nt}{div}\PYG{p}{\PYGZgt{}}
\end{sphinxVerbatim}



\fvset{hllines={, ,}}%
\begin{sphinxVerbatim}[commandchars=\\\{\}]
\PYG{p}{\PYGZlt{}}\PYG{n+nt}{div} \PYG{n+na}{class}\PYG{o}{=}\PYG{l+s}{\PYGZdq{}card text\PYGZhy{}center\PYGZdq{}} \PYG{n+na}{style}\PYG{o}{=}\PYG{l+s}{\PYGZdq{}width: 18rem;\PYGZdq{}}\PYG{p}{\PYGZgt{}}
  \PYG{p}{\PYGZlt{}}\PYG{n+nt}{div} \PYG{n+na}{class}\PYG{o}{=}\PYG{l+s}{\PYGZdq{}card\PYGZhy{}body\PYGZdq{}}\PYG{p}{\PYGZgt{}}
    \PYG{p}{\PYGZlt{}}\PYG{n+nt}{h5} \PYG{n+na}{class}\PYG{o}{=}\PYG{l+s}{\PYGZdq{}card\PYGZhy{}title\PYGZdq{}}\PYG{p}{\PYGZgt{}}Título\PYG{p}{\PYGZlt{}}\PYG{p}{/}\PYG{n+nt}{h5}\PYG{p}{\PYGZgt{}}
    \PYG{p}{\PYGZlt{}}\PYG{n+nt}{p} \PYG{n+na}{class}\PYG{o}{=}\PYG{l+s}{\PYGZdq{}card\PYGZhy{}text\PYGZdq{}}\PYG{p}{\PYGZgt{}}Contenido\PYG{p}{\PYGZlt{}}\PYG{p}{/}\PYG{n+nt}{p}\PYG{p}{\PYGZgt{}}
    \PYG{p}{\PYGZlt{}}\PYG{n+nt}{a} \PYG{n+na}{href}\PYG{o}{=}\PYG{l+s}{\PYGZdq{}\PYGZsh{}\PYGZdq{}} \PYG{n+na}{class}\PYG{o}{=}\PYG{l+s}{\PYGZdq{}btn btn\PYGZhy{}primary\PYGZdq{}}\PYG{p}{\PYGZgt{}}Botón\PYG{p}{\PYGZlt{}}\PYG{p}{/}\PYG{n+nt}{a}\PYG{p}{\PYGZgt{}}
  \PYG{p}{\PYGZlt{}}\PYG{p}{/}\PYG{n+nt}{div}\PYG{p}{\PYGZgt{}}
\PYG{p}{\PYGZlt{}}\PYG{p}{/}\PYG{n+nt}{div}\PYG{p}{\PYGZgt{}}
\end{sphinxVerbatim}



\fvset{hllines={, ,}}%
\begin{sphinxVerbatim}[commandchars=\\\{\}]
\PYG{p}{\PYGZlt{}}\PYG{n+nt}{div} \PYG{n+na}{class}\PYG{o}{=}\PYG{l+s}{\PYGZdq{}card text\PYGZhy{}right\PYGZdq{}} \PYG{n+na}{style}\PYG{o}{=}\PYG{l+s}{\PYGZdq{}width: 18rem;\PYGZdq{}}\PYG{p}{\PYGZgt{}}
  \PYG{p}{\PYGZlt{}}\PYG{n+nt}{div} \PYG{n+na}{class}\PYG{o}{=}\PYG{l+s}{\PYGZdq{}card\PYGZhy{}body\PYGZdq{}}\PYG{p}{\PYGZgt{}}
    \PYG{p}{\PYGZlt{}}\PYG{n+nt}{h5} \PYG{n+na}{class}\PYG{o}{=}\PYG{l+s}{\PYGZdq{}card\PYGZhy{}title\PYGZdq{}}\PYG{p}{\PYGZgt{}}Título\PYG{p}{\PYGZlt{}}\PYG{p}{/}\PYG{n+nt}{h5}\PYG{p}{\PYGZgt{}}
    \PYG{p}{\PYGZlt{}}\PYG{n+nt}{p} \PYG{n+na}{class}\PYG{o}{=}\PYG{l+s}{\PYGZdq{}card\PYGZhy{}text\PYGZdq{}}\PYG{p}{\PYGZgt{}}Contenido\PYG{p}{\PYGZlt{}}\PYG{p}{/}\PYG{n+nt}{p}\PYG{p}{\PYGZgt{}}
    \PYG{p}{\PYGZlt{}}\PYG{n+nt}{a} \PYG{n+na}{href}\PYG{o}{=}\PYG{l+s}{\PYGZdq{}\PYGZsh{}\PYGZdq{}} \PYG{n+na}{class}\PYG{o}{=}\PYG{l+s}{\PYGZdq{}btn btn\PYGZhy{}primary\PYGZdq{}}\PYG{p}{\PYGZgt{}}Botón\PYG{p}{\PYGZlt{}}\PYG{p}{/}\PYG{n+nt}{a}\PYG{p}{\PYGZgt{}}
  \PYG{p}{\PYGZlt{}}\PYG{p}{/}\PYG{n+nt}{div}\PYG{p}{\PYGZgt{}}
\PYG{p}{\PYGZlt{}}\PYG{p}{/}\PYG{n+nt}{div}\PYG{p}{\PYGZgt{}}
\end{sphinxVerbatim}




\subsection{Navegación}
\label{\detokenize{mas-componentes:navegacion}}
\fvset{hllines={, ,}}%
\begin{sphinxVerbatim}[commandchars=\\\{\}]
\PYG{p}{\PYGZlt{}}\PYG{n+nt}{div} \PYG{n+na}{class}\PYG{o}{=}\PYG{l+s}{\PYGZdq{}card text\PYGZhy{}center\PYGZdq{}} \PYG{n+na}{style}\PYG{o}{=}\PYG{l+s}{\PYGZdq{}width: 18rem\PYGZdq{}}\PYG{p}{\PYGZgt{}}
  \PYG{p}{\PYGZlt{}}\PYG{n+nt}{div} \PYG{n+na}{class}\PYG{o}{=}\PYG{l+s}{\PYGZdq{}card\PYGZhy{}header\PYGZdq{}}\PYG{p}{\PYGZgt{}}
    \PYG{p}{\PYGZlt{}}\PYG{n+nt}{ul} \PYG{n+na}{class}\PYG{o}{=}\PYG{l+s}{\PYGZdq{}nav nav\PYGZhy{}tabs card\PYGZhy{}header\PYGZhy{}tabs\PYGZdq{}}\PYG{p}{\PYGZgt{}}
      \PYG{p}{\PYGZlt{}}\PYG{n+nt}{li} \PYG{n+na}{class}\PYG{o}{=}\PYG{l+s}{\PYGZdq{}nav\PYGZhy{}item\PYGZdq{}}\PYG{p}{\PYGZgt{}}
        \PYG{p}{\PYGZlt{}}\PYG{n+nt}{a} \PYG{n+na}{class}\PYG{o}{=}\PYG{l+s}{\PYGZdq{}nav\PYGZhy{}link active\PYGZdq{}} \PYG{n+na}{href}\PYG{o}{=}\PYG{l+s}{\PYGZdq{}\PYGZsh{}\PYGZdq{}}\PYG{p}{\PYGZgt{}}Activo\PYG{p}{\PYGZlt{}}\PYG{p}{/}\PYG{n+nt}{a}\PYG{p}{\PYGZgt{}}
      \PYG{p}{\PYGZlt{}}\PYG{p}{/}\PYG{n+nt}{li}\PYG{p}{\PYGZgt{}}
      \PYG{p}{\PYGZlt{}}\PYG{n+nt}{li} \PYG{n+na}{class}\PYG{o}{=}\PYG{l+s}{\PYGZdq{}nav\PYGZhy{}item\PYGZdq{}}\PYG{p}{\PYGZgt{}}
        \PYG{p}{\PYGZlt{}}\PYG{n+nt}{a} \PYG{n+na}{class}\PYG{o}{=}\PYG{l+s}{\PYGZdq{}nav\PYGZhy{}link\PYGZdq{}} \PYG{n+na}{href}\PYG{o}{=}\PYG{l+s}{\PYGZdq{}\PYGZsh{}\PYGZdq{}}\PYG{p}{\PYGZgt{}}Link\PYG{p}{\PYGZlt{}}\PYG{p}{/}\PYG{n+nt}{a}\PYG{p}{\PYGZgt{}}
      \PYG{p}{\PYGZlt{}}\PYG{p}{/}\PYG{n+nt}{li}\PYG{p}{\PYGZgt{}}
      \PYG{p}{\PYGZlt{}}\PYG{n+nt}{li} \PYG{n+na}{class}\PYG{o}{=}\PYG{l+s}{\PYGZdq{}nav\PYGZhy{}item\PYGZdq{}}\PYG{p}{\PYGZgt{}}
        \PYG{p}{\PYGZlt{}}\PYG{n+nt}{a} \PYG{n+na}{class}\PYG{o}{=}\PYG{l+s}{\PYGZdq{}nav\PYGZhy{}link disabled\PYGZdq{}} \PYG{n+na}{href}\PYG{o}{=}\PYG{l+s}{\PYGZdq{}\PYGZsh{}\PYGZdq{}}\PYG{p}{\PYGZgt{}}Inactivo\PYG{p}{\PYGZlt{}}\PYG{p}{/}\PYG{n+nt}{a}\PYG{p}{\PYGZgt{}}
      \PYG{p}{\PYGZlt{}}\PYG{p}{/}\PYG{n+nt}{li}\PYG{p}{\PYGZgt{}}
    \PYG{p}{\PYGZlt{}}\PYG{p}{/}\PYG{n+nt}{ul}\PYG{p}{\PYGZgt{}}
  \PYG{p}{\PYGZlt{}}\PYG{p}{/}\PYG{n+nt}{div}\PYG{p}{\PYGZgt{}}
  \PYG{p}{\PYGZlt{}}\PYG{n+nt}{div} \PYG{n+na}{class}\PYG{o}{=}\PYG{l+s}{\PYGZdq{}card\PYGZhy{}body\PYGZdq{}}\PYG{p}{\PYGZgt{}}
    \PYG{p}{\PYGZlt{}}\PYG{n+nt}{h5} \PYG{n+na}{class}\PYG{o}{=}\PYG{l+s}{\PYGZdq{}card\PYGZhy{}title\PYGZdq{}}\PYG{p}{\PYGZgt{}}Título\PYG{p}{\PYGZlt{}}\PYG{p}{/}\PYG{n+nt}{h5}\PYG{p}{\PYGZgt{}}
    \PYG{p}{\PYGZlt{}}\PYG{n+nt}{p} \PYG{n+na}{class}\PYG{o}{=}\PYG{l+s}{\PYGZdq{}card\PYGZhy{}text\PYGZdq{}}\PYG{p}{\PYGZgt{}}Contenido\PYG{p}{\PYGZlt{}}\PYG{p}{/}\PYG{n+nt}{p}\PYG{p}{\PYGZgt{}}
    \PYG{p}{\PYGZlt{}}\PYG{n+nt}{a} \PYG{n+na}{href}\PYG{o}{=}\PYG{l+s}{\PYGZdq{}\PYGZsh{}\PYGZdq{}} \PYG{n+na}{class}\PYG{o}{=}\PYG{l+s}{\PYGZdq{}btn btn\PYGZhy{}primary\PYGZdq{}}\PYG{p}{\PYGZgt{}}Botón\PYG{p}{\PYGZlt{}}\PYG{p}{/}\PYG{n+nt}{a}\PYG{p}{\PYGZgt{}}
  \PYG{p}{\PYGZlt{}}\PYG{p}{/}\PYG{n+nt}{div}\PYG{p}{\PYGZgt{}}
\PYG{p}{\PYGZlt{}}\PYG{p}{/}\PYG{n+nt}{div}\PYG{p}{\PYGZgt{}}
\end{sphinxVerbatim}



\fvset{hllines={, ,}}%
\begin{sphinxVerbatim}[commandchars=\\\{\}]
\PYG{p}{\PYGZlt{}}\PYG{n+nt}{div} \PYG{n+na}{class}\PYG{o}{=}\PYG{l+s}{\PYGZdq{}card text\PYGZhy{}center\PYGZdq{}} \PYG{n+na}{style}\PYG{o}{=}\PYG{l+s}{\PYGZdq{}width: 18rem\PYGZdq{}}\PYG{p}{\PYGZgt{}}
  \PYG{p}{\PYGZlt{}}\PYG{n+nt}{div} \PYG{n+na}{class}\PYG{o}{=}\PYG{l+s}{\PYGZdq{}card\PYGZhy{}header\PYGZdq{}}\PYG{p}{\PYGZgt{}}
    \PYG{p}{\PYGZlt{}}\PYG{n+nt}{ul} \PYG{n+na}{class}\PYG{o}{=}\PYG{l+s}{\PYGZdq{}nav nav\PYGZhy{}pills card\PYGZhy{}header\PYGZhy{}pills\PYGZdq{}}\PYG{p}{\PYGZgt{}}
      \PYG{p}{\PYGZlt{}}\PYG{n+nt}{li} \PYG{n+na}{class}\PYG{o}{=}\PYG{l+s}{\PYGZdq{}nav\PYGZhy{}item\PYGZdq{}}\PYG{p}{\PYGZgt{}}
        \PYG{p}{\PYGZlt{}}\PYG{n+nt}{a} \PYG{n+na}{class}\PYG{o}{=}\PYG{l+s}{\PYGZdq{}nav\PYGZhy{}link active\PYGZdq{}} \PYG{n+na}{href}\PYG{o}{=}\PYG{l+s}{\PYGZdq{}\PYGZsh{}\PYGZdq{}}\PYG{p}{\PYGZgt{}}Activo\PYG{p}{\PYGZlt{}}\PYG{p}{/}\PYG{n+nt}{a}\PYG{p}{\PYGZgt{}}
      \PYG{p}{\PYGZlt{}}\PYG{p}{/}\PYG{n+nt}{li}\PYG{p}{\PYGZgt{}}
      \PYG{p}{\PYGZlt{}}\PYG{n+nt}{li} \PYG{n+na}{class}\PYG{o}{=}\PYG{l+s}{\PYGZdq{}nav\PYGZhy{}item\PYGZdq{}}\PYG{p}{\PYGZgt{}}
        \PYG{p}{\PYGZlt{}}\PYG{n+nt}{a} \PYG{n+na}{class}\PYG{o}{=}\PYG{l+s}{\PYGZdq{}nav\PYGZhy{}link\PYGZdq{}} \PYG{n+na}{href}\PYG{o}{=}\PYG{l+s}{\PYGZdq{}\PYGZsh{}\PYGZdq{}}\PYG{p}{\PYGZgt{}}Link\PYG{p}{\PYGZlt{}}\PYG{p}{/}\PYG{n+nt}{a}\PYG{p}{\PYGZgt{}}
      \PYG{p}{\PYGZlt{}}\PYG{p}{/}\PYG{n+nt}{li}\PYG{p}{\PYGZgt{}}
      \PYG{p}{\PYGZlt{}}\PYG{n+nt}{li} \PYG{n+na}{class}\PYG{o}{=}\PYG{l+s}{\PYGZdq{}nav\PYGZhy{}item\PYGZdq{}}\PYG{p}{\PYGZgt{}}
        \PYG{p}{\PYGZlt{}}\PYG{n+nt}{a} \PYG{n+na}{class}\PYG{o}{=}\PYG{l+s}{\PYGZdq{}nav\PYGZhy{}link disabled\PYGZdq{}} \PYG{n+na}{href}\PYG{o}{=}\PYG{l+s}{\PYGZdq{}\PYGZsh{}\PYGZdq{}}\PYG{p}{\PYGZgt{}}Inactivo\PYG{p}{\PYGZlt{}}\PYG{p}{/}\PYG{n+nt}{a}\PYG{p}{\PYGZgt{}}
      \PYG{p}{\PYGZlt{}}\PYG{p}{/}\PYG{n+nt}{li}\PYG{p}{\PYGZgt{}}
    \PYG{p}{\PYGZlt{}}\PYG{p}{/}\PYG{n+nt}{ul}\PYG{p}{\PYGZgt{}}
  \PYG{p}{\PYGZlt{}}\PYG{p}{/}\PYG{n+nt}{div}\PYG{p}{\PYGZgt{}}
  \PYG{p}{\PYGZlt{}}\PYG{n+nt}{div} \PYG{n+na}{class}\PYG{o}{=}\PYG{l+s}{\PYGZdq{}card\PYGZhy{}body\PYGZdq{}}\PYG{p}{\PYGZgt{}}
    \PYG{p}{\PYGZlt{}}\PYG{n+nt}{h5} \PYG{n+na}{class}\PYG{o}{=}\PYG{l+s}{\PYGZdq{}card\PYGZhy{}title\PYGZdq{}}\PYG{p}{\PYGZgt{}}Título\PYG{p}{\PYGZlt{}}\PYG{p}{/}\PYG{n+nt}{h5}\PYG{p}{\PYGZgt{}}
    \PYG{p}{\PYGZlt{}}\PYG{n+nt}{p} \PYG{n+na}{class}\PYG{o}{=}\PYG{l+s}{\PYGZdq{}card\PYGZhy{}text\PYGZdq{}}\PYG{p}{\PYGZgt{}}Contenido\PYG{p}{\PYGZlt{}}\PYG{p}{/}\PYG{n+nt}{p}\PYG{p}{\PYGZgt{}}
    \PYG{p}{\PYGZlt{}}\PYG{n+nt}{a} \PYG{n+na}{href}\PYG{o}{=}\PYG{l+s}{\PYGZdq{}\PYGZsh{}\PYGZdq{}} \PYG{n+na}{class}\PYG{o}{=}\PYG{l+s}{\PYGZdq{}btn btn\PYGZhy{}primary\PYGZdq{}}\PYG{p}{\PYGZgt{}}Botón\PYG{p}{\PYGZlt{}}\PYG{p}{/}\PYG{n+nt}{a}\PYG{p}{\PYGZgt{}}
  \PYG{p}{\PYGZlt{}}\PYG{p}{/}\PYG{n+nt}{div}\PYG{p}{\PYGZgt{}}
\PYG{p}{\PYGZlt{}}\PYG{p}{/}\PYG{n+nt}{div}\PYG{p}{\PYGZgt{}}
\end{sphinxVerbatim}




\subsection{Estilos de cartas}
\label{\detokenize{mas-componentes:estilos-de-cartas}}
\fvset{hllines={, ,}}%
\begin{sphinxVerbatim}[commandchars=\\\{\}]
\PYG{p}{\PYGZlt{}}\PYG{n+nt}{div} \PYG{n+na}{class}\PYG{o}{=}\PYG{l+s}{\PYGZdq{}card text\PYGZhy{}white bg\PYGZhy{}primary mb\PYGZhy{}3\PYGZdq{}} \PYG{n+na}{style}\PYG{o}{=}\PYG{l+s}{\PYGZdq{}max\PYGZhy{}width: 18rem;\PYGZdq{}}\PYG{p}{\PYGZgt{}}
  \PYG{p}{\PYGZlt{}}\PYG{n+nt}{div} \PYG{n+na}{class}\PYG{o}{=}\PYG{l+s}{\PYGZdq{}card\PYGZhy{}header\PYGZdq{}}\PYG{p}{\PYGZgt{}}Cabecera\PYG{p}{\PYGZlt{}}\PYG{p}{/}\PYG{n+nt}{div}\PYG{p}{\PYGZgt{}}
  \PYG{p}{\PYGZlt{}}\PYG{n+nt}{div} \PYG{n+na}{class}\PYG{o}{=}\PYG{l+s}{\PYGZdq{}card\PYGZhy{}body\PYGZdq{}}\PYG{p}{\PYGZgt{}}
    \PYG{p}{\PYGZlt{}}\PYG{n+nt}{h5} \PYG{n+na}{class}\PYG{o}{=}\PYG{l+s}{\PYGZdq{}card\PYGZhy{}title\PYGZdq{}}\PYG{p}{\PYGZgt{}}Título\PYG{p}{\PYGZlt{}}\PYG{p}{/}\PYG{n+nt}{h5}\PYG{p}{\PYGZgt{}}
    \PYG{p}{\PYGZlt{}}\PYG{n+nt}{p} \PYG{n+na}{class}\PYG{o}{=}\PYG{l+s}{\PYGZdq{}card\PYGZhy{}text\PYGZdq{}}\PYG{p}{\PYGZgt{}}Cuerpo de carta\PYG{p}{\PYGZlt{}}\PYG{p}{/}\PYG{n+nt}{p}\PYG{p}{\PYGZgt{}}
  \PYG{p}{\PYGZlt{}}\PYG{p}{/}\PYG{n+nt}{div}\PYG{p}{\PYGZgt{}}
\PYG{p}{\PYGZlt{}}\PYG{p}{/}\PYG{n+nt}{div}\PYG{p}{\PYGZgt{}}

\PYG{p}{\PYGZlt{}}\PYG{n+nt}{br}\PYG{p}{\PYGZgt{}}

\PYG{p}{\PYGZlt{}}\PYG{n+nt}{div} \PYG{n+na}{class}\PYG{o}{=}\PYG{l+s}{\PYGZdq{}card text\PYGZhy{}white bg\PYGZhy{}secondary mb\PYGZhy{}3\PYGZdq{}} \PYG{n+na}{style}\PYG{o}{=}\PYG{l+s}{\PYGZdq{}max\PYGZhy{}width: 18rem;\PYGZdq{}}\PYG{p}{\PYGZgt{}}
  \PYG{p}{\PYGZlt{}}\PYG{n+nt}{div} \PYG{n+na}{class}\PYG{o}{=}\PYG{l+s}{\PYGZdq{}card\PYGZhy{}header\PYGZdq{}}\PYG{p}{\PYGZgt{}}Cabecera\PYG{p}{\PYGZlt{}}\PYG{p}{/}\PYG{n+nt}{div}\PYG{p}{\PYGZgt{}}
  \PYG{p}{\PYGZlt{}}\PYG{n+nt}{div} \PYG{n+na}{class}\PYG{o}{=}\PYG{l+s}{\PYGZdq{}card\PYGZhy{}body\PYGZdq{}}\PYG{p}{\PYGZgt{}}
    \PYG{p}{\PYGZlt{}}\PYG{n+nt}{h5} \PYG{n+na}{class}\PYG{o}{=}\PYG{l+s}{\PYGZdq{}card\PYGZhy{}title\PYGZdq{}}\PYG{p}{\PYGZgt{}}Título\PYG{p}{\PYGZlt{}}\PYG{p}{/}\PYG{n+nt}{h5}\PYG{p}{\PYGZgt{}}
    \PYG{p}{\PYGZlt{}}\PYG{n+nt}{p} \PYG{n+na}{class}\PYG{o}{=}\PYG{l+s}{\PYGZdq{}card\PYGZhy{}text\PYGZdq{}}\PYG{p}{\PYGZgt{}}Cuerpo de carta\PYG{p}{\PYGZlt{}}\PYG{p}{/}\PYG{n+nt}{p}\PYG{p}{\PYGZgt{}}
  \PYG{p}{\PYGZlt{}}\PYG{p}{/}\PYG{n+nt}{div}\PYG{p}{\PYGZgt{}}
\PYG{p}{\PYGZlt{}}\PYG{p}{/}\PYG{n+nt}{div}\PYG{p}{\PYGZgt{}}

\PYG{p}{\PYGZlt{}}\PYG{n+nt}{br}\PYG{p}{\PYGZgt{}}

\PYG{p}{\PYGZlt{}}\PYG{n+nt}{div} \PYG{n+na}{class}\PYG{o}{=}\PYG{l+s}{\PYGZdq{}card text\PYGZhy{}white bg\PYGZhy{}success mb\PYGZhy{}3\PYGZdq{}} \PYG{n+na}{style}\PYG{o}{=}\PYG{l+s}{\PYGZdq{}max\PYGZhy{}width: 18rem;\PYGZdq{}}\PYG{p}{\PYGZgt{}}
  \PYG{p}{\PYGZlt{}}\PYG{n+nt}{div} \PYG{n+na}{class}\PYG{o}{=}\PYG{l+s}{\PYGZdq{}card\PYGZhy{}header\PYGZdq{}}\PYG{p}{\PYGZgt{}}Cabecera\PYG{p}{\PYGZlt{}}\PYG{p}{/}\PYG{n+nt}{div}\PYG{p}{\PYGZgt{}}
  \PYG{p}{\PYGZlt{}}\PYG{n+nt}{div} \PYG{n+na}{class}\PYG{o}{=}\PYG{l+s}{\PYGZdq{}card\PYGZhy{}body\PYGZdq{}}\PYG{p}{\PYGZgt{}}
    \PYG{p}{\PYGZlt{}}\PYG{n+nt}{h5} \PYG{n+na}{class}\PYG{o}{=}\PYG{l+s}{\PYGZdq{}card\PYGZhy{}title\PYGZdq{}}\PYG{p}{\PYGZgt{}}Título\PYG{p}{\PYGZlt{}}\PYG{p}{/}\PYG{n+nt}{h5}\PYG{p}{\PYGZgt{}}
    \PYG{p}{\PYGZlt{}}\PYG{n+nt}{p} \PYG{n+na}{class}\PYG{o}{=}\PYG{l+s}{\PYGZdq{}card\PYGZhy{}text\PYGZdq{}}\PYG{p}{\PYGZgt{}}Cuerpo de carta\PYG{p}{\PYGZlt{}}\PYG{p}{/}\PYG{n+nt}{p}\PYG{p}{\PYGZgt{}}
  \PYG{p}{\PYGZlt{}}\PYG{p}{/}\PYG{n+nt}{div}\PYG{p}{\PYGZgt{}}
\PYG{p}{\PYGZlt{}}\PYG{p}{/}\PYG{n+nt}{div}\PYG{p}{\PYGZgt{}}

\PYG{p}{\PYGZlt{}}\PYG{n+nt}{br}\PYG{p}{\PYGZgt{}}

\PYG{p}{\PYGZlt{}}\PYG{n+nt}{div} \PYG{n+na}{class}\PYG{o}{=}\PYG{l+s}{\PYGZdq{}card text\PYGZhy{}white bg\PYGZhy{}danger mb\PYGZhy{}3\PYGZdq{}} \PYG{n+na}{style}\PYG{o}{=}\PYG{l+s}{\PYGZdq{}max\PYGZhy{}width: 18rem;\PYGZdq{}}\PYG{p}{\PYGZgt{}}
  \PYG{p}{\PYGZlt{}}\PYG{n+nt}{div} \PYG{n+na}{class}\PYG{o}{=}\PYG{l+s}{\PYGZdq{}card\PYGZhy{}header\PYGZdq{}}\PYG{p}{\PYGZgt{}}Cabecera\PYG{p}{\PYGZlt{}}\PYG{p}{/}\PYG{n+nt}{div}\PYG{p}{\PYGZgt{}}
  \PYG{p}{\PYGZlt{}}\PYG{n+nt}{div} \PYG{n+na}{class}\PYG{o}{=}\PYG{l+s}{\PYGZdq{}card\PYGZhy{}body\PYGZdq{}}\PYG{p}{\PYGZgt{}}
    \PYG{p}{\PYGZlt{}}\PYG{n+nt}{h5} \PYG{n+na}{class}\PYG{o}{=}\PYG{l+s}{\PYGZdq{}card\PYGZhy{}title\PYGZdq{}}\PYG{p}{\PYGZgt{}}Título\PYG{p}{\PYGZlt{}}\PYG{p}{/}\PYG{n+nt}{h5}\PYG{p}{\PYGZgt{}}
    \PYG{p}{\PYGZlt{}}\PYG{n+nt}{p} \PYG{n+na}{class}\PYG{o}{=}\PYG{l+s}{\PYGZdq{}card\PYGZhy{}text\PYGZdq{}}\PYG{p}{\PYGZgt{}}Cuerpo de carta\PYG{p}{\PYGZlt{}}\PYG{p}{/}\PYG{n+nt}{p}\PYG{p}{\PYGZgt{}}
  \PYG{p}{\PYGZlt{}}\PYG{p}{/}\PYG{n+nt}{div}\PYG{p}{\PYGZgt{}}
\PYG{p}{\PYGZlt{}}\PYG{p}{/}\PYG{n+nt}{div}\PYG{p}{\PYGZgt{}}

\PYG{p}{\PYGZlt{}}\PYG{n+nt}{br}\PYG{p}{\PYGZgt{}}

\PYG{p}{\PYGZlt{}}\PYG{n+nt}{div} \PYG{n+na}{class}\PYG{o}{=}\PYG{l+s}{\PYGZdq{}card text\PYGZhy{}white bg\PYGZhy{}warning mb\PYGZhy{}3\PYGZdq{}} \PYG{n+na}{style}\PYG{o}{=}\PYG{l+s}{\PYGZdq{}max\PYGZhy{}width: 18rem;\PYGZdq{}}\PYG{p}{\PYGZgt{}}
  \PYG{p}{\PYGZlt{}}\PYG{n+nt}{div} \PYG{n+na}{class}\PYG{o}{=}\PYG{l+s}{\PYGZdq{}card\PYGZhy{}header\PYGZdq{}}\PYG{p}{\PYGZgt{}}Cabecera\PYG{p}{\PYGZlt{}}\PYG{p}{/}\PYG{n+nt}{div}\PYG{p}{\PYGZgt{}}
  \PYG{p}{\PYGZlt{}}\PYG{n+nt}{div} \PYG{n+na}{class}\PYG{o}{=}\PYG{l+s}{\PYGZdq{}card\PYGZhy{}body\PYGZdq{}}\PYG{p}{\PYGZgt{}}
    \PYG{p}{\PYGZlt{}}\PYG{n+nt}{h5} \PYG{n+na}{class}\PYG{o}{=}\PYG{l+s}{\PYGZdq{}card\PYGZhy{}title\PYGZdq{}}\PYG{p}{\PYGZgt{}}Título\PYG{p}{\PYGZlt{}}\PYG{p}{/}\PYG{n+nt}{h5}\PYG{p}{\PYGZgt{}}
    \PYG{p}{\PYGZlt{}}\PYG{n+nt}{p} \PYG{n+na}{class}\PYG{o}{=}\PYG{l+s}{\PYGZdq{}card\PYGZhy{}text\PYGZdq{}}\PYG{p}{\PYGZgt{}}Cuerpo de carta\PYG{p}{\PYGZlt{}}\PYG{p}{/}\PYG{n+nt}{p}\PYG{p}{\PYGZgt{}}
  \PYG{p}{\PYGZlt{}}\PYG{p}{/}\PYG{n+nt}{div}\PYG{p}{\PYGZgt{}}
\PYG{p}{\PYGZlt{}}\PYG{p}{/}\PYG{n+nt}{div}\PYG{p}{\PYGZgt{}}

\PYG{p}{\PYGZlt{}}\PYG{n+nt}{br}\PYG{p}{\PYGZgt{}}

\PYG{p}{\PYGZlt{}}\PYG{n+nt}{div} \PYG{n+na}{class}\PYG{o}{=}\PYG{l+s}{\PYGZdq{}card text\PYGZhy{}white bg\PYGZhy{}info mb\PYGZhy{}3\PYGZdq{}} \PYG{n+na}{style}\PYG{o}{=}\PYG{l+s}{\PYGZdq{}max\PYGZhy{}width: 18rem;\PYGZdq{}}\PYG{p}{\PYGZgt{}}
  \PYG{p}{\PYGZlt{}}\PYG{n+nt}{div} \PYG{n+na}{class}\PYG{o}{=}\PYG{l+s}{\PYGZdq{}card\PYGZhy{}header\PYGZdq{}}\PYG{p}{\PYGZgt{}}Cabecera\PYG{p}{\PYGZlt{}}\PYG{p}{/}\PYG{n+nt}{div}\PYG{p}{\PYGZgt{}}
  \PYG{p}{\PYGZlt{}}\PYG{n+nt}{div} \PYG{n+na}{class}\PYG{o}{=}\PYG{l+s}{\PYGZdq{}card\PYGZhy{}body\PYGZdq{}}\PYG{p}{\PYGZgt{}}
    \PYG{p}{\PYGZlt{}}\PYG{n+nt}{h5} \PYG{n+na}{class}\PYG{o}{=}\PYG{l+s}{\PYGZdq{}card\PYGZhy{}title\PYGZdq{}}\PYG{p}{\PYGZgt{}}Título\PYG{p}{\PYGZlt{}}\PYG{p}{/}\PYG{n+nt}{h5}\PYG{p}{\PYGZgt{}}
    \PYG{p}{\PYGZlt{}}\PYG{n+nt}{p} \PYG{n+na}{class}\PYG{o}{=}\PYG{l+s}{\PYGZdq{}card\PYGZhy{}text\PYGZdq{}}\PYG{p}{\PYGZgt{}}Cuerpo de carta\PYG{p}{\PYGZlt{}}\PYG{p}{/}\PYG{n+nt}{p}\PYG{p}{\PYGZgt{}}
  \PYG{p}{\PYGZlt{}}\PYG{p}{/}\PYG{n+nt}{div}\PYG{p}{\PYGZgt{}}
\PYG{p}{\PYGZlt{}}\PYG{p}{/}\PYG{n+nt}{div}\PYG{p}{\PYGZgt{}}

\PYG{p}{\PYGZlt{}}\PYG{n+nt}{br}\PYG{p}{\PYGZgt{}}

\PYG{p}{\PYGZlt{}}\PYG{n+nt}{div} \PYG{n+na}{class}\PYG{o}{=}\PYG{l+s}{\PYGZdq{}card bg\PYGZhy{}light mb\PYGZhy{}3\PYGZdq{}} \PYG{n+na}{style}\PYG{o}{=}\PYG{l+s}{\PYGZdq{}max\PYGZhy{}width: 18rem;\PYGZdq{}}\PYG{p}{\PYGZgt{}}
  \PYG{p}{\PYGZlt{}}\PYG{n+nt}{div} \PYG{n+na}{class}\PYG{o}{=}\PYG{l+s}{\PYGZdq{}card\PYGZhy{}header\PYGZdq{}}\PYG{p}{\PYGZgt{}}Cabecera\PYG{p}{\PYGZlt{}}\PYG{p}{/}\PYG{n+nt}{div}\PYG{p}{\PYGZgt{}}
  \PYG{p}{\PYGZlt{}}\PYG{n+nt}{div} \PYG{n+na}{class}\PYG{o}{=}\PYG{l+s}{\PYGZdq{}card\PYGZhy{}body\PYGZdq{}}\PYG{p}{\PYGZgt{}}
    \PYG{p}{\PYGZlt{}}\PYG{n+nt}{h5} \PYG{n+na}{class}\PYG{o}{=}\PYG{l+s}{\PYGZdq{}card\PYGZhy{}title\PYGZdq{}}\PYG{p}{\PYGZgt{}}Título\PYG{p}{\PYGZlt{}}\PYG{p}{/}\PYG{n+nt}{h5}\PYG{p}{\PYGZgt{}}
    \PYG{p}{\PYGZlt{}}\PYG{n+nt}{p} \PYG{n+na}{class}\PYG{o}{=}\PYG{l+s}{\PYGZdq{}card\PYGZhy{}text\PYGZdq{}}\PYG{p}{\PYGZgt{}}Cuerpo de carta\PYG{p}{\PYGZlt{}}\PYG{p}{/}\PYG{n+nt}{p}\PYG{p}{\PYGZgt{}}
  \PYG{p}{\PYGZlt{}}\PYG{p}{/}\PYG{n+nt}{div}\PYG{p}{\PYGZgt{}}
\PYG{p}{\PYGZlt{}}\PYG{p}{/}\PYG{n+nt}{div}\PYG{p}{\PYGZgt{}}

\PYG{p}{\PYGZlt{}}\PYG{n+nt}{br}\PYG{p}{\PYGZgt{}}

\PYG{p}{\PYGZlt{}}\PYG{n+nt}{div} \PYG{n+na}{class}\PYG{o}{=}\PYG{l+s}{\PYGZdq{}card text\PYGZhy{}white bg\PYGZhy{}dark mb\PYGZhy{}3\PYGZdq{}} \PYG{n+na}{style}\PYG{o}{=}\PYG{l+s}{\PYGZdq{}max\PYGZhy{}width: 18rem;\PYGZdq{}}\PYG{p}{\PYGZgt{}}
  \PYG{p}{\PYGZlt{}}\PYG{n+nt}{div} \PYG{n+na}{class}\PYG{o}{=}\PYG{l+s}{\PYGZdq{}card\PYGZhy{}header\PYGZdq{}}\PYG{p}{\PYGZgt{}}Cabecera\PYG{p}{\PYGZlt{}}\PYG{p}{/}\PYG{n+nt}{div}\PYG{p}{\PYGZgt{}}
  \PYG{p}{\PYGZlt{}}\PYG{n+nt}{div} \PYG{n+na}{class}\PYG{o}{=}\PYG{l+s}{\PYGZdq{}card\PYGZhy{}body\PYGZdq{}}\PYG{p}{\PYGZgt{}}
    \PYG{p}{\PYGZlt{}}\PYG{n+nt}{h5} \PYG{n+na}{class}\PYG{o}{=}\PYG{l+s}{\PYGZdq{}card\PYGZhy{}title\PYGZdq{}}\PYG{p}{\PYGZgt{}}Título\PYG{p}{\PYGZlt{}}\PYG{p}{/}\PYG{n+nt}{h5}\PYG{p}{\PYGZgt{}}
    \PYG{p}{\PYGZlt{}}\PYG{n+nt}{p} \PYG{n+na}{class}\PYG{o}{=}\PYG{l+s}{\PYGZdq{}card\PYGZhy{}text\PYGZdq{}}\PYG{p}{\PYGZgt{}}Cuerpo de carta\PYG{p}{\PYGZlt{}}\PYG{p}{/}\PYG{n+nt}{p}\PYG{p}{\PYGZgt{}}
  \PYG{p}{\PYGZlt{}}\PYG{p}{/}\PYG{n+nt}{div}\PYG{p}{\PYGZgt{}}
\PYG{p}{\PYGZlt{}}\PYG{p}{/}\PYG{n+nt}{div}\PYG{p}{\PYGZgt{}}
\end{sphinxVerbatim}




\subsection{Estilos de cartas (bordes)}
\label{\detokenize{mas-componentes:estilos-de-cartas-bordes}}
\fvset{hllines={, ,}}%
\begin{sphinxVerbatim}[commandchars=\\\{\}]
\PYG{p}{\PYGZlt{}}\PYG{n+nt}{div} \PYG{n+na}{class}\PYG{o}{=}\PYG{l+s}{\PYGZdq{}card border\PYGZhy{}primary mb\PYGZhy{}3\PYGZdq{}} \PYG{n+na}{style}\PYG{o}{=}\PYG{l+s}{\PYGZdq{}max\PYGZhy{}width: 18rem;\PYGZdq{}}\PYG{p}{\PYGZgt{}}
  \PYG{p}{\PYGZlt{}}\PYG{n+nt}{div} \PYG{n+na}{class}\PYG{o}{=}\PYG{l+s}{\PYGZdq{}card\PYGZhy{}header\PYGZdq{}}\PYG{p}{\PYGZgt{}}Cabecera\PYG{p}{\PYGZlt{}}\PYG{p}{/}\PYG{n+nt}{div}\PYG{p}{\PYGZgt{}}
  \PYG{p}{\PYGZlt{}}\PYG{n+nt}{div} \PYG{n+na}{class}\PYG{o}{=}\PYG{l+s}{\PYGZdq{}card\PYGZhy{}body\PYGZdq{}}\PYG{p}{\PYGZgt{}}
    \PYG{p}{\PYGZlt{}}\PYG{n+nt}{h5} \PYG{n+na}{class}\PYG{o}{=}\PYG{l+s}{\PYGZdq{}card\PYGZhy{}title\PYGZdq{}}\PYG{p}{\PYGZgt{}}Título\PYG{p}{\PYGZlt{}}\PYG{p}{/}\PYG{n+nt}{h5}\PYG{p}{\PYGZgt{}}
    \PYG{p}{\PYGZlt{}}\PYG{n+nt}{p} \PYG{n+na}{class}\PYG{o}{=}\PYG{l+s}{\PYGZdq{}card\PYGZhy{}text\PYGZdq{}}\PYG{p}{\PYGZgt{}}Cuerpo de carta\PYG{p}{\PYGZlt{}}\PYG{p}{/}\PYG{n+nt}{p}\PYG{p}{\PYGZgt{}}
  \PYG{p}{\PYGZlt{}}\PYG{p}{/}\PYG{n+nt}{div}\PYG{p}{\PYGZgt{}}
\PYG{p}{\PYGZlt{}}\PYG{p}{/}\PYG{n+nt}{div}\PYG{p}{\PYGZgt{}}

\PYG{p}{\PYGZlt{}}\PYG{n+nt}{br}\PYG{p}{\PYGZgt{}}

\PYG{p}{\PYGZlt{}}\PYG{n+nt}{div} \PYG{n+na}{class}\PYG{o}{=}\PYG{l+s}{\PYGZdq{}card border\PYGZhy{}secondary mb\PYGZhy{}3\PYGZdq{}} \PYG{n+na}{style}\PYG{o}{=}\PYG{l+s}{\PYGZdq{}max\PYGZhy{}width: 18rem;\PYGZdq{}}\PYG{p}{\PYGZgt{}}
  \PYG{p}{\PYGZlt{}}\PYG{n+nt}{div} \PYG{n+na}{class}\PYG{o}{=}\PYG{l+s}{\PYGZdq{}card\PYGZhy{}header\PYGZdq{}}\PYG{p}{\PYGZgt{}}Cabecera\PYG{p}{\PYGZlt{}}\PYG{p}{/}\PYG{n+nt}{div}\PYG{p}{\PYGZgt{}}
  \PYG{p}{\PYGZlt{}}\PYG{n+nt}{div} \PYG{n+na}{class}\PYG{o}{=}\PYG{l+s}{\PYGZdq{}card\PYGZhy{}body\PYGZdq{}}\PYG{p}{\PYGZgt{}}
    \PYG{p}{\PYGZlt{}}\PYG{n+nt}{h5} \PYG{n+na}{class}\PYG{o}{=}\PYG{l+s}{\PYGZdq{}card\PYGZhy{}title\PYGZdq{}}\PYG{p}{\PYGZgt{}}Título\PYG{p}{\PYGZlt{}}\PYG{p}{/}\PYG{n+nt}{h5}\PYG{p}{\PYGZgt{}}
    \PYG{p}{\PYGZlt{}}\PYG{n+nt}{p} \PYG{n+na}{class}\PYG{o}{=}\PYG{l+s}{\PYGZdq{}card\PYGZhy{}text\PYGZdq{}}\PYG{p}{\PYGZgt{}}Cuerpo de carta\PYG{p}{\PYGZlt{}}\PYG{p}{/}\PYG{n+nt}{p}\PYG{p}{\PYGZgt{}}
  \PYG{p}{\PYGZlt{}}\PYG{p}{/}\PYG{n+nt}{div}\PYG{p}{\PYGZgt{}}
\PYG{p}{\PYGZlt{}}\PYG{p}{/}\PYG{n+nt}{div}\PYG{p}{\PYGZgt{}}

\PYG{p}{\PYGZlt{}}\PYG{n+nt}{br}\PYG{p}{\PYGZgt{}}

\PYG{p}{\PYGZlt{}}\PYG{n+nt}{div} \PYG{n+na}{class}\PYG{o}{=}\PYG{l+s}{\PYGZdq{}card border\PYGZhy{}success mb\PYGZhy{}3\PYGZdq{}} \PYG{n+na}{style}\PYG{o}{=}\PYG{l+s}{\PYGZdq{}max\PYGZhy{}width: 18rem;\PYGZdq{}}\PYG{p}{\PYGZgt{}}
  \PYG{p}{\PYGZlt{}}\PYG{n+nt}{div} \PYG{n+na}{class}\PYG{o}{=}\PYG{l+s}{\PYGZdq{}card\PYGZhy{}header\PYGZdq{}}\PYG{p}{\PYGZgt{}}Cabecera\PYG{p}{\PYGZlt{}}\PYG{p}{/}\PYG{n+nt}{div}\PYG{p}{\PYGZgt{}}
  \PYG{p}{\PYGZlt{}}\PYG{n+nt}{div} \PYG{n+na}{class}\PYG{o}{=}\PYG{l+s}{\PYGZdq{}card\PYGZhy{}body\PYGZdq{}}\PYG{p}{\PYGZgt{}}
    \PYG{p}{\PYGZlt{}}\PYG{n+nt}{h5} \PYG{n+na}{class}\PYG{o}{=}\PYG{l+s}{\PYGZdq{}card\PYGZhy{}title\PYGZdq{}}\PYG{p}{\PYGZgt{}}Título\PYG{p}{\PYGZlt{}}\PYG{p}{/}\PYG{n+nt}{h5}\PYG{p}{\PYGZgt{}}
    \PYG{p}{\PYGZlt{}}\PYG{n+nt}{p} \PYG{n+na}{class}\PYG{o}{=}\PYG{l+s}{\PYGZdq{}card\PYGZhy{}text\PYGZdq{}}\PYG{p}{\PYGZgt{}}Cuerpo de carta\PYG{p}{\PYGZlt{}}\PYG{p}{/}\PYG{n+nt}{p}\PYG{p}{\PYGZgt{}}
  \PYG{p}{\PYGZlt{}}\PYG{p}{/}\PYG{n+nt}{div}\PYG{p}{\PYGZgt{}}
\PYG{p}{\PYGZlt{}}\PYG{p}{/}\PYG{n+nt}{div}\PYG{p}{\PYGZgt{}}

\PYG{p}{\PYGZlt{}}\PYG{n+nt}{br}\PYG{p}{\PYGZgt{}}

\PYG{p}{\PYGZlt{}}\PYG{n+nt}{div} \PYG{n+na}{class}\PYG{o}{=}\PYG{l+s}{\PYGZdq{}card border\PYGZhy{}danger mb\PYGZhy{}3\PYGZdq{}} \PYG{n+na}{style}\PYG{o}{=}\PYG{l+s}{\PYGZdq{}max\PYGZhy{}width: 18rem;\PYGZdq{}}\PYG{p}{\PYGZgt{}}
  \PYG{p}{\PYGZlt{}}\PYG{n+nt}{div} \PYG{n+na}{class}\PYG{o}{=}\PYG{l+s}{\PYGZdq{}card\PYGZhy{}header\PYGZdq{}}\PYG{p}{\PYGZgt{}}Cabecera\PYG{p}{\PYGZlt{}}\PYG{p}{/}\PYG{n+nt}{div}\PYG{p}{\PYGZgt{}}
  \PYG{p}{\PYGZlt{}}\PYG{n+nt}{div} \PYG{n+na}{class}\PYG{o}{=}\PYG{l+s}{\PYGZdq{}card\PYGZhy{}body\PYGZdq{}}\PYG{p}{\PYGZgt{}}
    \PYG{p}{\PYGZlt{}}\PYG{n+nt}{h5} \PYG{n+na}{class}\PYG{o}{=}\PYG{l+s}{\PYGZdq{}card\PYGZhy{}title\PYGZdq{}}\PYG{p}{\PYGZgt{}}Título\PYG{p}{\PYGZlt{}}\PYG{p}{/}\PYG{n+nt}{h5}\PYG{p}{\PYGZgt{}}
    \PYG{p}{\PYGZlt{}}\PYG{n+nt}{p} \PYG{n+na}{class}\PYG{o}{=}\PYG{l+s}{\PYGZdq{}card\PYGZhy{}text\PYGZdq{}}\PYG{p}{\PYGZgt{}}Cuerpo de carta\PYG{p}{\PYGZlt{}}\PYG{p}{/}\PYG{n+nt}{p}\PYG{p}{\PYGZgt{}}
  \PYG{p}{\PYGZlt{}}\PYG{p}{/}\PYG{n+nt}{div}\PYG{p}{\PYGZgt{}}
\PYG{p}{\PYGZlt{}}\PYG{p}{/}\PYG{n+nt}{div}\PYG{p}{\PYGZgt{}}

\PYG{p}{\PYGZlt{}}\PYG{n+nt}{br}\PYG{p}{\PYGZgt{}}

\PYG{p}{\PYGZlt{}}\PYG{n+nt}{div} \PYG{n+na}{class}\PYG{o}{=}\PYG{l+s}{\PYGZdq{}card border\PYGZhy{}warning mb\PYGZhy{}3\PYGZdq{}} \PYG{n+na}{style}\PYG{o}{=}\PYG{l+s}{\PYGZdq{}max\PYGZhy{}width: 18rem;\PYGZdq{}}\PYG{p}{\PYGZgt{}}
  \PYG{p}{\PYGZlt{}}\PYG{n+nt}{div} \PYG{n+na}{class}\PYG{o}{=}\PYG{l+s}{\PYGZdq{}card\PYGZhy{}header\PYGZdq{}}\PYG{p}{\PYGZgt{}}Cabecera\PYG{p}{\PYGZlt{}}\PYG{p}{/}\PYG{n+nt}{div}\PYG{p}{\PYGZgt{}}
  \PYG{p}{\PYGZlt{}}\PYG{n+nt}{div} \PYG{n+na}{class}\PYG{o}{=}\PYG{l+s}{\PYGZdq{}card\PYGZhy{}body\PYGZdq{}}\PYG{p}{\PYGZgt{}}
    \PYG{p}{\PYGZlt{}}\PYG{n+nt}{h5} \PYG{n+na}{class}\PYG{o}{=}\PYG{l+s}{\PYGZdq{}card\PYGZhy{}title\PYGZdq{}}\PYG{p}{\PYGZgt{}}Título\PYG{p}{\PYGZlt{}}\PYG{p}{/}\PYG{n+nt}{h5}\PYG{p}{\PYGZgt{}}
    \PYG{p}{\PYGZlt{}}\PYG{n+nt}{p} \PYG{n+na}{class}\PYG{o}{=}\PYG{l+s}{\PYGZdq{}card\PYGZhy{}text\PYGZdq{}}\PYG{p}{\PYGZgt{}}Cuerpo de carta\PYG{p}{\PYGZlt{}}\PYG{p}{/}\PYG{n+nt}{p}\PYG{p}{\PYGZgt{}}
  \PYG{p}{\PYGZlt{}}\PYG{p}{/}\PYG{n+nt}{div}\PYG{p}{\PYGZgt{}}
\PYG{p}{\PYGZlt{}}\PYG{p}{/}\PYG{n+nt}{div}\PYG{p}{\PYGZgt{}}

\PYG{p}{\PYGZlt{}}\PYG{n+nt}{br}\PYG{p}{\PYGZgt{}}

\PYG{p}{\PYGZlt{}}\PYG{n+nt}{div} \PYG{n+na}{class}\PYG{o}{=}\PYG{l+s}{\PYGZdq{}card border\PYGZhy{}info mb\PYGZhy{}3\PYGZdq{}} \PYG{n+na}{style}\PYG{o}{=}\PYG{l+s}{\PYGZdq{}max\PYGZhy{}width: 18rem;\PYGZdq{}}\PYG{p}{\PYGZgt{}}
  \PYG{p}{\PYGZlt{}}\PYG{n+nt}{div} \PYG{n+na}{class}\PYG{o}{=}\PYG{l+s}{\PYGZdq{}card\PYGZhy{}header\PYGZdq{}}\PYG{p}{\PYGZgt{}}Cabecera\PYG{p}{\PYGZlt{}}\PYG{p}{/}\PYG{n+nt}{div}\PYG{p}{\PYGZgt{}}
  \PYG{p}{\PYGZlt{}}\PYG{n+nt}{div} \PYG{n+na}{class}\PYG{o}{=}\PYG{l+s}{\PYGZdq{}card\PYGZhy{}body\PYGZdq{}}\PYG{p}{\PYGZgt{}}
    \PYG{p}{\PYGZlt{}}\PYG{n+nt}{h5} \PYG{n+na}{class}\PYG{o}{=}\PYG{l+s}{\PYGZdq{}card\PYGZhy{}title\PYGZdq{}}\PYG{p}{\PYGZgt{}}Título\PYG{p}{\PYGZlt{}}\PYG{p}{/}\PYG{n+nt}{h5}\PYG{p}{\PYGZgt{}}
    \PYG{p}{\PYGZlt{}}\PYG{n+nt}{p} \PYG{n+na}{class}\PYG{o}{=}\PYG{l+s}{\PYGZdq{}card\PYGZhy{}text\PYGZdq{}}\PYG{p}{\PYGZgt{}}Cuerpo de carta\PYG{p}{\PYGZlt{}}\PYG{p}{/}\PYG{n+nt}{p}\PYG{p}{\PYGZgt{}}
  \PYG{p}{\PYGZlt{}}\PYG{p}{/}\PYG{n+nt}{div}\PYG{p}{\PYGZgt{}}
\PYG{p}{\PYGZlt{}}\PYG{p}{/}\PYG{n+nt}{div}\PYG{p}{\PYGZgt{}}

\PYG{p}{\PYGZlt{}}\PYG{n+nt}{br}\PYG{p}{\PYGZgt{}}

\PYG{p}{\PYGZlt{}}\PYG{n+nt}{div} \PYG{n+na}{class}\PYG{o}{=}\PYG{l+s}{\PYGZdq{}card bg\PYGZhy{}light mb\PYGZhy{}3\PYGZdq{}} \PYG{n+na}{style}\PYG{o}{=}\PYG{l+s}{\PYGZdq{}max\PYGZhy{}width: 18rem;\PYGZdq{}}\PYG{p}{\PYGZgt{}}
  \PYG{p}{\PYGZlt{}}\PYG{n+nt}{div} \PYG{n+na}{class}\PYG{o}{=}\PYG{l+s}{\PYGZdq{}card\PYGZhy{}header\PYGZdq{}}\PYG{p}{\PYGZgt{}}Cabecera\PYG{p}{\PYGZlt{}}\PYG{p}{/}\PYG{n+nt}{div}\PYG{p}{\PYGZgt{}}
  \PYG{p}{\PYGZlt{}}\PYG{n+nt}{div} \PYG{n+na}{class}\PYG{o}{=}\PYG{l+s}{\PYGZdq{}card\PYGZhy{}body\PYGZdq{}}\PYG{p}{\PYGZgt{}}
    \PYG{p}{\PYGZlt{}}\PYG{n+nt}{h5} \PYG{n+na}{class}\PYG{o}{=}\PYG{l+s}{\PYGZdq{}card\PYGZhy{}title\PYGZdq{}}\PYG{p}{\PYGZgt{}}Título\PYG{p}{\PYGZlt{}}\PYG{p}{/}\PYG{n+nt}{h5}\PYG{p}{\PYGZgt{}}
    \PYG{p}{\PYGZlt{}}\PYG{n+nt}{p} \PYG{n+na}{class}\PYG{o}{=}\PYG{l+s}{\PYGZdq{}card\PYGZhy{}text\PYGZdq{}}\PYG{p}{\PYGZgt{}}Cuerpo de carta\PYG{p}{\PYGZlt{}}\PYG{p}{/}\PYG{n+nt}{p}\PYG{p}{\PYGZgt{}}
  \PYG{p}{\PYGZlt{}}\PYG{p}{/}\PYG{n+nt}{div}\PYG{p}{\PYGZgt{}}
\PYG{p}{\PYGZlt{}}\PYG{p}{/}\PYG{n+nt}{div}\PYG{p}{\PYGZgt{}}

\PYG{p}{\PYGZlt{}}\PYG{n+nt}{br}\PYG{p}{\PYGZgt{}}

\PYG{p}{\PYGZlt{}}\PYG{n+nt}{div} \PYG{n+na}{class}\PYG{o}{=}\PYG{l+s}{\PYGZdq{}card border\PYGZhy{}dark mb\PYGZhy{}3\PYGZdq{}} \PYG{n+na}{style}\PYG{o}{=}\PYG{l+s}{\PYGZdq{}max\PYGZhy{}width: 18rem;\PYGZdq{}}\PYG{p}{\PYGZgt{}}
  \PYG{p}{\PYGZlt{}}\PYG{n+nt}{div} \PYG{n+na}{class}\PYG{o}{=}\PYG{l+s}{\PYGZdq{}card\PYGZhy{}header\PYGZdq{}}\PYG{p}{\PYGZgt{}}Cabecera\PYG{p}{\PYGZlt{}}\PYG{p}{/}\PYG{n+nt}{div}\PYG{p}{\PYGZgt{}}
  \PYG{p}{\PYGZlt{}}\PYG{n+nt}{div} \PYG{n+na}{class}\PYG{o}{=}\PYG{l+s}{\PYGZdq{}card\PYGZhy{}body\PYGZdq{}}\PYG{p}{\PYGZgt{}}
    \PYG{p}{\PYGZlt{}}\PYG{n+nt}{h5} \PYG{n+na}{class}\PYG{o}{=}\PYG{l+s}{\PYGZdq{}card\PYGZhy{}title\PYGZdq{}}\PYG{p}{\PYGZgt{}}Título\PYG{p}{\PYGZlt{}}\PYG{p}{/}\PYG{n+nt}{h5}\PYG{p}{\PYGZgt{}}
    \PYG{p}{\PYGZlt{}}\PYG{n+nt}{p} \PYG{n+na}{class}\PYG{o}{=}\PYG{l+s}{\PYGZdq{}card\PYGZhy{}text\PYGZdq{}}\PYG{p}{\PYGZgt{}}Cuerpo de carta\PYG{p}{\PYGZlt{}}\PYG{p}{/}\PYG{n+nt}{p}\PYG{p}{\PYGZgt{}}
  \PYG{p}{\PYGZlt{}}\PYG{p}{/}\PYG{n+nt}{div}\PYG{p}{\PYGZgt{}}
\PYG{p}{\PYGZlt{}}\PYG{p}{/}\PYG{n+nt}{div}\PYG{p}{\PYGZgt{}}
\end{sphinxVerbatim}




\section{Grupo de cartas}
\label{\detokenize{mas-componentes:grupo-de-cartas}}
\fvset{hllines={, ,}}%
\begin{sphinxVerbatim}[commandchars=\\\{\}]
\PYG{p}{\PYGZlt{}}\PYG{n+nt}{div} \PYG{n+na}{class}\PYG{o}{=}\PYG{l+s}{\PYGZdq{}card\PYGZhy{}group\PYGZdq{}}\PYG{p}{\PYGZgt{}}
  \PYG{p}{\PYGZlt{}}\PYG{n+nt}{div} \PYG{n+na}{class}\PYG{o}{=}\PYG{l+s}{\PYGZdq{}card\PYGZdq{}}\PYG{p}{\PYGZgt{}}
    \PYG{p}{\PYGZlt{}}\PYG{n+nt}{img} \PYG{n+na}{class}\PYG{o}{=}\PYG{l+s}{\PYGZdq{}card\PYGZhy{}img\PYGZhy{}top\PYGZdq{}} \PYG{n+na}{src}\PYG{o}{=}\PYG{l+s}{\PYGZdq{}galleries/cew/500\PYGZhy{}500\PYGZhy{}1.jpeg\PYGZdq{}} \PYG{n+na}{alt}\PYG{o}{=}\PYG{l+s}{\PYGZdq{}Texto alternativo\PYGZdq{}}\PYG{p}{\PYGZgt{}}
    \PYG{p}{\PYGZlt{}}\PYG{n+nt}{div} \PYG{n+na}{class}\PYG{o}{=}\PYG{l+s}{\PYGZdq{}card\PYGZhy{}body\PYGZdq{}}\PYG{p}{\PYGZgt{}}
      \PYG{p}{\PYGZlt{}}\PYG{n+nt}{h5} \PYG{n+na}{class}\PYG{o}{=}\PYG{l+s}{\PYGZdq{}card\PYGZhy{}title\PYGZdq{}}\PYG{p}{\PYGZgt{}}Título\PYG{p}{\PYGZlt{}}\PYG{p}{/}\PYG{n+nt}{h5}\PYG{p}{\PYGZgt{}}
      \PYG{p}{\PYGZlt{}}\PYG{n+nt}{p} \PYG{n+na}{class}\PYG{o}{=}\PYG{l+s}{\PYGZdq{}card\PYGZhy{}text\PYGZdq{}}\PYG{p}{\PYGZgt{}}Contenido.\PYG{p}{\PYGZlt{}}\PYG{p}{/}\PYG{n+nt}{p}\PYG{p}{\PYGZgt{}}
    \PYG{p}{\PYGZlt{}}\PYG{p}{/}\PYG{n+nt}{div}\PYG{p}{\PYGZgt{}}
    \PYG{p}{\PYGZlt{}}\PYG{n+nt}{div} \PYG{n+na}{class}\PYG{o}{=}\PYG{l+s}{\PYGZdq{}card\PYGZhy{}footer\PYGZdq{}}\PYG{p}{\PYGZgt{}}
      \PYG{p}{\PYGZlt{}}\PYG{n+nt}{small} \PYG{n+na}{class}\PYG{o}{=}\PYG{l+s}{\PYGZdq{}text\PYGZhy{}muted\PYGZdq{}}\PYG{p}{\PYGZgt{}}Pie\PYG{p}{\PYGZlt{}}\PYG{p}{/}\PYG{n+nt}{small}\PYG{p}{\PYGZgt{}}
    \PYG{p}{\PYGZlt{}}\PYG{p}{/}\PYG{n+nt}{div}\PYG{p}{\PYGZgt{}}
  \PYG{p}{\PYGZlt{}}\PYG{p}{/}\PYG{n+nt}{div}\PYG{p}{\PYGZgt{}}
  \PYG{p}{\PYGZlt{}}\PYG{n+nt}{div} \PYG{n+na}{class}\PYG{o}{=}\PYG{l+s}{\PYGZdq{}card\PYGZdq{}}\PYG{p}{\PYGZgt{}}
    \PYG{p}{\PYGZlt{}}\PYG{n+nt}{img} \PYG{n+na}{class}\PYG{o}{=}\PYG{l+s}{\PYGZdq{}card\PYGZhy{}img\PYGZhy{}top\PYGZdq{}} \PYG{n+na}{src}\PYG{o}{=}\PYG{l+s}{\PYGZdq{}galleries/cew/500\PYGZhy{}500\PYGZhy{}1.jpeg\PYGZdq{}} \PYG{n+na}{alt}\PYG{o}{=}\PYG{l+s}{\PYGZdq{}Texto alternativo\PYGZdq{}}\PYG{p}{\PYGZgt{}}
    \PYG{p}{\PYGZlt{}}\PYG{n+nt}{div} \PYG{n+na}{class}\PYG{o}{=}\PYG{l+s}{\PYGZdq{}card\PYGZhy{}body\PYGZdq{}}\PYG{p}{\PYGZgt{}}
      \PYG{p}{\PYGZlt{}}\PYG{n+nt}{h5} \PYG{n+na}{class}\PYG{o}{=}\PYG{l+s}{\PYGZdq{}card\PYGZhy{}title\PYGZdq{}}\PYG{p}{\PYGZgt{}}Título\PYG{p}{\PYGZlt{}}\PYG{p}{/}\PYG{n+nt}{h5}\PYG{p}{\PYGZgt{}}
      \PYG{p}{\PYGZlt{}}\PYG{n+nt}{p} \PYG{n+na}{class}\PYG{o}{=}\PYG{l+s}{\PYGZdq{}card\PYGZhy{}text\PYGZdq{}}\PYG{p}{\PYGZgt{}}Contenido.\PYG{p}{\PYGZlt{}}\PYG{p}{/}\PYG{n+nt}{p}\PYG{p}{\PYGZgt{}}
    \PYG{p}{\PYGZlt{}}\PYG{p}{/}\PYG{n+nt}{div}\PYG{p}{\PYGZgt{}}
    \PYG{p}{\PYGZlt{}}\PYG{n+nt}{div} \PYG{n+na}{class}\PYG{o}{=}\PYG{l+s}{\PYGZdq{}card\PYGZhy{}footer\PYGZdq{}}\PYG{p}{\PYGZgt{}}
      \PYG{p}{\PYGZlt{}}\PYG{n+nt}{small} \PYG{n+na}{class}\PYG{o}{=}\PYG{l+s}{\PYGZdq{}text\PYGZhy{}muted\PYGZdq{}}\PYG{p}{\PYGZgt{}}Pie\PYG{p}{\PYGZlt{}}\PYG{p}{/}\PYG{n+nt}{small}\PYG{p}{\PYGZgt{}}
    \PYG{p}{\PYGZlt{}}\PYG{p}{/}\PYG{n+nt}{div}\PYG{p}{\PYGZgt{}}
  \PYG{p}{\PYGZlt{}}\PYG{p}{/}\PYG{n+nt}{div}\PYG{p}{\PYGZgt{}}
  \PYG{p}{\PYGZlt{}}\PYG{n+nt}{div} \PYG{n+na}{class}\PYG{o}{=}\PYG{l+s}{\PYGZdq{}card\PYGZdq{}}\PYG{p}{\PYGZgt{}}
    \PYG{p}{\PYGZlt{}}\PYG{n+nt}{img} \PYG{n+na}{class}\PYG{o}{=}\PYG{l+s}{\PYGZdq{}card\PYGZhy{}img\PYGZhy{}top\PYGZdq{}} \PYG{n+na}{src}\PYG{o}{=}\PYG{l+s}{\PYGZdq{}galleries/cew/500\PYGZhy{}500\PYGZhy{}1.jpeg\PYGZdq{}} \PYG{n+na}{alt}\PYG{o}{=}\PYG{l+s}{\PYGZdq{}Texto alternativo\PYGZdq{}}\PYG{p}{\PYGZgt{}}
    \PYG{p}{\PYGZlt{}}\PYG{n+nt}{div} \PYG{n+na}{class}\PYG{o}{=}\PYG{l+s}{\PYGZdq{}card\PYGZhy{}body\PYGZdq{}}\PYG{p}{\PYGZgt{}}
      \PYG{p}{\PYGZlt{}}\PYG{n+nt}{h5} \PYG{n+na}{class}\PYG{o}{=}\PYG{l+s}{\PYGZdq{}card\PYGZhy{}title\PYGZdq{}}\PYG{p}{\PYGZgt{}}Título\PYG{p}{\PYGZlt{}}\PYG{p}{/}\PYG{n+nt}{h5}\PYG{p}{\PYGZgt{}}
      \PYG{p}{\PYGZlt{}}\PYG{n+nt}{p} \PYG{n+na}{class}\PYG{o}{=}\PYG{l+s}{\PYGZdq{}card\PYGZhy{}text\PYGZdq{}}\PYG{p}{\PYGZgt{}}Contenido.\PYG{p}{\PYGZlt{}}\PYG{p}{/}\PYG{n+nt}{p}\PYG{p}{\PYGZgt{}}
    \PYG{p}{\PYGZlt{}}\PYG{p}{/}\PYG{n+nt}{div}\PYG{p}{\PYGZgt{}}
    \PYG{p}{\PYGZlt{}}\PYG{n+nt}{div} \PYG{n+na}{class}\PYG{o}{=}\PYG{l+s}{\PYGZdq{}card\PYGZhy{}footer\PYGZdq{}}\PYG{p}{\PYGZgt{}}
      \PYG{p}{\PYGZlt{}}\PYG{n+nt}{small} \PYG{n+na}{class}\PYG{o}{=}\PYG{l+s}{\PYGZdq{}text\PYGZhy{}muted\PYGZdq{}}\PYG{p}{\PYGZgt{}}Pie\PYG{p}{\PYGZlt{}}\PYG{p}{/}\PYG{n+nt}{small}\PYG{p}{\PYGZgt{}}
    \PYG{p}{\PYGZlt{}}\PYG{p}{/}\PYG{n+nt}{div}\PYG{p}{\PYGZgt{}}
  \PYG{p}{\PYGZlt{}}\PYG{p}{/}\PYG{n+nt}{div}\PYG{p}{\PYGZgt{}}
\PYG{p}{\PYGZlt{}}\PYG{p}{/}\PYG{n+nt}{div}\PYG{p}{\PYGZgt{}}
\end{sphinxVerbatim}




\subsection{Card deck}
\label{\detokenize{mas-componentes:card-deck}}
\fvset{hllines={, ,}}%
\begin{sphinxVerbatim}[commandchars=\\\{\}]
\PYG{p}{\PYGZlt{}}\PYG{n+nt}{div} \PYG{n+na}{class}\PYG{o}{=}\PYG{l+s}{\PYGZdq{}card\PYGZhy{}deck\PYGZdq{}}\PYG{p}{\PYGZgt{}}
  \PYG{p}{\PYGZlt{}}\PYG{n+nt}{div} \PYG{n+na}{class}\PYG{o}{=}\PYG{l+s}{\PYGZdq{}card\PYGZdq{}}\PYG{p}{\PYGZgt{}}
    \PYG{p}{\PYGZlt{}}\PYG{n+nt}{img} \PYG{n+na}{class}\PYG{o}{=}\PYG{l+s}{\PYGZdq{}card\PYGZhy{}img\PYGZhy{}top\PYGZdq{}} \PYG{n+na}{src}\PYG{o}{=}\PYG{l+s}{\PYGZdq{}galleries/cew/500\PYGZhy{}500\PYGZhy{}1.jpeg\PYGZdq{}} \PYG{n+na}{alt}\PYG{o}{=}\PYG{l+s}{\PYGZdq{}Texto alternativo\PYGZdq{}}\PYG{p}{\PYGZgt{}}
    \PYG{p}{\PYGZlt{}}\PYG{n+nt}{div} \PYG{n+na}{class}\PYG{o}{=}\PYG{l+s}{\PYGZdq{}card\PYGZhy{}body\PYGZdq{}}\PYG{p}{\PYGZgt{}}
      \PYG{p}{\PYGZlt{}}\PYG{n+nt}{h5} \PYG{n+na}{class}\PYG{o}{=}\PYG{l+s}{\PYGZdq{}card\PYGZhy{}title\PYGZdq{}}\PYG{p}{\PYGZgt{}}Título\PYG{p}{\PYGZlt{}}\PYG{p}{/}\PYG{n+nt}{h5}\PYG{p}{\PYGZgt{}}
      \PYG{p}{\PYGZlt{}}\PYG{n+nt}{p} \PYG{n+na}{class}\PYG{o}{=}\PYG{l+s}{\PYGZdq{}card\PYGZhy{}text\PYGZdq{}}\PYG{p}{\PYGZgt{}}Contenido.\PYG{p}{\PYGZlt{}}\PYG{p}{/}\PYG{n+nt}{p}\PYG{p}{\PYGZgt{}}
    \PYG{p}{\PYGZlt{}}\PYG{p}{/}\PYG{n+nt}{div}\PYG{p}{\PYGZgt{}}
    \PYG{p}{\PYGZlt{}}\PYG{n+nt}{div} \PYG{n+na}{class}\PYG{o}{=}\PYG{l+s}{\PYGZdq{}card\PYGZhy{}footer\PYGZdq{}}\PYG{p}{\PYGZgt{}}
      \PYG{p}{\PYGZlt{}}\PYG{n+nt}{small} \PYG{n+na}{class}\PYG{o}{=}\PYG{l+s}{\PYGZdq{}text\PYGZhy{}muted\PYGZdq{}}\PYG{p}{\PYGZgt{}}Pie\PYG{p}{\PYGZlt{}}\PYG{p}{/}\PYG{n+nt}{small}\PYG{p}{\PYGZgt{}}
    \PYG{p}{\PYGZlt{}}\PYG{p}{/}\PYG{n+nt}{div}\PYG{p}{\PYGZgt{}}
  \PYG{p}{\PYGZlt{}}\PYG{p}{/}\PYG{n+nt}{div}\PYG{p}{\PYGZgt{}}
  \PYG{p}{\PYGZlt{}}\PYG{n+nt}{div} \PYG{n+na}{class}\PYG{o}{=}\PYG{l+s}{\PYGZdq{}card\PYGZdq{}}\PYG{p}{\PYGZgt{}}
    \PYG{p}{\PYGZlt{}}\PYG{n+nt}{img} \PYG{n+na}{class}\PYG{o}{=}\PYG{l+s}{\PYGZdq{}card\PYGZhy{}img\PYGZhy{}top\PYGZdq{}} \PYG{n+na}{src}\PYG{o}{=}\PYG{l+s}{\PYGZdq{}galleries/cew/500\PYGZhy{}500\PYGZhy{}1.jpeg\PYGZdq{}} \PYG{n+na}{alt}\PYG{o}{=}\PYG{l+s}{\PYGZdq{}Texto alternativo\PYGZdq{}}\PYG{p}{\PYGZgt{}}
    \PYG{p}{\PYGZlt{}}\PYG{n+nt}{div} \PYG{n+na}{class}\PYG{o}{=}\PYG{l+s}{\PYGZdq{}card\PYGZhy{}body\PYGZdq{}}\PYG{p}{\PYGZgt{}}
      \PYG{p}{\PYGZlt{}}\PYG{n+nt}{h5} \PYG{n+na}{class}\PYG{o}{=}\PYG{l+s}{\PYGZdq{}card\PYGZhy{}title\PYGZdq{}}\PYG{p}{\PYGZgt{}}Título\PYG{p}{\PYGZlt{}}\PYG{p}{/}\PYG{n+nt}{h5}\PYG{p}{\PYGZgt{}}
      \PYG{p}{\PYGZlt{}}\PYG{n+nt}{p} \PYG{n+na}{class}\PYG{o}{=}\PYG{l+s}{\PYGZdq{}card\PYGZhy{}text\PYGZdq{}}\PYG{p}{\PYGZgt{}}Contenido.\PYG{p}{\PYGZlt{}}\PYG{p}{/}\PYG{n+nt}{p}\PYG{p}{\PYGZgt{}}
    \PYG{p}{\PYGZlt{}}\PYG{p}{/}\PYG{n+nt}{div}\PYG{p}{\PYGZgt{}}
    \PYG{p}{\PYGZlt{}}\PYG{n+nt}{div} \PYG{n+na}{class}\PYG{o}{=}\PYG{l+s}{\PYGZdq{}card\PYGZhy{}footer\PYGZdq{}}\PYG{p}{\PYGZgt{}}
      \PYG{p}{\PYGZlt{}}\PYG{n+nt}{small} \PYG{n+na}{class}\PYG{o}{=}\PYG{l+s}{\PYGZdq{}text\PYGZhy{}muted\PYGZdq{}}\PYG{p}{\PYGZgt{}}Pie\PYG{p}{\PYGZlt{}}\PYG{p}{/}\PYG{n+nt}{small}\PYG{p}{\PYGZgt{}}
    \PYG{p}{\PYGZlt{}}\PYG{p}{/}\PYG{n+nt}{div}\PYG{p}{\PYGZgt{}}
  \PYG{p}{\PYGZlt{}}\PYG{p}{/}\PYG{n+nt}{div}\PYG{p}{\PYGZgt{}}
  \PYG{p}{\PYGZlt{}}\PYG{n+nt}{div} \PYG{n+na}{class}\PYG{o}{=}\PYG{l+s}{\PYGZdq{}card\PYGZdq{}}\PYG{p}{\PYGZgt{}}
    \PYG{p}{\PYGZlt{}}\PYG{n+nt}{img} \PYG{n+na}{class}\PYG{o}{=}\PYG{l+s}{\PYGZdq{}card\PYGZhy{}img\PYGZhy{}top\PYGZdq{}} \PYG{n+na}{src}\PYG{o}{=}\PYG{l+s}{\PYGZdq{}galleries/cew/500\PYGZhy{}500\PYGZhy{}1.jpeg\PYGZdq{}} \PYG{n+na}{alt}\PYG{o}{=}\PYG{l+s}{\PYGZdq{}Texto alternativo\PYGZdq{}}\PYG{p}{\PYGZgt{}}
    \PYG{p}{\PYGZlt{}}\PYG{n+nt}{div} \PYG{n+na}{class}\PYG{o}{=}\PYG{l+s}{\PYGZdq{}card\PYGZhy{}body\PYGZdq{}}\PYG{p}{\PYGZgt{}}
      \PYG{p}{\PYGZlt{}}\PYG{n+nt}{h5} \PYG{n+na}{class}\PYG{o}{=}\PYG{l+s}{\PYGZdq{}card\PYGZhy{}title\PYGZdq{}}\PYG{p}{\PYGZgt{}}Título\PYG{p}{\PYGZlt{}}\PYG{p}{/}\PYG{n+nt}{h5}\PYG{p}{\PYGZgt{}}
      \PYG{p}{\PYGZlt{}}\PYG{n+nt}{p} \PYG{n+na}{class}\PYG{o}{=}\PYG{l+s}{\PYGZdq{}card\PYGZhy{}text\PYGZdq{}}\PYG{p}{\PYGZgt{}}Contenido.\PYG{p}{\PYGZlt{}}\PYG{p}{/}\PYG{n+nt}{p}\PYG{p}{\PYGZgt{}}
    \PYG{p}{\PYGZlt{}}\PYG{p}{/}\PYG{n+nt}{div}\PYG{p}{\PYGZgt{}}
    \PYG{p}{\PYGZlt{}}\PYG{n+nt}{div} \PYG{n+na}{class}\PYG{o}{=}\PYG{l+s}{\PYGZdq{}card\PYGZhy{}footer\PYGZdq{}}\PYG{p}{\PYGZgt{}}
      \PYG{p}{\PYGZlt{}}\PYG{n+nt}{small} \PYG{n+na}{class}\PYG{o}{=}\PYG{l+s}{\PYGZdq{}text\PYGZhy{}muted\PYGZdq{}}\PYG{p}{\PYGZgt{}}Pie\PYG{p}{\PYGZlt{}}\PYG{p}{/}\PYG{n+nt}{small}\PYG{p}{\PYGZgt{}}
    \PYG{p}{\PYGZlt{}}\PYG{p}{/}\PYG{n+nt}{div}\PYG{p}{\PYGZgt{}}
  \PYG{p}{\PYGZlt{}}\PYG{p}{/}\PYG{n+nt}{div}\PYG{p}{\PYGZgt{}}
\PYG{p}{\PYGZlt{}}\PYG{p}{/}\PYG{n+nt}{div}\PYG{p}{\PYGZgt{}}
\end{sphinxVerbatim}




\chapter{Dibujando Formas en 2D}
\label{\detokenize{dibujando-en-2d::doc}}\label{\detokenize{dibujando-en-2d:dibujando-formas-en-2d}}
Hasta ahora la mayoría del contenido que creamos consiste principalmente en
texto y "cajas", es decir, cuadrados dentro de cuadrados.

Pero que pasa si quiero una linea, un triangulo o un circulo en mi pagina?

Para eso existe un set de tags llamado SVG que nos permite crear dibujos
"vectoriales", es decir que su contenido son las formas en si y se ven bien en
cualquier resolución de pantalla, no como las imágenes hechas de pixeles, donde
si la imagen es chica y la agrandamos empezamos a perder calidad.

Empecemos con un ejemplo simple:

\fvset{hllines={, ,}}%
\begin{sphinxVerbatim}[commandchars=\\\{\}]
\PYG{p}{\PYGZlt{}}\PYG{n+nt}{svg} \PYG{n+na}{height}\PYG{o}{=}\PYG{l+s}{\PYGZdq{}100\PYGZdq{}} \PYG{n+na}{width}\PYG{o}{=}\PYG{l+s}{\PYGZdq{}100\PYGZdq{}}\PYG{p}{\PYGZgt{}}
    \PYG{p}{\PYGZlt{}}\PYG{n+nt}{circle} \PYG{n+na}{cx}\PYG{o}{=}\PYG{l+s}{\PYGZdq{}50\PYGZdq{}} \PYG{n+na}{cy}\PYG{o}{=}\PYG{l+s}{\PYGZdq{}50\PYGZdq{}} \PYG{n+na}{r}\PYG{o}{=}\PYG{l+s}{\PYGZdq{}40\PYGZdq{}} \PYG{n+na}{stroke}\PYG{o}{=}\PYG{l+s}{\PYGZdq{}black\PYGZdq{}} \PYG{n+na}{stroke\PYGZhy{}width}\PYG{o}{=}\PYG{l+s}{\PYGZdq{}3\PYGZdq{}} \PYG{n+na}{fill}\PYG{o}{=}\PYG{l+s}{\PYGZdq{}red\PYGZdq{}} \PYG{p}{/}\PYG{p}{\PYGZgt{}}
\PYG{p}{\PYGZlt{}}\PYG{p}{/}\PYG{n+nt}{svg}\PYG{p}{\PYGZgt{}}
\end{sphinxVerbatim}



Los tags son nuevos y específicos de SVG, es decir, solo los podemos usar dentro
de un tag raíz \sphinxhref{https://developer.mozilla.org/es/docs/Web/SVG/Element/svg}{svg},
lo bueno es que son bastante descriptivos.

En el ejemplo arriba decimos que queremos dibujar en svg, dentro de un cuadro
de 100x100.

El dibujo consiste de un circulo (\sphinxhref{https://developer.mozilla.org/es/docs/Web/SVG/Element/circle}{circle} en ingles) con centro x=50 y centro y=50, con un radio de
40, borde de 3 negro y relleno rojo.

Veamos algunos otros ejemplos:

\fvset{hllines={, ,}}%
\begin{sphinxVerbatim}[commandchars=\\\{\}]
\PYG{p}{\PYGZlt{}}\PYG{n+nt}{svg} \PYG{n+na}{width}\PYG{o}{=}\PYG{l+s}{\PYGZdq{}400\PYGZdq{}} \PYG{n+na}{height}\PYG{o}{=}\PYG{l+s}{\PYGZdq{}180\PYGZdq{}}\PYG{p}{\PYGZgt{}}
  \PYG{p}{\PYGZlt{}}\PYG{n+nt}{rect} \PYG{n+na}{x}\PYG{o}{=}\PYG{l+s}{\PYGZdq{}50\PYGZdq{}} \PYG{n+na}{y}\PYG{o}{=}\PYG{l+s}{\PYGZdq{}20\PYGZdq{}} \PYG{n+na}{rx}\PYG{o}{=}\PYG{l+s}{\PYGZdq{}20\PYGZdq{}} \PYG{n+na}{ry}\PYG{o}{=}\PYG{l+s}{\PYGZdq{}20\PYGZdq{}} \PYG{n+na}{width}\PYG{o}{=}\PYG{l+s}{\PYGZdq{}150\PYGZdq{}} \PYG{n+na}{height}\PYG{o}{=}\PYG{l+s}{\PYGZdq{}150\PYGZdq{}} \PYG{n+na}{style}\PYG{o}{=}\PYG{l+s}{\PYGZdq{}fill:blue;stroke:pink;stroke\PYGZhy{}width:5;opacity:0.5\PYGZdq{}} \PYG{p}{/}\PYG{p}{\PYGZgt{}}
\PYG{p}{\PYGZlt{}}\PYG{p}{/}\PYG{n+nt}{svg}\PYG{p}{\PYGZgt{}}
\end{sphinxVerbatim}



Acá vemos un rectángulo (\sphinxhref{https://developer.mozilla.org/es/docs/Web/SVG/Element/rect}{rect}) posicionado en x=50, y=20, con ancho de 150 y alto de 150 y bordes
redondeados.

Como veras el resto de las propiedades se las define con el atributo \sphinxtitleref{style} al
igual que en HTML, algunos atributos son nuevos pero el resto sigue aplicando.

Algunos mas:

\fvset{hllines={, ,}}%
\begin{sphinxVerbatim}[commandchars=\\\{\}]
\PYG{p}{\PYGZlt{}}\PYG{n+nt}{svg} \PYG{n+na}{height}\PYG{o}{=}\PYG{l+s}{\PYGZdq{}250\PYGZdq{}} \PYG{n+na}{width}\PYG{o}{=}\PYG{l+s}{\PYGZdq{}500\PYGZdq{}}\PYG{p}{\PYGZgt{}}
    \PYG{p}{\PYGZlt{}}\PYG{n+nt}{ellipse} \PYG{n+na}{cx}\PYG{o}{=}\PYG{l+s}{\PYGZdq{}240\PYGZdq{}} \PYG{n+na}{cy}\PYG{o}{=}\PYG{l+s}{\PYGZdq{}100\PYGZdq{}} \PYG{n+na}{rx}\PYG{o}{=}\PYG{l+s}{\PYGZdq{}220\PYGZdq{}} \PYG{n+na}{ry}\PYG{o}{=}\PYG{l+s}{\PYGZdq{}30\PYGZdq{}} \PYG{n+na}{style}\PYG{o}{=}\PYG{l+s}{\PYGZdq{}fill:purple\PYGZdq{}} \PYG{p}{/}\PYG{p}{\PYGZgt{}}
    \PYG{p}{\PYGZlt{}}\PYG{n+nt}{ellipse} \PYG{n+na}{cx}\PYG{o}{=}\PYG{l+s}{\PYGZdq{}220\PYGZdq{}} \PYG{n+na}{cy}\PYG{o}{=}\PYG{l+s}{\PYGZdq{}70\PYGZdq{}} \PYG{n+na}{rx}\PYG{o}{=}\PYG{l+s}{\PYGZdq{}190\PYGZdq{}} \PYG{n+na}{ry}\PYG{o}{=}\PYG{l+s}{\PYGZdq{}20\PYGZdq{}} \PYG{n+na}{style}\PYG{o}{=}\PYG{l+s}{\PYGZdq{}fill:lime\PYGZdq{}} \PYG{p}{/}\PYG{p}{\PYGZgt{}}
    \PYG{p}{\PYGZlt{}}\PYG{n+nt}{ellipse} \PYG{n+na}{cx}\PYG{o}{=}\PYG{l+s}{\PYGZdq{}210\PYGZdq{}} \PYG{n+na}{cy}\PYG{o}{=}\PYG{l+s}{\PYGZdq{}45\PYGZdq{}} \PYG{n+na}{rx}\PYG{o}{=}\PYG{l+s}{\PYGZdq{}170\PYGZdq{}} \PYG{n+na}{ry}\PYG{o}{=}\PYG{l+s}{\PYGZdq{}15\PYGZdq{}} \PYG{n+na}{style}\PYG{o}{=}\PYG{l+s}{\PYGZdq{}fill:yellow\PYGZdq{}} \PYG{p}{/}\PYG{p}{\PYGZgt{}}

    \PYG{p}{\PYGZlt{}}\PYG{n+nt}{polygon} \PYG{n+na}{points}\PYG{o}{=}\PYG{l+s}{\PYGZdq{}200,10 250,190 160,210\PYGZdq{}} \PYG{n+na}{style}\PYG{o}{=}\PYG{l+s}{\PYGZdq{}fill:lime;stroke:purple;stroke\PYGZhy{}width:1\PYGZdq{}} \PYG{p}{/}\PYG{p}{\PYGZgt{}}

    \PYG{p}{\PYGZlt{}}\PYG{n+nt}{line} \PYG{n+na}{x1}\PYG{o}{=}\PYG{l+s}{\PYGZdq{}0\PYGZdq{}} \PYG{n+na}{y1}\PYG{o}{=}\PYG{l+s}{\PYGZdq{}0\PYGZdq{}} \PYG{n+na}{x2}\PYG{o}{=}\PYG{l+s}{\PYGZdq{}200\PYGZdq{}} \PYG{n+na}{y2}\PYG{o}{=}\PYG{l+s}{\PYGZdq{}200\PYGZdq{}} \PYG{n+na}{style}\PYG{o}{=}\PYG{l+s}{\PYGZdq{}stroke:rgb(255,0,0);stroke\PYGZhy{}width:2\PYGZdq{}} \PYG{p}{/}\PYG{p}{\PYGZgt{}}
    \PYG{p}{\PYGZlt{}}\PYG{n+nt}{polyline} \PYG{n+na}{points}\PYG{o}{=}\PYG{l+s}{\PYGZdq{}0,40 40,40 40,80 80,80 80,120 120,120 120,160\PYGZdq{}} \PYG{n+na}{style}\PYG{o}{=}\PYG{l+s}{\PYGZdq{}fill:white;stroke:red;stroke\PYGZhy{}width:4\PYGZdq{}} \PYG{p}{/}\PYG{p}{\PYGZgt{}}

    \PYG{p}{\PYGZlt{}}\PYG{n+nt}{text} \PYG{n+na}{x}\PYG{o}{=}\PYG{l+s}{\PYGZdq{}0\PYGZdq{}} \PYG{n+na}{y}\PYG{o}{=}\PYG{l+s}{\PYGZdq{}15\PYGZdq{}} \PYG{n+na}{fill}\PYG{o}{=}\PYG{l+s}{\PYGZdq{}red\PYGZdq{}} \PYG{n+na}{transform}\PYG{o}{=}\PYG{l+s}{\PYGZdq{}rotate(30 20,40)\PYGZdq{}}\PYG{p}{\PYGZgt{}}Texto en SVG\PYG{p}{\PYGZlt{}}\PYG{p}{/}\PYG{n+nt}{text}\PYG{p}{\PYGZgt{}}
\PYG{p}{\PYGZlt{}}\PYG{p}{/}\PYG{n+nt}{svg}\PYG{p}{\PYGZgt{}}
\end{sphinxVerbatim}



Pero lo mejor que tiene SVG es que hay editores libres y gratuitos que nos
permiten dibujar como cualquier editor de imágenes y luego ver el código SVG
generado.

Este editor se llama \sphinxhref{https://inkscape.org/es/}{Inkscape} y lo podes
descargar desde la pagina.

Luego de instalarlo se ve algo así:

\begin{figure}[htbp]
\centering

\noindent\sphinxincludegraphics[width=1.000\linewidth]{{inkscape}.png}
\end{figure}

Abriendo el editor XML de Inkscape podemos ver como se crea cada forma, que
tags y atributos se usan.

\begin{figure}[htbp]
\centering

\noindent\sphinxincludegraphics{{inkscape-xml}.gif}
\end{figure}

Si usamos un editor para crear un SVG que queremos insertar en nuestra pagina
tenemos dos opciones:
\begin{itemize}
\item {} 
Guardar el dibujo como un archivo SVG y copiar el contenido del archivo abriendolo con un editor de texto y pegandolo en nuestra pagina

\item {} 
Insertandolo como una imagen externa

\end{itemize}

Ya vimos como insertar SVG directamente en el HTML, ahora veamos como incluirlo
como una imagen externa:

\fvset{hllines={, ,}}%
\begin{sphinxVerbatim}[commandchars=\\\{\}]
\PYG{p}{\PYGZlt{}}\PYG{n+nt}{img} \PYG{n+na}{src}\PYG{o}{=}\PYG{l+s}{\PYGZdq{}../galleries/cew/10/example.svg\PYGZdq{}} \PYG{n+na}{width}\PYG{o}{=}\PYG{l+s}{\PYGZdq{}475\PYGZdq{}} \PYG{n+na}{height}\PYG{o}{=}\PYG{l+s}{\PYGZdq{}336\PYGZdq{}}\PYG{p}{\PYGZgt{}}
\end{sphinxVerbatim}



\sphinxhref{https://commons.wikimedia.org/wiki/File:\%D0\%9F\%D1\%80\%D0\%B8\%D0\%BC\%D0\%B5\%D1\%80\_\%D1\%87\%D0\%B5\%D1\%80\%D1\%82\%D0\%B5\%D0\%B6\%D0\%B0\_\%D0\%B2\_SVG\_\%D1\%84\%D0\%BE\%D1\%80\%D0\%BC\%D0\%B0\%D1\%82\%D0\%B5.svg}{Fuente}

Una ultima observación sobre SVG, si bien son parecidos a HTML, SVG es mas
estricto en cuanto a los nombres y atributos de tags permitidos y con la
necesidad de "cerrar" todos los tags, si cometemos un error en HTML, el
navegador va a hacer lo mejor que pueda para presentar el contenido igual, en
SVG muy probablemente no se dibuje nada.


\chapter{Dibujando Formas en 3D}
\label{\detokenize{dibujando-en-3d::doc}}\label{\detokenize{dibujando-en-3d:dibujando-formas-en-3d}}
Ya dibujamos en 2D, se podrá en 3D?

La respuesta es si, pero como es un área que todavía esta en desarrollo no
es soportada 100\% y de forma simple en todos los navegadores por lo que nos
vamos a ayudar de una librería llamada \sphinxhref{https://aframe.io/}{aframe}.

Una librería es uno o mas archivos que al cargarlos en nuestra pagina le
agregan funcionalidades.

En este caso nos permiten crear escenas en 3D como si estuvieramos escribiendo
HTML.

Como dije esto no esta completamente estandarizado así que aframe define sus
propios tags que no son parte de ningún estándar que soporten todos los
navegadores.

Entonces empecemos cargando la librería aframe:

\fvset{hllines={, ,}}%
\begin{sphinxVerbatim}[commandchars=\\\{\}]
\PYG{p}{\PYGZlt{}}\PYG{n+nt}{script} \PYG{n+na}{src}\PYG{o}{=}\PYG{l+s}{\PYGZdq{}https://aframe.io/releases/0.8.0/aframe.min.js\PYGZdq{}}\PYG{p}{\PYGZgt{}}\PYG{p}{\PYGZlt{}}\PYG{p}{/}\PYG{n+nt}{script}\PYG{p}{\PYGZgt{}}
\end{sphinxVerbatim}



Y luego creamos nuestra escena:

\fvset{hllines={, ,}}%
\begin{sphinxVerbatim}[commandchars=\\\{\}]
\PYG{p}{\PYGZlt{}}\PYG{n+nt}{div} \PYG{n+na}{style}\PYG{o}{=}\PYG{l+s}{\PYGZdq{}width: 100\PYGZpc{}; height: 25em\PYGZdq{}}\PYG{p}{\PYGZgt{}}
        \PYG{p}{\PYGZlt{}}\PYG{n+nt}{a\PYGZhy{}scene} \PYG{n+na}{embedded}\PYG{p}{\PYGZgt{}}
          \PYG{p}{\PYGZlt{}}\PYG{n+nt}{a\PYGZhy{}box} \PYG{n+na}{position}\PYG{o}{=}\PYG{l+s}{\PYGZdq{}\PYGZhy{}1 0.5 \PYGZhy{}3\PYGZdq{}} \PYG{n+na}{rotation}\PYG{o}{=}\PYG{l+s}{\PYGZdq{}0 45 0\PYGZdq{}} \PYG{n+na}{color}\PYG{o}{=}\PYG{l+s}{\PYGZdq{}blue\PYGZdq{}}\PYG{p}{\PYGZgt{}}\PYG{p}{\PYGZlt{}}\PYG{p}{/}\PYG{n+nt}{a\PYGZhy{}box}\PYG{p}{\PYGZgt{}}
          \PYG{p}{\PYGZlt{}}\PYG{n+nt}{a\PYGZhy{}sphere} \PYG{n+na}{position}\PYG{o}{=}\PYG{l+s}{\PYGZdq{}0 1.25 \PYGZhy{}5\PYGZdq{}} \PYG{n+na}{radius}\PYG{o}{=}\PYG{l+s}{\PYGZdq{}1.45\PYGZdq{}} \PYG{n+na}{color}\PYG{o}{=}\PYG{l+s}{\PYGZdq{}yellow\PYGZdq{}}\PYG{p}{\PYGZgt{}}\PYG{p}{\PYGZlt{}}\PYG{p}{/}\PYG{n+nt}{a\PYGZhy{}sphere}\PYG{p}{\PYGZgt{}}
          \PYG{p}{\PYGZlt{}}\PYG{n+nt}{a\PYGZhy{}cylinder} \PYG{n+na}{position}\PYG{o}{=}\PYG{l+s}{\PYGZdq{}1 0.75 \PYGZhy{}3\PYGZdq{}} \PYG{n+na}{radius}\PYG{o}{=}\PYG{l+s}{\PYGZdq{}0.5\PYGZdq{}} \PYG{n+na}{height}\PYG{o}{=}\PYG{l+s}{\PYGZdq{}1.5\PYGZdq{}} \PYG{n+na}{color}\PYG{o}{=}\PYG{l+s}{\PYGZdq{}\PYGZsh{}FFC65D\PYGZdq{}}\PYG{p}{\PYGZgt{}}\PYG{p}{\PYGZlt{}}\PYG{p}{/}\PYG{n+nt}{a\PYGZhy{}cylinder}\PYG{p}{\PYGZgt{}}
          \PYG{p}{\PYGZlt{}}\PYG{n+nt}{a\PYGZhy{}plane} \PYG{n+na}{position}\PYG{o}{=}\PYG{l+s}{\PYGZdq{}0 0 \PYGZhy{}4\PYGZdq{}} \PYG{n+na}{rotation}\PYG{o}{=}\PYG{l+s}{\PYGZdq{}\PYGZhy{}90 0 0\PYGZdq{}} \PYG{n+na}{width}\PYG{o}{=}\PYG{l+s}{\PYGZdq{}4\PYGZdq{}} \PYG{n+na}{height}\PYG{o}{=}\PYG{l+s}{\PYGZdq{}4\PYGZdq{}} \PYG{n+na}{color}\PYG{o}{=}\PYG{l+s}{\PYGZdq{}\PYGZsh{}7BC8A4\PYGZdq{}}\PYG{p}{\PYGZgt{}}\PYG{p}{\PYGZlt{}}\PYG{p}{/}\PYG{n+nt}{a\PYGZhy{}plane}\PYG{p}{\PYGZgt{}}
          \PYG{p}{\PYGZlt{}}\PYG{n+nt}{a\PYGZhy{}sky} \PYG{n+na}{color}\PYG{o}{=}\PYG{l+s}{\PYGZdq{}\PYGZsh{}BBBBBB\PYGZdq{}}\PYG{p}{\PYGZgt{}}\PYG{p}{\PYGZlt{}}\PYG{p}{/}\PYG{n+nt}{a\PYGZhy{}sky}\PYG{p}{\PYGZgt{}}
        \PYG{p}{\PYGZlt{}}\PYG{p}{/}\PYG{n+nt}{a\PYGZhy{}scene}\PYG{p}{\PYGZgt{}}
\PYG{p}{\PYGZlt{}}\PYG{p}{/}\PYG{n+nt}{div}\PYG{p}{\PYGZgt{}}
\end{sphinxVerbatim}



El tag raíz se llama \sphinxhref{https://aframe.io/docs/0.8.0/core/scene.html\#sidebar}{a-scene}, que traducido seria "una escena", el cual contiene \sphinxhref{https://aframe.io/docs/0.8.0/primitives/a-box.html\#sidebar}{a-box} "una caja", \sphinxhref{https://aframe.io/docs/0.8.0/primitives/a-sphere.html\#sidebar}{a-shere} "una esfera", \sphinxhref{https://aframe.io/docs/0.8.0/primitives/a-cylinder.html\#sidebar}{a-cylinder} "un cilindro", \sphinxhref{https://aframe.io/docs/0.8.0/primitives/a-plane.html\#sidebar}{a-plane} "un plano" y \sphinxhref{https://aframe.io/docs/0.8.0/primitives/a-sky.html\#sidebar}{a-sky} "un cielo".

Cada uno con atributos especificando la posicion en 3 dimensiones, su tamaño,
ya sea con su radio o su alto y su ancho y su color.

Fijate que con el mouse y las flechas del teclado podes moverte dentro de la
escena, no es algo fijo sino algo que podes explorar. Si tenes un dispositivo
que soporte Realidad Virtual (los últimos smartphones o anteojos de realidad
virtual) apretando el icono en la esquina inferior derecha podes "sumergirte"
en la escena en realidad virtual.

La escena de arriba en un proyecto de thimble así podes modificar los colores,
formas, posiciones y tamaños.

Abrí \sphinxurl{https://thimbleprojects.org/marianoguerra/510288/} y hace click en \sphinxtitleref{Remix}
para copiar el proyecto a tu cuenta.

Aframe también puede usarse para imágenes y videos panorámicos, veamos un ejemplo:

\fvset{hllines={, ,}}%
\begin{sphinxVerbatim}[commandchars=\\\{\}]
\PYG{p}{\PYGZlt{}}\PYG{n+nt}{div} \PYG{n+na}{style}\PYG{o}{=}\PYG{l+s}{\PYGZdq{}width: 100\PYGZpc{}; height: 25em\PYGZdq{}}\PYG{p}{\PYGZgt{}}
        \PYG{p}{\PYGZlt{}}\PYG{n+nt}{a\PYGZhy{}scene} \PYG{n+na}{embedded}\PYG{p}{\PYGZgt{}}
          \PYG{p}{\PYGZlt{}}\PYG{n+nt}{a\PYGZhy{}sky} \PYG{n+na}{src}\PYG{o}{=}\PYG{l+s}{\PYGZdq{}https://raw.githubusercontent.com/aframevr/aframe/v0.7.0/examples/boilerplate/panorama/puydesancy.jpg\PYGZdq{}} \PYG{n+na}{rotation}\PYG{o}{=}\PYG{l+s}{\PYGZdq{}0 \PYGZhy{}130 0\PYGZdq{}}\PYG{p}{\PYGZgt{}}\PYG{p}{\PYGZlt{}}\PYG{p}{/}\PYG{n+nt}{a\PYGZhy{}sky}\PYG{p}{\PYGZgt{}}

          \PYG{p}{\PYGZlt{}}\PYG{n+nt}{a\PYGZhy{}text} \PYG{n+na}{value}\PYG{o}{=}\PYG{l+s}{\PYGZdq{}Puy de Sancy, Francia\PYGZdq{}} \PYG{n+na}{width}\PYG{o}{=}\PYG{l+s}{\PYGZdq{}6\PYGZdq{}} \PYG{n+na}{position}\PYG{o}{=}\PYG{l+s}{\PYGZdq{}\PYGZhy{}2.5 0.25 \PYGZhy{}1.5\PYGZdq{}}
                          \PYG{n+na}{rotation}\PYG{o}{=}\PYG{l+s}{\PYGZdq{}0 15 0\PYGZdq{}}\PYG{p}{\PYGZgt{}}\PYG{p}{\PYGZlt{}}\PYG{p}{/}\PYG{n+nt}{a\PYGZhy{}text}\PYG{p}{\PYGZgt{}}
        \PYG{p}{\PYGZlt{}}\PYG{p}{/}\PYG{n+nt}{a\PYGZhy{}scene}\PYG{p}{\PYGZgt{}}
\PYG{p}{\PYGZlt{}}\PYG{p}{/}\PYG{n+nt}{div}\PYG{p}{\PYGZgt{}}
\end{sphinxVerbatim}

Por el momento aframe solo soporta una escena por pagina, por lo que no muestro
el resultado directamente, abrí
\sphinxurl{https://thimbleprojects.org/marianoguerra/511087/} y hace click en \sphinxtitleref{Remix} para
copiar el proyecto a tu cuenta.

Podes ver mas ejemplos en \sphinxurl{https://aframe.io/examples/showcase/snowglobe/}


\chapter{Audio y Video}
\label{\detokenize{audio-y-video::doc}}\label{\detokenize{audio-y-video:audio-y-video}}
Incluir un video o audio seria algo tan simple como un tag y la ubicación del
archivo.

Por cuestiones históricas hay muchos formatos de audio y video y las
organizaciones que desarrollan los navegadores mas usados (Microsoft: Internet
Explorer 11 y Microsoft Edge, Apple: Safari, Google: Chrome, Mozilla: Firefox)
tienen distintos objetivos e intereses que hacen que soporten algunos formatos
y otros no.


\section{Video}
\label{\detokenize{audio-y-video:video}}
Empecemos con un ejemplo que según esta \sphinxhref{https://caniuse.com/\#feat=webm}{tabla de compatibilidad para el formato webm}, no va a andar en IE 11 y Safari.

\fvset{hllines={, ,}}%
\begin{sphinxVerbatim}[commandchars=\\\{\}]
\PYG{p}{\PYGZlt{}}\PYG{n+nt}{video} \PYG{n+na}{src}\PYG{o}{=}\PYG{l+s}{\PYGZdq{}/cew\PYGZus{}files/12/example.webm\PYGZdq{}} \PYG{n+na}{type}\PYG{o}{=}\PYG{l+s}{\PYGZdq{}video/webm\PYGZdq{}} \PYG{n+na}{controls}\PYG{p}{\PYGZgt{}}\PYG{p}{\PYGZlt{}}\PYG{p}{/}\PYG{n+nt}{video}\PYG{p}{\PYGZgt{}}
\end{sphinxVerbatim}



Viendo la \sphinxhref{https://caniuse.com/\#feat=mpeg4}{tabla de compatibilidad para el formato mp4} vemos que podemos hacerlo funcionar en mas versiones pero no necesariamente todas.

\fvset{hllines={, ,}}%
\begin{sphinxVerbatim}[commandchars=\\\{\}]
\PYG{p}{\PYGZlt{}}\PYG{n+nt}{video} \PYG{n+na}{src}\PYG{o}{=}\PYG{l+s}{\PYGZdq{}/cew\PYGZus{}files/12/example.mp4\PYGZdq{}} \PYG{n+na}{type}\PYG{o}{=}\PYG{l+s}{\PYGZdq{}video/mp4\PYGZdq{}} \PYG{n+na}{controls}\PYG{p}{\PYGZgt{}}\PYG{p}{\PYGZlt{}}\PYG{p}{/}\PYG{n+nt}{video}\PYG{p}{\PYGZgt{}}
\end{sphinxVerbatim}



Que pasa si queremos hacerlo funcionar en la mayor cantidad de plataformas
posibles priorizando formatos mas livianos y con mejor calidad?

Podemos especificar los videos en orden de preferencia, el navegador va a
intentar en orden del primero al ultimo cargarlos, cuando encuentre uno que
sirve lo va a usar.

\fvset{hllines={, ,}}%
\begin{sphinxVerbatim}[commandchars=\\\{\}]
\PYG{p}{\PYGZlt{}}\PYG{n+nt}{video} \PYG{n+na}{controls}\PYG{p}{\PYGZgt{}}
    \PYG{p}{\PYGZlt{}}\PYG{n+nt}{source} \PYG{n+na}{src}\PYG{o}{=}\PYG{l+s}{\PYGZdq{}/cew\PYGZus{}files/12/example.webm\PYGZdq{}} \PYG{n+na}{type}\PYG{o}{=}\PYG{l+s}{\PYGZdq{}video/webm\PYGZdq{}}\PYG{p}{\PYGZgt{}}
    \PYG{p}{\PYGZlt{}}\PYG{n+nt}{source} \PYG{n+na}{src}\PYG{o}{=}\PYG{l+s}{\PYGZdq{}/cew\PYGZus{}files/12/example.mp4\PYGZdq{}} \PYG{n+na}{type}\PYG{o}{=}\PYG{l+s}{\PYGZdq{}video/mp4\PYGZdq{}}\PYG{p}{\PYGZgt{}}
\PYG{p}{\PYGZlt{}}\PYG{p}{/}\PYG{n+nt}{video}\PYG{p}{\PYGZgt{}}
\end{sphinxVerbatim}



Otro formato que suele usarse es ogv, acá la \sphinxhref{https://caniuse.com/\#feat=ogv}{tabla de compatibilidad del formato ogv}.


\subsection{Vista Previa}
\label{\detokenize{audio-y-video:vista-previa}}
Cuando la pagina carga y el video esta en pausa el navegador va a elegir una
vista previa automáticamente, si queremos tener mas control sobre la imagen
mostrada podemos especificarsela explícitamente con el atributo \sphinxtitleref{poster}:

\fvset{hllines={, ,}}%
\begin{sphinxVerbatim}[commandchars=\\\{\}]
\PYG{p}{\PYGZlt{}}\PYG{n+nt}{video} \PYG{n+na}{controls} \PYG{n+na}{poster}\PYG{o}{=}\PYG{l+s}{\PYGZdq{}/cew\PYGZus{}files/12/poster.png\PYGZdq{}}\PYG{p}{\PYGZgt{}}
    \PYG{p}{\PYGZlt{}}\PYG{n+nt}{source} \PYG{n+na}{src}\PYG{o}{=}\PYG{l+s}{\PYGZdq{}/cew\PYGZus{}files/12/example.webm\PYGZdq{}} \PYG{n+na}{type}\PYG{o}{=}\PYG{l+s}{\PYGZdq{}video/webm\PYGZdq{}}\PYG{p}{\PYGZgt{}}
    \PYG{p}{\PYGZlt{}}\PYG{n+nt}{source} \PYG{n+na}{src}\PYG{o}{=}\PYG{l+s}{\PYGZdq{}/cew\PYGZus{}files/12/example.mp4\PYGZdq{}} \PYG{n+na}{type}\PYG{o}{=}\PYG{l+s}{\PYGZdq{}video/mp4\PYGZdq{}}\PYG{p}{\PYGZgt{}}
\PYG{p}{\PYGZlt{}}\PYG{p}{/}\PYG{n+nt}{video}\PYG{p}{\PYGZgt{}}
\end{sphinxVerbatim}




\subsection{Subtítulos}
\label{\detokenize{audio-y-video:subtitulos}}
Ya sea por cuestiones de accesibilidad o para traducir o explicar el contenido
del video, podemos agregar subtítulos a un video usando el tag \sphinxtitleref{track}.

El formato del archivo es bastante simple:

\fvset{hllines={, ,}}%
\begin{sphinxVerbatim}[commandchars=\\\{\}]
\PYG{n}{WEBVTT}

\PYG{l+m+mi}{00}\PYG{p}{:}\PYG{l+m+mf}{01.000} \PYG{o}{\PYGZhy{}}\PYG{o}{\PYGZhy{}}\PYG{o}{\PYGZgt{}} \PYG{l+m+mi}{00}\PYG{p}{:}\PYG{l+m+mf}{04.000}
\PYG{n}{Primer} \PYG{n}{mensaje}\PYG{p}{,} \PYG{k}{del} \PYG{n}{segundo} \PYG{l+m+mi}{1} \PYG{n}{al} \PYG{l+m+mi}{4}

\PYG{l+m+mi}{00}\PYG{p}{:}\PYG{l+m+mf}{05.000} \PYG{o}{\PYGZhy{}}\PYG{o}{\PYGZhy{}}\PYG{o}{\PYGZgt{}} \PYG{l+m+mi}{00}\PYG{p}{:}\PYG{l+m+mf}{08.000}
\PYG{n}{Segundo} \PYG{n}{mensaje}\PYG{p}{,} \PYG{k}{del} \PYG{n}{segundo} \PYG{l+m+mi}{5} \PYG{n}{al} \PYG{l+m+mi}{8}

\PYG{o}{.}\PYG{o}{.}\PYG{o}{.}
\end{sphinxVerbatim}

Empieza con \sphinxtitleref{WEBVTT} en la primera linea, un salto de linea y luego
tantas veces como sea necesario:

\fvset{hllines={, ,}}%
\begin{sphinxVerbatim}[commandchars=\\\{\}]
\PYG{p}{[}\PYG{n}{Tiempo} \PYG{n}{comienzo}\PYG{p}{]} \PYG{o}{\PYGZhy{}}\PYG{o}{\PYGZhy{}}\PYG{o}{\PYGZgt{}} \PYG{p}{[}\PYG{n}{Tiempo} \PYG{n}{fin}\PYG{p}{]}
\PYG{n}{Texto} \PYG{k}{del} \PYG{n}{subtítulo}
\end{sphinxVerbatim}

Podemos tener mas de un tag track para agregar subtitulos en distintos idiomas
y marcar uno por defecto, aca un ejemplo con subtítulos en Español:

\fvset{hllines={, ,}}%
\begin{sphinxVerbatim}[commandchars=\\\{\}]
\PYG{p}{\PYGZlt{}}\PYG{n+nt}{video} \PYG{n+na}{controls} \PYG{n+na}{poster}\PYG{o}{=}\PYG{l+s}{\PYGZdq{}/cew\PYGZus{}files/12/poster.png\PYGZdq{}}\PYG{p}{\PYGZgt{}}
    \PYG{p}{\PYGZlt{}}\PYG{n+nt}{source} \PYG{n+na}{src}\PYG{o}{=}\PYG{l+s}{\PYGZdq{}/cew\PYGZus{}files/12/example.webm\PYGZdq{}} \PYG{n+na}{type}\PYG{o}{=}\PYG{l+s}{\PYGZdq{}video/webm\PYGZdq{}}\PYG{p}{\PYGZgt{}}
    \PYG{p}{\PYGZlt{}}\PYG{n+nt}{source} \PYG{n+na}{src}\PYG{o}{=}\PYG{l+s}{\PYGZdq{}/cew\PYGZus{}files/12/example.mp4\PYGZdq{}} \PYG{n+na}{type}\PYG{o}{=}\PYG{l+s}{\PYGZdq{}video/mp4\PYGZdq{}}\PYG{p}{\PYGZgt{}}

    \PYG{p}{\PYGZlt{}}\PYG{n+nt}{track} \PYG{n+na}{src}\PYG{o}{=}\PYG{l+s}{\PYGZdq{}/cew\PYGZus{}files/12/subtitulo.vtt\PYGZdq{}}
        \PYG{n+na}{label}\PYG{o}{=}\PYG{l+s}{\PYGZdq{}Subtitulos en Español\PYGZdq{}}
        \PYG{n+na}{kind}\PYG{o}{=}\PYG{l+s}{\PYGZdq{}captions\PYGZdq{}}
        \PYG{n+na}{srclang}\PYG{o}{=}\PYG{l+s}{\PYGZdq{}es\PYGZdq{}}
        \PYG{n+na}{default}\PYG{p}{\PYGZgt{}}

\PYG{p}{\PYGZlt{}}\PYG{p}{/}\PYG{n+nt}{video}\PYG{p}{\PYGZgt{}}
\end{sphinxVerbatim}




\subsection{Fragmentos}
\label{\detokenize{audio-y-video:fragmentos}}
Que pasa si tenemos un video bastante largo pero solo queremos mostrar un fragmento?

Para eso podemos especificarle el principio y/o final del fragmento que nos
interesa.

Notar que al momento de escribir esto es una característica bastante nueva, (ver \sphinxhref{https://caniuse.com/\#feat=media-fragments}{tabla de compatibilidad de media fragments} al momento de leer esto para
ver si sigue siendo nueva y poco soportada).

Podemos indicarle el comienzo (segundo 10) y que reproduzca hasta el final:

\fvset{hllines={, ,}}%
\begin{sphinxVerbatim}[commandchars=\\\{\}]
\PYG{n}{t}\PYG{o}{=}\PYG{l+m+mi}{10}
\end{sphinxVerbatim}

Indicar solo el final, que reproduzca del principio y reproduzca hasta el segundo 20:

\fvset{hllines={, ,}}%
\begin{sphinxVerbatim}[commandchars=\\\{\}]
\PYG{n}{t}\PYG{o}{=}\PYG{p}{,}\PYG{l+m+mi}{20}
\end{sphinxVerbatim}

O el principio y el final, que arranque en el segundo 10 y reproduzca hasta el segundo 20:

\fvset{hllines={, ,}}%
\begin{sphinxVerbatim}[commandchars=\\\{\}]
\PYG{n}{t}\PYG{o}{=}\PYG{l+m+mi}{10}\PYG{p}{,}\PYG{l+m+mi}{20}
\end{sphinxVerbatim}

Veamoslo en nuestro video, que reproduzca desde el segundo 3 al 8 (puede que
no funcione en tu navegador).

\fvset{hllines={, ,}}%
\begin{sphinxVerbatim}[commandchars=\\\{\}]
\PYG{p}{\PYGZlt{}}\PYG{n+nt}{video} \PYG{n+na}{src}\PYG{o}{=}\PYG{l+s}{\PYGZdq{}/cew\PYGZus{}files/12/example.mp4\PYGZsh{}t=3,8\PYGZdq{}} \PYG{n+na}{type}\PYG{o}{=}\PYG{l+s}{\PYGZdq{}video/mp4\PYGZdq{}} \PYG{n+na}{controls}\PYG{p}{\PYGZgt{}}\PYG{p}{\PYGZlt{}}\PYG{p}{/}\PYG{n+nt}{video}\PYG{p}{\PYGZgt{}}
\end{sphinxVerbatim}




\subsection{Embebiendo}
\label{\detokenize{audio-y-video:embebiendo}}
Y que pasa si quiero poner en mi pagina un video que esta en una pagina de
videos como youtube?

Para eso podemos \sphinxstyleemphasis{embeber} (embed en ingles) el contenido en nuestra pagina.

Si miras el video de los ejemplos de arriba, podrás ver que si vamos a share y
luego seleccionamos embed, youtube nos da un HTML que podemos incluir en
nuestra pagina para incluir el video directamente desde youtube.

\fvset{hllines={, ,}}%
\begin{sphinxVerbatim}[commandchars=\\\{\}]
\PYG{p}{\PYGZlt{}}\PYG{n+nt}{iframe} \PYG{n+na}{width}\PYG{o}{=}\PYG{l+s}{\PYGZdq{}560\PYGZdq{}} \PYG{n+na}{height}\PYG{o}{=}\PYG{l+s}{\PYGZdq{}315\PYGZdq{}}
    \PYG{n+na}{src}\PYG{o}{=}\PYG{l+s}{\PYGZdq{}https://www.youtube.com/embed/XM3eaJPB2Cc\PYGZdq{}}
    \PYG{n+na}{frameborder}\PYG{o}{=}\PYG{l+s}{\PYGZdq{}0\PYGZdq{}}
    \PYG{n+na}{allow}\PYG{o}{=}\PYG{l+s}{\PYGZdq{}autoplay; encrypted\PYGZhy{}media\PYGZdq{}}
    \PYG{n+na}{allowfullscreen}\PYG{p}{\PYGZgt{}}\PYG{p}{\PYGZlt{}}\PYG{p}{/}\PYG{n+nt}{iframe}\PYG{p}{\PYGZgt{}}
\end{sphinxVerbatim}



Podemos ver un video de youtube embebido que muestra un video de como embeber
un video de youtube :)

El dialogo en youtube nos permite configurar algunos parametros que cambian el
HTML que nos muestra, en el resultado de arriba vemos que podemos modificar el
ancho, alto, si tiene borde, si hace auto play y si permite ponerlo en pantalla
completa.


\section{Audio}
\label{\detokenize{audio-y-video:audio}}
Como con video, hay muchos formatos de audio y cada navegador soporta un subset
distinto, dado que hay mas formato de audio en uso listo las tablas de
compatibilidad primero:
\begin{itemize}
\item {} 
\sphinxhref{https://caniuse.com/\#feat=wav}{wav}

\item {} 
\sphinxhref{https://caniuse.com/\#feat=mp3}{mp3}

\item {} 
\sphinxhref{https://caniuse.com/\#feat=ogg-vorbis}{ogg}

\item {} 
\sphinxhref{https://caniuse.com/\#feat=aac}{aac}

\item {} 
\sphinxhref{https://caniuse.com/\#feat=opus}{opus}

\item {} 
\sphinxhref{https://caniuse.com/\#feat=flac}{flac}

\end{itemize}

La canción que vamos a usar de ejemplo es \sphinxhref{https://www.tribeofnoise.com/Solstar}{Rough Patches de Solstar}.

Empezamos con un audio en formato ogg:

\fvset{hllines={, ,}}%
\begin{sphinxVerbatim}[commandchars=\\\{\}]
\PYG{p}{\PYGZlt{}}\PYG{n+nt}{audio} \PYG{n+na}{controls} \PYG{n+na}{src}\PYG{o}{=}\PYG{l+s}{\PYGZdq{}/cew\PYGZus{}files/12/example.ogg\PYGZdq{}}\PYG{p}{\PYGZgt{}}\PYG{p}{\PYGZlt{}}\PYG{p}{/}\PYG{n+nt}{audio}\PYG{p}{\PYGZgt{}}
\end{sphinxVerbatim}



Como veras el HTML es bastante similar al tag video.

Si no funciona o si tenes un mp3:

\fvset{hllines={, ,}}%
\begin{sphinxVerbatim}[commandchars=\\\{\}]
\PYG{p}{\PYGZlt{}}\PYG{n+nt}{audio} \PYG{n+na}{controls} \PYG{n+na}{src}\PYG{o}{=}\PYG{l+s}{\PYGZdq{}/cew\PYGZus{}files/12/example.mp3\PYGZdq{}}\PYG{p}{\PYGZgt{}}\PYG{p}{\PYGZlt{}}\PYG{p}{/}\PYG{n+nt}{audio}\PYG{p}{\PYGZgt{}}
\end{sphinxVerbatim}



Pero si viste las tablas de compatibilidad y queres soportar la mayor cantidad
de navegadores, al igual que con el tag video se puede incluir mas de un
archivo.

\fvset{hllines={, ,}}%
\begin{sphinxVerbatim}[commandchars=\\\{\}]
\PYG{p}{\PYGZlt{}}\PYG{n+nt}{audio} \PYG{n+na}{controls}\PYG{p}{\PYGZgt{}}
    \PYG{p}{\PYGZlt{}}\PYG{n+nt}{source} \PYG{n+na}{src}\PYG{o}{=}\PYG{l+s}{\PYGZdq{}/cew\PYGZus{}files/12/example.ogg\PYGZdq{}} \PYG{n+na}{type}\PYG{o}{=}\PYG{l+s}{\PYGZdq{}audio/ogg\PYGZdq{}}\PYG{p}{/}\PYG{p}{\PYGZgt{}}
    \PYG{p}{\PYGZlt{}}\PYG{n+nt}{source} \PYG{n+na}{src}\PYG{o}{=}\PYG{l+s}{\PYGZdq{}/cew\PYGZus{}files/12/example.mp3\PYGZdq{}} \PYG{n+na}{type}\PYG{o}{=}\PYG{l+s}{\PYGZdq{}audio/mpeg\PYGZdq{}}\PYG{p}{/}\PYG{p}{\PYGZgt{}}
\PYG{p}{\PYGZlt{}}\PYG{p}{/}\PYG{n+nt}{audio}\PYG{p}{\PYGZgt{}}
\end{sphinxVerbatim}




\subsection{Embebiendo}
\label{\detokenize{audio-y-video:id1}}
Como con videos, hay paginas web que brindan audios y nos permiten embeberlos,
en este caso uno de los mas usados es soundcloud, al igual que en youtube, si
hacemos click en share y luego en embed, nos da un fragmento de HTML que podemos
incluir en nuestra pagina:

\fvset{hllines={, ,}}%
\begin{sphinxVerbatim}[commandchars=\\\{\}]
\PYG{p}{\PYGZlt{}}\PYG{n+nt}{iframe}
    \PYG{n+na}{width}\PYG{o}{=}\PYG{l+s}{\PYGZdq{}100\PYGZpc{}\PYGZdq{}}
    \PYG{n+na}{height}\PYG{o}{=}\PYG{l+s}{\PYGZdq{}300\PYGZdq{}}
    \PYG{n+na}{scrolling}\PYG{o}{=}\PYG{l+s}{\PYGZdq{}no\PYGZdq{}}
    \PYG{n+na}{frameborder}\PYG{o}{=}\PYG{l+s}{\PYGZdq{}no\PYGZdq{}}
    \PYG{n+na}{allow}\PYG{o}{=}\PYG{l+s}{\PYGZdq{}autoplay\PYGZdq{}}
    \PYG{n+na}{src}\PYG{o}{=}\PYG{l+s}{\PYGZdq{}https://w.soundcloud.com/player/?url=https\PYGZpc{}3A//api.soundcloud.com/tracks/72505324\PYGZam{}color=\PYGZpc{}23ff5500\PYGZam{}auto\PYGZus{}play=false\PYGZam{}hide\PYGZus{}related=false\PYGZam{}show\PYGZus{}comments=true\PYGZam{}show\PYGZus{}user=true\PYGZam{}show\PYGZus{}reposts=false\PYGZam{}show\PYGZus{}teaser=true\PYGZam{}visual=true\PYGZdq{}}\PYG{p}{\PYGZgt{}}
\PYG{p}{\PYGZlt{}}\PYG{p}{/}\PYG{n+nt}{iframe}\PYG{p}{\PYGZgt{}}
\end{sphinxVerbatim}



Aca hay un video de como obtener el HTML:




\chapter{Recursos online}
\label{\detokenize{recursos-online::doc}}\label{\detokenize{recursos-online:recursos-online}}
Si venís siguiente todas las secciones de esta serie habrás notado un patrón:
\begin{enumerate}
\item {} 
Esto parece bastante repetitivo

\item {} 
Abra alguien que haya hecho algo para facilitar esto?

\item {} 
Si!

\end{enumerate}

En esta sección vamos a ver algunos recursos que nos van a hacer mas
fácil empezar y adaptar los recursos disponibles a lo que necesitemos.


\section{Adaptando boostrap a nuestros gustos}
\label{\detokenize{recursos-online:adaptando-boostrap-a-nuestros-gustos}}
\sphinxhref{http://whootstrap.themes.guide/}{Whoostrap} cuenta con una lista de temas
para aplicar a bootstrap y cambiar su aspecto básico.

Acá hay un ejemplo de como usarlo:



Guardamos el texto del CSS en un archivo y lo incluimos en nuestro proyecto, aca hay un ejemplo: \sphinxurl{https://thimbleprojects.org/marianoguerra/512478/}

La pagina tambien provee algunos themes predefinidos \sphinxhref{http://themes.guide/}{themes.guide}

Otras paginas quen nos brinda themes gratuitos que podemos descargar y
usar: \sphinxhref{https://hackerthemes.com/}{hackerthemes} y \sphinxhref{https://demos.creative-tim.com/now-ui-kit/index.html}{Now UI Kit}


\section{Copiando fragmentos de HTML}
\label{\detokenize{recursos-online:copiando-fragmentos-de-html}}
Muchas de las partes de una pagina son generales y se repiten, por ejemplo
la barra de navegación superior, el pie de pagina, una lista de productos o
características, como esas cosas son repetitivas pero no hay una forma simple
de "abstraerlas" sin tener que aprender javascript, hay paginas que nos muestran
distintos fragmentos de HTML para componentes comunes. En ingles le llaman
cheatsheets, acá hay una de bootstrap que es muy útil: \sphinxhref{https://hackerthemes.com/bootstrap-cheatsheet/}{Bootstrap Cheatsheet}

Hace click en el componente que querés ver y te va a mostrar el HTML a la
izquierda y como se ve a la derecha.


\chapter{Un poco de lógica a la vista}
\label{\detokenize{un-poco-de-logica-a-la-vista:un-poco-de-logica-a-la-vista}}\label{\detokenize{un-poco-de-logica-a-la-vista::doc}}
Hasta el momento las paginas que creamos carecen de interactividad, el
contenido se muestra, pero no responde a ninguna acción de nuestra parte.

También sucede que son estáticas, todo el contenido de la pagina esta en el
HTML, no hay forma de usar el mismo HTML para mostrar información que varíe en
el tiempo o según contexto.

Para agregar dinamicidad y que la pagina muestre contenido distinto según el
contexto vamos a hacer uso de una herramienta llamada plantillas (template en
ingles) o también vistas (view en ingles).

Estas plantillas nos permiten describir el HTML con "huecos" indicando donde
van los datos que necesitamos, pero los datos provienen de otro lugar, la
plantilla toma los datos y los reemplaza en los "huecos".

Esto nos permite también ahorrarnos trabajo cuando tenemos que mostrar muchos
datos que tienen la misma estructura, definimos la plantilla para un elemento y
le indicamos a la plantilla que lo muestre tantas veces como elementos haya en
una lista.

Para agregar interactividad, es decir, que la pagina reaccione a acciones del
usuario, vamos a usar un nuevo lenguaje llamado javascript, que nos permite
indicar rutinas que modifican los datos en respuesta a acciones iniciadas por
el usuario. Nuestras plantillas son notificadas de los cambios en los datos y
actualizan su contenido.


\section{Contemos}
\label{\detokenize{un-poco-de-logica-a-la-vista:contemos}}
El primer ejemplo va a empezar simple, nuestro dato va a ser un numero, que
indica cuantas veces se apretó un botón, es decir, un contador.

Cuando el usuario hace click en el botón, incrementamos en 1 el contador.

Empecemos con el HTML que ya conocemos:

\fvset{hllines={, ,}}%
\begin{sphinxVerbatim}[commandchars=\\\{\}]
\PYG{p}{\PYGZlt{}}\PYG{n+nt}{div}\PYG{p}{\PYGZgt{}}
    \PYG{p}{\PYGZlt{}}\PYG{n+nt}{p}\PYG{p}{\PYGZgt{}}Contador: \PYG{p}{\PYGZlt{}}\PYG{n+nt}{span}\PYG{p}{\PYGZgt{}}0\PYG{p}{\PYGZlt{}}\PYG{p}{/}\PYG{n+nt}{span}\PYG{p}{\PYGZgt{}}\PYG{p}{\PYGZlt{}}\PYG{p}{/}\PYG{n+nt}{p}\PYG{p}{\PYGZgt{}}
    \PYG{p}{\PYGZlt{}}\PYG{n+nt}{button}\PYG{p}{\PYGZgt{}}Incrementar\PYG{p}{\PYGZlt{}}\PYG{p}{/}\PYG{n+nt}{button}\PYG{p}{\PYGZgt{}}
\PYG{p}{\PYGZlt{}}\PYG{p}{/}\PYG{n+nt}{div}\PYG{p}{\PYGZgt{}}
\end{sphinxVerbatim}

Que resulta en:



Lo único nuevo es el tag \sphinxtitleref{button} que no habíamos visto hasta ahora porque
no sirve de mucho si no sabemos como hacerlo interactivo.

Muy linda aplicación, pero notaras que si hacemos click en el botón no pasa nada...

Es porque no le indicamos dos cosas:
\begin{itemize}
\item {} 
Que sucede cuando se hace click en el botón

\item {} 
De donde saca el valor del contador y donde lo muestra

\end{itemize}

Para lo primero necesitamos indicarle al botón "cuando esto sucede, hace esto otro",
Para lo segundo necesitamos indicarle:
\begin{itemize}
\item {} 
El estado inicial de nuestros datos
\begin{itemize}
\item {} 
En este caso, un contador inicialmente en cero

\end{itemize}

\item {} 
Donde mostrar ese contador en nuestro HTML

\end{itemize}

Para eso vamos a usar un proyecto llamado \sphinxhref{https://vuejs.org/}{vuejs}, que
nos permite hacer esto y mucho mas.

Primero tenemos que incluir vuejs en nuestra pagina, así su funcionalidad esta
disponible, esto lo hacemos con un tag \sphinxtitleref{script} dentro del \sphinxtitleref{head} de nuestra pagina:

\fvset{hllines={, ,}}%
\begin{sphinxVerbatim}[commandchars=\\\{\}]
\PYG{p}{\PYGZlt{}}\PYG{n+nt}{script} \PYG{n+na}{src}\PYG{o}{=}\PYG{l+s}{\PYGZdq{}https://cdnjs.cloudflare.com/ajax/libs/vue/2.5.17/vue.min.js\PYGZdq{}}\PYG{p}{\PYGZgt{}}\PYG{p}{\PYGZlt{}}\PYG{p}{/}\PYG{n+nt}{script}\PYG{p}{\PYGZgt{}}
\end{sphinxVerbatim}



Lo segundo que necesitamos hacer, es indicarle a vuejs, que parte de nuestro HTML
es su responsabilidad, ya que podemos tener distintas partes de la pagina manejadas
por distintas aplicaciones. Esto se lo indicamos agregando un identificador al tag
raíz de nuestra aplicación e indicando ese identificador cuando inicializamos la
aplicación.

\fvset{hllines={, ,}}%
\begin{sphinxVerbatim}[commandchars=\\\{\}]
\PYG{p}{\PYGZlt{}}\PYG{n+nt}{div} \PYG{n+na}{id}\PYG{o}{=}\PYG{l+s}{\PYGZdq{}mi\PYGZhy{}app\PYGZhy{}1\PYGZdq{}}\PYG{p}{\PYGZgt{}}
    \PYG{c}{\PYGZlt{}!\PYGZhy{}\PYGZhy{}}\PYG{c}{ nuestra aplicación va acá }\PYG{c}{\PYGZhy{}\PYGZhy{}\PYGZgt{}}
\PYG{p}{\PYGZlt{}}\PYG{p}{/}\PYG{n+nt}{div}\PYG{p}{\PYGZgt{}}
\end{sphinxVerbatim}

Luego necesitamos inicializar nuestra app, para esto le indicamos cual es su
estado inicial y cual es el id de la raíz del HTML de nuestra app.

Esta parte va a requerir bastante explicación ya que vamos a usar un nuevo
lenguaje que quizás hayas oído mencionar: javascript.

\fvset{hllines={, ,}}%
\begin{sphinxVerbatim}[commandchars=\\\{\}]
\PYG{p}{\PYGZlt{}}\PYG{n+nt}{script}\PYG{p}{\PYGZgt{}}
  \PYG{k}{new} \PYG{n+nx}{Vue}\PYG{p}{(}\PYG{p}{\PYGZob{}}
    \PYG{n+nx}{el}\PYG{o}{:} \PYG{l+s+s1}{\PYGZsq{}\PYGZsh{}mi\PYGZhy{}app\PYGZhy{}1\PYGZsq{}}\PYG{p}{,}
    \PYG{n+nx}{data}\PYG{o}{:} \PYG{p}{\PYGZob{}}\PYG{n+nx}{count}\PYG{o}{:} \PYG{l+m+mi}{0}\PYG{p}{\PYGZcb{}}
  \PYG{p}{\PYGZcb{}}\PYG{p}{)}\PYG{p}{;}
\PYG{p}{\PYGZlt{}}\PYG{p}{/}\PYG{n+nt}{script}\PYG{p}{\PYGZgt{}}
\end{sphinxVerbatim}

El tag \sphinxtitleref{script} te parecerá conocido ya que lo usamos para incluir aplicaciones
de otros, como vuejs o a-frame en capítulos anteriores. Ahora vamos a escribir
nuestros propios programas. Empezando con uno corto.

Al incluir vue.min.js lo que hicimos fue cargar un archivo con código
javascript adentro, que lo que hace es registrar un identificador llamado \sphinxtitleref{Vue}
en nuestra pagina.

Este identificador es lo que se llama un \sphinxtitleref{constructor}, un constructor es una
función especial que al invocarla con la instrucción \sphinxtitleref{new} nos devuelve un nuevo
\sphinxtitleref{objecto}. No te preocupes, son muchos conceptos que vamos a ir explorando en
breve, pero por ahora sabe que para crear una nueva app usando vuejs, tenemos
que crear un nuevo objeto de tipo \sphinxtitleref{Vue}, el cual esta disponible porque incluimos
el script \sphinxtitleref{vue.min.js}.

El \sphinxtitleref{constructor} \sphinxtitleref{Vue} necesita información para crear una aplicación,
mínimamente necesita saber:
\begin{itemize}
\item {} 
Cual es el id del tag raíz donde la app va a correr

\end{itemize}

En nuestro caso el id es mi-app-1, como hay muchas formas de indicar el tag
raíz, para que vue sepa que es un id le ponemos \# al principio.
\begin{itemize}
\item {} 
Cual es el estado inicial de nuestra aplicación

\end{itemize}

En nuestro caso un solo campo, llamado \sphinxtitleref{count} inicializado a \sphinxtitleref{0}.

No te preocupes por ahora con los detalles de javascript, para las siguientes
apps vas a poder copiar el código y solo cambiar el id y los datos iniciales.

Ok, inicializamos nuestra app, pero el HTML esta vacío, como le decimos "mostrá el valor de \sphinxtitleref{count} acá"?

Como habrás visto hasta ahora, cuando un programador necesita decirle algo a
una computadora, en lugar de usar un formato existente, inventa un lenguaje
nuevo, hasta ahora aprendimos HTML, CSS y javascript, todos con formatos
distintos, con nuestro lenguaje de plantillas no iba a ser la excepción, pero
por suerte es bastante simple.

También vale la pena aclarar que con estos 4 lenguajes (HTML, CSS, javascript y
un lenguaje de plantillas) es suficiente para hacer cualquier tipo de
aplicación como las que usas día a día en internet.

Para indicar los "huecos" donde van los datos, usamos el siguiente formato:

\fvset{hllines={, ,}}%
\begin{sphinxVerbatim}[commandchars=\\\{\}]
\PYG{p}{\PYGZlt{}}\PYG{n+nt}{div} \PYG{n+na}{id}\PYG{o}{=}\PYG{l+s}{\PYGZdq{}mi\PYGZhy{}app\PYGZhy{}1\PYGZdq{}}\PYG{p}{\PYGZgt{}}
    \PYG{p}{\PYGZlt{}}\PYG{n+nt}{p}\PYG{p}{\PYGZgt{}}Contador: \PYG{p}{\PYGZlt{}}\PYG{n+nt}{span}\PYG{p}{\PYGZgt{}}\PYGZob{}\PYGZob{}count\PYGZcb{}\PYGZcb{}\PYG{p}{\PYGZlt{}}\PYG{p}{/}\PYG{n+nt}{span}\PYG{p}{\PYGZgt{}}\PYG{p}{\PYGZlt{}}\PYG{p}{/}\PYG{n+nt}{p}\PYG{p}{\PYGZgt{}}
    \PYG{p}{\PYGZlt{}}\PYG{n+nt}{button}\PYG{p}{\PYGZgt{}}Incrementar\PYG{p}{\PYGZlt{}}\PYG{p}{/}\PYG{n+nt}{button}\PYG{p}{\PYGZgt{}}
\PYG{p}{\PYGZlt{}}\PYG{p}{/}\PYG{n+nt}{div}\PYG{p}{\PYGZgt{}}
\end{sphinxVerbatim}

Como veras dentro del tag \sphinxtitleref{span} escribimos \sphinxtitleref{\{\{count\}\}} lo que significa "pone el valor del campo \sphinxtitleref{count} acá".

El resultado es:





Un avance, pero el botón aun no hace nada...

Lo que le queremos indicar es "cuando el usuario haga click, hace esto",
lo cual se logra agregando un atributo especial al botón, especial porque
lo entiende vuejs, ese atributo se llama \sphinxtitleref{@click}:

\fvset{hllines={, ,}}%
\begin{sphinxVerbatim}[commandchars=\\\{\}]
\PYG{p}{\PYGZlt{}}\PYG{n+nt}{div} \PYG{n+na}{id}\PYG{o}{=}\PYG{l+s}{\PYGZdq{}mi\PYGZhy{}app\PYGZhy{}1\PYGZdq{}}\PYG{p}{\PYGZgt{}}
    \PYG{p}{\PYGZlt{}}\PYG{n+nt}{p}\PYG{p}{\PYGZgt{}}Contador: \PYG{p}{\PYGZlt{}}\PYG{n+nt}{span}\PYG{p}{\PYGZgt{}}\PYGZob{}\PYGZob{}count\PYGZcb{}\PYGZcb{}\PYG{p}{\PYGZlt{}}\PYG{p}{/}\PYG{n+nt}{span}\PYG{p}{\PYGZgt{}}\PYG{p}{\PYGZlt{}}\PYG{p}{/}\PYG{n+nt}{p}\PYG{p}{\PYGZgt{}}
    \PYG{p}{\PYGZlt{}}\PYG{n+nt}{button} \PYG{n+na}{@click}\PYG{o}{=}\PYG{l+s}{\PYGZdq{}count = count + 1\PYGZdq{}}\PYG{p}{\PYGZgt{}}Incrementar\PYG{p}{\PYGZlt{}}\PYG{p}{/}\PYG{n+nt}{button}\PYG{p}{\PYGZgt{}}
\PYG{p}{\PYGZlt{}}\PYG{p}{/}\PYG{n+nt}{div}\PYG{p}{\PYGZgt{}}
\end{sphinxVerbatim}



La magia esta acá:

\fvset{hllines={, ,}}%
\begin{sphinxVerbatim}[commandchars=\\\{\}]
\PYG{p}{\PYGZlt{}}\PYG{n+nt}{button} \PYG{n+na}{@click}\PYG{o}{=}\PYG{l+s}{\PYGZdq{}count = count + 1\PYGZdq{}}\PYG{p}{\PYGZgt{}}Incrementar\PYG{p}{\PYGZlt{}}\PYG{p}{/}\PYG{n+nt}{button}\PYG{p}{\PYGZgt{}}
\end{sphinxVerbatim}

Le decimos "cuando el usuario haga click en este botón, corre el código
\sphinxtitleref{count = count + 1}, es decir, el nuevo valor de count es igual al viejo valor de
count mas 1.

Probalo haciendo click, debería incrementarse.


\section{Una Lista (de tareas)}
\label{\detokenize{un-poco-de-logica-a-la-vista:una-lista-de-tareas}}
Hasta ahora mostramos un valor y agregamos interactividad a nuestra pagina,
pero de la introducción aun falta una cosa: evitar repetición.

Eso vamos a ver ahora haciendo una aplicación para listar tareas.

Como siempre necesitamos tener un tag raíz para nuestra aplicación, un estado
inicial, mostrar los datos y agregar interactividad.

Nuestro estado inicial va a ser una lista con 0 o mas datos de tipo texto
indicando la tarea a realizar, empecemos con un par de tareas iniciales así
podemos practicar repetición antes de agregar interactividad:

\fvset{hllines={, ,}}%
\begin{sphinxVerbatim}[commandchars=\\\{\}]
\PYG{p}{\PYGZlt{}}\PYG{n+nt}{script}\PYG{p}{\PYGZgt{}}
  \PYG{k}{new} \PYG{n+nx}{Vue}\PYG{p}{(}\PYG{p}{\PYGZob{}}
    \PYG{n+nx}{el}\PYG{o}{:} \PYG{l+s+s1}{\PYGZsq{}\PYGZsh{}todo\PYGZhy{}app\PYGZsq{}}\PYG{p}{,}
    \PYG{n+nx}{data}\PYG{o}{:} \PYG{p}{\PYGZob{}}
        \PYG{n+nx}{tareas}\PYG{o}{:} \PYG{p}{[}
            \PYG{l+s+s1}{\PYGZsq{}Conquistar el mundo\PYGZsq{}}\PYG{p}{,}
            \PYG{l+s+s1}{\PYGZsq{}Abolir el patriarcado\PYGZsq{}}\PYG{p}{,}
            \PYG{l+s+s1}{\PYGZsq{}Comprar pan\PYGZsq{}}
        \PYG{p}{]}
    \PYG{p}{\PYGZcb{}}
  \PYG{p}{\PYGZcb{}}\PYG{p}{)}
\PYG{p}{\PYGZlt{}}\PYG{p}{/}\PYG{n+nt}{script}\PYG{p}{\PYGZgt{}}
\end{sphinxVerbatim}

Nuestro estado inicial contiene un campo llamado \sphinxtitleref{tareas} que es una lista de
tres valores, los tres son de tipo texto (otro tipo que vimos es el tipo
numérico para el contador)

Ahora con nuestra lista de tareas inicializada, podemos mostrarla en la pantalla,
si fuéramos a hacerlo a la vieja usanza, haríamos algo así:

\fvset{hllines={, ,}}%
\begin{sphinxVerbatim}[commandchars=\\\{\}]
\PYG{p}{\PYGZlt{}}\PYG{n+nt}{ul}\PYG{p}{\PYGZgt{}}
    \PYG{p}{\PYGZlt{}}\PYG{n+nt}{li}\PYG{p}{\PYGZgt{}}Conquistar el mundo\PYG{p}{\PYGZlt{}}\PYG{p}{/}\PYG{n+nt}{li}\PYG{p}{\PYGZgt{}}
    \PYG{p}{\PYGZlt{}}\PYG{n+nt}{li}\PYG{p}{\PYGZgt{}}Abolir el patriarcado\PYG{p}{\PYGZlt{}}\PYG{p}{/}\PYG{n+nt}{li}\PYG{p}{\PYGZgt{}}
    \PYG{p}{\PYGZlt{}}\PYG{n+nt}{li}\PYG{p}{\PYGZgt{}}Comprar pan\PYG{p}{\PYGZlt{}}\PYG{p}{/}\PYG{n+nt}{li}\PYG{p}{\PYGZgt{}}
\PYG{p}{\PYGZlt{}}\PYG{p}{/}\PYG{n+nt}{ul}\PYG{p}{\PYGZgt{}}
\end{sphinxVerbatim}

Que se vería así:



Pero obviamente esto no nos sirve, queremos listar las tareas de nuestra lista
de datos, con lo que aprendimos hasta ahora podríamos intentar:

\fvset{hllines={, ,}}%
\begin{sphinxVerbatim}[commandchars=\\\{\}]
\PYG{p}{\PYGZlt{}}\PYG{n+nt}{ul}\PYG{p}{\PYGZgt{}}
    \PYG{p}{\PYGZlt{}}\PYG{n+nt}{li}\PYG{p}{\PYGZgt{}}\PYGZob{}\PYGZob{}tarea\PYGZcb{}\PYGZcb{}\PYG{p}{\PYGZlt{}}\PYG{p}{/}\PYG{n+nt}{li}\PYG{p}{\PYGZgt{}}
\PYG{p}{\PYGZlt{}}\PYG{p}{/}\PYG{n+nt}{ul}\PYG{p}{\PYGZgt{}}
\end{sphinxVerbatim}

Pero esto no funciona porque no tenemos una sola tarea, sino muchas, y el
identificador tarea no esta definido, tenemos el identificador tareas, sin embargo, estamos bastante cerca..., para repetir un fragmento de HTML cuyo contenido
se encuentra en una lista tenemos que indicarle a vue algo así como "para cada tarea en la lista de tareas, mostrá este HTML", veamos como se hace:

\fvset{hllines={, ,}}%
\begin{sphinxVerbatim}[commandchars=\\\{\}]
\PYG{p}{\PYGZlt{}}\PYG{n+nt}{div} \PYG{n+na}{id}\PYG{o}{=}\PYG{l+s}{\PYGZdq{}todo\PYGZhy{}app\PYGZdq{}}\PYG{p}{\PYGZgt{}}
    \PYG{p}{\PYGZlt{}}\PYG{n+nt}{ul}\PYG{p}{\PYGZgt{}}
        \PYG{p}{\PYGZlt{}}\PYG{n+nt}{li} \PYG{n+na}{v\PYGZhy{}for}\PYG{o}{=}\PYG{l+s}{\PYGZdq{}tarea in tareas\PYGZdq{}}\PYG{p}{\PYGZgt{}}\PYGZob{}\PYGZob{}tarea\PYGZcb{}\PYGZcb{}\PYG{p}{\PYGZlt{}}\PYG{p}{/}\PYG{n+nt}{li}\PYG{p}{\PYGZgt{}}
    \PYG{p}{\PYGZlt{}}\PYG{p}{/}\PYG{n+nt}{ul}\PYG{p}{\PYGZgt{}}
\PYG{p}{\PYGZlt{}}\PYG{p}{/}\PYG{n+nt}{div}\PYG{p}{\PYGZgt{}}
\end{sphinxVerbatim}

Lo que resulta en:



Si entendés un poco de ingles veras que nuestra idea "para cada tarea en la
lista de tareas, mostrá este HTML" se traduce bastante similar.

Usamos el atributo \sphinxtitleref{v-for} (la v es de vue, \sphinxtitleref{v-for} es un atributo que vue
entiende, como \sphinxtitleref{@click} antes), dentro del valor del atributo le decimos,
"tarea in tareas", lo cual completo \sphinxtitleref{v-for="tarea in tareas"} leído de corrido
casi se lee "for tarea in tareas" que se traduce "para tarea en tareas".

El tag con el atributo \sphinxtitleref{v-for} y sus tags hijos se van a repetir tantas veces
como elementos haya en la lista \sphinxtitleref{tareas}, en nuestro caso 3 veces.


\subsection{Agregando nuevas tareas}
\label{\detokenize{un-poco-de-logica-a-la-vista:agregando-nuevas-tareas}}
Como agregamos nuevas tareas a nuestra lista? para eso vamos a necesitar un
lugar donde podamos escribir la descripción de la nueva tarea y un botón para
agregar la tarea a la lista.

El campo de texto lo creamos con el tag \sphinxtitleref{input}, el botón como ya vimos antes,
con el tag \sphinxtitleref{button}, probemos un poco el HTML:

\fvset{hllines={, ,}}%
\begin{sphinxVerbatim}[commandchars=\\\{\}]
\PYG{p}{\PYGZlt{}}\PYG{n+nt}{input}\PYG{p}{\PYGZgt{}}
\PYG{p}{\PYGZlt{}}\PYG{n+nt}{button}\PYG{p}{\PYGZgt{}}Crear nueva tarea\PYG{p}{\PYGZlt{}}\PYG{p}{/}\PYG{n+nt}{button}\PYG{p}{\PYGZgt{}}
\end{sphinxVerbatim}



Muy bonito pero eso no hace nada, como "conecto" el contenido del tag \sphinxtitleref{input}
a un campo en mis datos?

primero necesitamos crear un nuevo campo en nuestros datos iniciales para
el contenido de la tarea a agregar, luego necesitamos indicarle al tag \sphinxtitleref{input}
que su contenido es el valor del campo, llamemoslo \sphinxtitleref{tituloNuevo}:

\fvset{hllines={, ,}}%
\begin{sphinxVerbatim}[commandchars=\\\{\}]
\PYG{p}{\PYGZlt{}}\PYG{n+nt}{div} \PYG{n+na}{id}\PYG{o}{=}\PYG{l+s}{\PYGZdq{}todo\PYGZhy{}app\PYGZdq{}}\PYG{p}{\PYGZgt{}}
  \PYG{p}{\PYGZlt{}}\PYG{n+nt}{input} \PYG{n+na}{v\PYGZhy{}model}\PYG{o}{=}\PYG{l+s}{\PYGZdq{}tituloNuevo\PYGZdq{}}\PYG{p}{\PYGZgt{}}
  \PYG{p}{\PYGZlt{}}\PYG{n+nt}{button} \PYG{n+na}{@click}\PYG{o}{=}\PYG{l+s}{\PYGZdq{}tareas.push(tituloNuevo); tituloNuevo = \PYGZsq{}\PYGZsq{};\PYGZdq{}}\PYG{p}{\PYGZgt{}}Crear nueva tarea\PYG{p}{\PYGZlt{}}\PYG{p}{/}\PYG{n+nt}{button}\PYG{p}{\PYGZgt{}}

  \PYG{p}{\PYGZlt{}}\PYG{n+nt}{ul}\PYG{p}{\PYGZgt{}}
    \PYG{p}{\PYGZlt{}}\PYG{n+nt}{li} \PYG{n+na}{v\PYGZhy{}for}\PYG{o}{=}\PYG{l+s}{\PYGZdq{}tarea in tareas\PYGZdq{}}\PYG{p}{\PYGZgt{}}\PYGZob{}\PYGZob{}tarea\PYGZcb{}\PYGZcb{}\PYG{p}{\PYGZlt{}}\PYG{p}{/}\PYG{n+nt}{li}\PYG{p}{\PYGZgt{}}
  \PYG{p}{\PYGZlt{}}\PYG{p}{/}\PYG{n+nt}{ul}\PYG{p}{\PYGZgt{}}
\PYG{p}{\PYGZlt{}}\PYG{p}{/}\PYG{n+nt}{div}\PYG{p}{\PYGZgt{}}
\PYG{p}{\PYGZlt{}}\PYG{n+nt}{script}\PYG{p}{\PYGZgt{}}
  \PYG{k}{new} \PYG{n+nx}{Vue}\PYG{p}{(}\PYG{p}{\PYGZob{}}
    \PYG{n+nx}{el}\PYG{o}{:} \PYG{l+s+s1}{\PYGZsq{}\PYGZsh{}todo\PYGZhy{}app\PYGZsq{}}\PYG{p}{,}
    \PYG{n+nx}{data}\PYG{o}{:} \PYG{p}{\PYGZob{}}
        \PYG{n+nx}{tituloNuevo}\PYG{o}{:} \PYG{l+s+s1}{\PYGZsq{}\PYGZsq{}}\PYG{p}{,}
        \PYG{n+nx}{tareas}\PYG{o}{:} \PYG{p}{[}
            \PYG{l+s+s1}{\PYGZsq{}Conquistar el mundo\PYGZsq{}}\PYG{p}{,}
            \PYG{l+s+s1}{\PYGZsq{}Abolir el patriarcado\PYGZsq{}}\PYG{p}{,}
            \PYG{l+s+s1}{\PYGZsq{}Comprar pan\PYGZsq{}}
        \PYG{p}{]}
    \PYG{p}{\PYGZcb{}}
  \PYG{p}{\PYGZcb{}}\PYG{p}{)}
\PYG{p}{\PYGZlt{}}\PYG{p}{/}\PYG{n+nt}{script}\PYG{p}{\PYGZgt{}}
\end{sphinxVerbatim}

Probemos:



Vamos por partes:

\fvset{hllines={, ,}}%
\begin{sphinxVerbatim}[commandchars=\\\{\}]
\PYG{p}{\PYGZlt{}}\PYG{n+nt}{input} \PYG{n+na}{v\PYGZhy{}model}\PYG{o}{=}\PYG{l+s}{\PYGZdq{}tituloNuevo\PYGZdq{}}\PYG{p}{\PYGZgt{}}
\end{sphinxVerbatim}

Agregamos el atributo \sphinxtitleref{v-model} para indicarle a vue "el contenido de este tag
esta conectado al valor de \sphinxtitleref{tituloNuevo} en nuestros datos.

\fvset{hllines={, ,}}%
\begin{sphinxVerbatim}[commandchars=\\\{\}]
\PYG{p}{\PYGZlt{}}\PYG{n+nt}{button} \PYG{n+na}{@click}\PYG{o}{=}\PYG{l+s}{\PYGZdq{}tareas.push(tituloNuevo); tituloNuevo = \PYGZsq{}\PYGZsq{};\PYGZdq{}}\PYG{p}{\PYGZgt{}}Crear nueva tarea\PYG{p}{\PYGZlt{}}\PYG{p}{/}\PYG{n+nt}{button}\PYG{p}{\PYGZgt{}}
\end{sphinxVerbatim}

Nuestro viejo amigo \sphinxtitleref{@click} en el botón hace dos cosas, primero agrega un
elemento al final de la lista \sphinxtitleref{tareas}, usando el método \sphinxtitleref{push}, que agrega un
elemento al final de la lista que esta antes del punto, el elemento a agregar
se lo indicamos entre paréntesis, en este caso queremos agregar el valor
contenido en \sphinxtitleref{tituloNuevo}.

Pero eso no es todo, también queremos limpiar el contenido del tag \sphinxtitleref{input} así
el usuario puede escribir una tarea nueva sin tener que borrar el contenido
que ya se agrego a la lista.

Para eso necesitamos indicar una instrucción nueva, y como ya tenemos una, necesitamos separarla, en javascript las instrucciones se separan con punto y coma.

La segunda instrucción actualiza el valor de \sphinxtitleref{tituloNuevo} a el texto vacío,
indicado con dos comillas juntas \sphinxtitleref{''}.


\subsection{Borrando tareas}
\label{\detokenize{un-poco-de-logica-a-la-vista:borrando-tareas}}
Agregar tareas sin poderlas eliminar no suena a una buena experiencia de
usuario, necesitamos poder borrar tareas, para eso al lado de cada tarea vamos
a agregar un botón que al ser clickeado va a borrar dicha tarea.

Pero para poder borrar la tarea necesitamos saber su posición en la lista
para poder decir "borra el elemento numero 2 de la lista \sphinxtitleref{tareas}", para
eso vamos a hacer uso de una variación del atributo \sphinxtitleref{v-for} que nos permite
obtener la posición (indice) del valor que estamos mostrando, el formato es:

\fvset{hllines={, ,}}%
\begin{sphinxVerbatim}[commandchars=\\\{\}]
\PYG{p}{\PYGZlt{}}\PYG{n+nt}{li} \PYG{n+na}{v\PYGZhy{}for}\PYG{o}{=}\PYG{l+s}{\PYGZdq{}(tarea, pos) in tareas\PYGZdq{}}\PYG{p}{\PYGZgt{}}
    \PYG{p}{\PYGZlt{}}\PYG{n+nt}{span}\PYG{p}{\PYGZgt{}}\PYGZob{}\PYGZob{}pos\PYGZcb{}\PYGZcb{}: \PYGZob{}\PYGZob{}tarea\PYGZcb{}\PYGZcb{}\PYG{p}{\PYGZlt{}}\PYG{p}{/}\PYG{n+nt}{span}\PYG{p}{\PYGZgt{}}
    \PYG{p}{\PYGZlt{}}\PYG{n+nt}{button} \PYG{n+na}{@click}\PYG{o}{=}\PYG{l+s}{\PYGZdq{}\PYGZdl{}delete(tareas, pos)\PYGZdq{}} \PYG{n+na}{style}\PYG{o}{=}\PYG{l+s}{\PYGZdq{}float: right\PYGZdq{}}\PYG{p}{\PYGZgt{}}X\PYG{p}{\PYGZlt{}}\PYG{p}{/}\PYG{n+nt}{button}\PYG{p}{\PYGZgt{}}
\PYG{p}{\PYGZlt{}}\PYG{p}{/}\PYG{n+nt}{li}\PYG{p}{\PYGZgt{}}
\end{sphinxVerbatim}

Vamos por partes, primero usamos el formato \sphinxtitleref{(tarea, pos) in tareas} para que vue
nos devuelva no solo cada elemento en la lista sino si posición (indice), el cual
vamos a nombrar \sphinxtitleref{pos}.

\fvset{hllines={, ,}}%
\begin{sphinxVerbatim}[commandchars=\\\{\}]
\PYG{p}{\PYGZlt{}}\PYG{n+nt}{li} \PYG{n+na}{v\PYGZhy{}for}\PYG{o}{=}\PYG{l+s}{\PYGZdq{}(tarea, pos) in tareas\PYGZdq{}}\PYG{p}{\PYGZgt{}}
\end{sphinxVerbatim}

Luego para ver que estamos haciendo las cosas bien, mostramos el valor de \sphinxtitleref{pos}
antes de la descripción de cada tarea:

\fvset{hllines={, ,}}%
\begin{sphinxVerbatim}[commandchars=\\\{\}]
\PYG{p}{\PYGZlt{}}\PYG{n+nt}{span}\PYG{p}{\PYGZgt{}}\PYGZob{}\PYGZob{}pos\PYGZcb{}\PYGZcb{}: \PYGZob{}\PYGZob{}tarea\PYGZcb{}\PYGZcb{}\PYG{p}{\PYGZlt{}}\PYG{p}{/}\PYG{n+nt}{span}\PYG{p}{\PYGZgt{}}
\end{sphinxVerbatim}

Finalmente agregamos un botón, que al ser clickeado llama a la función
\sphinxtitleref{\$delete} de vuejs, que recibe dos parámetros, el primero es la lista a la que
le queremos remover un elemento, el segundo parámetro es la posición o indice
que queremos remover.

\fvset{hllines={, ,}}%
\begin{sphinxVerbatim}[commandchars=\\\{\}]
\PYG{p}{\PYGZlt{}}\PYG{n+nt}{button} \PYG{n+na}{@click}\PYG{o}{=}\PYG{l+s}{\PYGZdq{}\PYGZdl{}delete(tareas, pos)\PYGZdq{}} \PYG{n+na}{style}\PYG{o}{=}\PYG{l+s}{\PYGZdq{}float: right\PYGZdq{}}\PYG{p}{\PYGZgt{}}X\PYG{p}{\PYGZlt{}}\PYG{p}{/}\PYG{n+nt}{button}\PYG{p}{\PYGZgt{}}
\end{sphinxVerbatim}

El resultado con un poco mas de formato:



Código completo:

\fvset{hllines={, ,}}%
\begin{sphinxVerbatim}[commandchars=\\\{\}]
\PYG{c+cp}{\PYGZlt{}!doctype html\PYGZgt{}}
\PYG{p}{\PYGZlt{}}\PYG{n+nt}{html}\PYG{p}{\PYGZgt{}}
  \PYG{p}{\PYGZlt{}}\PYG{n+nt}{head}\PYG{p}{\PYGZgt{}}
        \PYG{p}{\PYGZlt{}}\PYG{n+nt}{meta} \PYG{n+na}{charset}\PYG{o}{=}\PYG{l+s}{\PYGZdq{}utf\PYGZhy{}8\PYGZdq{}}\PYG{p}{\PYGZgt{}}
        \PYG{p}{\PYGZlt{}}\PYG{n+nt}{title}\PYG{p}{\PYGZgt{}}Vue: Lista de Tareas\PYG{p}{\PYGZlt{}}\PYG{p}{/}\PYG{n+nt}{title}\PYG{p}{\PYGZgt{}}
        \PYG{p}{\PYGZlt{}}\PYG{n+nt}{script} \PYG{n+na}{src}\PYG{o}{=}\PYG{l+s}{\PYGZdq{}https://cdnjs.cloudflare.com/ajax/libs/vue/2.5.17/vue.min.js\PYGZdq{}}\PYG{p}{\PYGZgt{}}\PYG{p}{\PYGZlt{}}\PYG{p}{/}\PYG{n+nt}{script}\PYG{p}{\PYGZgt{}}
        \PYG{p}{\PYGZlt{}}\PYG{n+nt}{link} \PYG{n+na}{rel}\PYG{o}{=}\PYG{l+s}{\PYGZdq{}stylesheet\PYGZdq{}} \PYG{n+na}{href}\PYG{o}{=}\PYG{l+s}{\PYGZdq{}https://stackpath.bootstrapcdn.com/bootstrap/4.1.3/css/bootstrap.min.css\PYGZdq{}}\PYG{p}{\PYGZgt{}}
  \PYG{p}{\PYGZlt{}}\PYG{p}{/}\PYG{n+nt}{head}\PYG{p}{\PYGZgt{}}
  \PYG{p}{\PYGZlt{}}\PYG{n+nt}{body} \PYG{n+na}{class}\PYG{o}{=}\PYG{l+s}{\PYGZdq{}m\PYGZhy{}3\PYGZdq{}}\PYG{p}{\PYGZgt{}}
        \PYG{p}{\PYGZlt{}}\PYG{n+nt}{div} \PYG{n+na}{id}\PYG{o}{=}\PYG{l+s}{\PYGZdq{}todo\PYGZhy{}app\PYGZdq{}}\PYG{p}{\PYGZgt{}}
          \PYG{p}{\PYGZlt{}}\PYG{n+nt}{input} \PYG{n+na}{v\PYGZhy{}model}\PYG{o}{=}\PYG{l+s}{\PYGZdq{}tituloNuevo\PYGZdq{}} \PYG{n+na}{type}\PYG{o}{=}\PYG{l+s}{\PYGZdq{}text\PYGZdq{}} \PYG{n+na}{class}\PYG{o}{=}\PYG{l+s}{\PYGZdq{}form\PYGZhy{}control\PYGZdq{}} \PYG{n+na}{id}\PYG{o}{=}\PYG{l+s}{\PYGZdq{}tareaDesc\PYGZdq{}} \PYG{n+na}{placeholder}\PYG{o}{=}\PYG{l+s}{\PYGZdq{}Descripción de tarea\PYGZdq{}}\PYG{p}{\PYGZgt{}}

          \PYG{p}{\PYGZlt{}}\PYG{n+nt}{div} \PYG{n+na}{class}\PYG{o}{=}\PYG{l+s}{\PYGZdq{}my\PYGZhy{}2 text\PYGZhy{}center\PYGZdq{}}\PYG{p}{\PYGZgt{}}
                \PYG{p}{\PYGZlt{}}\PYG{n+nt}{button} \PYG{n+na}{@click}\PYG{o}{=}\PYG{l+s}{\PYGZdq{}tareas.push(tituloNuevo); tituloNuevo = \PYGZsq{}\PYGZsq{};\PYGZdq{}}
                                \PYG{n+na}{class}\PYG{o}{=}\PYG{l+s}{\PYGZdq{}btn btn\PYGZhy{}primary\PYGZdq{}}\PYG{p}{\PYGZgt{}}Crear nueva tarea\PYG{p}{\PYGZlt{}}\PYG{p}{/}\PYG{n+nt}{button}\PYG{p}{\PYGZgt{}}
          \PYG{p}{\PYGZlt{}}\PYG{p}{/}\PYG{n+nt}{div}\PYG{p}{\PYGZgt{}}

          \PYG{p}{\PYGZlt{}}\PYG{n+nt}{table} \PYG{n+na}{class}\PYG{o}{=}\PYG{l+s}{\PYGZdq{}table table\PYGZhy{}bordered table\PYGZhy{}striped table\PYGZhy{}hover\PYGZdq{}}\PYG{p}{\PYGZgt{}}
                \PYG{p}{\PYGZlt{}}\PYG{n+nt}{tbody}\PYG{p}{\PYGZgt{}}
                  \PYG{p}{\PYGZlt{}}\PYG{n+nt}{tr} \PYG{n+na}{v\PYGZhy{}for}\PYG{o}{=}\PYG{l+s}{\PYGZdq{}(tarea, pos) in tareas\PYGZdq{}}\PYG{p}{\PYGZgt{}}
                        \PYG{p}{\PYGZlt{}}\PYG{n+nt}{td}\PYG{p}{\PYGZgt{}}\PYGZob{}\PYGZob{}pos\PYGZcb{}\PYGZcb{}\PYG{p}{\PYGZlt{}}\PYG{p}{/}\PYG{n+nt}{td}\PYG{p}{\PYGZgt{}}
                        \PYG{p}{\PYGZlt{}}\PYG{n+nt}{td}\PYG{p}{\PYGZgt{}}\PYGZob{}\PYGZob{}tarea\PYGZcb{}\PYGZcb{}\PYG{p}{\PYGZlt{}}\PYG{p}{/}\PYG{n+nt}{td}\PYG{p}{\PYGZgt{}}
                        \PYG{p}{\PYGZlt{}}\PYG{n+nt}{td} \PYG{n+na}{class}\PYG{o}{=}\PYG{l+s}{\PYGZdq{}text\PYGZhy{}center\PYGZdq{}}\PYG{p}{\PYGZgt{}}
                          \PYG{p}{\PYGZlt{}}\PYG{n+nt}{button} \PYG{n+na}{@click}\PYG{o}{=}\PYG{l+s}{\PYGZdq{}\PYGZdl{}delete(tareas, pos)\PYGZdq{}} \PYG{n+na}{class}\PYG{o}{=}\PYG{l+s}{\PYGZdq{}btn btn\PYGZhy{}danger btn\PYGZhy{}sm\PYGZdq{}}\PYG{p}{\PYGZgt{}}X\PYG{p}{\PYGZlt{}}\PYG{p}{/}\PYG{n+nt}{button}\PYG{p}{\PYGZgt{}}
                        \PYG{p}{\PYGZlt{}}\PYG{p}{/}\PYG{n+nt}{td}\PYG{p}{\PYGZgt{}}
                  \PYG{p}{\PYGZlt{}}\PYG{p}{/}\PYG{n+nt}{tr}\PYG{p}{\PYGZgt{}}
                \PYG{p}{\PYGZlt{}}\PYG{p}{/}\PYG{n+nt}{tbody}\PYG{p}{\PYGZgt{}}
          \PYG{p}{\PYGZlt{}}\PYG{p}{/}\PYG{n+nt}{table}\PYG{p}{\PYGZgt{}}
        \PYG{p}{\PYGZlt{}}\PYG{p}{/}\PYG{n+nt}{div}\PYG{p}{\PYGZgt{}}
        \PYG{p}{\PYGZlt{}}\PYG{n+nt}{script}\PYG{p}{\PYGZgt{}}
          \PYG{k}{new} \PYG{n+nx}{Vue}\PYG{p}{(}\PYG{p}{\PYGZob{}}
                \PYG{n+nx}{el}\PYG{o}{:} \PYG{l+s+s1}{\PYGZsq{}\PYGZsh{}todo\PYGZhy{}app\PYGZsq{}}\PYG{p}{,}
                \PYG{n+nx}{data}\PYG{o}{:} \PYG{p}{\PYGZob{}}
                  \PYG{n+nx}{tituloNuevo}\PYG{o}{:} \PYG{l+s+s1}{\PYGZsq{}\PYGZsq{}}\PYG{p}{,}
                  \PYG{n+nx}{tareas}\PYG{o}{:} \PYG{p}{[}
                        \PYG{l+s+s1}{\PYGZsq{}Conquistar el mundo\PYGZsq{}}\PYG{p}{,}
                        \PYG{l+s+s1}{\PYGZsq{}Abolir el patriarcado\PYGZsq{}}\PYG{p}{,}
                        \PYG{l+s+s1}{\PYGZsq{}Comprar pan\PYGZsq{}}
                  \PYG{p}{]}
                \PYG{p}{\PYGZcb{}}
          \PYG{p}{\PYGZcb{}}\PYG{p}{)}
        \PYG{p}{\PYGZlt{}}\PYG{p}{/}\PYG{n+nt}{script}\PYG{p}{\PYGZgt{}}
  \PYG{p}{\PYGZlt{}}\PYG{p}{/}\PYG{n+nt}{body}\PYG{p}{\PYGZgt{}}
\PYG{p}{\PYGZlt{}}\PYG{p}{/}\PYG{n+nt}{html}\PYG{p}{\PYGZgt{}}
\end{sphinxVerbatim}


\chapter{Datos con Javascript}
\label{\detokenize{datos-con-javascript::doc}}\label{\detokenize{datos-con-javascript:datos-con-javascript}}
En el capítulo anterior {\hyperref[\detokenize{un-poco-de-logica-a-la-vista::doc}]{\sphinxcrossref{\DUrole{doc}{Un poco de lógica a la vista}}}}
usamos un poco de javascript pero sin una introducción formal, ya es tiempo de
conocerlo un poco mejor, empezando con la forma de expresar distintos tipos de
datos, esto nos va a permitir definir el estado inicial de nuestra aplicación.

Antes de empezar de lleno te recomiendo que escribas los ejemplos que voy
poniendo así te vas acostumbrando a programar, es muy común pensar que solo
leyendo uno entiende las cosas, pero es muy fácil de darse cuenta que no es tan
así cuando intentamos escribirlo por nuestra cuenta y empezamos a notar
detalles importantes que se nos pasaron de largo al leerlo.

Para poder probar hay dos formas, la primera es abrir la sección de
consola interactiva en las herramientas de desarrollo de tu navegador.

Las herramientas de desarrollo se abren apretando la tecla F12 en Firefox o
Chrome, esto nos va a abrir un panel en la parte inferior de la pantalla, la
cual tiene múltiples secciones, la que vamos a usar hoy es la que normalmente
es la tercera de izquierda a derecha, llamada "Consola", hace click en el tab
"Consola" o "Console" según el idioma de tu navegador.

En la parte inferior hay un símbolo \sphinxtitleref{\textgreater{}} o \sphinxtitleref{\textgreater{}\textgreater{}}, si hacemos click a la derecha
va a aparecer un cursor, ahí podemos escribir javascript, el cual se va a
ejecutar cuando apretemos enter y nos va a mostrar el resultado (o un error).

La otra forma es usar alguna consola interactiva online, una conocida es
\sphinxhref{http://repl.it/languages/javascript}{repl.it}, hace click y debería abrirte
una pagina donde podes escribir en el panel del medio y ejecutarlo apretando la
tecla Control y Enter a la vez, el resultado debería aparecer en el panel de la
derecha.

Acá hay un video mostrando como usarlo:



Ya tenemos lo que necesitamos para empezar.


\section{Números Decimales}
\label{\detokenize{datos-con-javascript:numeros-decimales}}
Un tipo de dato muy usado, tenemos dos tipos, los enteros (integer en ingles),
son números sin coma decimal, veamos algunos:

Escribiendo lo siguiente:

\fvset{hllines={, ,}}%
\begin{sphinxVerbatim}[commandchars=\\\{\}]
\PYG{l+m+mi}{1}
\end{sphinxVerbatim}

Si apretamos enter en la consola del navegador o Ctrl+Enter en repl.it el
resultado que nos va a devolver es:

\fvset{hllines={, ,}}%
\begin{sphinxVerbatim}[commandchars=\\\{\}]
\PYG{o}{\PYGZlt{}} \PYG{l+m+mi}{1}
\end{sphinxVerbatim}

Nada mágico, decimos que javascript evaluó el código \sphinxtitleref{1} y el resultado fue
\sphinxtitleref{1}, aunque no lo creas eso es un programa valido, aunque no uno muy útil.

Veamos otros números, de ahora en mas voy a poner el código y el resultado, se
entiende que tenes que escribirlo y apretar enter o Ctrl+enter según donde
estés escribiendo el código para ver el resultado.

\fvset{hllines={, ,}}%
\begin{sphinxVerbatim}[commandchars=\\\{\}]
\PYG{l+m+mi}{2}
\end{sphinxVerbatim}

\fvset{hllines={, ,}}%
\begin{sphinxVerbatim}[commandchars=\\\{\}]
\PYG{o}{\PYGZlt{}} \PYG{l+m+mi}{2}
\end{sphinxVerbatim}

\fvset{hllines={, ,}}%
\begin{sphinxVerbatim}[commandchars=\\\{\}]
\PYG{l+m+mi}{0}
\end{sphinxVerbatim}

\fvset{hllines={, ,}}%
\begin{sphinxVerbatim}[commandchars=\\\{\}]
\PYG{o}{\PYGZlt{}} \PYG{l+m+mi}{0}
\end{sphinxVerbatim}

\fvset{hllines={, ,}}%
\begin{sphinxVerbatim}[commandchars=\\\{\}]
\PYG{l+m+mi}{42}
\end{sphinxVerbatim}

\fvset{hllines={, ,}}%
\begin{sphinxVerbatim}[commandchars=\\\{\}]
\PYG{o}{\PYGZlt{}} \PYG{l+m+mi}{42}
\end{sphinxVerbatim}

También se pueden números negativos:

\fvset{hllines={, ,}}%
\begin{sphinxVerbatim}[commandchars=\\\{\}]
\PYG{o}{\PYGZhy{}}\PYG{l+m+mi}{42}
\end{sphinxVerbatim}

\fvset{hllines={, ,}}%
\begin{sphinxVerbatim}[commandchars=\\\{\}]
\PYG{o}{\PYGZlt{}} \PYG{o}{\PYGZhy{}}\PYG{l+m+mi}{42}
\end{sphinxVerbatim}

Números grandes

\fvset{hllines={, ,}}%
\begin{sphinxVerbatim}[commandchars=\\\{\}]
\PYG{l+m+mi}{1234567890}
\end{sphinxVerbatim}

\fvset{hllines={, ,}}%
\begin{sphinxVerbatim}[commandchars=\\\{\}]
\PYG{o}{\PYGZlt{}} \PYG{l+m+mi}{1234567890}
\end{sphinxVerbatim}


\section{Números Decimales}
\label{\detokenize{datos-con-javascript:id1}}
También podemos escribir números con decimales, en programación los vas a
escuchar nombrar como "float", "floating point number", o "double", en lugar de
una coma se usa un punto:

\fvset{hllines={, ,}}%
\begin{sphinxVerbatim}[commandchars=\\\{\}]
\PYG{l+m+mf}{0.5}
\end{sphinxVerbatim}

\fvset{hllines={, ,}}%
\begin{sphinxVerbatim}[commandchars=\\\{\}]
\PYG{o}{\PYGZlt{}} \PYG{l+m+mf}{0.5}
\end{sphinxVerbatim}

\fvset{hllines={, ,}}%
\begin{sphinxVerbatim}[commandchars=\\\{\}]
\PYG{l+m+mf}{10.5}
\end{sphinxVerbatim}

\fvset{hllines={, ,}}%
\begin{sphinxVerbatim}[commandchars=\\\{\}]
\PYG{o}{\PYGZlt{}} \PYG{l+m+mf}{10.5}
\end{sphinxVerbatim}

\fvset{hllines={, ,}}%
\begin{sphinxVerbatim}[commandchars=\\\{\}]
\PYG{o}{\PYGZhy{}}\PYG{l+m+mf}{0.5}
\end{sphinxVerbatim}

\fvset{hllines={, ,}}%
\begin{sphinxVerbatim}[commandchars=\\\{\}]
\PYG{o}{\PYGZlt{}} \PYG{o}{\PYGZhy{}}\PYG{l+m+mf}{0.5}
\end{sphinxVerbatim}


\section{Texto}
\label{\detokenize{datos-con-javascript:texto}}
Otro tipo de dato muy importante es el texto, como vas a ir viendo a los
programadores les gusta poner nombres raros a las cosas y nunca decidir
cambiarlos, por mas que la razón del nombre ya se haya olvidado.

En este caso, al tipo de dato texto en programación se le suele llamar cadena
de texto, en ingles "string".

En javascript el texto es cualquier cosa envuelta en comillas simples \sphinxtitleref{'} o
dobles, \sphinxtitleref{"}, a la computadora le da igual cual uses.

Empecemos con el texto mas simple, texto vacío:

\fvset{hllines={, ,}}%
\begin{sphinxVerbatim}[commandchars=\\\{\}]
\PYG{l+s+s2}{\PYGZdq{}\PYGZdq{}}
\end{sphinxVerbatim}

\fvset{hllines={, ,}}%
\begin{sphinxVerbatim}[commandchars=\\\{\}]
\PYG{o}{\PYGZlt{}} \PYG{l+s+s2}{\PYGZdq{}}\PYG{l+s+s2}{\PYGZdq{}}
\end{sphinxVerbatim}

En Firefox cuando escribo con comillas simples me lo muestra de vuelta en
comillas dobles, así que parece que tiene una preferencia :)

\fvset{hllines={, ,}}%
\begin{sphinxVerbatim}[commandchars=\\\{\}]
\PYG{l+s+s1}{\PYGZsq{}\PYGZsq{}}
\end{sphinxVerbatim}

\fvset{hllines={, ,}}%
\begin{sphinxVerbatim}[commandchars=\\\{\}]
\PYG{o}{\PYGZlt{}} \PYG{l+s+s2}{\PYGZdq{}}\PYG{l+s+s2}{\PYGZdq{}}
\end{sphinxVerbatim}

El texto mas común que vas a encontrar en \sphinxhref{https://es.wikipedia.org/wiki/Hola\_mundo}{introducciones a programación}:

\fvset{hllines={, ,}}%
\begin{sphinxVerbatim}[commandchars=\\\{\}]
\PYG{l+s+s1}{\PYGZsq{}Hola mundo!\PYGZsq{}}
\end{sphinxVerbatim}

\fvset{hllines={, ,}}%
\begin{sphinxVerbatim}[commandchars=\\\{\}]
\PYG{o}{\PYGZlt{}} \PYG{l+s+s2}{\PYGZdq{}}\PYG{l+s+s2}{Hola mundo!}\PYG{l+s+s2}{\PYGZdq{}}
\end{sphinxVerbatim}

Si queremos tener comillas dentro del texto, tenemos que "escaparlas" con una
barra invertida antes de la comilla si es del mismo tipo que estamos usando
para envolver el texto, así la computadora sabe que no se termina el texto aun,
esta es una buena razón para usar un tipo de comillas sobre el otro, para evitar
tener que escapar las comillas internas:

\fvset{hllines={, ,}}%
\begin{sphinxVerbatim}[commandchars=\\\{\}]
\PYG{l+s+s1}{\PYGZsq{}Esto es \PYGZdq{}texto\PYGZdq{}\PYGZsq{}}
\end{sphinxVerbatim}

\fvset{hllines={, ,}}%
\begin{sphinxVerbatim}[commandchars=\\\{\}]
\PYG{o}{\PYGZlt{}} \PYG{l+s+s2}{\PYGZdq{}}\PYG{l+s+s2}{Esto es }\PYG{l+s+se}{\PYGZbs{}\PYGZdq{}}\PYG{l+s+s2}{texto}\PYG{l+s+se}{\PYGZbs{}\PYGZdq{}}\PYG{l+s+s2}{\PYGZdq{}}
\end{sphinxVerbatim}

Si queremos usar el mismo tipo de comillas las tenemos que escapar:

\fvset{hllines={, ,}}%
\begin{sphinxVerbatim}[commandchars=\\\{\}]
\PYG{l+s+s2}{\PYGZdq{}Esto es \PYGZbs{}\PYGZdq{}texto\PYGZbs{}\PYGZdq{}\PYGZdq{}}
\end{sphinxVerbatim}

\fvset{hllines={, ,}}%
\begin{sphinxVerbatim}[commandchars=\\\{\}]
\PYG{o}{\PYGZlt{}} \PYG{l+s+s2}{\PYGZdq{}}\PYG{l+s+s2}{Esto es }\PYG{l+s+se}{\PYGZbs{}\PYGZdq{}}\PYG{l+s+s2}{texto}\PYG{l+s+se}{\PYGZbs{}\PYGZdq{}}\PYG{l+s+s2}{\PYGZdq{}}
\end{sphinxVerbatim}

Igual para comillas simples:

\fvset{hllines={, ,}}%
\begin{sphinxVerbatim}[commandchars=\\\{\}]
\PYG{l+s+s2}{\PYGZdq{}Esto es \PYGZsq{}texto\PYGZsq{}\PYGZdq{}}
\end{sphinxVerbatim}

\fvset{hllines={, ,}}%
\begin{sphinxVerbatim}[commandchars=\\\{\}]
\PYG{o}{\PYGZlt{}} \PYG{l+s+s2}{\PYGZdq{}}\PYG{l+s+s2}{Esto es }\PYG{l+s+s2}{\PYGZsq{}}\PYG{l+s+s2}{texto}\PYG{l+s+s2}{\PYGZsq{}}\PYG{l+s+s2}{\PYGZdq{}}
\end{sphinxVerbatim}

\fvset{hllines={, ,}}%
\begin{sphinxVerbatim}[commandchars=\\\{\}]
\PYG{l+s+s1}{\PYGZsq{}Esto es \PYGZbs{}\PYGZsq{}texto\PYGZbs{}\PYGZsq{}\PYGZsq{}}
\end{sphinxVerbatim}

\fvset{hllines={, ,}}%
\begin{sphinxVerbatim}[commandchars=\\\{\}]
\PYG{o}{\PYGZlt{}} \PYG{l+s+s2}{\PYGZdq{}}\PYG{l+s+s2}{Esto es }\PYG{l+s+s2}{\PYGZsq{}}\PYG{l+s+s2}{texto}\PYG{l+s+s2}{\PYGZsq{}}\PYG{l+s+s2}{\PYGZdq{}}
\end{sphinxVerbatim}


\section{Verdadero y Falso}
\label{\detokenize{datos-con-javascript:verdadero-y-falso}}
En programación se toman muchas decisiones, esas decisiones se toman según
condiciones, las cuales evalúan a verdadero o falso, y según el resultado
decidimos hacer una cosa, otra, o ninguna.

Debido a que esto es algo muy común, existe un tipo de dato muy simple, que
se llama \sphinxhref{https://es.wikipedia.org/wiki/Tipo\_de\_dato\_l\%C3\%B3gico}{Lógico} pero
lo vas a encontrar usualmente mencionado como booleano (pronunciado buleano) o directamente en ingles, bool (pronunciado bul),
en honor a \sphinxhref{https://es.wikipedia.org/wiki/George\_Boole}{George Boole}

Hay solo dos valores para este tipo de datos:

Verdadero:

\fvset{hllines={, ,}}%
\begin{sphinxVerbatim}[commandchars=\\\{\}]
\PYG{k+kc}{true}
\end{sphinxVerbatim}

\fvset{hllines={, ,}}%
\begin{sphinxVerbatim}[commandchars=\\\{\}]
\PYG{o}{\PYGZlt{}} \PYG{n}{true}
\end{sphinxVerbatim}

Falso:

\fvset{hllines={, ,}}%
\begin{sphinxVerbatim}[commandchars=\\\{\}]
\PYG{k+kc}{false}
\end{sphinxVerbatim}

\fvset{hllines={, ,}}%
\begin{sphinxVerbatim}[commandchars=\\\{\}]
\PYG{o}{\PYGZlt{}} \PYG{n}{false}
\end{sphinxVerbatim}


\section{Indicando ausencia de datos}
\label{\detokenize{datos-con-javascript:indicando-ausencia-de-datos}}
Como hacemos si queremos indicar que a un dato no lo tenemos?

Un tal Tony Hoare se pregunto lo mismo en 1965 y tuvo una idea, un tipo de dato
al cual luego iba a llamar \sphinxhref{https://es.wikipedia.org/wiki/C.\_A.\_R.\_Hoare\#cite\_ref-4}{"El error de los mil millones de dólares"} , este es el tipo
de dato nulo, el cual tiene un solo valor:

\fvset{hllines={, ,}}%
\begin{sphinxVerbatim}[commandchars=\\\{\}]
\PYG{k+kc}{null}
\end{sphinxVerbatim}

\fvset{hllines={, ,}}%
\begin{sphinxVerbatim}[commandchars=\\\{\}]
\PYG{o}{\PYGZlt{}} \PYG{n}{null}
\end{sphinxVerbatim}

El problema surge cuando pensamos que vamos a recibir un número, texto u otro
tipo de dato y alguien nos envía un \sphinxtitleref{null} para indicarnos que no tiene ningún
valor, nosotros operamos sobre ese valor asumiendo que tiene un valor y lo que
obtenemos a cambio es un error. Así que primer consejo, intenta minimizar la
cantidad de veces que usas este tipo de datos, aunque muchas veces no quede
alternativa.

Hasta aquí llegamos y ya aprendimos todos los tipos de datos simples en
javascript, ahora pasemos a los tipos de datos "compuestos", es decir, estos
tipos de datos contienen otros datos.


\section{Lista}
\label{\detokenize{datos-con-javascript:lista}}
En nuestro ejemplo del capitulo anterior teníamos una lista de tareas, cada
tarea era un valor de tipo texto.

Una lista se define rodeando sus elementos entre corchetes \sphinxtitleref{{[}} para empezar la
lista y \sphinxtitleref{{]}} para terminarla.

Por razones históricas el tipo de dato lista también suele llamarse en ingles
\sphinxhref{https://es.wikipedia.org/wiki/Vector\_(inform\%C3\%A1tica)}{"array"}.

Empecemos por la lista mas simple, una lista sin elementos:

\fvset{hllines={, ,}}%
\begin{sphinxVerbatim}[commandchars=\\\{\}]
\PYG{p}{[}\PYG{p}{]}
\end{sphinxVerbatim}

Firefox me lo muestra así:

\fvset{hllines={, ,}}%
\begin{sphinxVerbatim}[commandchars=\\\{\}]
\PYG{o}{\PYGZlt{}} \PYG{n}{Array} \PYG{p}{[}\PYG{p}{]}
\end{sphinxVerbatim}

Una lista de un elemento:

\fvset{hllines={, ,}}%
\begin{sphinxVerbatim}[commandchars=\\\{\}]
\PYG{p}{[}\PYG{l+m+mi}{1}\PYG{p}{]}
\end{sphinxVerbatim}

\fvset{hllines={, ,}}%
\begin{sphinxVerbatim}[commandchars=\\\{\}]
\PYG{o}{\PYGZlt{}} \PYG{n}{Array} \PYG{p}{[} \PYG{l+m+mi}{1} \PYG{p}{]}
\end{sphinxVerbatim}

Una lista de dos elementos, separamos cada elemento con una coma \sphinxtitleref{,}:

\fvset{hllines={, ,}}%
\begin{sphinxVerbatim}[commandchars=\\\{\}]
\PYG{p}{[}\PYG{l+m+mi}{1}\PYG{p}{,} \PYG{k+kc}{true}\PYG{p}{]}
\end{sphinxVerbatim}

\fvset{hllines={, ,}}%
\begin{sphinxVerbatim}[commandchars=\\\{\}]
\PYG{o}{\PYGZlt{}} \PYG{n}{Array} \PYG{p}{[} \PYG{l+m+mi}{1}\PYG{p}{,} \PYG{n}{true} \PYG{p}{]}
\end{sphinxVerbatim}

Como veras una lista puede contener cualquier tipo de dato, no hace falta que
todos sean del mismo tipo, veamos un caso extremo con todos los tipos de datos
que aprendimos hasta ahora, te desafío a escribirlo sin cometer un error al
primero intento, yo cometí uno (pista: me olvide de cerrar un corchete :).

\fvset{hllines={, ,}}%
\begin{sphinxVerbatim}[commandchars=\\\{\}]
\PYG{p}{[}\PYG{o}{\PYGZhy{}}\PYG{l+m+mi}{1}\PYG{p}{,} \PYG{l+m+mi}{0}\PYG{p}{,} \PYG{l+m+mi}{1}\PYG{p}{,} \PYG{l+m+mi}{42}\PYG{p}{,} \PYG{l+m+mf}{0.5}\PYG{p}{,} \PYG{k+kc}{true}\PYG{p}{,} \PYG{k+kc}{false}\PYG{p}{,} \PYG{k+kc}{null}\PYG{p}{,} \PYG{l+s+s2}{\PYGZdq{}\PYGZdq{}}\PYG{p}{,} \PYG{l+s+s1}{\PYGZsq{}hola\PYGZsq{}}\PYG{p}{,} \PYG{p}{[}\PYG{p}{]}\PYG{p}{,} \PYG{p}{[}\PYG{p}{[}\PYG{l+s+s2}{\PYGZdq{}si, listas dentro de listas\PYGZdq{}}\PYG{p}{]}\PYG{p}{]}\PYG{p}{]}
\end{sphinxVerbatim}

Firefox me lo muestra indicándome que tiene 12 elementos y me muestra algunos,
no todos, no te preocupes, todos los elementos están ahí, si hago click en la
flecha \sphinxtitleref{\textgreater{}} me expande los elementos así los puedo explorar:

\fvset{hllines={, ,}}%
\begin{sphinxVerbatim}[commandchars=\\\{\}]
\PYGZlt{} Array(12) [ \PYGZhy{}1, 0, 1, 42, 0.5, true, false, null, \PYGZdq{}\PYGZdq{}, \PYGZdq{}hola\PYGZdq{}, … ]
\end{sphinxVerbatim}


\section{Objeto}
\label{\detokenize{datos-con-javascript:objeto}}
Felicitaciones, llegamos al ultimo tipo de dato de javascript, el objeto,
llamado de muchas formas, en otro lado lo veras como "map", "hash", "record",
en javascript se lo suele llamar objeto, así que vamos a seguir nombrandolo
así.

Un objeto es un tipo de dato que nos permite ponerle nombres a los valores que
contiene, entonces si queremos, por ejemplo, tener un valor que represente un
personaje en una serie y queremos tener su nombre, su tipo y su color, podemos
hacerlo con un objeto, pero no nos adelantemos, empecemos con el objeto mas
simple, uno que no tiene ningún valor, sí, aunque no lo creas es muy útil.

\fvset{hllines={, ,}}%
\begin{sphinxVerbatim}[commandchars=\\\{\}]
\PYG{p}{\PYGZob{}}\PYG{p}{\PYGZcb{}}
\end{sphinxVerbatim}

Por razones que aprenderemos mas adelante, firefox se piensa que estoy
intentando hacer otra cosa y me devuelve:

\fvset{hllines={, ,}}%
\begin{sphinxVerbatim}[commandchars=\\\{\}]
\PYG{o}{\PYGZlt{}} \PYG{n}{undefined}
\end{sphinxVerbatim}

Quizás cuando pruebes esto ya no se comporte así, Chrome no se confunde y devuelve:

\fvset{hllines={, ,}}%
\begin{sphinxVerbatim}[commandchars=\\\{\}]
\PYG{o}{\PYGZlt{}} \PYG{p}{\PYGZob{}}\PYG{p}{\PYGZcb{}}
\end{sphinxVerbatim}

Te preguntaras que es ese \sphinxtitleref{undefined} que devolvió?

Es un tipo de dato que no mencione hasta ahora porque es un valor que
representa la ausencia de valores, no es como null que es un valor que dice "no
hay valor", undefined es un poco mas confuso y abstracto y es usado cuando una
operación no devuelve ningún valor, o cuando por ejemplo intentamos obtener un
elemento en una lista vacía o un campo que no existe en un objeto, en ese caso
null no es útil porque ese campo no esta definido como null, simplemente no
esta definido, en esos casos obtenemos a su primo: \sphinxtitleref{undefined}.

No hace falta que entiendas el significado de undefined aun, ya vas a
encontrartelo mas seguido de lo deseado.

Sigamos con un objeto con un solo campo:

\fvset{hllines={, ,}}%
\begin{sphinxVerbatim}[commandchars=\\\{\}]
\PYG{p}{\PYGZob{}}\PYG{l+s+s2}{\PYGZdq{}nombre\PYGZdq{}}\PYG{o}{:} \PYG{l+s+s2}{\PYGZdq{}Bob\PYGZdq{}}\PYG{p}{\PYGZcb{}}
\end{sphinxVerbatim}

\fvset{hllines={, ,}}%
\begin{sphinxVerbatim}[commandchars=\\\{\}]
\PYG{o}{\PYGZlt{}} \PYG{p}{\PYGZob{}}\PYG{n}{nombre}\PYG{p}{:} \PYG{l+s+s2}{\PYGZdq{}}\PYG{l+s+s2}{Bob}\PYG{l+s+s2}{\PYGZdq{}}\PYG{p}{\PYGZcb{}}
\end{sphinxVerbatim}

Para tener mas de un campo separamos los pares clave valor (los recordaras de CSS) con comas:

\fvset{hllines={, ,}}%
\begin{sphinxVerbatim}[commandchars=\\\{\}]
\PYG{p}{\PYGZob{}}\PYG{l+s+s2}{\PYGZdq{}nombre\PYGZdq{}}\PYG{o}{:} \PYG{l+s+s2}{\PYGZdq{}Bob\PYGZdq{}}\PYG{p}{,} \PYG{l+s+s2}{\PYGZdq{}color\PYGZdq{}}\PYG{o}{:} \PYG{l+s+s2}{\PYGZdq{}amarillo\PYGZdq{}}\PYG{p}{,} \PYG{l+s+s2}{\PYGZdq{}tipo\PYGZdq{}}\PYG{o}{:} \PYG{l+s+s2}{\PYGZdq{}esponja\PYGZdq{}}\PYG{p}{\PYGZcb{}}
\end{sphinxVerbatim}

\fvset{hllines={, ,}}%
\begin{sphinxVerbatim}[commandchars=\\\{\}]
\PYG{o}{\PYGZlt{}} \PYG{p}{\PYGZob{}}\PYG{n}{nombre}\PYG{p}{:} \PYG{l+s+s2}{\PYGZdq{}}\PYG{l+s+s2}{Bob}\PYG{l+s+s2}{\PYGZdq{}}\PYG{p}{,} \PYG{n}{color}\PYG{p}{:} \PYG{l+s+s2}{\PYGZdq{}}\PYG{l+s+s2}{amarillo}\PYG{l+s+s2}{\PYGZdq{}}\PYG{p}{,} \PYG{n}{tipo}\PYG{p}{:} \PYG{l+s+s2}{\PYGZdq{}}\PYG{l+s+s2}{esponja}\PYG{l+s+s2}{\PYGZdq{}}\PYG{p}{\PYGZcb{}}
\end{sphinxVerbatim}

Listo! ya sabes todos los tipos de datos necesarios para programar en
javascript, y de paso aprendiste sobre \sphinxhref{https://es.wikipedia.org/wiki/JSON}{JSON} (pronunciado yeison), que es un
formato que nos permite transmitir datos entre computadoras, el cual se extrajo
de los tipos de datos de javascript, así que cuando alguien te diga que un
programa genera/recibe/produce datos en JSON, ya sabes a que se refieren, y
ahora podes leerlo y escribirlo.

También deberías entender y poder escribir los datos iniciales en el capitulo
anterior.



\renewcommand{\indexname}{Índice}
\printindex
\end{document}